%%%%%%%%%%%%%%%%%%%%%%%%%%%%%%%%%%%%%%%%%%%%%%%%%%%%%%%%%%%%%%%%%%%%%%%%%%%%%%%%
% fontes.tex
%
% Arquivo de configuração de packages para uso o LaTeX, conforme minhas
% preferências e modelos pessoais.
%
% Para maiores informações, visite:
%    https://github.com/abrantesasf/latex
%
% NÃO ALTERE SE NÃO SOUBER O QUE ESTÁ FAZENDO!
%%%%%%%%%%%%%%%%%%%%%%%%%%%%%%%%%%%%%%%%%%%%%%%%%%%%%%%%%%%%%%%%%%%%%%%%%%%%%%%%


%%%%%%%%%%%%%%%%%%%%%%%%%%%%%%%%%%%%%%%%%%%%%%%%%%%%%%%%%%%%%%%%%%%%%%%%%%%%%%%%
%%% Configurações de encodings e fontes:
\ifxetex
  % Se usar XeLaTeX, usa fontes específicas:
  %\usepackage[tuenc,no-math]{fontspec}
  \usepackage[tuenc]{fontspec}
  \setmainfont{equity-text-a-regular.otf}[
    Path           = /home/abrantesasf/.local/share/fonts/ ,
    BoldFont       = equity-text-a-bold.otf                ,
    ItalicFont     = equity-text-a-italic.otf              ,
    BoldItalicFont = equity-text-a-bold-italic.otf]
  \setsansfont{equity-text-b-regular.otf}[
    Path           = /home/abrantesasf/.local/share/fonts/ ,
    BoldFont       = equity-text-b-bold.otf                ,
    ItalicFont     = equity-text-b-italic.otf              ,
    BoldItalicFont = equity-text-b-bold-italic.otf]      
  \setmonofont{triplicate-t4-regular.otf}[
    Path           = /home/abrantesasf/.local/share/fonts/ ,
    BoldFont       = triplicate-t4-bold.otf                ,
    ItalicFont     = triplicate-t4-italic.otf              ,
    BoldItalicFont = triplicate-t4-bold-italic.otf]
  \usepackage{fontawesome5}
\else
  % Se não for XeLaTeX, vai com as normais mesmo:
  \usepackage[T1]{fontenc}
  \usepackage[utf8]{inputenc}
  \usepackage{lmodern}
  % Altera a fonte padrão do documento (nem todas funcionam em modo math):
  %   phv = Helvetica
  %   ptm = Times
  %   ppl = Palatino
  %   pbk = bookman
  %   pag = AdobeAvantGarde
  %   pnc = Adobe NewCenturySchoolBook
  %\renewcommand{\familydefault}{ppl}
\fi
