%%%%%%%%%%%%%%%%%%%%%%%%%%%%%%%%%%%%%%%%%%%%%%%%%%%%%%%%%%%%%%%%%%%%%%%%%%%%%%%%
% matematica.tex
%
% Arquivo de configuração de packages para uso o LaTeX, conforme minhas
% preferências e modelos pessoais.
%
% Para maiores informações, visite:
%    https://github.com/abrantesasf/latex
%
% NÃO ALTERE SE NÃO SOUBER O QUE ESTÁ FAZENDO!
%%%%%%%%%%%%%%%%%%%%%%%%%%%%%%%%%%%%%%%%%%%%%%%%%%%%%%%%%%%%%%%%%%%%%%%%%%%%%%%%


%%%%%%%%%%%%%%%%%%%%%%%%%%%%%%%%%%%%%%%%%%%%%%%%%%%%%%%%%%%%%%%%%%%%%%%%%%%%%%%%
%%% Carrega bibliotecas e símbolos matemáticos, fontes adicionais e configura
%%% algumas outras opções
\usepackage{amsmath}
\usepackage{amssymb}
\usepackage{amsthm}
\usepackage{amsfonts}
\usepackage{mathrsfs}
\usepackage{proof}
\usepackage{siunitx}
  \sisetup{group-separator = {\,}}
  \sisetup{group-digits = {integer}}
  \sisetup{output-decimal-marker = {,}}
  \sisetup{separate-uncertainty}
  \sisetup{multi-part-units = single}
  \sisetup{binary-units = true}
  \sisetup{list-final-separator = { e }}
\usepackage{syllogism}
  \setsylpuncpa{}
  \setsylpuncpb{}
  \setsylpuncc{}
  \setsylergosign{}
\usepackage{bm}
\usepackage{cancel}
\usepackage{esvect}
\usepackage{mathtools}
\usepackage{icomma}
\usepackage{nicefrac}
%\usepackage{units}

% Altera separador decimal via comando, se necessário (prefira o siunitx):
%\mathchardef\period=\mathcode`.
%\DeclareMathSymbol{.}{\mathord}{letters}{"3B}

% Declara grau Fahrenheit
\DeclareSIUnit{\degreeFahrenheit}{\unit{\degree}F}


%%%%%%%%%%%%%%%%%%%%%%%%%%%%%%%%%%%%%%%%%%%%%%%%%%%%%%%%%%%%%%%%%%%%%%%%%%%%%%%%
%%% Definições para teoremas, etc.

% Para article:
\makeatletter
\@ifclassloaded{article}{
  \theoremstyle{definition}
  \newtheorem{definicao}{Definição}[section]
  \newtheorem{conjecture}{Conjectura}[section]
  \newtheorem{teorema}{Teorema}[section]
  \newtheorem{lemma}{Lema}[section]
  \newtheorem{corolario}{Corolário}[section]
  \theoremstyle{remark}
  \newtheorem*{nota}{Nota}
  \newtheorem*{observacao}{Observação:}
}{}

% Para book:
\@ifclassloaded{book}{
  \theoremstyle{definition}
  \newtheorem{definicao}{Definição}[chapter]
  \newtheorem{conjecture}{Conjectura}[chapter]
  \newtheorem{teorema}{Teorema}[chapter]
  \newtheorem{lemma}{Lema}[chapter]
  \newtheorem{corolario}{Corolário}[chapter]
  \theoremstyle{remark}
  \newtheorem*{nota}{Nota}
  \newtheorem*{observacao}{Observação:}
}{}
\makeatother

