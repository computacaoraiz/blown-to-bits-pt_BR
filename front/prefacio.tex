%%%%%%%%%%%%%%%%%%%%%%%%%%%%%%%%%%%%%%%%%%%%%%%%%%%%%%%%%%%%%%%%%%%%%%%%%%%%%%%%
\chapter{Prefácio}
\label{chap:prefacio}

Há mais de uma década, decidimos escrever este livro porque a revolução digital
estava produzindo mudanças profundas em quase todos os aspectos de nossas vidas,
e escolhas sociais inteligentes requeriam o entendimento dos princípios
subjacentes da tecnologia digital e suas implicações para as instituições
humanas. O que faltava não era um livro sobre como os computadores funcionam,
mas um que fizesse uma análise da revolução digital a partir de uma perspectiva
humana.

Este livro se manteve surpreendentemente atual por mais de uma década. O
progresso contínuo da tecnologia e sua ampla adoção, entretanto, exigiam uma
segunda edição. Alguns tópicos merecem menção especial.

Não é surpresa que o impacto da tecnologia na privacidade --- ou na falta dela
--- tenha se acelerado. A tecnologia de reconhecimento facial era incipiente
quando escrevemos este livro pela primeira vez, e agora é onipresente. Os
aplicativos de celular rastreiam todos os nossos movimentos. Softwares de
reconhecimento de voz são amplamente utilizados, tanto por governos quanto por
organizações comerciais (veja o Capítulo 2, ``Despido à Luz do Sol: Privacidade
Perdida, Privacidade Abandonada'').

As aplicações de inteligência artificial são agora comuns. Não há necessidade de
uma coleção de CDs (se você ainda se lembrar o que são CDs). Basta dizer ao seu
dispositivo o que gostaria de ouvir. Ou dizer ``Siri, quando eu chegar à
\ingles{Home Depot} em Waltham, me lembre de comprar algumas
lâmpadas''. Assistentes automatizados estão integrados aos controles remotos de
TVs e geladeiras (veja o Capítulo 9, ``A Próxima Fronteira: IA e o Mundo dos
Bits do Futuro'').

Em 2008, o Facebook era popular para se conectar com amigos, e o Twitter mal
havia começado. Hoje, essas e outras plataformas de mídia social têm um impacto
profundo na sociedade, facilitando revoltas sociais, influenciando eleições e
oferecendo palanques e megafones para políticos (veja o Capítulo 3, ``Quem
Possui Sua Privacidade? A Comercialização de Dados Pessoais'').

Nenhum de nós poderia prever a atenção que a pandemia do coronavírus de 2020
traria para as implicações e impactos da revolução digital. Em questão de
semanas, as escolas, do jardim de infância até os programas de
pós-graduação, se tornaram ``virtuais''. Crianças de seis anos dominaram
videoconferências, e a compra online se tornou o principal meio de suprir as
necessidades diárias. O trabalho remoto se tornou a norma, algo que teria sido
difícil de imaginar uma década antes e impossível uma década antes disso.

Estamos otimistas de que um entendimento mais profundo do ``o que é'' e de
``como funciona'' o mundo digital levará a escolhas sábias à medida que a
revolução continua.

Todos os dias, bilhões de fotografias, notícias, músicas, raios-X, programas de
TV, chamadas telefônicas e e-mails são espalhados pelo mundo como sequências de
zeros e uns: bits. Listas telefônicas, jornais, CDs, cartas escritas à mão --- e
privacidade --- são relíquias da era pré-digital.

Não podemos escapar dessa explosão de informações digitais, e poucos de nós
querem isso --- os benefícios são muito sedutores. A tecnologia digital
possibilitou inovações, colaborações, entretenimento e participação democrática
sem precedentes.

Mas as mesmas maravilhas da engenharia estão rompendo suposições centenárias
sobre privacidade, identidade, liberdade de expressão e controle pessoal, à
medida que mais e mais detalhes de nossas vidas são capturados como dados
digitais.

Você pode controlar quem vê todas as informações pessoais sobre você? Alguma
coisa pode ser confidencial hoje em dia, quando nada mais parece ser privado? A
Internet deve ser censurada da mesma forma que o rádio e a TV? Quando você
pesquisa algo, quem decide o que mostrar para você? Como saber o que é
``verdadeiro'' quando vivemos em uma câmara de ecos digitais com uma infinidade
de fontes de informação --- e desinformação? Você ainda tem liberdade de
expressão no mundo digital? Você tem voz na definição de políticas
governamentais ou corporativas sobre tudo isso? No mundo da 
Inteligência Artificial, como sabemos por que as máquinas decidem fazer as
coisas? Um pequeno grupo de corporações poderosas influencia o que sabemos e
como percebemos o mundo? Já perdemos o controle?

\ingles{Blown to Bits} guia você pelo cenário digital, oferecendo respostas
provocativas a essas perguntas por meio de histórias reais
intrigantes. Compreender o potencial e as armadilhas deste mundo transformado é
informação essencial para todos.

Este livro é um chamado para as consequências humanas da explosão digital.
