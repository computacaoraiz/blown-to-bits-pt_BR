\documentclass{book}
\usepackage{url}
\newcommand{\ingles}[1]{\textit{#1}}

\begin{document}

\chapter[Os Bits Estão No Ar]{Os Bits Estão No Ar\\\large\textit{Velhas Metáforas, Novas Tecnologias, e Liberdade de Expressão}}
\label{cap8:os}

\section{Censurando o candidato}
\label{cap8:os-censurando}
Em 8 de outubro de 2016, apenas seis semanas antes da eleição presidencial dos
Estados Unidos, o \ingles{The Washington Post} lançou uma bomba\footnote{David A.
Fahrenthold, ``Trump Recorded Having Extremely Lewd Conversation About Women in
2005'', The Washington Post, October 8, 2016,
\url{https://www.washingtonpost.com/politics/trump-recorded-having-extremely-lewd-conversation-about-women-in-2005/2016/10/07/3b9ce776-8cb4-11e6-bf8a-3d26847eeed4_story.html}.}. 
O Post havia recebido uma gravação de vídeo na qual o então candidato Donald Trump
usava linguagem grosseira ao se vangloriar de agressão sexual contra mulheres.
Conforme relatado na matéria, o Sr. Trump disse:

\begin{quote}
``Eu tentei e \verb|f---| ela. Ela era casada.''\ldots ``E quando você é uma estrela,
eles deixam você fazer isso\ldots Você pode fazer qualquer coisa\ldots Agarrá-las
pela \verb|b----a|''.
\end{quote}

Foi assim que o \ingles{Post} o publicou, com hifens substituindo certas letras,
mas com contexto suficiente para que as palavras pudessem ser reconstruídas de
forma inequívoca. A matéria continha um link para o vídeo completo e não expurgado,
para que qualquer pessoa com conexão à internet pudesse ouvir o que o Sr. Trump
mais tarde caracterizou como sua ``conversa de vestiário''.

O \ingles{The New York Times} optou por imprimir as palavras ofensivas na íntegra.\footnote{
Alexander Burns et al., ``Donald Trump Apology Caps Day of Outrage over Lewd
Tape'', The New York Times, October 7, 2016,
\url{https://www.nytimes.com/2016/10/08/us/politics/donald-trump-women.html}; Al
Tompkins, ``As Profanity-Laced Video Leaks, Outlets Grapple with Trump's Language'',
Poynter, October 7, 2016,
\url{https://www.poynter.org/reporting-editing/2016/as-profanity-laced-videoleaks-outlets-grapple-with-trumps-language/}.}
A rede de televisão a cabo CNN exibiu o vídeo completo, mas seus repórteres usaram
``a palavra F'' e ``a palavra B'' onde o Post havia usado hifens. Essas decisões
eram questões de julgamento editorial e poderiam ter seguido qualquer direção.O
editor do Post, Marty Baron, disse: ``Tomamos nossos melhores julgamentos ao
equilibrar o gosto em relação à clareza do que foi dito'', enquanto a editora do
\ingles{Times}, Carolyn Ryan, explicou: ``decidimos que as palavras vulgares em
si eram noticiosas e omiti-las ou descrevê-las de outra forma teria sido menos
honesto''.

Esses veículos de mídia poderiam chegar a conclusões diferentes sobre o que
reportar, mas nenhuma lei ou regulamentação governamental eram relevantes.
A Primeira Emenda protegia o direito dos jornais e das emissoras de televisão a
cabo de reportarem sobre a fita como desejavam.

Mas não as emissoras de televisão aberta. Embora a maioria dos americanos agora
assista aos seus programas da ABC, CBS e NBC por meio de caixas de TV a cabo ou
pela Internet, os termos ``transmissão'' e ``via ar'' ainda diferenciam as estações
que podem ser recebidas por antenas captando sinais de rádio emitidos por antenas
transmissoras gigantes. E essas estações estão sujeitas à regulamentação da
Comissão Federal de Comunicações (FCC): Nenhuma linguagem indecente ou profana
durante as horas em que as crianças podem ouvir---e a linguagem do Sr. Trump
provavelmente ultrapassou esse limite. De acordo com a FCC, conteúdo indecente
retrata órgãos ou atividades sexuais ou excretoras de uma maneira que não atende
ao teste Miller para obscenidade, e conteúdo profano inclui linguagem ``extremamente
ofensiva'' que é considerada um incômodo público.

As emissoras de transmissão censuraram as palavras ofensivas quando exibiram o
vídeo. Elas não tiveram escolha: poderiam ter sido sujeitas a multas substanciais
da FCC ou até mesmo à perda de suas licenças de transmissão se não o tivessem
feito.

Sob a Primeira Emenda, o governo geralmente não atua para restringir o discurso.
Não pode impor seus julgamentos editoriais sobre jornais, nem mesmo para aumentar
a gama de informações disponíveis para os leitores. A Suprema Corte considerou
inconstitucional uma lei da Flórida que garantia aos candidatos políticos um
simples ``direito de resposta'' a ataques de jornais contra eles. As emissoras de
televisão a cabo não precisavam se preocupar com reclamações à FCC, embora algumas
tenham sido apresentadas; esse meio está amplamente além do alcance da censura
governamental.

No entanto, em 2016, uma agência do governo federal estava censurando palavras
na televisão aberta, usando regras que cobriam até mesmo os comentários de um
candidato presidencial no meio de uma campanha política. Estamos em uma era de
sensibilidade aumentada em relação à programação que as crianças podem ver, mas
os americanos em geral continuam sendo contrários a ter o governo como ``babá''
de seus programas de televisão. Por que a FCC tem o poder de regular o que pode
ser dito pelas ondas do ar?

\section{Como a transmissão se tornou regulamentada}
\label{cap8:os-como}
A FCC (Comissão Federal de Comunicações) obteve sua autoridade sobre o que é
dito em transmissões de rádio e televisão quando havia menos formas de
distribuição de informações. A teoria era que as vias públicas de transmissão
eram escassas e o governo precisava garantir que fossem usadas no interesse
público. Conforme o rádio e a televisão se tornaram universalmente acessíveis,
surgiu uma segunda justificativa para a regulamentação governamental do discurso
nas transmissões. Porque as mídias de transmissões têm ``uma presença
singularmente pervasiva na vida de todos os americanos'', como afirmou a Suprema
Corte em 1978, o governo tinha um interesse especial em proteger um público
indefeso de conteúdo de rádio e televisão objetável.

A explosão nas tecnologias de comunicação confundiu ambas as justificativas. Na
era digital, há muito mais maneiras de os dados alcançarem o consumidor, então o
rádio e a televisão não são mais únicos em sua pervasividade. Com tecnologia
mínima, qualquer pessoa pode sentar em casa ou em uma cafeteria e escolher entre
bilhões de páginas da web e dezenas de milhões de blogs. O locutor Howard Stern
deixou o rádio de transmissão terrestre para a rádio via satélite, onde a FCC não tem
autoridade para regular o que ele diz. Quase 90\% dos  telespectadores americanos
obtêm seu sinal de TV por meio de cabo ou satélite igualmente não regulamentados,
em vez de antenas no  telhado.\footnote{``Rise in Broadband Only Television
Homes'', Informitv, July 12, 2017,
\url{https://informitv.com/2017/07/12/rise-in-broadband-only-television-homes/}}
\ingles{Feeds RSS} fornecem informações atualizadas para milhões de usuários de
celulares em movimento. Estações de rádio e canais de televisão hoje não são mais
escassos nem singularmente pervasivos.\\

%%%%%%%%%%%%%%% BLOCO DO LADO %%%%%%%%%%%%%%%
Na era digital, há muitas mais maneiras para os dados alcançarem o consumidor,
então o rádio e a televisão de transmissão não são mais únicos em sua pervasividade.\\
%%%%%%%%%%%%%%% BLOCO DO LADO %%%%%%%%%%%%%%%

Para que o governo proteja as crianças de todas as informações ofensivas que
chegam por meio de qualquer meio de comunicação, sua autoridade teria que ser
ampliada significativamente e atualizada continuamente. Embora algumas propostas
tenham sido feitas, o Congresso não aprovou nenhuma lei que estendesse as
regulamentações de indecência da FCC para mídia de satélite e televisão a cabo.
O que é exibido na TV a cabo e via satélite é limitado pelo que os espectadores
e anunciantes aceitarão, mas não pelo que qualquer autoridade governamental
possa ditar.

A explosão nas comunicações levanta outra possibilidade, no entanto. Se
praticamente qualquer pessoa pode agora enviar informações que muitas pessoas
podem receber, talvez o interesse do governo em restringir as transmissões deva
ser menor do que costumava ser, e não maior. Na ausência de escassez, talvez o
governo não deva ter mais autoridade sobre o que é dito no rádio e na TV do que
sobre o que é impresso nos jornais. Nesse caso, em vez de expandir a autoridade
de censura da FCC, o Congresso deveria eliminá-la completamente, assim como a
Suprema Corte encerrou a regulamentação do conteúdo de jornais na Flórida.

As partes que já possuem lugares no dial de rádio e na programação de TV
argumentam que o espectro---as vias públicas de transmissão---continua sendo um
recurso limitado que requer proteção governamental. A teoria é que ninguém está
criando mais espectro de rádio, e ele precisa ser usado no interesse público.

Mas olhe ao seu redor. Ainda existem apenas algumas estações no dial AM e FM.
Mas milhares, talvez dezenas de milhares, de comunicações de rádio estão
ocorrendo no ar ao seu redor. A maioria dos americanos carrega rádios
bidirecionais em seus bolsos---dispositivos que chamamos de celulares. Para ser
preciso, carregamos celulares em nossas mãos, já que muitos de nós preferem
arriscar bater em postes de luz do que atrasar a leitura de nossas mensagens de
texto por apenas alguns segundos. Se você estiver ouvindo música em um fone de
ouvido Bluetooth e navegando na Web por uma conexão Wi-Fi, são mais duas conexões
de rádio que você está usando. Rádios e televisores poderiam ser muito mais
inteligentes do que são atualmente e poderiam fazer um uso melhor das vias
públicas de transmissão, assim como os celulares fazem.

Os avanços na engenharia minaram a capacidade do governo de anular a Primeira
Emenda na rádio e televisão. A Constituição exige, sob essas circunstâncias
alteradas, que o governo pare de regular o discurso verbal. De fato, quando a
Suprema Corte dos Estados Unidos anulou as multas que a FCC estava impondo quando
celebridades proferiam um ``palavrão fugaz'' em seus comentários transmitidos ao
vivo, ela restringiu o escopo de sua decisão, mas insinuou que poderia ser hora
de reavaliar toda a questão da censura de transmissões.

Como argumento científico, a alegação de que o espectro é necessariamente escasso
agora é muito fraca. No entanto, essa visão ainda é fortemente defendida pela
própria indústria que está sendo regulamentada. Os titulares de licenças
incumbentes---estações de transmissão e redes existentes---têm incentivo para
proteger seu ``território'' no espectro contra qualquer risco, real ou imaginário,
de que seus sinais possam ser corrompidos. Ao desencorajar a inovação tecnológica,
os incumbentes podem limitar a concorrência e evitar investimentos de capital.
Esses fios estranhamente entrelaçados---o interesse do governo na escassez
artificial para justificar a regulamentação do discurso e o interesse dos
titulares de licenças incumbentes na escassez artificial para limitar a
concorrência e os custos---hoje prejudicam tanto a criatividade cultural quanto
a tecnológica, em detrimento da sociedade.

Para entender as forças confluentes que criaram o mundo da censura de rádio e
televisão de hoje, é preciso voltar aos inventores da tecnologia.

\subsection{Do telégrafo sem fio para o caos sem fio}
\label{cap8:os-como-do}
Vermelho, laranja, amarelo, verde, azul---as cores do arco-íris---são todas
diferentes, mas ao mesmo tempo são todas iguais. Qualquer criança com uma caixa
de lápis de cor sabe que elas são diferentes. Elas são iguais porque todas são o
resultado da radiação eletromagnética atingindo nossos olhos. A radiação viaja em
ondas que oscilam muito rapidamente. A única diferença física entre o vermelho e
o azul é que as ondas vermelhas oscilam cerca de 450 trilhões de vezes por segundo
e as ondas azuis cerca de 50\% mais rápido.

Como o espectro da luz visível é contínuo, existe uma infinidade de cores entre
o vermelho e o azul. Misturar luz de diferentes frequências cria outras cores---por
exemplo, metade de ondas azuis e metade de ondas vermelhas cria uma tonalidade de
rosa conhecida como magenta, que não aparece no arco-íris.

Na década de 1860, o físico britânico James Clerk Maxwell percebeu que a luz
consiste em ondas eletromagnéticas. Suas equações previram que poderia haver
ondas de outras frequências---ondas que as pessoas não poderiam sentir. De fato,
tais ondas passaram por nós desde o início dos tempos. Elas caem invisivelmente
do sol e das estrelas e irradiam quando ocorrem raios. Ninguém suspeitava de sua
existência até que as equações de Maxwell dissessem que elas deveriam existir. Na
verdade, deve haver todo um espectro de ondas invisíveis de diferentes frequências,
todas viajando com a mesma grande velocidade da luz visível.

Em 1887, a era do rádio começou com uma demonstração de Henrich Hertz. Ele dobrou
um fio em um círculo, deixando uma pequena lacuna entre as duas extremidades.
Quando ele disparou uma grande faísca elétrica a alguns metros de distância, uma
pequena faísca saltou pela lacuna do fio quase completamente circular. A grande
faísca havia provocado uma chuva de ondas eletromagnéticas invisíveis, que
viajaram pelo espaço e causaram o fluxo de corrente elétrica no outro fio. A
pequena faísca era a corrente que completava o circuito. Hertz havia criado a
primeira antena e revelado as ondas de rádio que a atingiram. A unidade de
frequência leva seu nome: Um ciclo por segundo é 1 hertz, ou Hz, para abreviar.
Um kHz (quilohertz) é mil ciclos por segundo, e um MHz (megahertz) é um milhão de
ciclos por segundo. Essas são as unidades nos dial de rádio AM e FM.

Guglielmo Marconi não era matemático nem cientista. Ele era um inventor curioso.
Com apenas 13 anos na época do experimento de Hertz, Marconi passou a próxima
década desenvolvendo, por tentativa e erro, maneiras melhores de criar explosões
de ondas de rádio e antenas para detectá-las a distâncias maiores.

Em 1901, Marconi estava em \ingles{Newfoundland} e recebeu uma única letra em
código Morse transmitida da Inglaterra. Com base nesse sucesso, a \ingles{Marconi
Wireless Telegraph Company} em breve estava possibilitando que navios se
comunicassem entre si e com a costa. Quando o Titanic partiu em sua viagem
fatídica em 1912, estava equipado com equipamentos Marconi. O principal trabalho
dos operadores de rádio do navio era transmitir mensagens pessoais para e de
passageiros, mas eles também receberam pelo menos 20 avisos de outros navios sobre
os icebergs que estavam à frente.\footnote{knjazmilos, ``The Ice Warnings Received
By Titanic'', Titanic-Titanic.Com, June 17, 2019,
\url{http://www.titanic-titanic.com/tag/ice/}}

As palavras ``Telegrafia Sem Fio'' no nome da empresa de Marconi sugerem a maior
limitação do rádio inicial. A tecnologia foi concebida como um dispositivo para
comunicação ponto a ponto. O rádio resolveu o pior problema da telegrafia. Nenhuma
calamidade, sabotagem ou guerra poderia interromper as transmissões sem fio
cortando cabos. Mas havia uma desvantagem compensatória: qualquer pessoa poderia
ouvir. O enorme poder da radiodifusão para alcançar milhares de pessoas ao mesmo
tempo foi inicialmente visto como uma responsabilidade. Quem pagaria para enviar
uma mensagem a outra pessoa quando qualquer um poderia ouvi-la?

Conforme a telegrafia sem fio se tornou popular, outro problema surgiu---um que
moldou o desenvolvimento do rádio e da televisão desde então. Se várias pessoas
estivessem transmitindo simultaneamente na mesma área geográfica, seus sinais não
poderiam ser mantidos separados. O desastre do Titanic demonstrou a confusão que
poderia resultar. Na manhã seguinte ao choque do navio com o iceberg, os jornais
americanos relataram com entusiasmo que todos os passageiros haviam sido salvos
e o navio estava sendo rebocado para a costa. O erro resultou da fusão confusa de
dois segmentos não relacionados de código Morse por um operador de rádio. Um navio
perguntou ``todos os passageiros do Titanic estão seguros?'' E um navio
completamente diferente relatou que estava ``a 300 milhas a oeste do Titanic e
rebocando um tanque de óleo para Halifax''.\footnote{Karl Baarslag, S O S to the
Rescue (Oxford University Press, 1935), p. 72.} Todos os navios tinham rádios e
operadores de rádio. Mas não havia regras ou convenções sobre se, como ou quando
usá-los.

Os ouvintes dos transmissores iniciais de Marconi eram facilmente confundidos
porque não tinham como ``sintonizar'' uma comunicação específica. Apesar de todo
o talento de Marconi em estender o alcance da transmissão, ele estava usando
essencialmente o método de Hertz para gerar ondas de rádio: grandes faíscas. Os
faíscos espalhavam energia eletromagnética por todo o espectro de rádio. A energia
podia ser interrompida e reiniciada para se transformar em pontos e traços, mas
não havia mais nada para controlar. O ruído de um operador de rádio era como o de
qualquer outro. Quando vários transmitiam simultaneamente, resultava em caos.

As muitas cores da luz visível parecem brancas quando todas estão misturadas. Um
filtro de cor permite passar algumas frequências de luz visível, mas não outras.
Se você olhar para o mundo através de um filtro vermelho, tudo terá uma tonalidade
mais clara ou mais escura de vermelho, porque apenas a luz vermelha passará. O
que o rádio precisava era algo semelhante para o espectro de rádio: uma maneira
de produzir ondas de rádio de uma única frequência, ou pelo menos uma faixa
estreita de frequências, e um receptor que pudesse permitir essas frequências e
eliminar o resto. Na verdade, essa tecnologia já existia.

Em 1907, Lee De Forest patenteou uma tecnologia fundamental para a \ingles{De
Forest Radio Telephone Company}---dedicada a enviar voz e até música pelas ondas
de rádio. Quando ele transmitiu Enrico Caruso do \ingles{Metropolitan Opera House}
em Nova York cantando Pagliacci em 13 de janeiro de 1910, o canto alcançou navios
no mar. Amadores se reuniram ao redor de receptores em Nova York e Nova Jersey.
O efeito foi sensacional. Centenas de radiodifusores amadores entraram em ação
nos próximos anos, dizendo ansiosamente o que queriam e tocando a música que
podiam para qualquer pessoa que estivesse ouvindo.\\

%%%%%%%%%%%%%%% BLOCO DO LADO %%%%%%%%%%%%%%%
Sem um entendimento claro sobre quais frequências usar, a comunicação por rádio
era uma questão de tentativa e erro.\\
%%%%%%%%%%%%%%% BLOCO DO LADO %%%%%%%%%%%%%%%

Mas, sem entender claramente quais frequências usar, a comunicação por rádio era
uma questão de tentativa e erro. Mesmo o que o \ingles{The New York Times} descreveu
como as ``ondas de música sem lar'' da transmissão de Caruso entraram em conflito
com outra estação que, ``apesar de todas as súplicas'', insistiu em transmitir na
frequência idêntica de 350 kHz. Algumas pessoas podiam ``capturar o êxtase'' da
voz de Caruso, mas outras só recebiam algum irritante código Morse do outro
radiodifusor: ``Acabei de tomar uma cerveja agora''.\footnote{``Wireless Melody
Jarred'', New York Times, January 14, 1910.}

\subsection{Ondas de Rádio em Seus Canais}
\label{cap8:os-como-ondas}
A emergente indústria de rádio não poderia crescer sob tais condições. Interesses
comerciais se somaram às preocupações da Marinha dos Estados Unidos sobre a
interferência de amadores em suas comunicações marítimas. O desastre do Titanic,
embora tenha tido pouco a ver com as falhas do rádio, catalisou ação governamental.
Em 12 de maio de 1912, William Alden Smith pediu a regulamentação do rádio no
plenário do Senado dos EUA.~``Quando o mundo chora junto por uma perda comum\ldots'',
proclamou o senador, ``por que as nações não deveriam limpar o mar de seus idiomas
conflitantes e regular com sabedoria este novo servo da humanidade?''\footnote{
William Alden Smith, ``United States Senate Inquiry Report'', Titanic Inquiry
Project, May 28, 1912, \url{https://www.titanicinquiry.org/USInq/USReport/AmInqRepSmith01.php}}\\

%%%%%%%%%%%%%%% BLOCO DO LADO %%%%%%%%%%%%%%%
ALTAS FREQUÊNCIAS

Ao longo dos anos, melhorias tecnológicas tornaram possível o uso de frequências
cada vez mais elevadas. As primeiras transmissões de TV eram feitas em frequências
consideradas na época como ``muito altas'' (VHF), porque eram mais elevadas do
que as do rádio AM.~A tecnologia avançou novamente, e mais estações apareceram
em ``frequências ultra altas'' (UHF). A frequência mais alta em uso comercial hoje
é de 77 GHz, ou seja, 77 gigahertz, o que equivale a 77.000 MHz. Em geral, os
sinais de alta frequência enfraquecem com a distância mais do que os sinais de
baixa frequência, e, portanto, são principalmente úteis em ambientes localizados
ou urbanos. As ondas curtas correspondem a frequências altas porque todas as ondas
de rádio viajam à mesma velocidade, que é a velocidade da luz.\\
%%%%%%%%%%%%%%% BLOCO DO LADO %%%%%%%%%%%%%%%

A Lei do Rádio de 1912\footnote{``An Act to Regulate Radio Communication'',
August 13, 1912, \url{http://earlyradiohistory.us/1912act.htm}.} limitou as
transmissões aos titulares de licenças. As licenças de rádio seriam ``concedidas
pelo Secretário de Comércio e Trabalho mediante solicitação''. Ao conceder uma
licença, o secretário especificaria as frequências ``autorizadas para uso pela
estação para a prevenção de interferências e as horas para as quais a estação
está licenciada para funcionar''. A lei reservou para uso governamental as
frequências escolhidas entre cerca de 200 e 500 kHz, que permitiam as comunicações
mais claras em longas distâncias. Os amadores foram empurrados para frequências
de ``ondas curtas'' acima de 1500 kHz, consideradas inúteis por razões tecnológicas.
A frequência de 1000 kHz foi reservada para chamadas de socorro, e as estações
licenciadas eram obrigadas a ouvi-la a cada 15 minutos (a única disposição que
poderia ter ajudado o Titanic, já que os operadores de rádio de um navio próximo
haviam saído de serviço e perderam os apelos de socorro do Titanic). O restante
do espectro poderia ser designado pelo secretário para estações de rádio comerciais
e empresas privadas. Enfatizando a natureza do rádio como ``telegrafia sem fio'',
a lei tornou crime divulgar uma mensagem de rádio para qualquer pessoa que não
fosse seu destinatário pretendido.

Muita coisa mudou desde 1912. O uso das ondas de rádio se tornou mais variado, a
alocação de blocos de espectro mudou e a faixa de frequências utilizáveis cresceu.
A imagem atual de alocação de espectro se transformou em um denso e desorganizado
mosaico, resultado de décadas de decisões da FCC (Comissão Federal de Comunicações)
que lembram o julgamento de Salomão (veja a Figura 8.1). No entanto, ainda é o
governo dos EUA que estipula quais partes do espectro podem ser usadas para quais
propósitos. Ele impede que os usuários interfiram uns com os outros e com as
comunicações governamentais, exigindo que transmitam com potência limitada e
apenas em suas frequências atribuídas. Enquanto não havia muitas estações de rádio,
a promessa implícita na lei de 1912 de que as licenças seriam concedidas ``mediante
solicitação'' não causou problemas. Com os amadores afastados para uma área remota
do espectro de rádio, havia muito espectro disponível para uso comercial, militar
e de segurança.\\

%%%%%%%%%%%%%%%%%%%%%%%%%%
%%%%%%% IMAGEM 8.1 %%%%%%%
%%%%%%%%%%%%%%%%%%%%%%%%%% 
FIGURA 8.1 Alocação de frequências no espectro de rádio dos Estados Unidos. O
espectro de 3 kHz a 300 GHz está disposto da esquerda para a direita e de cima
para baixo, com a escala 10 vezes mais densa em cada fila sucessiva. Por exemplo,
o grande bloco na segunda fila é o dial de rádio AM, com cerca de 1 MHz de largura.
A mesma quantidade de espectro teria aproximadamente 0,00002 polegada de largura
na última fila.\footnote{``United States Frequency Allocation Chart'', National
Telecommunications and Information Administration, 2003,
\url{https://www.ntia.doc.gov/files/ntia/publications/2003-allochrt.pdf}.}\\

Em menos de uma década, essa imagem mudou drasticamente. Em 2 de novembro de 1920,
uma estação de rádio de Detroit transmitiu a eleição de Warren Harding como
presidente dos Estados Unidos, transmitindo para sua pequena audiência de rádio
os resultados que estava recebendo por telégrafo. O rádio já não era apenas uma
forma de comunicação ponto a ponto. Um ano depois, uma estação de rádio de Nova
York transmitiu a Série Mundial entre os Giants e os Yankees, jogada a jogada. A
transmissão esportiva nasceu, com o locutor repetindo de forma entediante as
informações de bola e strike transmitidas por telefone por um repórter de jornal
no campo de beisebol.\footnote{Erik Barnouw, A Tower in Babel (Oxford University
 Press, 1966), 69.}

A compreensão pública das possibilidades cresceu rapidamente. As primeiras cinco
estações de rádio foram licenciadas para transmissão em 1921. Em um ano, havia
670 delas.\footnote{Barnouw, 91.} O número de receptores de rádio aumentou em um
ano de menos de 50.000 para mais de 600.000, talvez um milhão.\footnote{``Asks
Radio Experts to Chart the Ether'', New York Times, February 28, 1922.} As estações
que usavam a mesma frequência na mesma cidade dividiram as horas do dia. A transmissão
de rádio se tornou um negócio lucrativo, mas o crescimento não poderia continuar
indefinidamente.

Em 12 de novembro de 1921, a licença de transmissão da \ingles{Intercity Radio
Co.} na cidade de Nova York expirou. Herbert Hoover, na época o secretário de
Comércio, recusou-se a renová-la, alegando que não havia frequência na qual a
Intercity pudesse transmitir no espaço aéreo da cidade sem interferir com estações
governamentais ou privadas. A Intercity processou Hoover para ter sua licença
restaurada---e venceu.\footnote{Hoover, Secretary of Commerce, v. Intercity Radio
Co, Inc., Decision No. 3766 (52 App.D.C.339, 286 Fed. 1003).} O tribunal afirmou
que Hoover poderia escolher a frequência, mas não tinha discricionariedade para
negar a licença. Como o comitê do Congresso que propôs a Lei do Rádio de 1912
havia colocado, o sistema de licenciamento era ``substancialmente o mesmo que o
usado para documentar mais de 25.000 embarcações mercantes''. A metáfora implícita
era que Hoover deveria acompanhar as estações como navios no oceano. Ele poderia
dizer a eles em quais rotas de navegação deveriam estar, mas não poderia impedi-los
de entrar na água.

A indústria de rádio implorava por ordem. Hoover convocou uma Conferência Nacional
de Rádio em 1922 na tentativa de alcançar um consenso sobre novas regulamentações
antes que o caos se instalasse. O espectro era ``um grande ativo nacional'',
segundo ele, e ``torna-se de interesse público primário dizer quem fará a
transmissão, em que circunstâncias e com que tipo de material''.\footnote{Herbert
Hoover, Memoirs, vol. 2 (The Macmillan Company, 1952), p. 140,
\url{https://hoover.archives.gov/research/ebooks}.}
``A grande massa de assinantes precisa de proteção quanto aos ruídos que preenchem
seus aparelhos'', e as ondas do ar precisam de ``um policial'' para detectar
``abusadores que estão colocando em perigo o tráfego''.\footnote{``Asks Radio
Experts to Chart the Ether''.}

Hoover dividiu o espectro de 550 kHz a 1350 kHz em faixas de 10 kHz---chamadas
``canais'', em consonância com a metáfora náutica---para acomodar mais estações.
``Bandas de guarda'' vazias foram deixadas em cada lado das bandas alocadas porque
os sinais de transmissão inevitavelmente se espalham, reduzindo a quantidade de
espectro utilizável. A persuasão e a conformidade voluntária ajudaram Hoover a
limitar a interferência. À medida que as estações se estabeleciam, encontravam
vantagem em cumprir as prescrições de Hoover. Startups tinham mais dificuldade em
entrar no mercado. Hoover convenceu representantes de um grupo religioso de que,
para advertir sobre o apocalipse iminente, eles deveriam comprar tempo em estações
existentes em vez de construir uma estação própria. Afinal, seu dinheiro renderia
mais dessa forma: em seis meses, após o fim do mundo, eles não teriam mais uso
para um transmissor.\footnote{The Reminiscences of Herbert Clark Hoover, vol.
Radio Unit, Oral History Research Project, 1951, 12.} A eficácia de Hoover fez
com que o Congresso ficasse complacente; o sistema estava funcionando bem o
suficiente sem leis.

Mas à medida que a divisão de frequências se tornava mais precisa, os problemas
pioravam.\footnote{``End Cincinnati Radio Row'', The New York Times, February 15,
1925.} Em 1924, as estações WLW e WMH em Cincinnati transmitiram na mesma frequência
até que Hoover intermediou um acordo para que três estações compartilhassem duas
frequências em intervalos de tempo rotativos. Finalmente, o sistema entrou em
colapso. Em 1925, a \ingles{Zenith Radio Corporation} recebeu uma licença para
usar a frequência de 930 kHz em Chicago, mas apenas nas noites de quinta-feira,
apenas das 22h à meia-noite e apenas se uma estação de Denver não quisesse
transmitir naquele horário. Sem permissão, a Zenith começou a transmitir em 910
kHz, uma frequência que estava mais disponível porque havia sido cedida por
tratado ao Canadá. Hoover multou a Zenith; a Zenith desafiou a autoridade de
Hoover para regular as frequências e venceu na justiça.\footnote{United States v.
Zenith Radio Corporation, 12 F.2d 614 (N.D. Ill. 1926),
\url{https://law.justia.com/cases/federal/district-courts/F2/12/614/1490149/}.}
O secretário então recebeu uma notícia ainda pior do procurador-geral dos Estados
Unidos: a Lei de 1912, redigida antes mesmo de a radiodifusão ser concebida, era
tão ambígua que provavelmente não lhe dava autoridade para regulamentar nada sobre
o rádio de transmissão---frequência, potência ou horário do dia.

Hoover levantou as mãos. Qualquer pessoa poderia iniciar uma estação e escolher
uma frequência---havia 600 pedidos pendentes---mas ao fazer isso, estavam
``prosseguindo inteiramente por sua própria conta e risco''.\footnote{``Hoover
Asks Help to Avoid Air Chaos'', The New York Times, July 19, 1926.} O resultado
foi o ``caos no ar'' que Hoover havia previsto. Era pior do que antes da Lei de
1912 porque existiam muitos mais transmissores e eram muito mais potentes.
Estações surgiam, pulavam por todo o espectro de frequência em busca de espaço
disponível e aumentavam a potência de transmissão ao máximo para abafar os sinais
concorrentes. O rádio se tornou virtualmente inútil, especialmente nas cidades.
O Congresso finalmente foi forçado a agir.

\subsection{A Nacionalização do Espectro}
\label{cap8:os-como-nacionalizacao}
As premissas da Lei do Rádio de 1927\footnote{Radio Act of 1927, Pub. L. No. HR
9971 (1927), \url{https://www.fcc.gov/document/radio-act-1927-established-federal-radio-commission}.}
ainda estão em vigor. O espectro tem sido tratado como um recurso nacional escasso
desde então, gerenciado pelo governo. O objetivo da lei era:

\begin{quote}
manter o controle dos Estados Unidos sobre todos os canais de transmissão de
rádio; e permitir o uso de tais canais, mas não a propriedade dos mesmos, por
indivíduos, empresas ou corporações, por períodos limitados de tempo, sob
licenças concedidas por autoridade federal.    
\end{quote}

O público poderia usar o espectro, sob condições estipuladas pelo governo, mas
não poderia ser dono dele. Uma nova autoridade, a Comissão Federal de Rádio (FRC),
tomou decisões sobre licenciamento. O público tinha uma expectativa qualificada
de que os pedidos de licença seriam concedidos:

\begin{quote}
    A autoridade de licenciamento, se o interesse, a conveniência pública ou a
    necessidade assim o exigirem, concederá a qualquer requerente uma licença de
    estação.    
\end{quote}

A lei reconhecia que a demanda por licenças poderia exceder a oferta de espectro.
Em caso de competição entre requerentes

\begin{quote}
    a autoridade de licenciamento deverá fazer uma distribuição de licenças,
    faixas de frequência\ldots, períodos de tempo de operação e de potência entre
    os diferentes estados e comunidades, a fim de proporcionar um serviço de rádio
    justo, eficiente e equitativo para cada um.\footnote{Radio Act of 1927, 1.}\\
\end{quote}


%%%%%%%%%%%%%%% BLOCO DO LADO %%%%%%%%%%%%%%%
A ``Comissão de Rádio'' se expande

Em 1934, o nome da FRC foi alterado para Comissão Federal de Comunicações---a 
FCC---quando a regulamentação de telefone e telégrafo passou a ser supervisionada
pela comissão. Quando um segmento separado do espectro de rádio foi alocado para
a televisão, a FCC assumiu a autoridade sobre as transmissões de vídeo também.\\
%%%%%%%%%%%%%%% BLOCO DO LADO %%%%%%%%%%%%%%%

A linguagem sobre ``conveniência pública, interesse ou necessidade'' ecoa o discurso
de Hoover em 1922 sobre um ``ativo nacional'' e o ``interesse público''. Também
não é por acaso que esta lei foi redigida quando o Escândalo \ingles{Teapot Dome}
estava atingindo o ápice. Reservas de petróleo em terras federais em Wyoming haviam
sido arrendadas para a \ingles{Sinclair Oil} em 1923, com a assistência de subornos
pagos ao secretário do Interior. Levou vários anos para que as investigações do
Congresso e os casos judiciais federais expusessem a má conduta; o secretário
acabou sendo preso. No início de 1927, o uso justo de recursos nacionais no
interesse público era uma grande preocupação nos Estados Unidos.

Com a aprovação da lei de 1927, o espectro de rádio se tornou uma terra federal.
Tratados internacionais subsequentes foram estabelecidos para limitar a
interferência perto das fronteiras nacionais. Mas nos Estados Unidos, assim como
Hoover havia solicitado cinco anos antes, o governo federal assumiu o controle
sobre quem seria autorizado a transmitir, quais ondas de rádio eles poderiam
usar---e até mesmo o que eles poderiam dizer.

\subsection{Glandulas de cabra e a primeira Emenda}
\label{cap8:os-como-glandulas}
A Lei do Rádio de 1927 estipulou que a FRC (Comissão Federal de Rádio, na sigla
em inglês) não poderia restringir a liberdade de expressão pelo rádio:

\begin{quote}
    Nada nesta Lei deverá ser entendido ou interpretado como dando à autoridade
    de licenciamento o poder de censura\ldots, e nenhuma regulamentação ou
    condição\ldots deverá interferir no direito à liberdade de expressão por meio
    de comunicações de rádio.\footnote{Radio Act of 1927, 29.}  
\end{quote}

Inevitavelmente, um caso surgiria expondo o conflito implícito: Por um lado, a
comissão tinha que usar um padrão de interesse público ao conceder e renovar
licenças. Por outro lado, tinha que evitar a censura. O caso pivotal foi sobre a
licença da rádio KFKB, a estação do médico de glândulas de cabra do Kansas, John
Romulus Brinkley (veja Figura 8.2). A ira que recaiu sobre a CBS em 2004 por
mostrar um flash do seio de Janet Jackson\footnote{Cecilia Kang, ``Court Knocks
Down FCC's Fine for Janet Jackson's `Wardrobe Malfunction' '', The Washington Post,
November 2, 2011, \url{https://www.washingtonpost.com/business/economy/court-knocks-down-fccs-fine-forjanet-jacksons-wardrobe-malfunction/2011/11/02/gIQA98BpgM_story.html}.}---e
a multa de \$325.000\footnote{Ralph Berrier, ``FCC Hits WDBJ with Proposed \$325,000
Indecency Fine'', Roanoke Times, March 23, 2015, 
\url{https://roanoke.com/news/local/fcc-hits-wdbj-with-proposed-325-000-indecency-fine/article_f9c2a1b6-0f9a-50a9-8f9b-d02c0f2079ac.html}.}
imposta à emissora de TV WBDJ de Roanoke, Virginia, em 2015, por mostrar
acidentalmente uma imagem gráfica por três segundos---descende da ação da FRC
contra este charlatão clássico americano.\\

%%%%%%%%%%%%%%%%%%%%%%%%%%
%%%%%%% IMAGEM 8.2 %%%%%%%
%%%%%%%%%%%%%%%%%%%%%%%%%%
FIGURA 8.2 Um artigo de jornal plantado sobre a clínica de glândulas de cabra do
``Dr.''~Brinkley. O próprio médico está mostrado à esquerda, segurando o primeiro
bebê---chamado ``Billy'', é claro---concebido após um transplante de glândulas de
cabra.~(\ingles{New York Evening Journal}, 11 de setembro de 1926. Microfilme
cortesia da Biblioteca do Congresso.)\\

Brinkley, nascido em 1885, tornou-se um ``médico'' com licença para praticar em
Kansas ao comprar um diploma da \ingles{Eclectic Medical University} em Kansas
City. Ele trabalhou brevemente como médico para a \ingles{Swift \& Co.}, os
empacotadores de carne. Em 1917, ele estabeleceu sua prática médica em Milford,
uma pequena cidade a cerca de 70 milhas a oeste de Topeka. Um dia, um homem veio
em busca de conselhos sobre sua virilidade falhando, descrevendo-se como um ``pneu
vazio''. Baseando-se em sua memória do comportamento das cabras de seus dias no
matadouro, Brinkley disse: ``Você não teria problemas se tivesse um par dessas
glândulas de bode em você.'' ``Bem, por que você não coloca elas em mim?'' perguntou
o paciente. Brinkley fez o transplante em uma sala dos fundos, e um negócio nasceu.
Logo ele estava realizando 50 transplantes por mês, a \$750 por cirurgia. Com o
tempo, ele descobriu que prometer desempenho sexual era ainda mais lucrativo do
que prometer fertilidade.\footnote{Barnouw, A Tower in Babel, p. 169; Gerald Carson,
The Roguish World of Doctor Brinkley (Rinehart, 1960), p. 33; Pope Brock, Charlatan:
America's Most Dangerous Huckster, the Man Who Pursued Him, and the Age of Flimflam
(Crown Publishers, 2008).}

Quando jovem, Brinkley havia trabalhado em um escritório de telégrafo, então ele
conhecia o potencial da tecnologia de comunicação. Em 1923, ele abriu a primeira
estação de rádio do Kansas, a KFKB---\ingles{``Kansas First, Kansas Best''} radio,
ou às vezes \ingles{``Kansas Folks Know Best''}. A estação transmitia uma mistura
de música country, pregação fundamentalista e conselhos médicos do Dr.~Brinkley.
Os ouvintes enviavam suas reclamações, e o conselho quase sempre era comprar alguns
dos medicamentos patenteados de Dr.~Brinkley por correspondência. Um segmento típico
ia assim:

\begin{quote}
    Ela diz que teve uma operação, teve alguns problemas há 10 anos. Acho que
    a operação foi desnecessária, e não faz muito sentido remover um ovário com
    a expectativa de resultar em maternidade. Meu conselho para você é usar o
    Tônico Feminino nº 50, 67 e 61. Essa combinação fará por você o que você
    deseja que qualquer combinação faça, após três meses de uso persistente.\footnote{KFKB
    Broadcasting Association v. Federal Radio Commission (D.C. Cir. 1930),
    \url{http://archive.org/details/dc_circ_1930_5240_kfkb_broad_assn_v_fed_radio_commn}.}
\end{quote}

A KFKB tinha um transmissor extremamente poderoso, audível até a metade do Atlântico.
Em uma pesquisa nacional, foi a estação mais popular da América, com quatro vezes
mais votos do que a segunda colocada.\footnote{Carson, The Roguish World of Doctor
Brinkley, p. 143} Brinkley estava recebendo 3.000 cartas por dia e era uma sensação
em todo o centro-oeste dos Estados Unidos. Em um bom dia, 500 pessoas poderiam
aparecer em Milford. No entanto, a Associação Médica Americana---incentivada por
uma estação de rádio local concorrente---protestou contra a charlatanice dele. A
FRC concluiu que ``o interesse público, a conveniência ou a necessidade'' não
seriam atendidos para renovação da licença. Brinkley argumentou que o cancelamento
era nada menos que censura.

Um tribunal de apelações apoiou a FRC em uma decisão histórica. A censura, explicou
o tribunal, era uma restrição prévia, o que não estava em questão no caso de
Brinkley. A FRC tinha ``apenas exercido seu inquestionável direito de tomar nota
da conduta passada do apelante''. Um ponto discutível---como Albert Gallatin disse
mais de 200 anos atrás sobre a restrição prévia da imprensa, era ``absurdo dizer
que punir um certo ato não era uma restrição à liberdade de realizar esse ato''.\footnote{Anthony
Lewis, Make No Law: The Sullivan Case and the First Amendment (Vintage Books,
1992), p. 60.}

O tribunal usou a metáfora das terras públicas para justificar a ação da FRC:\@

\begin{quote}
    Porque o número de frequências de radiodifusão disponíveis é limitado, a
    comissão é necessariamente chamada para considerar o caráter e a qualidade do
    serviço a ser prestado\ldots Obviamente, não há espaço na banda de transmissão
    para todos os negócios ou escolas de pensamento.
\end{quote}

``Necessariamente'' e ``obviamente''. Sempre é sábio examinar os argumentos que
proclamam em voz alta o quanto são autoevidentes.

O Juiz Felix Frankfurter, em uma opinião sobre um caso diferente em 1943, reafirmou
o princípio de uma forma que frequentemente tem sido citada:

\begin{quote}
    A situação em que o rádio se encontrava antes de 1927 era atribuível a certos
    fatos básicos sobre o rádio como meio de comunicação---suas instalações são
    limitadas; elas não estão disponíveis para todos que desejem usá-las; o
    espectro de rádio simplesmente não é grande o suficiente para acomodar todos.
    Há uma limitação natural fixa no número de estações que podem operar sem
    interferir umas com as outras.\footnote{National Broadcasting Co., Inc.~v.
    United States, 319 U.S. 190 (1943),
    \url{https://supreme.justia.com/cases/federal/us/319/190/}.}
\end{quote}

Esses eram fatos da tecnologia da época. Eles eram verdadeiros, mas eram verdades
contingentes da engenharia. Eles nunca foram leis universais da física e não são
mais limitações da tecnologia. Graças às inovações de engenharia, praticamente
não há uma ``limitação natural'' significativa no número de estações de transmissão.
Argumentos sobre escassez inevitável não podem mais justificar a negação do governo
dos Estados Unidos quanto ao uso das ondas do ar.

A vasta infraestrutura regulatória, construída para racionalizar o uso do espectro
por uma tecnologia de rádio muito mais limitada, se ajustou lentamente---como
quase inevitavelmente deve acontecer: as burocracias não se movem tão rapidamente
quanto os inovadores tecnológicos. A FCC tenta antecipar as necessidades de recursos
centralmente e com muita antecedência. No entanto, a tecnologia pode causar mudanças
abruptas na oferta, e as forças de mercado podem causar mudanças abruptas na demanda.
O planejamento centralizado não funciona melhor para a FCC do que funcionou para
a União Soviética.

Além disso, muitos interessados na tecnologia antiga estão satisfeitos em ver as
regras permanecerem inalteradas. Assim como os locatários desfrutando de
arrendamentos em terras públicas, os detentores de licenças de rádio incumbentes
não têm motivo para incentivar o uso concorrente dos ativos que controlam. Quanto
mais dinheiro estiver em jogo, maior será a alavancagem das empreitadas lucrativas.
As licenças de rádio tinham valor quase desde o início, e à medida que a escassez
aumentava, o preço também aumentava. Em 1925, uma licença em Chicago foi vendida
por \$50.000. À medida que a publicidade se expandiu e as estações se uniram em
redes, as transações atingiram sete dígitos. Após a lei de 1927, as disputas entre
estações tinham que ser resolvidas por meio de litígios, viagens a Washington e
pressão de representantes congressistas amigáveis---tudo mais viável para estações
com bolsos fundos do que para as demais. No início, havia muitas estações
universitárias, mas a FRC as pressionou à medida que o valor das ondas do ar
aumentava. Como organizações sem fins lucrativos, as estações universitárias não
conseguiram manter sua posição. Eventualmente, a maioria das estações educacionais
foi vendida para radiodifusores comerciais.~\ingles{De facto}, como um historiador
colocou, ``enquanto falava em termos de interesse público\ldots~a comissão na
verdade escolheu promover os interesses dos radiodifusores comerciais.''\footnote{E.
Herring, ``Politics and Radio Regulation,'' Harvard Business Review 13 (1935):167--178.}

\section{O caminho para a desregulamentação do espectro}
\label{cap8:os-caminho}
Hoje, todos nós somos radiodifusores e receptores de rádio. O smartphone no seu
bolso está utilizando ondas de rádio para postar suas fotos no Instagram, enviar
sua consulta de pesquisa para o Google, enviar mensagens de texto para seu irmão
e, é claro, transmitir sua voz sem fio, caso você faça uma chamada telefônica à
moda antiga. Mas você utiliza ondas de rádio para inúmeros outros dispositivos
que não se afastam muito do seu corpo. Seus fones de ouvido Bluetooth usam sinais
de rádio de curto alcance para transmitir músicas codificadas vindas do seu
smartphone ou iPod. O chaveiro sem chave no seu bolso usa outro sinal para
destravar seu carro, em vez de um carro estacionado ao lado. Um de nós usa uma
bomba de insulina que se comunica sem fio com seu sensor de glicose no sangue
para simular, de forma muito aproximada, um ciclo bioquímico regulatório humano
que falhou nele.

Cada um desses sinais utiliza um pedaço do espectro. Eles obedecem às mesmas leis
físicas básicas das transmissões de rádio da WBZ em Boston, que estão ocorrendo
continuamente desde que a WBZ se tornou a primeira estação comercial do leste
dos EUA em 1921. No entanto, as novas transmissões de rádio são diferentes em
dois aspectos críticos. Existem bilhões delas acontecendo todos os dias. E,
enquanto a potência de transmissão da WBZ é de 50.000 watts, a potência de um
chaveiro de carro é inferior a 0,0002 watts.

Se o governo ainda tivesse que conceder licenças para cada transmissor de rádio,
como o Congresso autorizou na sequência do caos de rádio dos anos 1920, nem as
chaves de rádio nem centenas de outros usos inovadores de rádio de baixa potência
poderiam ter surgido. A lei e a burocracia teriam sufocado essa parte da explosão
digital.

Outro desenvolvimento também estava por trás da explosão sem fio. A tecnologia
teve que mudar para que o espectro disponível pudesse ser utilizado de forma mais
eficiente. A digitalização e a miniaturização mudaram o mundo das comunicações.
A história dos telefones celulares e da Internet sem fio e muitas conveniências
ainda não imaginadas é um emaranhado de política, tecnologia e direito. Você não
pode entender o emaranhado sem entender os fios, mas no futuro, os fios não
precisarão permanecer amarrados da mesma forma como estão hoje.

\subsection{De poucos alto-falantes para bilhões de sussurros}
\label{cap8:os-caminho-pouco}
Há quarenta anos, não existiam celulares. Apenas alguns executivos de negócios
tinham telefones móveis, mas os dispositivos eram volumosos e caros. A
miniaturização ajudou a transformar o telefone celular de um privilégio de alguns
magnatas corporativos em um direito de nascença de todos os adolescentes
americanos. Mas o avanço principal ocorreu na alocação de espectro, ao repensar
a forma como o espectro de rádio era utilizado.

Na era dos grandes e desajeitados telefones celulares, a empresa de telefonia
móvel tinha uma grande antena e obtinha da FCC o direito de usar algumas
frequências em uma área urbana. O telefone do executivo era uma pequena estação
de rádio que transmitia sua chamada. O telefone celular precisava ser poderoso o
suficiente para alcançar a antena da empresa, onde quer que o telefone estivesse
na cidade. O número de chamadas simultâneas estava limitado ao número de
frequências alocadas para a empresa. A tecnologia era a mesma usada pelas estações
de rádio de transmissão, exceto que os rádios dos telefones celulares eram
bidirecionais. A escassez de espectro, ainda citada hoje como limitadora do número
de canais de transmissão, então limitava o número de telefones celulares. Hoover
entendeu isso lá em 1922. ``Obviamente'', disse ele, ``se 10.000.000 de assinantes
de telefone estão clamando pelo ar por seus companheiros\ldots a atmosfera será
preenchida com um caos frenético, sem nenhuma comunicação 
possível.''\footnote{``Asks Radio Experts to Chart the Ether.''}

Celulares, Wi-Fi, Bluetooth---essas tecnologias exploram a Lei de Moore.
Transmissores e receptores de rádio se tornaram mais rápidos, mais baratos e
menores. Pegue os celulares, por exemplo. Como as torres de celular estão a apenas
cerca de uma milha de distância uma da outra, os celulares só precisam ser
poderosos o suficiente para enviar seus sinais a menos de uma milha. Uma vez
recebido por uma antena, o sinal é enviado para a empresa de telefonia celular
por ``linha terrestre''---ou seja, por cabos de cobre ou fibra óptica em postes
ou subterrâneos. Só é necessário espectro de rádio suficiente para lidar com as
chamadas dentro da ``célula'' que cerca uma torre, uma vez que as mesmas
frequências podem ser usadas simultaneamente para lidar com chamadas em outras
células. Muitos arranjos devem ser feitos para evitar que uma chamada seja
interrompida quando um telefone ativo é movido de uma célula para outra, mas os
computadores, incluindo os pequenos computadores dentro dos celulares, são
inteligentes e rápidos o suficiente para acompanhar essas rearrumações.

A tecnologia de telefones celulares ilustra uma mudança importante no uso do
espectro de rádio. A maioria das comunicações por rádio agora ocorre em curtas
distâncias. São transmissões entre torres de celular e telefones celulares.
Entre roteadores sem fio e os computadores de trabalhadores de escritório e
frequentadores de cafeterias. Entre os fones de ouvido de telefones sem fio e
suas bases. Entre as cabines de pedágio de estradas e os transponders instalados
nos para-brisas dos motoristas. Entre chaves eletrônicas e os carros que elas
destrancam. Entre jogadores de videogame e seus jogos. Entre celulares e fones
de ouvido Bluetooth.

Até mesmo as transmissões de ``rádio via satélite'' às vezes partem de uma antena
próxima para o receptor do cliente, e não diretamente de um satélite orbitando
no espaço exterior. Em áreas urbanas, tantos prédios estão entre o receptor e o
satélite que as empresas de rádio instalaram ``repetidores''---antenas conectadas
entre si por fio. Quando você ouve a \ingles{SiriusXM0} no seu carro enquanto
dirige pela cidade, o sinal provavelmente está chegando até você de uma antena a
algumas quadras de distância.\footnote{Mark Lloyd, ``The Strange Case of Satellite
Radio,'' Center for American Progress, February 8, 2006,
\url{https://www.americanprogress.org/issues/democracy/news/2006/02/08/1829/the-strange-case-of-satellite-radio/}.}

A tecnologia celular 5G é outro desenvolvimento do mesmo tipo. O 5G alcança taxas
de dados mais altas usando uma parte diferente e de frequência mais alta do
espectro em comparação com sua predecessora, a tecnologia 4G. Um preço deve ser
pago por usar esse rádio de alta frequência e alta taxa de dados: os sinais
enfraquecem rapidamente com a distância do transmissor. Portanto, as células 5G
são menores do que as células 4G, e muitos mais transmissores celulares devem ser
instalados. É por isso que o 5G é mais amplamente implantado em áreas urbanas.

Detalhes à parte, o espectro de rádio não é mais principalmente para sinalização
em longas distâncias. As políticas de espectro foram estabelecidas quando o uso
principal do rádio era para transmissões de navio para costa, sinalização de SOS
a grandes distâncias e transmissões em grandes áreas geográficas. À medida que a
nação se tornou mais conectada por fios, a maioria dos sinais de rádio viaja
apenas alguns metros ou algumas centenas de metros. Nessas condições alteradas,
as antigas regras para o gerenciamento do espectro não fazem sentido.

\subsection{Podemos apenas dividir a propriedade de forma diferente?}
\label{cap8:os-caminho-podemos}
Mesmo partes do espectro que são ``alocadas'' para licenciados podem ser
drasticamente subutilizadas na prática. Um Relatório da Comissão de Comunicações
Federal coloca dessa forma: ``A escassez de espectro muitas vezes é um problema
de acesso ao espectro. Ou seja, o recurso do espectro está disponível, mas seu
uso é compartimentado por políticas tradicionais baseadas em tecnologias
tradicionais''.\footnote{``Report of the Spectrum Efficiency Working Group,''
Federal Communications Commission, Spectrum Policy Task Force, November 15, 2002,
\url{https://transition.fcc.gov/sptf/files/SEWGFinalReport_1.pdf}.} A comissão
chegou a essa conclusão em parte ouvindo as ondas do ar em vários blocos de
frequência para testar com que frequência nada estava sendo transmitido. Na maior
parte do tempo, mesmo em ambientes urbanos densos como San Diego, Atlanta e
Chicago, bandas de espectro importantes estavam quase 100\% ociosas. O público
seria melhor atendido se outras pessoas pudessem usar o espectro que, de outra
forma, estaria ocioso.

Há cerca de 20 anos, a FCC tem utilizado o ``marketing secundário de espectro''.
Alguém que deseje algum espectro para uso temporário pode ser capaz de alugá-lo
de uma parte que tenha o direito de usá-lo, mas esteja disposta a cedê-lo em
troca de um pagamento. Uma estação de rádio universitária, por exemplo, pode
precisar da capacidade de transmitir com alta potência apenas em alguns sábados
à tarde para cobrir jogos de futebol importantes---um momento em que os mercados
de ações estão fechados e parte do espectro não está sendo amplamente utilizado
por empresas financeiras. Ou talvez, em vez de reservar uma faixa exclusivamente
para transmissões de emergência, ela possa ser disponibilizada para outros fins
de entretenimento, com a compreensão---reforçada por códigos incorporados nos
transmissores---de que a frequência seria cedida sob demanda para transmissões
de segurança pública.

Leilões informatizados podem resultar na distribuição muito eficiente de bens,
seja para itens usados no eBay ou para faixas de espectro muito pequenas. O uso
de partes específicas do espectro---em momentos específicos e em áreas geográficas
específicas---cria eficiências se os detentores de licenças de faixas de espectro
subutilizadas tiverem incentivos para vender parte de seu tempo a outras partes.

No entanto, os mercados secundários não alteram o modelo básico: uma banda de
frequência pertence a uma parte de cada vez. Essas ideias de leilão alteram o
esquema de alocação. Em vez de ter uma agência governamental licenciando o
espectro de forma estática para uma única parte com direitos exclusivos, várias
partes podem dividi-lo e fazer negociações. Mas esses esquemas mantêm a noção
fundamental de que o espectro é como a terra, a ser dividida entre aqueles que
desejam usá-lo.

\subsection{Compartilhando o espectro}
\label{cap8:os-caminho-compartilhando}
Em sua opinião de 1943, o juiz Frankfurter usou uma analogia que apontava
inadvertidamente para outra maneira de pensar. O espectro era inevitavelmente
escasso, ele opinou: ``A regulamentação do rádio, portanto, era tão vital para
seu desenvolvimento quanto o controle de tráfego era para o desenvolvimento do
automóvel.''

Assim como se diz que o espectro é, as rodovias são um patrimônio nacional. Elas
são controladas por governos federais, estaduais e locais, que estabelecem regras
para seu uso. Você não pode dirigir muito rápido. Seu veículo não pode exceder
limites de altura e peso, que podem depender da estrada.

Mas todos compartilham as estradas. Não existem rodovias especiais reservadas
para veículos governamentais. Empresas de transporte rodoviário não podem obter
licenças para usar estradas específicas e impedir a concorrência. Todos
compartilham a capacidade das estradas para transportar o 
tráfego.\footnote{Eli Noam, ``Taking the Next Step Beyond Spectrum Auctions: Open
Spectrum Access,'' 1995,\url{http://www.columbia.edu/dlc/wp/citi/citinoam21.html}.}

As estradas são o que é conhecido em termos legais como um ``bem comum'' (uma
noção introduzida no Capítulo 6, ``Equilíbrio Desestabilizado''). O oceano também
é um bem comum, um recurso compartilhado sujeito a acordos internacionais de
pesca. Em teoria, pelo menos, o oceano não precisaria ser um bem comum. Os barcos
de pesca poderiam ter direitos exclusivos de pesca em setores separados da
superfície do oceano. Se as regiões fossem grandes o suficiente, poderia ser
possível ganhar a vida com a pesca nessas condições. Mas tal alocação dos
recursos do oceano seria terrivelmente ineficiente para a sociedade como um todo.
Os oceanos atendem melhor às necessidades humanas quando são tratados como um bem
comum e os barcos de pesca se movem com os peixes, sob limites acordados quanto
à intensidade da pesca.\\

%%%%%%%%%%%%%%% BLOCO DO LADO %%%%%%%%%%%%%%%
O site de Yochai Benkler, \url{http://www.benkler.org}, possui vários artigos
importantes e de fácil leitura para download gratuito, incluindo o clássico
``Superando a Agorafobia''\footnote{Yochai Benkler, ``Overcoming Agoraphobia:
Building the Commons of the Digitally Networked Environment,'' Harvard Journal
of Law and Technology 11 (1998),\url{http://www.benkler.org/agoraphobia.pdf}.}. 
Seu livro ``A Riqueza das Redes''\footnote{Yochai Benkler, The Wealth of Networks
(Yale University Press, 2007).} detalha esses e outros conceitos.\\
%%%%%%%%%%%%%%% BLOCO DO LADO %%%%%%%%%%%%%%%

O espectro pode ser compartilhado em vez de ser dividido em partes. Há um
precedente nas comunicações eletrônicas. A Internet é um bem digital comum. Os
pacotes de todos se misturam com os de todos os outros nas fibras ópticas e nas
ligações via satélite da infraestrutura central da Internet. Os pacotes são
codificados. Qual pacote pertence a quem é separado nas extremidades. Qualquer
coisa confidencial pode ser criptografada. Na formulação de políticas de espectro,
há uma escolha entre a alocação de espectro---talvez com alguma capacidade de
negociação para melhorar a utilização---e uma abordagem mais aberta, baseada em
um modelo comum, semelhante à Internet. Conforme a tecnologia sem fio avançou
nas duas primeiras décadas do século XXI, o Congresso moveu-se para liberar
algumas seções subutilizadas do espectro, anteriormente atribuídas à televisão
aberta, e adotou uma combinação das abordagens ``licenciada'' e ``aberta'' para
seu uso.

Para fazer qualquer tipo de compartilhamento de espectro funcionar, duas ideias
são a chave: em primeiro lugar, o uso de uma grande largura de banda pode
evitar interferências e aumentar muito a capacidade de transmissão; e em segundo
lugar, colocar computadores nos receptores de rádio pode melhorar
significativamente a utilização do espectro.

\section{A inventora mais linda do mundo}
\label{cap8:os-inventora}
A técnica de espectro espalhada foi descoberta e esquecida várias vezes e em
vários países.\footnote{R. A. Scholtz, ``The Origins of Spread-Spectrum
Communications'', IEEE Transactions on Communications 30, no. 5 (1982): 822--854,
\url{https://doi.org/} 10.1109/TCOM.1982.1095547; R. Scholtz, ``Notes on 
Spread-Spectrum History'', IEEE Transactions on Communications 31, no. 1 (1983):
82--84, \url{https://doi.org/10.1109/TCOM.1983.1095718}; R. Price, ``Further
Notes and Anecdotes on Spread-Spectrum Origins'', IEEE Transactions on
Communications 31, no. 1 (1983): 85--97,
\url{https://doi.org/10.1109/TCOM.1983.1095725}; Rob Walters, Spread Spectrum
(Book surge LLC, 2005).} Empresas (como ITT, Sylvania e Magnavox), universidades
(especialmente o MIT) e laboratórios governamentais que realizavam pesquisas
classificadas, compartilhavam a criação desse componente fundamental das
telecomunicações modernas, muitas vezes sem conhecimento das atividades uns dos
outros.

De longe, o precedente mais notável para o espectro espalhado foi uma invenção
patenteada pela atriz de Hollywood Hedy Lamarr, ``a mulher mais bonita do mundo'',
nas palavras do magnata do cinema Louis Mayer, e George Antheil, um compositor
vanguardista conhecido como ``o garoto malvado da música''.

Lamarr ganhou notoriedade na Europa ao aparecer nua em 1933, aos 19 anos, no
filme tcheco \ingles{Ecstasy}. Ela se tornou a esposa de troféu de Fritz Mandl,
um fabricante de munições austríaco cujos clientes incluíam Hitler e Mussolini.
Em 1937, ela se disfarçou de empregada e escapou da casa de Mandl, fugindo
primeiro para Paris e depois para Londres. Lá, ela conheceu Mayer, que a levou
para Hollywood. Ela se tornou uma estrela e a beleza icônica de sua geração no
cinema (veja a Figura 8.3).

Em 1940, Lamarr marcou um encontro com Antheil. Ela achava que sua parte superior
do corpo poderia precisar de melhorias e esperava que Antheil pudesse dar alguns
conselhos. Antheil era um autoproclamado especialista em endocrinologia feminina
e havia escrito uma série de artigos para a revista \ingles{Esquire} com títulos 
como \ingles{The Glandbook for the Questing Male}.\footnote{George Antheil, 
``Glands on a Hobby Horse'', Esquire, April 1936; George Antheil, ``Glandbook
for the Questing Male'', Esquire, May 1936; George Antheil, ``The Glandbook in
Practical Use'', Esquire, June 1936.} Antheil sugeriu extratos 
glandulares.\footnote{George Antheil, Bad Boy of Music (Doubleday, Doran \& Co.,
1945) 327.} A conversa deles então se voltou para outros assuntos,
especificamente, a guerra de torpedos.

Um torpedo, apenas uma bomba com uma hélice, poderia afundar um navio maciço.
Torpedos controlados por rádio já haviam sido desenvolvidos no final da Primeira
Guerra Mundial, mas estavam longe de serem infalíveis. Uma contramedida eficaz
era interferir no sinal que controlava o torpedo, transmitindo ruído de rádio
alto na frequência do sinal de controle. O torpedo ficaria descontrolado e
provavelmente perderia seu alvo. Observando os negócios de Mandl, Lamarr havia
aprendido sobre torpedos e por que era difícil controlá-los.

Lamarr havia se tornado ferozmente pró-americana e desejava ajudar no desempenho
dos Aliados na guerra. Ela concebeu a ideia de transmitir o sinal de controle do
torpedo em rajadas curtas em diferentes frequências. O código para a sequência de
frequências seria mantido de forma idêntica no torpedo e no navio de controle.
Como a sequência seria desconhecida para o inimigo, a transmissão não poderia ser
bloqueada inundando o espectro de rádio com ruído em uma banda de frequência
limitada. Seria necessário muito poder para bloquear todas as frequências
possíveis simultaneamente.\\ 

%%%%%%%%%%%%%%%%%%%%%%%%%%
%%%%%%% IMAGEM 8.3 %%%%%%%
%%%%%%%%%%%%%%%%%%%%%%%%%%
FIGURA 8.3 Hedy Lamarr, por volta da idade em que ela e George Antheil fizeram
sua descoberta de espectro espalhado.\\

A contribuição de Antheil foi controlar a sequência de saltos de frequência por
meio de um mecanismo de piano mecânico---com o qual ele estava familiarizado,
pois havia composto sua obra-prima, ``Ballet Mécanique'', para pianos mecânicos
sincronizados. Conforme ele e Lamarr conceberam o dispositivo (que nunca foi
construído), o sinal saltaria entre 88 frequências, como as 88 teclas de um
teclado de piano. O navio e o torpedo teriam rolos de piano idênticos,
efetivamente criptografando o sinal de transmissão.

Em 1941, Lamarr e Antheil cederam sua patente (veja a Figura 8.4) para a Marinha.
Um pequeno artigo na página de ``Entretenimento'' do \ingles{New York Times}
citou um engenheiro do exército descrevendo a invenção deles como ``ultra-secreta''
a ponto de não poder dizer para que servia, exceto que estava ``relacionada ao
controle remoto de aparelhos usados na guerra''.\footnote{``Hedy Lamarr Inventor'',
The New York Times, October 1, 1941.} No entanto, a Marinha parece não ter feito
nada com a invenção na época. Em vez disso, Lamarr passou a vender títulos de
guerra. Chamando a si mesma de ``apenas uma garimpeira simples para o Tio Sam'',
ela vendia beijos e uma vez arrecadou US\$ 4,5 milhões em um único
almoço.\footnote{``\$4,547,000 Bonds'', The New York Times, September 2, 1942;
``Hollywood Puts on a Show'', Time, October 12, 1942.} A patente foi ignorada
por mais de uma década. Romuald Ireneus Scibor-Marchocki, que era engenheiro de
uma contratada naval na década de 1950, lembra de ter recebido uma cópia quando
foi designado para trabalhar em um dispositivo para localizar submarinos inimigos.
Ele não reconheceu a detentora da patente porque ela não usou seu nome artístico.\\

%%%%%%%%%%%%%%% BLOCO DO LADO %%%%%%%%%%%%%%%
A história de Antheil e Lamarr, e o lugar de sua invenção na história do espectro
espalhado, é contada no livro \ingles{``Spread Spectrum''} por Rob
Walters.\footnote{Walters, Spread Spectrum.}\\
%%%%%%%%%%%%%%% BLOCO DO LADO %%%%%%%%%%%%%%%

%%%%%%%%%%%%%%%%%%%%%%%%%%
%%%%%%% IMAGEM 8.4 %%%%%%%
%%%%%%%%%%%%%%%%%%%%%%%%%%
FIGURA 8.4 Patente original de espectro espalhado de Hedy Lamarr (anteriormente
Kiesler---Gene Markey foi seu segundo marido, de seis) e George Antheil. À
esquerda, o início da própria patente. À direita, um diagrama do mecanismo de
piano mecânico incluído como ilustração na patente. (Escritório de Patentes dos
EUA)\\

E isso, em poucas palavras, é a estranha história de serendipidade, trabalho em
equipe, vaidade e patriotismo que levou à descoberta do espectro espalhado por
Lamarr e Antheil. A conexão desses dois com a descoberta do espectro espalhado
só foi feita na década de 1990. Naquele momento, a influência de seu trabalho
havia se entrelaçado com várias linhas de pesquisa militar classificada. Se Hedy
Lamarr foi mais uma Leif Erikson do que um Cristóvão Colombo desse novo território
conceitual, ela certamente foi a mais improvável de seus descobridores. Em 1997,
a \ingles{Electronic Frontier Foundation} a homenageou por sua descoberta; ela
recebeu o prêmio dizendo: ``Já era hora''. Quando questionada sobre suas
realizações duplas, ela comentou: ``Os filmes têm um lugar certo em um determinado
período de tempo. A tecnologia é para sempre''.

\end{document}