\documentclass{book}
\usepackage{url}
\newcommand{\ingles}[1]{\textit{#1}}

\begin{document}

\chapter[Os Bits Estão No Ar]{Os Bits Estão No Ar\\\large\textit{Velhas Metáforas, Novas Tecnologias, e Liberdade de Expressão}}
\label{cap8:os}

\section{Censurando o candidato}
\label{cap8:os-censurando}
Em 8 de outubro de 2016, apenas seis semanas antes da eleição presidencial dos
Estados Unidos, o \ingles{The Washington Post} lançou uma bomba\footnote{David A.
Fahrenthold, ``Trump Recorded Having Extremely Lewd Conversation About Women in
2005'', The Washington Post, October 8, 2016,
\url{https://www.washingtonpost.com/politics/trump-recorded-having-extremely-lewd-conversation-about-women-in-2005/2016/10/07/3b9ce776-8cb4-11e6-bf8a-3d26847eeed4_story.html}.}. 
O Post havia recebido uma gravação de vídeo na qual o então candidato Donald Trump
usava linguagem grosseira ao se vangloriar de agressão sexual contra mulheres.
Conforme relatado na matéria, o Sr. Trump disse:

\begin{quote}
``Eu tentei e \verb|f---| ela. Ela era casada.''\ldots ``E quando você é uma estrela,
eles deixam você fazer isso\ldots Você pode fazer qualquer coisa\ldots Agarrá-las
pela \verb|b----a|''.
\end{quote}

Foi assim que o \ingles{Post} o publicou, com hifens substituindo certas letras,
mas com contexto suficiente para que as palavras pudessem ser reconstruídas de
forma inequívoca. A matéria continha um link para o vídeo completo e não expurgado,
para que qualquer pessoa com conexão à internet pudesse ouvir o que o Sr. Trump
mais tarde caracterizou como sua ``conversa de vestiário''.

O \ingles{The New York Times} optou por imprimir as palavras ofensivas na íntegra.\footnote{
Alexander Burns et al., ``Donald Trump Apology Caps Day of Outrage over Lewd
Tape'', The New York Times, October 7, 2016,
\url{https://www.nytimes.com/2016/10/08/us/politics/donald-trump-women.html}; Al
Tompkins, ``As Profanity-Laced Video Leaks, Outlets Grapple with Trump's Language'',
Poynter, October 7, 2016,
\url{https://www.poynter.org/reporting-editing/2016/as-profanity-laced-videoleaks-outlets-grapple-with-trumps-language/}.}
A rede de televisão a cabo CNN exibiu o vídeo completo, mas seus repórteres usaram
``a palavra F'' e ``a palavra B'' onde o Post havia usado hifens. Essas decisões
eram questões de julgamento editorial e poderiam ter seguido qualquer direção.O
editor do Post, Marty Baron, disse: ``Tomamos nossos melhores julgamentos ao
equilibrar o gosto em relação à clareza do que foi dito'', enquanto a editora do
\ingles{Times}, Carolyn Ryan, explicou: ``decidimos que as palavras vulgares em
si eram noticiosas e omiti-las ou descrevê-las de outra forma teria sido menos
honesto''.

Esses veículos de mídia poderiam chegar a conclusões diferentes sobre o que
reportar, mas nenhuma lei ou regulamentação governamental eram relevantes.
A Primeira Emenda protegia o direito dos jornais e das emissoras de televisão a
cabo de reportarem sobre a fita como desejavam.

Mas não as emissoras de televisão aberta. Embora a maioria dos americanos agora
assista aos seus programas da ABC, CBS e NBC por meio de caixas de TV a cabo ou
pela Internet, os termos ``transmissão'' e ``via ar'' ainda diferenciam as estações
que podem ser recebidas por antenas captando sinais de rádio emitidos por antenas
transmissoras gigantes. E essas estações estão sujeitas à regulamentação da
Comissão Federal de Comunicações (FCC): Nenhuma linguagem indecente ou profana
durante as horas em que as crianças podem ouvir---e a linguagem do Sr. Trump
provavelmente ultrapassou esse limite. De acordo com a FCC, conteúdo indecente
retrata órgãos ou atividades sexuais ou excretoras de uma maneira que não atende
ao teste Miller para obscenidade, e conteúdo profano inclui linguagem ``extremamente
ofensiva'' que é considerada um incômodo público.

As emissoras de transmissão censuraram as palavras ofensivas quando exibiram o
vídeo. Elas não tiveram escolha: poderiam ter sido sujeitas a multas substanciais
da FCC ou até mesmo à perda de suas licenças de transmissão se não o tivessem
feito.

Sob a Primeira Emenda, o governo geralmente não atua para restringir o discurso.
Não pode impor seus julgamentos editoriais sobre jornais, nem mesmo para aumentar
a gama de informações disponíveis para os leitores. A Suprema Corte considerou
inconstitucional uma lei da Flórida que garantia aos candidatos políticos um
simples ``direito de resposta'' a ataques de jornais contra eles. As emissoras de
televisão a cabo não precisavam se preocupar com reclamações à FCC, embora algumas
tenham sido apresentadas; esse meio está amplamente além do alcance da censura
governamental.

No entanto, em 2016, uma agência do governo federal estava censurando palavras
na televisão aberta, usando regras que cobriam até mesmo os comentários de um
candidato presidencial no meio de uma campanha política. Estamos em uma era de
sensibilidade aumentada em relação à programação que as crianças podem ver, mas
os americanos em geral continuam sendo contrários a ter o governo como ``babá''
de seus programas de televisão. Por que a FCC tem o poder de regular o que pode
ser dito pelas ondas do ar?

\section{Como a transmissão se tornou regulamentada}
\label{cap8:os-como}
A FCC (Comissão Federal de Comunicações) obteve sua autoridade sobre o que é
dito em transmissões de rádio e televisão quando havia menos formas de
distribuição de informações. A teoria era que as vias públicas de transmissão
eram escassas e o governo precisava garantir que fossem usadas no interesse
público. Conforme o rádio e a televisão se tornaram universalmente acessíveis,
surgiu uma segunda justificativa para a regulamentação governamental do discurso
nas transmissões. Porque as mídias de transmissões têm ``uma presença
singularmente pervasiva na vida de todos os americanos'', como afirmou a Suprema
Corte em 1978, o governo tinha um interesse especial em proteger um público
indefeso de conteúdo de rádio e televisão objetável.

A explosão nas tecnologias de comunicação confundiu ambas as justificativas. Na
era digital, há muito mais maneiras de os dados alcançarem o consumidor, então o
rádio e a televisão não são mais únicos em sua pervasividade. Com tecnologia
mínima, qualquer pessoa pode sentar em casa ou em uma cafeteria e escolher entre
bilhões de páginas da web e dezenas de milhões de blogs. O locutor Howard Stern
deixou o rádio de transmissão terrestre para a rádio via satélite, onde a FCC não tem
autoridade para regular o que ele diz. Quase 90\% dos  telespectadores americanos
obtêm seu sinal de TV por meio de cabo ou satélite igualmente não regulamentados,
em vez de antenas no  telhado.\footnote{``Rise in Broadband Only Television
Homes'', Informitv, July 12, 2017,
\url{https://informitv.com/2017/07/12/rise-in-broadband-only-television-homes/}}
\ingles{Feeds RSS} fornecem informações atualizadas para milhões de usuários de
celulares em movimento. Estações de rádio e canais de televisão hoje não são mais
escassos nem singularmente pervasivos.\\

%%%%%%%%%%%%%%% BLOCO DO LADO %%%%%%%%%%%%%%%
Na era digital, há muitas mais maneiras para os dados alcançarem o consumidor,
então o rádio e a televisão de transmissão não são mais únicos em sua pervasividade.\\
%%%%%%%%%%%%%%% BLOCO DO LADO %%%%%%%%%%%%%%%

Para que o governo proteja as crianças de todas as informações ofensivas que
chegam por meio de qualquer meio de comunicação, sua autoridade teria que ser
ampliada significativamente e atualizada continuamente. Embora algumas propostas
tenham sido feitas, o Congresso não aprovou nenhuma lei que estendesse as
regulamentações de indecência da FCC para mídia de satélite e televisão a cabo.
O que é exibido na TV a cabo e via satélite é limitado pelo que os espectadores
e anunciantes aceitarão, mas não pelo que qualquer autoridade governamental
possa ditar.

A explosão nas comunicações levanta outra possibilidade, no entanto. Se
praticamente qualquer pessoa pode agora enviar informações que muitas pessoas
podem receber, talvez o interesse do governo em restringir as transmissões deva
ser menor do que costumava ser, e não maior. Na ausência de escassez, talvez o
governo não deva ter mais autoridade sobre o que é dito no rádio e na TV do que
sobre o que é impresso nos jornais. Nesse caso, em vez de expandir a autoridade
de censura da FCC, o Congresso deveria eliminá-la completamente, assim como a
Suprema Corte encerrou a regulamentação do conteúdo de jornais na Flórida.

As partes que já possuem lugares no dial de rádio e na programação de TV
argumentam que o espectro---as vias públicas de transmissão---continua sendo um
recurso limitado que requer proteção governamental. A teoria é que ninguém está
criando mais espectro de rádio, e ele precisa ser usado no interesse público.

Mas olhe ao seu redor. Ainda existem apenas algumas estações no dial AM e FM.
Mas milhares, talvez dezenas de milhares, de comunicações de rádio estão
ocorrendo no ar ao seu redor. A maioria dos americanos carrega rádios
bidirecionais em seus bolsos---dispositivos que chamamos de celulares. Para ser
preciso, carregamos celulares em nossas mãos, já que muitos de nós preferem
arriscar bater em postes de luz do que atrasar a leitura de nossas mensagens de
texto por apenas alguns segundos. Se você estiver ouvindo música em um fone de
ouvido Bluetooth e navegando na Web por uma conexão Wi-Fi, são mais duas conexões
de rádio que você está usando. Rádios e televisores poderiam ser muito mais
inteligentes do que são atualmente e poderiam fazer um uso melhor das vias
públicas de transmissão, assim como os celulares fazem.

Os avanços na engenharia minaram a capacidade do governo de anular a Primeira
Emenda na rádio e televisão. A Constituição exige, sob essas circunstâncias
alteradas, que o governo pare de regular o discurso verbal. De fato, quando a
Suprema Corte dos Estados Unidos anulou as multas que a FCC estava impondo quando
celebridades proferiam um ``palavrão fugaz'' em seus comentários transmitidos ao
vivo, ela restringiu o escopo de sua decisão, mas insinuou que poderia ser hora
de reavaliar toda a questão da censura de transmissões.

Como argumento científico, a alegação de que o espectro é necessariamente escasso
agora é muito fraca. No entanto, essa visão ainda é fortemente defendida pela
própria indústria que está sendo regulamentada. Os titulares de licenças
incumbentes---estações de transmissão e redes existentes---têm incentivo para
proteger seu ``território'' no espectro contra qualquer risco, real ou imaginário,
de que seus sinais possam ser corrompidos. Ao desencorajar a inovação tecnológica,
os incumbentes podem limitar a concorrência e evitar investimentos de capital.
Esses fios estranhamente entrelaçados---o interesse do governo na escassez
artificial para justificar a regulamentação do discurso e o interesse dos
titulares de licenças incumbentes na escassez artificial para limitar a
concorrência e os custos---hoje prejudicam tanto a criatividade cultural quanto
a tecnológica, em detrimento da sociedade.

Para entender as forças confluentes que criaram o mundo da censura de rádio e
televisão de hoje, é preciso voltar aos inventores da tecnologia.

\subsection{Do telégrafo sem fio para o caos sem fio}
\label{cap8:os-como-do}
Vermelho, laranja, amarelo, verde, azul---as cores do arco-íris---são todas
diferentes, mas ao mesmo tempo são todas iguais. Qualquer criança com uma caixa
de lápis de cor sabe que elas são diferentes. Elas são iguais porque todas são o
resultado da radiação eletromagnética atingindo nossos olhos. A radiação viaja em
ondas que oscilam muito rapidamente. A única diferença física entre o vermelho e
o azul é que as ondas vermelhas oscilam cerca de 450 trilhões de vezes por segundo
e as ondas azuis cerca de 50\% mais rápido.

Como o espectro da luz visível é contínuo, existe uma infinidade de cores entre
o vermelho e o azul. Misturar luz de diferentes frequências cria outras cores---por
exemplo, metade de ondas azuis e metade de ondas vermelhas cria uma tonalidade de
rosa conhecida como magenta, que não aparece no arco-íris.

Na década de 1860, o físico britânico James Clerk Maxwell percebeu que a luz
consiste em ondas eletromagnéticas. Suas equações previram que poderia haver
ondas de outras frequências---ondas que as pessoas não poderiam sentir. De fato,
tais ondas passaram por nós desde o início dos tempos. Elas caem invisivelmente
do sol e das estrelas e irradiam quando ocorrem raios. Ninguém suspeitava de sua
existência até que as equações de Maxwell dissessem que elas deveriam existir. Na
verdade, deve haver todo um espectro de ondas invisíveis de diferentes frequências,
todas viajando com a mesma grande velocidade da luz visível.

Em 1887, a era do rádio começou com uma demonstração de Henrich Hertz. Ele dobrou
um fio em um círculo, deixando uma pequena lacuna entre as duas extremidades.
Quando ele disparou uma grande faísca elétrica a alguns metros de distância, uma
pequena faísca saltou pela lacuna do fio quase completamente circular. A grande
faísca havia provocado uma chuva de ondas eletromagnéticas invisíveis, que
viajaram pelo espaço e causaram o fluxo de corrente elétrica no outro fio. A
pequena faísca era a corrente que completava o circuito. Hertz havia criado a
primeira antena e revelado as ondas de rádio que a atingiram. A unidade de
frequência leva seu nome: Um ciclo por segundo é 1 hertz, ou Hz, para abreviar.
Um kHz (quilohertz) é mil ciclos por segundo, e um MHz (megahertz) é um milhão de
ciclos por segundo. Essas são as unidades nos dial de rádio AM e FM.

Guglielmo Marconi não era matemático nem cientista. Ele era um inventor curioso.
Com apenas 13 anos na época do experimento de Hertz, Marconi passou a próxima
década desenvolvendo, por tentativa e erro, maneiras melhores de criar explosões
de ondas de rádio e antenas para detectá-las a distâncias maiores.

Em 1901, Marconi estava em \ingles{Newfoundland} e recebeu uma única letra em
código Morse transmitida da Inglaterra. Com base nesse sucesso, a \ingles{Marconi
Wireless Telegraph Company} em breve estava possibilitando que navios se
comunicassem entre si e com a costa. Quando o Titanic partiu em sua viagem
fatídica em 1912, estava equipado com equipamentos Marconi. O principal trabalho
dos operadores de rádio do navio era transmitir mensagens pessoais para e de
passageiros, mas eles também receberam pelo menos 20 avisos de outros navios sobre
os icebergs que estavam à frente.\footnote{knjazmilos, ``The Ice Warnings Received
By Titanic'', Titanic-Titanic.Com, June 17, 2019,
\url{http://www.titanic-titanic.com/tag/ice/}}

As palavras ``Telegrafia Sem Fio'' no nome da empresa de Marconi sugerem a maior
limitação do rádio inicial. A tecnologia foi concebida como um dispositivo para
comunicação ponto a ponto. O rádio resolveu o pior problema da telegrafia. Nenhuma
calamidade, sabotagem ou guerra poderia interromper as transmissões sem fio
cortando cabos. Mas havia uma desvantagem compensatória: qualquer pessoa poderia
ouvir. O enorme poder da radiodifusão para alcançar milhares de pessoas ao mesmo
tempo foi inicialmente visto como uma responsabilidade. Quem pagaria para enviar
uma mensagem a outra pessoa quando qualquer um poderia ouvi-la?

Conforme a telegrafia sem fio se tornou popular, outro problema surgiu---um que
moldou o desenvolvimento do rádio e da televisão desde então. Se várias pessoas
estivessem transmitindo simultaneamente na mesma área geográfica, seus sinais não
poderiam ser mantidos separados. O desastre do Titanic demonstrou a confusão que
poderia resultar. Na manhã seguinte ao choque do navio com o iceberg, os jornais
americanos relataram com entusiasmo que todos os passageiros haviam sido salvos
e o navio estava sendo rebocado para a costa. O erro resultou da fusão confusa de
dois segmentos não relacionados de código Morse por um operador de rádio. Um navio
perguntou ``todos os passageiros do Titanic estão seguros?'' E um navio
completamente diferente relatou que estava ``a 300 milhas a oeste do Titanic e
rebocando um tanque de óleo para Halifax''.\footnote{Karl Baarslag, S O S to the
Rescue (Oxford University Press, 1935), p. 72.} Todos os navios tinham rádios e
operadores de rádio. Mas não havia regras ou convenções sobre se, como ou quando
usá-los.

Os ouvintes dos transmissores iniciais de Marconi eram facilmente confundidos
porque não tinham como ``sintonizar'' uma comunicação específica. Apesar de todo
o talento de Marconi em estender o alcance da transmissão, ele estava usando
essencialmente o método de Hertz para gerar ondas de rádio: grandes faíscas. Os
faíscos espalhavam energia eletromagnética por todo o espectro de rádio. A energia
podia ser interrompida e reiniciada para se transformar em pontos e traços, mas
não havia mais nada para controlar. O ruído de um operador de rádio era como o de
qualquer outro. Quando vários transmitiam simultaneamente, resultava em caos.

As muitas cores da luz visível parecem brancas quando todas estão misturadas. Um
filtro de cor permite passar algumas frequências de luz visível, mas não outras.
Se você olhar para o mundo através de um filtro vermelho, tudo terá uma tonalidade
mais clara ou mais escura de vermelho, porque apenas a luz vermelha passará. O
que o rádio precisava era algo semelhante para o espectro de rádio: uma maneira
de produzir ondas de rádio de uma única frequência, ou pelo menos uma faixa
estreita de frequências, e um receptor que pudesse permitir essas frequências e
eliminar o resto. Na verdade, essa tecnologia já existia.

Em 1907, Lee De Forest patenteou uma tecnologia fundamental para a \ingles{De
Forest Radio Telephone Company}---dedicada a enviar voz e até música pelas ondas
de rádio. Quando ele transmitiu Enrico Caruso do \ingles{Metropolitan Opera House}
em Nova York cantando Pagliacci em 13 de janeiro de 1910, o canto alcançou navios
no mar. Amadores se reuniram ao redor de receptores em Nova York e Nova Jersey.
O efeito foi sensacional. Centenas de radiodifusores amadores entraram em ação
nos próximos anos, dizendo ansiosamente o que queriam e tocando a música que
podiam para qualquer pessoa que estivesse ouvindo.\\

%%%%%%%%%%%%%%% BLOCO DO LADO %%%%%%%%%%%%%%%
Sem um entendimento claro sobre quais frequências usar, a comunicação por rádio
era uma questão de tentativa e erro.\\
%%%%%%%%%%%%%%% BLOCO DO LADO %%%%%%%%%%%%%%%

Mas, sem entender claramente quais frequências usar, a comunicação por rádio era
uma questão de tentativa e erro. Mesmo o que o \ingles{The New York Times} descreveu
como as ``ondas de música sem lar'' da transmissão de Caruso entraram em conflito
com outra estação que, ``apesar de todas as súplicas'', insistiu em transmitir na
frequência idêntica de 350 kHz. Algumas pessoas podiam ``capturar o êxtase'' da
voz de Caruso, mas outras só recebiam algum irritante código Morse do outro
radiodifusor: ``Acabei de tomar uma cerveja agora''.\footnote{``Wireless Melody
Jarred,'' New York Times, January 14, 1910.}

\end{document}