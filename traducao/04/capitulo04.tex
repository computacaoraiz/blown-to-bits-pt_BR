\chapter[Gatekeepers]{Gatekeepers\\\large\textit{Quem está no comando aqui?}}
\label{gatekeepers}

\section{Quem Controla o Fluxo de Bits?}
\label{gatekeepers:fluxo-bits}

Quando o Sindicato dos Trabalhadores das Telecomunicações entrou em greve contra
a Telus, a principal empresa de telecomunicações do oeste do Canadá,\footnote{Ian
Austen, ``A Canadian Telecom's Labor Dispute Leads to Blocked Web Sites and
Questions of Censorship,'' \ingles{The New York Times}, August 1, 2005, \url{https://www.nytimes.com/2005/08/01/business/worldbusiness/a-canadian-telecoms-labor-dispute-leads-to-blocked.html}.}
uma discussão sobre a interrupção da greve surgiu em um site pró-sindicato operado
por um funcionário da Telus. Então, de repente, o site se tornou inacessível para
qualquer pessoa que estivesse usando o serviço de Internet da Telus. Os assinantes
da Telus podiam receber bits originados de lugares que vão do Afeganistão ao Zimbábue.
Eles podiam receber bits que representavam desde orquestras sinfônicas até pornografia.
Mas se eles quisessem ver a discussão sobre a resistência aos esforços da gerência
para interromper a greve, não podiam. A Telus tomou a posição de que, como os cabos
que entregavam os bits pertenciam à empresa, ela poderia escolher entregar ou não
os bits.

O sindicato estava furioso, e os especialistas jurídicos estavam confusos. Parecia
ser claro o suficiente que a Telus não poderia cortar o serviço telefônico para o
sindicato ou seus apoiadores se quisesse, mas as leis haviam sido escritas em uma
era pré-Internet. A Telus estava dentro de seus direitos ao interromper o serviço
de Internet nesse caso? A empresa observou que também havia bloqueado o site
telusscabs.ca, que mostrava fotos de gerentes e funcionários que estavam indo
trabalhar apesar da greve. A Telus afirmou que tinha responsabilidade pela segurança
deles e sentia-se obrigada a protegê-los. No entanto, descobriu-se que a Telus
bloqueou muitos mais do que esses 2 sites. O servidor web que hospedava esses 2
sites também hospedava outros 766 sites, incluindo um site de medicina alternativa
e um site de arrecadação de fundos para pesquisa sobre câncer de mama. Ao bloquear
com sucesso os 2 sites ofensivos, a Telus também bloqueou todos os outros.

A Telus voltou atrás depois de garantir que nenhum material ameaçador seria publicado.
Mas o incidente --- e outros semelhantes --- levantou questões que ainda hoje não têm
respostas claras e geralmente aceitas. Quem controla o que as pessoas podem fazer na
Internet?

\section{A Internet Aberta?}
\label{gatekeepers:internet-aberta}

Ninguém deveria estar encarregado da Internet. Nem mesmo era para ser algo que pudesse
ser possuído ou controlado. Era para ser mais como uma linguagem que diferentes pessoas
poderiam usar da maneira que pudessem imaginar, para conversar umas com as outras,
recitar poesias ou cantar músicas. Deveria ser como o ``éter luminífero'', a substância
invisível que preenchia o espaço, que os físicos costumavam pensar que deveria existir
porque a luz não poderia ir de um lugar para outro sem ele. A Internet deveria ser um
meio que possibilitasse a comunicação de qualquer pessoa com qualquer pessoa e de
qualquer lugar para qualquer lugar, mas o controle dessa comunicação era considerado
impossível, pois não haveria lugar para controlá-la. Qualquer pessoa que quisesse
participar de uma conversa poderia fazê-lo --- apenas falando a linguagem dos protocolos
da Internet.

Pergunte a Alex Jones se as coisas funcionaram dessa maneira. Um destacado teórico da
conspiração americano --- ou, como ele se considera, um ``criminoso de pensamento contra
o Grande Irmão'' --- Jones conquistou uma grande quantidade de seguidores no YouTube,
Facebook, LinkedIn e em outros sites de redes sociais. Milhões de pessoas seguiam cada
palavra dele --- e cada rumor maluco que ele promovia era enviado instantaneamente para
seus celulares para que pudessem ver. E então, de repente, muitos sites o baniram. A
Apple parou de fornecer o aplicativo de Jones aos usuários. Você ainda pode encontrar o
site dele se procurar, mas o Pinterest não vai sugerir isso para você.

Se isso não te convencer de que a Internet não é um paraíso participativo, use a Internet
para perguntar a qualquer pessoa na China o que aconteceu em 4 de junho. Seja por e-mail,
mensagens de texto ou Weibo (a versão chinesa do Twitter), é improvável que sua mensagem
chegue a alguém, pois a menção ao dia 4 de junho foi amplamente censurada. Pergunte a
qualquer pessoa em Hong Kong, e você nem precisará dizer qual ano; o massacre de Tiananmen
em 4 de junho de 1989 ainda é vividamente lembrado mesmo após mais de 30 anos. Mas na
Internet do continente, os bilhões de usuários nunca falam sobre o dia 4 de junho.
Mencioná-lo resulta em uma conversa sendo rapidamente apagada em vez de se espalhar
rapidamente. E o governo não é o único guardião que controla o compartilhamento eletrônico
de informações na China. Quando os manifestantes em Hong Kong usaram um aplicativo chamado
HKmap.live para se organizar, o governo chinês ficou furioso, e a Apple removeu o
aplicativo de sua App Store. O Google respondeu de forma semelhante a um pedido da polícia
de Hong Kong para retirar do ar um jogo que permitia aos usuários desempenharem o papel de
manifestantes.\footnote{Tripp Mickle et al., ``Apple, Google Pull Hong Kong Protest Apps
Amid China Uproar,'' \ingles{Wall Street Journal}, October 10, 2019,
\url{https://www.wsj.com/articles/apple-pulls-hong-kong-cop-tracking-map-app-after-china-uproar-11570681464}.}

Ou pergunte a Hasan Minhaj, um comediante americano cujo programa ``Patriot Act'' é
distribuído pela Netflix. Seus comentários críticos sobre o príncipe herdeiro saudita
Mohammed bin Salman podem ser assistidos em quase todo o mundo --- mas não onde eles
teriam mais significado, na Arábia Saudita. O governo saudita exigiu que o episódio fosse
retirado do ar, citando uma lei que criminaliza a ``produção, preparação, transmissão ou
armazenamento de material que afete a ordem pública, os valores religiosos, a moral pública
e a privacidade, através de redes de informação ou computadores.'' A Netflix respondeu
dizendo que apoia a liberdade artística --- mas, mesmo assim, removeu o vídeo.\footnote{Jim
Rutenberg, ``Netflix’s Bow to Saudi Censors Comes at a Cost to Free Speech,'' \ingles{The
New York Times}, January 6, 2019, \url{https://www.nytimes.com/2019/01/06/business/media/netflix-saudi-arabia-censorship-hasan-minhaj.html}.}
A liberdade artística de Minhaj se mostrou incompatível com a liberdade pessoal dos
funcionários da Netflix, que poderiam ser sujeitos a dez anos ou mais de prisão pelo crime
vagamente definido de transmitir material que afete a ordem pública.

Ou pergunte a qualquer pessoa que venda algo. O mecanismo de busca do Google é utilizado
em mais de 90\% das pesquisas na Internet; o Bing da Microsoft ocupa o segundo lugar, com
3\%. Se você quer vender ferramentas e não aparece na primeira página de resultados
quando as pessoas procuram por ``ferramenta,'' pode ser difícil atrair atenção. ``O
Google é o guardião da \ingles{World Wide Web}, da internet como a conhecemos,'' como
disse o advogado Gary Reback. ``Cada bit hoje é tão importante como o petróleo era quando
John D. Rockefeller estava o monopolizando.''\footnote{Steve Kroft, ``How Did Google Get
so Big?'' CBS News, May 21, 2018, \url{https://www.cbsnews.com/news/how-did-google-get-so-big/}.}
O Google se defendeu respondendo que não é um monopólio porque os resultados de busca da
Amazon também influenciam os resultados das compras. (O Google poderia ter observado que
isso acontece especialmente porque a Amazon concede um selo de ``Escolha da Amazon'' aos
produtos que favorece.) Mas a defesa do Google apenas enfatiza o fato de que o número de
guardiões é pequeno, mesmo que seja ligeiramente maior do que um. Pode haver 1.000 pequenas
empresas fabricando e vendendo ferramentas, mas apenas algumas que aparecem na primeira
página dos resultados de busca têm chances de obter negócios na Internet --- e elas estão
competindo por visibilidade contra empresas muito maiores.

Ou pergunte a qualquer pessoa em Browning, Montana, se qualquer um pode usar a Internet
para se comunicar com qualquer pessoa. Na Reserva Indígena Blackfoot, apenas 0,1\% da
população tem acesso à Internet de alta velocidade. A conectividade mais barata de
qualquer tipo é de 10 Mbps e custa quase US\$ 780 por ano\footnote{``Internet Providers
in Browning, Montana,'' Broadband Now, accessed April 27, 2020, \url{https://broadbandnow.com/Montana/Browning}.}
--- em um lugar com uma taxa de pobreza de 35\% e uma renda média anual familiar de menos
de US\$ 22.000.\footnote{``Browning, MT,'' Data USA, accessed April 27, 2020,
\url{https://datausa.io/profile/geo/browning-mt/}.} Na maioria das cidades americanas,
obter dados pela Internet é como abrir a torneira para obter água, e o mesmo acontece em
toda a Finlândia e Japão. O serviço de Internet é abundante em lugares onde é lucrativo
para fornecedores privados ou onde é fornecido como uma questão de política pública. Nada
disso é verdade em Browning, onde o serviço de Internet é praticamente indisponível.

Independentemente da história, da teoria e do potencial, a realidade é que algumas
corporações e alguns governos exercem um enorme controle sobre o que a maioria das pessoas
realmente vê na Internet e o que elas podem fazer com ela. Se essas empresas e instituições
não fornecerem a infraestrutura para entregar a sua mensagem, ou se eles derem menor
prioridade à entrega do seu anúncio, notícia ou crítica política, será como se você tivesse
gritado no deserto. Alguém pode ouvir o que você diz, mas provavelmente não muitas pessoas.
Isso é exatamente o oposto da forma como a Internet foi projetada para funcionar. A Internet
evoluiu desde seu design financiado publicamente como um sistema aberto com usos potenciais
ilimitados para um sistema em que algumas empresas privadas detêm um controle quase
monopolista sobre cada um de seus principais aspectos.

Este capítulo foca em três tipos de guardiões da Internet. O primeiro são os controladores
dos canais de dados pelos quais os bits fluem. Chamaremos esses de guardiões dos links. O
segundo são os controladores das ferramentas que usamos para encontrar coisas na Web.
Chamaremos esses de guardiões das buscas. E o terceiro são os controladores das conexões
sociais que são, para muitos de nós, nosso uso mais importante da Internet. Chamaremos esses
de guardiões sociais.

Os guardiões dos links controlam o meio físico através do qual os bits fluem, enquanto os
guardiões das buscas e os guardiões sociais controlam o que esses bits expressam --- ou seja,
eles são guardiões de conteúdo. No entanto, essas distinções não são tão nítidas quanto
possam parecer. Os guardiões dos links podem ser capazes, por exemplo, de censurar ou
favorecer determinados conteúdos em relação a outros ou determinados clientes em relação a
outros. Guardiões de conteúdo podem entrar no mercado de links se acharem que será vantajoso
resistir ao controle quase monopolista dos guardiões dos links --- independentemente de
consolidar o controle de links e conteúdo estar ou não no interesse público mais amplamente
concebido. Guardiões sociais adicionaram a busca dentro de suas plataformas sociais para minar
o controle quase monopolista dos guardiões das buscas.

Nos Estados Unidos, todas as três funções dos guardiões estão em grande parte nas mãos do
setor privado. Em outras partes do mundo, os governos assumiram alguns desses papéis de
guardiões. Debates familiares sobre serviços privados versus públicos têm acontecido como
parte da Internet quase desde o seu início. Uma argumentação conhecida é que a concorrência
entre partes privadas reduz os custos e melhora a qualidade; mas outros argumentam que a
consolidação resulta em eficiências de escala que superam os efeitos negativos da redução da
concorrência. De acordo com outra narrativa, o governo deve fornecer infraestrutura de
benefício geral para o povo, custeada através de impostos gerais em vez de compra privada; ele
deve fornecer o ``éter'', da mesma forma que fornece estradas e serviços postais, igualmente
para todos. Mas tais analogias apenas levantam a questão se a Internet realmente se assemelha
às estradas públicas, em que qualquer um pode dirigir, ou se se assemelha à televisão a cabo
ou cinemas, que são mais acessíveis em áreas urbanas do que rurais e não estão disponíveis
para pessoas que não estão dispostas a pagar as taxas.

Os resultados das diversas respostas possíveis para tais questões têm sido variadas e dependem
em certa medida de questões fundamentais de metas cívicas e econômicas. Em regimes autoritários,
comprometidos com a ``harmonia'' social em detrimento da liberdade individual, o controle de
conteúdo pode ser ainda mais centralizado do que nos Estados Unidos. Por outro lado,
investimentos substanciais do governo na infraestrutura fora dos Estados Unidos resultaram em
conectividade muito melhor em certos países democráticos e não democráticos. Os debates sobre
o nível correto de investimento e supervisão governamental da Internet não são mais simples do
que a história do envolvimento governamental na entrega de correspondência, eletricidade, serviço
telefônico, educação ou cuidados médicos. Após contar rapidamente a história de como a Internet
aberta caiu sob o controle de oligopólios de guardiões, levantaremos as questões com as quais a
sociedade fica sobre o que, se é que alguma coisa, fazer.

Vamos começar com como tudo funciona.