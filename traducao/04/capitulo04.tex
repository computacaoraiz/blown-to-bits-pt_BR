\chapter[Gatekeepers]{Gatekeepers\\\large\textit{Quem está no comando aqui?}}
\label{gatekeepers}

\section{Quem Controla o Fluxo de Bits?}
\label{gatekeepers:fluxo-bits}

Quando o Sindicato dos Trabalhadores das Telecomunicações entrou em greve contra
a Telus, a principal empresa de telecomunicações do oeste do Canadá,\footnote{Ian
Austen, ``A Canadian Telecom's Labor Dispute Leads to Blocked Web Sites and
Questions of Censorship,'' \ingles{The New York Times}, August 1, 2005,
\url{https://www.nytimes.com/2005/08/01/business/worldbusiness/a-canadian-telecoms-labor-dispute-leads-to-blocked.html}.}
uma discussão sobre a interrupção da greve surgiu em um site pró-sindicato
operado por um funcionário da Telus. Então, de repente, o site se tornou
inacessível para qualquer pessoa que estivesse usando o serviço de Internet da
Telus. Os assinantes da Telus podiam receber bits originados de lugares que vão
do Afeganistão ao Zimbábue. Eles podiam receber bits que representavam desde
orquestras sinfônicas até pornografia. Mas se eles quisessem ver a discussão
sobre a resistência aos esforços da gerência para interromper a greve, não podiam.
A Telus tomou a posição de que, como os cabos que entregavam os bits pertenciam à
empresa, ela poderia escolher entregar ou não os bits.

O sindicato estava furioso, e os especialistas jurídicos estavam confusos. Parecia
ser claro o suficiente que a Telus não poderia cortar o serviço telefônico para o
sindicato ou seus apoiadores se quisesse, mas as leis haviam sido escritas em uma
era pré-Internet. A Telus estava dentro de seus direitos ao interromper o serviço
de Internet nesse caso? A empresa observou que também havia bloqueado o site
telusscabs.ca, que mostrava fotos de gerentes e funcionários que estavam indo
trabalhar apesar da greve. A Telus afirmou que tinha responsabilidade pela
segurança deles e sentia-se obrigada a protegê-los. No entanto, descobriu-se que a
Telus bloqueou muitos mais do que esses 2 sites. O servidor web que hospedava esses
2 sites também hospedava outros 766 sites, incluindo um site de medicina
alternativa e um site de arrecadação de fundos para pesquisa sobre câncer de mama. Ao
bloquear com sucesso os 2 sites ofensivos, a Telus também bloqueou todos os outros.

A Telus voltou atrás depois de garantir que nenhum material ameaçador seria
publicado. Mas o incidente --- e outros semelhantes --- levantou questões que ainda
hoje não têm respostas claras e geralmente aceitas. Quem controla o que as pessoas
podem fazer na Internet?

\section{A Internet Aberta?}
\label{gatekeepers:internet-aberta}

Ninguém deveria estar encarregado da Internet. Nem mesmo era para ser algo que
pudesse ser possuído ou controlado. Era para ser mais como uma linguagem que
diferentes pessoas poderiam usar da maneira que pudessem imaginar, para conversar umas
com as outras, recitar poesias ou cantar músicas. Deveria ser como o ``éter luminífero'',
a substância invisível que preenchia o espaço, que os físicos costumavam pensar que
deveria existir porque a luz não poderia ir de um lugar para outro sem ele. A Internet
deveria ser um meio que possibilitasse a comunicação de qualquer pessoa com qualquer
pessoa e de qualquer lugar para qualquer lugar, mas o controle dessa comunicação era
considerado impossível, pois não haveria lugar para controlá-la. Qualquer pessoa que
quisesse participar de uma conversa poderia fazê-lo --- apenas falando a linguagem
dos protocolos da Internet.

Pergunte a Alex Jones se as coisas funcionaram dessa maneira. Um destacado teórico
da conspiração americano --- ou, como ele se considera, um ``criminoso de pensamento
contra o Grande Irmão'' --- Jones conquistou uma grande quantidade de seguidores no
YouTube, Facebook, LinkedIn e em outros sites de redes sociais. Milhões de pessoas
seguiam cada palavra dele --- e cada rumor maluco que ele promovia era enviado
instantaneamente para seus celulares para que pudessem ver. E então, de repente,
muitos sites o baniram. A Apple parou de fornecer o aplicativo de Jones aos usuários.
Você ainda pode encontrar o site dele se procurar, mas o Pinterest não vai sugerir
isso para você.

Se isso não te convencer de que a Internet não é um paraíso participativo, use a
Internet para perguntar a qualquer pessoa na China o que aconteceu em 4 de junho.
Seja por e-mail, mensagens de texto ou Weibo (a versão chinesa do Twitter), é
improvável que sua mensagem chegue a alguém, pois a menção ao dia 4 de junho foi
amplamente censurada. Pergunte a qualquer pessoa em Hong Kong, e você nem precisará
dizer qual ano; o massacre de Tiananmen em 4 de junho de 1989 ainda é vividamente
lembrado mesmo após mais de 30 anos. Mas na Internet do continente, os bilhões de
usuários nunca falam sobre o dia 4 de junho. Mencioná-lo resulta em uma conversa
sendo rapidamente apagada em vez de se espalhar rapidamente. E o governo não é o
único guardião que controla o compartilhamento eletrônico de informações na China.
Quando os manifestantes em Hong Kong usaram um aplicativo chamado HKmap.live para
se organizar, o governo chinês ficou furioso, e a Apple removeu o aplicativo de sua
App Store. O Google respondeu de forma semelhante a um pedido da polícia de Hong Kong
para retirar do ar um jogo que permitia aos usuários desempenharem o papel de
manifestantes.\footnote{Tripp Mickle et al., ``Apple, Google Pull Hong Kong Protest
Apps Amid China Uproar,'' \ingles{Wall Street Journal}, October 10, 2019,
\url{https://www.wsj.com/articles/apple-pulls-hong-kong-cop-tracking-map-app-after-china-uproar-11570681464}.}

Ou pergunte a Hasan Minhaj, um comediante americano cujo programa ``Patriot Act'' é
distribuído pela Netflix. Seus comentários críticos sobre o príncipe herdeiro saudita
Mohammed bin Salman podem ser assistidos em quase todo o mundo --- mas não onde eles
teriam mais significado, na Arábia Saudita. O governo saudita exigiu que o episódio
fosse retirado do ar, citando uma lei que criminaliza a ``produção, preparação,
transmissão ou armazenamento de material que afete a ordem pública, os valores
religiosos, a moral pública e a privacidade, através de redes de informação ou
computadores.'' A Netflix respondeu dizendo que apoia a liberdade artística --- mas,
mesmo assim, removeu o vídeo.\footnote{Jim Rutenberg, ``Netflix’s Bow to Saudi Censors
Comes at a Cost to Free Speech,'' \ingles{The New York Times}, January 6, 2019,
\url{https://www.nytimes.com/2019/01/06/business/media/netflix-saudi-arabia-censorship-hasan-minhaj.html}.}
A liberdade artística de Minhaj se mostrou incompatível com a liberdade pessoal dos
funcionários da Netflix, que poderiam ser sujeitos a dez anos ou mais de prisão pelo
crime vagamente definido de transmitir material que afete a ordem pública.

Ou pergunte a qualquer pessoa que venda algo. O mecanismo de busca do Google é
utilizado em mais de 90\% das pesquisas na Internet; o Bing da Microsoft ocupa o
segundo lugar, com 3\%. Se você quer vender ferramentas e não aparece na primeira
página de resultados quando as pessoas procuram por ``ferramenta,'' pode ser difícil
atrair atenção. ``O Google é o guardião da \ingles{World Wide Web}, da internet como
a conhecemos,'' como disse o advogado Gary Reback. ``Cada bit hoje é tão importante
como o petróleo era quando John D. Rockefeller estava o monopolizando.''\footnote{Steve
Kroft, ``How Did Google Get so Big?'' CBS News, May 21, 2018,
\url{https://www.cbsnews.com/news/how-did-google-get-so-big/}.}
O Google se defendeu respondendo que não é um monopólio porque os resultados de busca
da Amazon também influenciam os resultados das compras. (O Google poderia ter observado
que isso acontece especialmente porque a Amazon concede um selo de ``Escolha da Amazon''
aos produtos que favorece.) Mas a defesa do Google apenas enfatiza o fato de que o
número de guardiões é pequeno, mesmo que seja ligeiramente maior do que um. Pode haver
1.000 pequenas empresas fabricando e vendendo ferramentas, mas apenas algumas que
aparecem na primeira página dos resultados de busca têm chances de obter negócios na
Internet --- e elas estão competindo por visibilidade contra empresas muito maiores.

Ou pergunte a qualquer pessoa em Browning, Montana, se qualquer um pode usar a Internet
para se comunicar com qualquer pessoa. Na Reserva Indígena Blackfoot, apenas 0,1\% da
população tem acesso à Internet de alta velocidade. A conectividade mais barata de
qualquer tipo é de 10 Mbps e custa quase US\$ 780 por ano\footnote{``Internet Providers
in Browning, Montana,'' Broadband Now, accessed April 27, 2020, \url{https://broadbandnow.com/Montana/Browning}.}
--- em um lugar com uma taxa de pobreza de 35\% e uma renda média anual familiar de
menos de US\$ 22.000.\footnote{``Browning, MT,'' Data USA, accessed April 27, 2020,
\url{https://datausa.io/profile/geo/browning-mt/}.} Na maioria das cidades americanas,
obter dados pela Internet é como abrir a torneira para obter água, e o mesmo acontece
em toda a Finlândia e Japão. O serviço de Internet é abundante em lugares onde é
lucrativo para fornecedores privados ou onde é fornecido como uma questão de política
pública. Nada disso é verdade em Browning, onde o serviço de Internet é praticamente
indisponível.

Independentemente da história, da teoria e do potencial, a realidade é que algumas
corporações e alguns governos exercem um enorme controle sobre o que a maioria das
pessoas realmente vê na Internet e o que elas podem fazer com ela. Se essas empresas e
instituições não fornecerem a infraestrutura para entregar a sua mensagem, ou se eles
derem menor prioridade à entrega do seu anúncio, notícia ou crítica política, será como
se você tivesse gritado no deserto. Alguém pode ouvir o que você diz, mas provavelmente
não muitas pessoas. Isso é exatamente o oposto da forma como a Internet foi projetada
para funcionar. A Internet evoluiu desde seu design financiado publicamente como um
sistema aberto com usos potenciais ilimitados para um sistema em que algumas empresas
privadas detêm um controle quase monopolista sobre cada um de seus principais aspectos.

Este capítulo foca em três tipos de guardiões da Internet. O primeiro são os controladores
dos canais de dados pelos quais os bits fluem. Chamaremos esses de guardiões dos links.
O segundo são os controladores das ferramentas que usamos para encontrar coisas na Web.
Chamaremos esses de guardiões das buscas. E o terceiro são os controladores das conexões
sociais que são, para muitos de nós, nosso uso mais importante da Internet. Chamaremos
esses de guardiões sociais.

Os guardiões dos links controlam o meio físico através do qual os bits fluem, enquanto
os guardiões das buscas e os guardiões sociais controlam o que esses bits expressam
--- ou seja, eles são guardiões de conteúdo. No entanto, essas distinções não são tão
nítidas quanto possam parecer. Os guardiões dos links podem ser capazes, por exemplo,
de censurar ou favorecer determinados conteúdos em relação a outros ou determinados
clientes em relação a outros. Guardiões de conteúdo podem entrar no mercado de links se
acharem que será vantajoso resistir ao controle quase monopolista dos guardiões dos links
--- independentemente de consolidar o controle de links e conteúdo estar ou não no
interesse público mais amplamente concebido. Guardiões sociais adicionaram a busca dentro
de suas plataformas sociais para minar o controle quase monopolista dos guardiões das
buscas.

Nos Estados Unidos, todas as três funções dos guardiões estão em grande parte nas mãos
do setor privado. Em outras partes do mundo, os governos assumiram alguns desses papéis
de guardiões. Debates familiares sobre serviços privados versus públicos têm acontecido
como parte da Internet quase desde o seu início. Uma argumentação conhecida é que a
concorrência entre partes privadas reduz os custos e melhora a qualidade; mas outros
argumentam que a consolidação resulta em eficiências de escala que superam os efeitos
negativos da redução da concorrência. De acordo com outra narrativa, o governo deve
fornecer infraestrutura de benefício geral para o povo, custeada através de impostos
gerais em vez de compra privada; ele deve fornecer o ``éter'', da mesma forma que fornece
estradas e serviços postais, igualmente para todos. Mas tais analogias apenas levantam a
questão se a Internet realmente se assemelha às estradas públicas, em que qualquer um
pode dirigir, ou se se assemelha à televisão a cabo ou cinemas, que são mais acessíveis
em áreas urbanas do que rurais e não estão disponíveis para pessoas que não estão
dispostas a pagar as taxas.

Os resultados das diversas respostas possíveis para tais questões têm sido variadas
e dependem em certa medida de questões fundamentais de metas cívicas e econômicas.
Em regimes autoritários, comprometidos com a ``harmonia'' social em detrimento da
liberdade individual, o controle de conteúdo pode ser ainda mais centralizado do que
nos Estados Unidos. Por outro lado, investimentos substanciais do governo na
infraestrutura fora dos Estados Unidos resultaram em conectividade muito melhor em
certos países democráticos e não democráticos. Os debates sobre o nível correto de
investimento e supervisão governamental da Internet não são mais simples do que a
história do envolvimento governamental na entrega de correspondência, eletricidade,
serviço telefônico, educação ou cuidados médicos. Após contar rapidamente a história
de como a Internet aberta caiu sob o controle de oligopólios de guardiões,
levantaremos as questões com as quais a sociedade fica sobre o que, se é que alguma
coisa, fazer.

Vamos começar com como tudo funciona.

\section{Conectando os Pontos: Projetado para Compartilhar e Sobreviver}
\label{gatekeepers:conectando-pontos}

A Internet surgiu a partir do ARPANET, um projeto de rede de computadores do
Departamento de Defesa dos Estados Unidos (DoD) dos anos 1970. Por meio de sua
Agência de Projetos de Pesquisa Avançada (ARPA, posteriormente DARPA), o DoD estava
pagando direta ou indiretamente por máquinas de ponta em muitos laboratórios
acadêmicos e de pesquisa nacionais. A ARPA tinha duas preocupações.

Uma das preocupações era prosaica: a agência estava pagando por grandes e caros
computadores em todo o país, mas não havia maneira de que máquinas subutilizadas em
um local pudessem ser utilizadas para resolver problemas que os pesquisadores
precisavam resolver em outros locais. Assim, cada pesquisador queria o maior
computador possível, e muito tempo de computação estava sendo desperdiçado. Os
cientistas poderiam colocar seus dados em fitas e enviá-los pelo país por transporte
aéreo, mas não havia maneira de enviar os bits sem enviar os átomos. Então, a ARPA
queria melhorar a utilização dos computadores de pesquisa que estava financiando,
conectando-os em rede.

A outra preocupação da ARPA atingiu o cerne da missão militar. O DoD estava preocupado
há algum tempo que suas bases e navios espalhados não pudessem se comunicar se locais
críticos fossem destruídos em uma guerra nuclear. No início da década de 1960, a
preocupação era se a rede telefônica sobreviveria a um ataque que eliminasse alguns
centros de comutação chave, onde muitas linhas telefônicas de longa distância estavam
interconectadas.

Naquela época, o pesquisador Paul Baran estudou as propriedades de uma rede
descentralizada, na qual havia muitos pontos de junção, cada um conectado apenas a
alguns outros. (A rede telefônica, por outro lado, consistia de um pequeno número de
estações centrais de comutação conectadas a clientes, como raios se unindo ao centro de
uma roda até sua borda.) Na rede em forma de malha proposta por Baran, haveria muitos
caminhos entre quaisquer dois pontos, de modo que eliminar qualquer um dos pontos de
junção não impediria que outros pontos se comunicassem. Baran imaginou uma conexão
irregular de pontos de comutação, como mostrado abaixo em uma ilustração de seu artigo
de 1962.\footnote{Paul Baran, ``On Distributed Communications Networks,'' (RAND Corporation,
Santa Monica, CA, September 1962), Reprinted with permission.
\url{https://www.rand.org/pubs/papers/P2626.html}.}

%%%%%%%%%%% AQUI FICA UMA FIGURA, PÁGINA 80 %%%%
FIGURA\\
%%%%%%%%%%% AQUI FICA UMA FIGURA %%%%%%%%%%%%%%%

Baran contribuiu com uma segunda ideia importante: se um ponto de comutação ficasse
inativo, outra rota poderia ser encontrada que não passasse por ele, desde que o próprio
ponto de comutação não fosse um dos extremos da comunicação. Através da configuração
correta dos comutadores, a comunicação entre dois pontos poderia ser estabelecida ao
longo de um caminho específico. No entanto, se qualquer um dos pontos ao longo do caminho
fosse danificado, isso interromperia essa comunicação. Da mesma forma, uma falha de
hardware comum em qualquer um desses pontos intermediários também causaria interrupções.
Era importante proteger a integridade das comunicações individuais, mesmo quando os
componentes da rede falhassem de maneiras imprevisíveis.

Baran propôs dividir as comunicações em pequenos fragmentos de bits, o que hoje chamamos
de ``pacotes''. Além do ``\ingles{payload}'', um fragmento da própria comunicação, os
pacotes conteriam informações identificando a origem e o destino (semelhante às
informações de endereço em uma carta postal) e também um número de série para que o nó de
destino pudesse reuni-los na ordem correta, caso chegassem fora de sequência. Com todas
essas informações no ``envelope'', os pacotes que compõem uma única comunicação não
precisavam seguir o mesmo caminho. Se uma parte da rede estivesse indisponível, os nós da
rede poderiam direcionar os pacotes por um caminho diferente. Fazer tudo isso funcionar não
foi simples --- como os nós da rede saberiam em que direção encaminhar um pacote? --- mas,
em princípio, a ideia de Baran de uma interconexão em forma de malha e de comunicação em
forma de pacotes atenderia aos requisitos militares de sobrevivência.

\subsection{Protocolos: Como Apertar as Mãos com Estranhos}
\label{gatekeepers:protocolos}

Uma vez que a ARPANET estava operacional e conectava algumas dezenas de computadores,
começou a ficar claro que o que precisava ser conectado não eram computadores individuais,
mas redes de computadores existentes. Diferentes formas de interconectar computadores
poderiam coexistir, desde que as redes usassem uma linguagem comum para se comunicar entre
si. E nas décadas de 1970 e 1980, diferentes tipos de redes de computadores existiam, cada
uma utilizando os padrões de uma empresa de computadores diferente. A IBM tinha sua SNA
(Arquitetura de Rede de Sistemas). A Digital Equipment Corporation tinha a DECnet. A Apollo
Computer conectava suas máquinas em um anel, em vez de uma árvore ramificada ou malha. Cada
empresa divulgava as vantagens de seu esquema de rede, e algumas das reivindicações eram
válidas para casos de uso específicos. No entanto, nenhum dos fabricantes tinha incentivo
para tornar suas máquinas interoperáveis com as de outros fabricantes --- até que a ARPA
declarou que não pagaria por mais computadores a menos que eles pudessem ser interconectados.
Partindo do sucesso da ARPANET, Vinton Cerf e Robert Kahn projetaram os protocolos para
interconectar redes de computadores.\footnote{Vinton G. Cerf and Robert E Kahn, ``A Protocol
for Packet Network Intercommunication,'' \ingles{IEEE Transactions on Communications}, no. 5
(1974): 13.} Ou seja, eles projetaram a Internet.

A Internet são seus protocolos. A Internet não é uma máquina, nem mesmo uma coleção de
máquinas. Não é um software específico. É um conjunto de regras. Qualquer pessoa ou
organização pode construir hardware ou escrever software que siga essas regras e se tornar
uma parte funcional da Internet.

Protocolos são convenções de comunicação, como a convenção de que as pessoas apertam as
mãos direitas. Cumprimentar uns aos outros apertando as mãos esquerdas funcionaria
igualmente bem, mas a convenção estabelecida de aperto de mão com a mão direita permite que
estranhos se cumprimentem sem mediação prévia. Os protocolos da Internet são as convenções
pelas quais diferentes redes apertam as mãos para transmitir informações de uma rede para
outra. Cada rede pode operar internamente como desejar; apenas nos pontos em que as redes
estão conectadas é que os protocolos da Internet se tornam relevantes.

A decisão de tornar a ARPANET uma rede comutada por pacotes simplificou consideravelmente
o design da Internet. As redes eram conectadas à Internet por meio de pontos de conexão
chamados de \ingles{gateways}. Se um gateway se comportasse adequadamente, as informações
fluiriam por ele. Se ele não se comportasse corretamente, isso não causaria nenhum dano,
exceto isolar aquela rede do resto da Internet. Nenhum computador ou rede de computadores
precisava de permissão para ingressar na Internet. Se seguisse os padrões da Internet,
poderia ser compreendido por outros e interpretar mensagens direcionadas a ele.

Quando olhamos para a Internet hoje, ela parece variada e complicada: tantos tipos
diferentes de conteúdo, tantos tipos diferentes de dispositivos e tantos tipos diferentes
de conexões. Mas tudo é construído em cima de um único protocolo, conhecido simplesmente
como Protocolo de Internet (IP). É função do IP levar um único pacote de cerca de mil bits
de uma extremidade de um link de rede de comunicações para a outra. Os bits, ao serem
entregues, podem conter erros; nada físico funciona perfeitamente o tempo todo. Mas os
erros podem ser reconhecidos e, se necessário, tratados. Para fazer os pacotes atravessarem
a rede, o IP é usado repetidamente, como um estilo de corrente humana, com cada ponto de
comutação recebendo pacotes, verificando-os e, em seguida, enviando-os em direção ao destino
pretendido.

A simplicidade do design da Internet tornou possível construir protocolos sobre protocolos
para expandir o utilitário da Internet. Os usos iniciais da Internet eram para fazer login
em computadores de tempo compartilhado remotamente, mover arquivos de um lugar para outro
e enviar correio eletrônico. Todos esses serviços exigiam que os dados chegassem sem
erros, mas não necessariamente instantaneamente. Ninguém perceberia se uma transferência
de arquivo ou a entrega de um e-mail levasse uma fração de segundo extra, mas um único bit
mudar de 0 para 1 em trânsito poderia ter consequências catastróficas. Para tais
transferências, um protocolo foi desenvolvido para garantir que os pacotes enviados pela
fonte fossem recebidos corretamente e reconstituídos na ordem correta. Dada a falta de
confiabilidade dos nós intermediários da rede, isso requer algum registro tanto na origem
quanto no destino. Um pacote, uma vez recebido, é confirmado ao enviar um pacote especial
de volta do destino para a origem. A origem executa um temporizador; se um pacote enviado
não for confirmado antes que o temporizador zere, a origem conclui que o pacote se perdeu
de alguma forma e o retransmite.

Os detalhes são complicados, mas não são importantes para o panorama geral. O resultado é
que, desde que os comutadores estejam fazendo o melhor esforço para encaminhar os pacotes
em direção ao destino, qualquer mensagem enviada será recebida em perfeita ordem. O
protocolo que garante essas transmissões perfeitas é chamado de Protocolo de Controle de
Transmissão (TCP). Como o protocolo subjacente para mover pacotes ao longo de links
individuais da rede é o IP, TCP/IP é o nome comum para o par de convenções que tornam
comunicações confiáveis possíveis em uma rede não confiável.

Uma vez que não existem regras para ingressar na Internet, é válido questionar a suposição
de ``melhor esforço''. Será que um agente mal-intencionado poderia tentar sabotar a rede
adicionando comutadores que descartariam ou redirecionariam os pacotes em vez de enviá-los
para o destino pretendido? De fato, isso poderia acontecer, mas os pontos de comutação
vizinhos eventualmente perceberiam que os pacotes não estavam sendo entregues e começariam
a evitar os elementos maliciosos. O roteamento da Internet se autocorrige ao aprender a
evitar pontos problemáticos --- não apenas em caso de falhas de hardware, mas também em
caso de malícia. A Internet se torna mais confiável à medida que cresce em tamanho e se
torna mais interconectada.

A Internet funcionou porque, uma vez que um número suficientemente grande de partes
concordou em usá-la da maneira pretendida, os atores mal-intencionados poderiam, na prática,
ser excluídos, uma vez que eram poucos em número.

Além das informações de roteamento e \ingles{payload}, os pacotes também incluem alguns
bits redundantes para auxiliar na detecção de erros. Por exemplo, um único bit extra pode
ser adicionado a cada pacote para garantir que todos os pacotes transmitidos tenham um número
ímpar de bits 1. Se um pacote chegar com um número par de bits 1, ele pode ser reconhecido
como tendo sido corrompido durante a transmissão e descartado, para que o remetente possa
retransmiti-lo. Tais bits extras não podem garantir que cada pacote recebido esteja correto.
Mas eles garantem uma transmissão correta com probabilidade avassaladora e, do ponto de vista
prático, esse processo é suficiente para tornar a probabilidade de um erro não detectado
menor do que a probabilidade de uma catástrofe, como um impacto de meteorito, na fonte da
transmissão.

O IP, o protocolo de encaminhamento de pacotes de melhor esforço, também pode ser usado para
entregar mensagens de forma imperfeita, mas rápida. Por exemplo, pense em como a Internet
pode ser usada para transmitir comunicações de voz, como chamadas telefônicas. O sinal de voz
pode ser dividido em pequenos intervalos de tempo, cada um digitalizado e enviado pela
Internet. Mas em vez de usar o TCP, que garante a entrega, mas não a rapidez, um protocolo
diferente, chamado UDP, é utilizado. O UDP aceita alguma perda de pacotes em troca de uma
entrega mais rápida. Os tons de voz mudam lentamente de um instante para o próximo, de modo
que os pacotes de uma conversa telefônica podem ser um pouco embaralhados e alguns podem ser
omitidos totalmente, sem dificultar a compreensão da conversa --- desde que os pacotes que a
compõem cheguem a uma taxa aproximadamente correta.

Muitos outros protocolos foram projetados para outros fins e para serem sobrepostos a esses,
utilizando o TCP e o UDP para realizar tarefas de comunicação mais complexas. Por exemplo, o
Protocolo de Transferência de Hipertexto (HTTP) foi projetado para a comunicação entre um
navegador da web em um computador do usuário e um servidor web em qualquer outro lugar do
mundo. O HTTP depende do TCP para recuperar páginas da web com base em informações de
localização, como lewis.seas.harvard.edu. Portanto, sem precisar conhecer os detalhes de como
o TCP opera, qualquer pessoa poderia configurar um servidor web que entregasse páginas da web
em resposta a requisições recebidas.

\subsection{Quem está no comando?}
\label{gatekeepers:quem}

Não existem policiais da Internet para obrigar alguém a formatar seus pacotes como TCP, IP,
UDP ou outros protocolos estipulados. Ninguém o expulsará da Internet se você colocar o
endereço de origem onde o destino deveria estar e vice-versa. Se seus pacotes não estiverem
de acordo com os padrões, eles simplesmente não serão entregues, ou serão ignorados se forem
entregues.

No entanto, a Internet possui algumas autoridades governantes. Uma delas é o Internet
Engineering Task Force (IETF), que estabelece os padrões para os protocolos da Internet. O
IETF é uma organização notável. Está aberta a qualquer pessoa que queira participar e toma
decisões com base em ``consenso geral e código em funcionamento''. Em anos anteriores, o
IETF se reunia em uma sala e determinava o ``consenso geral'' ao fazer com que os membros
debatessem. Maiorias substanciais eram evidentes para todos, e as preferências individuais
desfrutavam de um nível de anonimato, pois em um grupo grande é difícil identificar quem
está concordando e quem não está. Como a maioria das mudanças nos protocolos da Internet são
aprimoramentos e adições que não alteram nada que já esteja funcionando, raramente há
necessidade de tomar uma decisão positiva sob pressão de tempo; o IETF pode adiar decisões,
permitir que as pessoas falem mais enquanto ajustam suas propostas e aguardar o
desenvolvimento de um consenso verdadeiro.

Assim, a Internet é aberta por design. Qualquer pessoa pode participar do processo de tomada
de decisão. Não estaria errado lembrar do utopismo comunal dos anos 1960. O membro inicial
do IETF, David Clark, popularmente disse: ``Rejeitamos reis, presidentes e votações.
Acreditamos em um consenso geral e código em funcionamento'' --- a última frase indicando a
preferência do engenheiro por provas de conceito em vez de conceitos isolados.\footnote{Pete
Resnick, ``On Consensus and Humming in the IETF,'' Internet Engineering Task Force, June 2014,
\url{https://tools.ietf.org/html/rfc7282}.} Claro, uma vez que a Internet se tornou amplamente
adotada, seria necessário fazer muita persuasão para desenvolver um consenso para mudar
qualquer coisa que se tornasse importante para muitas pessoas. Mas se você e eu estivéssemos
a meio mundo de distância um do outro e decidíssemos desenvolver nosso próprio protocolo
secreto para (digamos) duetos de xilofone transpacíficos, poderíamos programar nossos
computadores para trocar pacotes IP que mais ninguém saberia o que fazer. O IETF explica seu
papel desta maneira em sua declaração de missão:

\begin{quote}
Quando o IETF assume a propriedade de um protocolo ou função, ele aceita a responsabilidade
por todos os aspectos do protocolo, mesmo que alguns aspectos raramente ou nunca sejam vistos
na Internet. Por outro lado, quando o IETF não é responsável por um protocolo ou função, ele
não tenta exercer controle sobre ele, mesmo que às vezes possa tocar ou afetar a Internet
\footnote{Harald Tveit Alvestrand, ``A Mission Statement for the IETF,'' Internet Engineering
Task Force, October 2004, \url{https://tools.ietf.org/html/rfc3935}.}
\end{quote}

Esta é uma afirmação notável e mostra como a metáfora da ``Autoestrada da Informação'' se
desintegra quando aplicada à Internet. Se a Internet é uma estrada, é uma em que os veículos
motorizados aderem voluntariamente a certas convenções para poderem compartilhar a estrada com
segurança, mas ciclistas e skatistas também são bem-vindos a usar as vias --- embora por sua
própria conta e risco.

A Internet é aberta em outra direção também. Assim como o IP serve como a camada base para
uma hierarquia de protocolos, o IP em si é um protocolo lógico, não físico. Os pacotes da
Internet podem ser transmitidos por fios de cobre, por cabos de fibra óptica ou por ondas
de rádio. Se você é um usuário comum de computador pessoal comprando algo na Amazon, é
provável que os pacotes que vão e voltam entre você e a Amazon passem por todos os três e
mais, à medida que se movem do seu computador para o roteador sem fio, para o seu provedor
de serviços de Internet, através da Internet, para a rede corporativa da Amazon e para um
dos computadores da Amazon. Sempre que os engenheiros desenvolvem uma nova forma de
transmitir bits através de meios físicos, eles também podem desenvolver uma implementação do
IP que funcione nesse meio físico. Há até mesmo um protocolo de pombo-correio que, em
princípio, poderia ser usado para implementar o IP.

O IP, o formato pelo qual todos os pacotes passam pela Internet, desempenha um papel
semelhante ao design da tomada elétrica de 120V onipresente, com três buracos de forma e
dimensões especificadas. A fonte elétrica de um lado da tomada pode ser, em última instância
uma represa hidrelétrica a centenas de quilômetros de distância, painéis solares a apenas
alguns metros de distância ou baterias. Desde que a eletricidade esteja de acordo com os
padrões, a tomada está fazendo o seu trabalho. Os dispositivos que são conectados à tomada
podem ser geladeiras, escovas de dente, aspiradores de pó ou brocas dentárias. Desde que um
dispositivo seja equipado com a tomada certa e seja projetado para funcionar com corrente
alternada padrão, ele funcionará. Da mesma forma, o Protocolo de Internet age como um mediador
universal entre aplicativos e meios físicos.

Na verdade, a padronização do IP é a razão pela qual a Internet possui tantos usos que
inicialmente não foram previstos. O Zoom e o Facetime --- aplicações da Internet para
conectar pessoas por meio de links de áudio e vídeo ao vivo --- foram construídos com base
no IP, mesmo que não houvesse absolutamente nada sobre tais serviços no design original da
Internet. Os inventores do sistema de telefone pela Internet, o Skype --- um pequeno grupo de
engenheiros escandinavos e estonianos --- apenas precisaram adaptar os protocolos da Internet
para seus propósitos. E eles não precisaram pedir permissão à IETF ou a qualquer outra
autoridade para começar a usar o Skype ou incentivar outros a começarem a pagar por seu uso.

\section{A Internet Não Tem Guardiões?}
\label{gatekeepers:nao-tem-guardioes}

Sempre foi um exagero dizer que a Internet não possui guardiões, mas isso é menos verdade 
agora do que costumava ser. Como veremos em breve, em alguns países, os governos são os
principais guardiões, e em outros, como nos Estados Unidos, as corporações privadas
assumem papéis de guardiões. Vamos começar com as formas de guarda que existiram por
muito tempo.

\subsection{Nomes para Números: Qual é o Seu Endereço?}
\label{gatekeepers:nomes-para-numeros}

O primeiro fato da vida na Internet é que não adianta estar ``na'' Internet se ninguém
puder te encontrar. Os pacotes que fluem pela Internet possuem endereços numéricos. Alguma
entidade precisa traduzir os nomes simbólicos --- como cornell.edu e Skype.com --- em
números e acompanhar quais pontos de conexão têm quais números.

A Corporação da Internet para Atribuição de Nomes e Números (ICANN) é a entidade que
determina quais endereços numéricos são atribuídos à Universidade de Cornell ou à nação da
Austrália. Ela supervisiona a publicação de diretórios eletrônicos de tal forma que qualquer
pessoa que envie um e-mail para o endereço president@cornell.edu ou recupere uma página da
web de um endereço como http://anu.edu.au (a página inicial da Universidade Nacional
Australiana) seja direcionada para o local correto na Internet. As tabelas de tradução, de
letras para números, são mantidas em servidores do Sistema de Nomes de Domínio (DNS), que
outros computadores consultam para procurar os endereços numéricos a serem inseridos no campo
``destino'' dos pacotes IP antes de serem lançados na Internet. Se a Internet tem uma única
vulnerabilidade, é o controle sobre os servidores DNS. A nação insular de Tuvalu obtém seu
próprio domínio de alto nível na Internet, como .au para a Austrália? (Sim, e é muito valioso.
É .tv, e a nação, que costumava ganhar dinheiro vendendo selos, agora obtém receita ao permitir
que sites de vídeo como twitch.tv usem a extensão .tv.) Quem decide se a Coca-Cola tem direito
a cocacola.com ou, nesse sentido, cocacola.porcaria? Não deve ser surpresa que tais questões de
alto risco despertem grande interesse de governos e corporações multinacionais. Geralmente,
essas questões não podem ser resolvidas simplesmente concordando anonimamente ou por meio de
qualquer processo igualmente inclusivo.

No entanto, essas disputas territoriais são resolvidas sem o uso de força e sem fraturar a rede.
Na verdade, desde as primeiras máquinas conectadas à ARPANET em 1969, o número de dispositivos
conectados cresceu para bilhões hoje em dia. Qualquer um deles pode, em princípio, se conectar
a qualquer outro.\footnote{O design do protocolo de pacotes da Internet foi estabelecido antes
da era da miniaturização em massa e em um momento em que a memória do computador era limitada e
cara. Naquela época, ninguém imaginava a necessidade de conectar mais de 4 bilhões de
computadores à Internet, então apenas 32 bits foram reservados para os campos de endereço. Mas
agora relógios de pulso e refrigeradores têm seus próprios endereços IP, e o número total de
computadores conectados é maior do que pode ser distinguido usando 32 bits. Foram desenvolvidas
várias soluções alternativas, e um novo protocolo, o IPv6, que possui endereços de 128 bits,
está sendo lentamente implementado.} Os problemas sérios de guarda da Internet estão em outro
lugar.

\section{Guardiões dos Links: Conectando-se}
\label{gatekeepers:guardioes-links}

A Internet não é muito útil se você não consegue se conectar.

Se você dirigir para o oeste de Boston através do norte dos Estados Unidos, as opções nos
balcões de delicatessen dos supermercados mudam quando você chega ao Iowa. De repente, eles
apresentam saladas de gelatina em grande variedade, incorporando várias frutas picadas, camadas
coloridas e coberturas cremosas. Às vezes, a gelatina é moldada ao redor de peixe ou carne. A
moda persiste pelas Grandes Planícies e sobe pelas Montanhas Rochosas, mas desaparece nas
encostas descendentes. Quando você chega ao Oceano Pacífico, a gelatina novamente é principalmente
para crianças e pacientes hospitalares. O amor pelas criações de gelatina no coração rural é tão
prevalente que esses pratos são comumente apresentados nas capas de revistas de alimentos
disponíveis nos caixas dos supermercados costeiros e urbanos, mesmo que ninguém que compre lá
sonharia em servir tal coisa. Entre as elites costeiras e urbanas, as saladas de gelatina são
consideradas pouco sofisticadas.

A predileção do Meio-Oeste por saladas de gelatina está desaparecendo agora, à medida que a
cultura gastronômica, assim como o restante da cultura americana, está se tornando
geograficamente homogeneizada. Em algumas áreas, as receitas de salada de gelatina da vovó são
lembradas como chapéus de palha feitos à mão e vestidos de chita --- como artefatos de um
passado rural fora do lugar em tempos mais avançados. Mas as saladas de gelatina não foram
levadas para o Oeste em carroças cobertas; isso teria sido impossível, já que criá-las requer
refrigeração. Elas são, em vez disso, um subproduto de uma revolução tecnológica do século XX:
a eletrificação rural. As saladas de gelatina eram uma iguaria nas fazendas porque você não
poderia fazê-las se sua fazenda não fosse eletrificada. As pessoas que serviam saladas de
gelatina também tinham bombas de poço elétricas e luzes elétricas. Servir gelatina provava que
você estava tecnologicamente avançado.

A economia de difundir eletricidade e de difundir bits por meio de fios ou cabos é semelhante.
É caro instalar cabos em longas distâncias e, se houver poucos clientes no final da linha, não
é economicamente viável instalar o cabo. Também é caro puxar fios para muitas residências
individuais se estiverem distantes umas das outras. Os lucros são muito maiores ao conectar uma
cidade, porque assim que a linha é trazida até uma rua, cada unidade habitacional na rua se
torna um cliente, e a distância da linha principal para os clientes tende a ser curta. De fato,
as cidades foram eletrificadas primeiro, principalmente para a iluminação pública, que não exigia
a instalação de fiação dentro de edifícios particulares. Todos os outros usos da eletricidade
--- para iluminação interna, geladeiras, máquinas de lavar, lava-louças e rádios --- derivaram do
uso para a iluminação pública. Até hoje, um passeio pela Beacon Street, perto do Fenway Park em
Boston, leva você a uma estrutura que diz ``The Edison Electric Illuminating Company''.

Não teria feito sentido difundir a eletricidade em nível nacional apenas para que as pessoas
pudessem comer gelatina. Na verdade, não existia geladeira doméstica elétrica quando os primeiros
postes de iluminação elétrica foram acesos. A eletricidade doméstica era o que Jonathan Zittrain
chama de tecnologia geradora.\footnote{Jonathan Zittrain, \ingles{The Future of the Internet—and
How to Stop It} (Yale University Press, 2008).} Uma vez que a infraestrutura foi instalada,
pessoas criativas começaram a sonhar com usos para ela e desenvolveram tecnologias totalmente
novas que não poderiam ter existido sem ela. No processo, alguns setores antigos morreram, com
enormes custos para aqueles que lucraram com eles. Hoje em dia, ``caixa de gelo'' é, na melhor das
hipóteses, uma expressão nostálgica para uma geladeira, mas há um século havia um enorme setor
dedicado a cortar a superfície congelada dos lagos em blocos, transportá-los por longas distâncias
e distribuí-los nos lares americanos. As tecnologias geradoras também são tecnologias destrutivas.

A difusão da Internet seguiu em grande parte as mesmas etapas que a difusão da eletricidade já fez.
A iluminação foi a aplicação matadora para a eletricidade; as saladas de gelatina foram os vídeos
engraçados da rede elétrica. No entanto, a experiência dos Estados Unidos com a Internet tem sido
muito diferente do que foi com a rede elétrica.

Os Estados Unidos foram eletrificados rapidamente e de forma ubíqua, mas essa rápida expansão não
teria ocorrido sem um incentivo do governo federal. Fazer a instalação elétrica em áreas remotas
não era lucrativo, e teria sido ainda mais desvantajoso para um concorrente instalar um segundo
cabo para atender aos mesmos clientes. Portanto, a eletricidade estava prontamente disponível nas
cidades, mas extremamente cara em áreas remotas, quando estava disponível.

Franklin Roosevelt pôde perceber a diferença no preço da eletricidade quando se retirou de Nova
York e Washington, DC, para Warm Springs, Geórgia, onde buscava alívio e tratamentos de spa para
sua paralisia. Foi lá que ele concebeu a Lei de Eletrificação Rural e a assinou em lei em 1936.
Essa iniciativa estimulou não apenas a difusão da eletricidade, mas também a invenção de novas
maneiras para os consumidores comuns a utilizarem.

No início dos anos 1920, menos de 1\% das residências nos Estados Unidos tinham eletricidade. Seis
anos após a assinatura da Lei de Eletrificação Rural, seguindo uma terrível recessão econômica,
metade das residências nos Estados Unidos estavam eletrificadas. Em 1960, praticamente todo o país
tinha eletricidade.

Para fazer uma comparação sensata entre a difusão da Internet e a difusão da eletricidade,
precisamos definir nossos termos. A eletricidade fornecida às residências foi padronizada: nos
Estados Unidos, a eletricidade é corrente alternada, com frequência de 60 Hz e tensão de 120V.
Esses padrões são análogos ao IP para a Internet. Eles garantem que os mesmos aparelhos possam
ser conectados a tomadas em qualquer lugar do país.

Mas há outro parâmetro importante: a quantidade de energia elétrica usada por um aparelho ou
uma residência. A taxa na qual a eletricidade é usada é chamada de potência e é medida em
watts ou quilowatts (para milhares de watts). A quantidade de energia usada é medida em
quilowatt-hora; um quilowatt-hora (kWh) é a quantidade de energia consumida ao usá-la por uma
hora a uma taxa de 1.000 watts. Os códigos elétricos foram padronizados de forma que os
circuitos residenciais possam lidar com cerca de 2.000 watts; se você usar muito mais do que
isso, um disjuntor será desarmado, e o fio pode derreter se não houver um fusível ou disjuntor.
A fiação em uma casa antiga pode precisar ser atualizada quando um proprietário deseja usar
mais equipamentos elétricos ou equipamentos elétricos mais potentes, como ar-condicionado ou
uma banheira de hidromassagem. Por outro lado, os novos equipamentos tendem a usar menos
energia do que os antigos, portanto, o consumo médio por residência aumentou apenas lentamente
ao longo do tempo. As concessionárias de energia elétrica que realmente fornecem a energia
podem ter que atualizar sua rede de distribuição para acompanhar a demanda, mas uma combinação
de pressão do consumidor e normas federais geralmente tornam raro nos Estados Unidos ter
``apagões'', quando uma cidade inteira ou bairro tem energia insuficiente. Nos Estados Unidos,
a eletricidade é, em sua maioria, uma indústria regulamentada com sucesso.

O análogo da potência para a Internet é a taxa de bits, e aqui a experiência da Internet e da
rede elétrica se divergiram significativamente. O fornecimento de conectividade à Internet tem
sido deixado quase inteiramente para o setor privado, com regulamentação governamental mínima
e apoio governamental mínimo. Em quase nenhum lugar existe uma concorrência séria, então os
consumidores não podem mudar para provedores melhores. O provedor monopolista de serviços de
Internet pode oferecer uma escolha de velocidades, mas as velocidades mais altas provavelmente
serão exorbitantemente caras. Em uma palavra, em vez de fornecer Internet de alta velocidade,
a maioria dos provedores de serviços de Internet tenta nos convencer de que já a temos, e o
governo está ajudando em seu engano em vez de pressioná-los a melhorar seus serviços.

\subsection{Onde Estão os Gargalos e o Que Conta como ``Alta Velocidade''?}
\label{gatekeepers:gargalos}

A taxa na qual os bits completam sua transmissão --- por exemplo, de um servidor da web em
algum lugar para o navegador em execução no seu computador doméstico ou do seu computador de
escritório para a sala de bate-papo de vídeo na sede da sua empresa em Londres --- é a mais
lenta das taxas de qualquer um dos elos ao longo do caminho. A taxa na qual os bits fluem
através de um elo é afetada por alguns parâmetros físicos --- as propriedades elétricas ou
eletromagnéticas do cobre ou vidro dos quais o elo é composto --- e o quanto o tráfego é
pesado. Se a sua comunicação tiver o elo para si mesma, ela pode utilizar todos os bits por
segundo dos quais o meio de comunicação é capaz, mas se tiver que compartilhar essa capacidade
com um milhão de outras transmissões, a sua pode receber apenas um milionésimo.

Pense novamente no seu computador doméstico recuperando uma página da web de um servidor
pertencente a uma grande empresa. Seu pedido precisa chegar ao servidor corporativo, e os
pacotes que compõem a página da web precisam retornar ao seu computador doméstico.
Simplificando bastante, você pode pensar que os bits do seu pedido fazem três saltos. Eles
precisam ir do seu computador até a parede externa da sua casa; depois da sua casa para a
``rede de distribuição'', os cabos de longa distância que cruzam o país; e então através das
centenas ou milhares de milhas da rede de distribuição. A conexão da sua casa para a rede de
distribuição é comumente referida como a ``última milha''. A mesma hierarquia é, em princípio,
atravessada no outro extremo, exceto que a Amazon e o Google estão conectados diretamente à
rede de distribuição devido às suas enormes necessidades de capacidade. Se você estivesse se
comunicando com outra pessoa sentada em um computador doméstico, os bits teriam que percorrer
a ``última milha'' até a casa dessa pessoa.

Dentro de uma casa, a maioria das pessoas usa Wi-Fi, uma espécie de comunicação por rádio de
curta distância. As tecnologias Wi-Fi mais recentes podem atingir velocidades de gigabit, mas
na prática as conexões provavelmente serão mais lentas devido a interferências ou obstruções.
Usuários avançados ainda podem ter suas casas cabeadas para que possam conectar seus
computadores usando cabos Ethernet em vez de conexões sem fio.

Mas uma conexão sem fio lenta pode ser suficientemente rápida se a ``última milha'' já for
lenta de qualquer maneira. E nos Estados Unidos, quase certamente é lenta. Em termos globais,
o que é chamado nos Estados Unidos de ``Internet de alta velocidade'' simplesmente não o é.

A rede de distribuição da Internet é composto por cabos de fibra ótica. A fibra é incrível; o
próprio vidro possui uma capacidade de transporte de informações praticamente ilimitada. Sua
capacidade real não é limitada pelo vidro, mas sim pela eletrônica que conecta a rede nos
pontos de comutação. A eletrônica está constantemente sendo aprimorada; uma vez instalada, a
fibra nunca é substituída (a menos que seja danificada --- por exemplo, porque um barco de
pesca a pega).\footnote{For a full account of the story of fiber, see Susan Crawford, Fiber:
\ingles{The Coming Tech Revolution --- and Why America Might Miss It}, Yale University Press,
2018.}

Em algumas partes do mundo, a última milha também é composta por fibra, de modo que essas
incríveis capacidades de informações chegam diretamente às portas de residências e empresas.
Em Cingapura e na Suécia, praticamente todos têm acesso a velocidades de Internet na casa dos
bilhões de bits por segundo. Em contraste, talvez 15\% dos americanos tenham conexões de fibra
em suas residências, e essa porcentagem não está aumentando. A maioria de nós está conectada
por fios telefônicos legados, usando o chamado serviço DSL, ou por cabo coaxial que foi
instalado para trazer a TV a cabo para nossas casas. Até mesmo o serviço DSL está sendo
gradualmente descontinuado em alguns lugares por ser pouco lucrativo. E as provedoras de cabo
e telecomunicações efetivamente dividiram o mapa entre elas; em apenas poucos lugares um
consumidor pode escolher entre o serviço de cabo e DSL, muito menos entre múltiplos serviços
de cabo.

A própria Internet não se conecta à sua casa, escritório ou celular. As pessoas comuns
conectam seus dispositivos não à Internet, mas a um provedor de serviços de Internet (ISP).
Como os pacotes da Internet podem viajar por uma variedade de mídias físicas, em princípio,
não há limite para o número de ISPs que podem transportar pacotes para dentro e para fora de
sua casa, mediante pagamento. A realidade é muito diferente. Nos Estados Unidos, é provável
que seu ISP residencial seja a AT\&T, Time Warner, Comcast/Xfinity, Verizon ou Charter. A
razão pela qual há tão poucos é que cada uma delas é uma empresa de telecomunicações ou de
televisão a cabo, e elas estão usando a fiação ou cabo de fibra óptica que já trouxeram para
sua casa para fornecer serviço de telefone ou televisão. O acesso à Internet sem fio também é
possível --- por isso não se entende a conexão Wi-Fi entre seu computador e um roteador, que
está então conectado ao seu ISP, mas sim uma conexão sem fio diretamente com o ISP. A conexão
via satélite é possível em áreas rurais onde nenhum tipo de conexão com fio foi instalado,
mas a Internet via satélite é lenta e cara. Os celulares se conectam à Internet por meio da
rede de telefonia celular, mas a rede celular não é uma opção viável para uso residencial. E
os chamados sinais de rádio 5G, que estão sendo anunciados como o futuro da conectividade à
Internet, viajam apenas curtas distâncias. Portanto, uma infraestrutura 5G é realista apenas
em áreas densamente povoadas, onde é possível instalar muitos hubs de forma econômica.

Os americanos são bombardeados com anúncios de ``Internet de alta velocidade'', mas, na
realidade, até mesmo a definição do governo de ``alta velocidade'' é enganosa. A última vez
que foi atualizada foi em 2015, quando --- contra as objeções dos provedores de serviços de
Internet --- a FCC aumentou o padrão de 4 Mbps para 25 Mbps. Isso equivale a um quadragésimo
da velocidade de gigabit que agora é padrão no Japão e na Suécia, e que até mesmo a China
está difundindo em áreas rurais. E o padrão dos EUA de 25 Mbps se aplica apenas às velocidades
de download, como se a Internet fosse basicamente um meio de transmissão para os consumidores
receberem filmes da Netflix. Muitas aplicações, desde videochamadas até a transferência de
imagens médicas, exigem altas velocidades em ambas as direções. Empresas de todos os tipos
dependem da conectividade à Internet tanto para uploads quanto para downloads; eles usam a
Internet para obter informações sobre seus produtos e serviços, e até mesmo os próprios
produtos e serviços, para seus clientes. Portanto, é muito difícil iniciar um negócio em uma
área onde a conectividade é fraca ou limitada a downloads rápidos e uploads lentos. E, no
entanto, nos Estados Unidos, a Internet foi otimizada como substituta da televisão --- como
uma forma para poucos fornecerem conteúdo para muitos.

Uma segunda forma de semântica criativa distorce as estatísticas governamentais sobre a
disponibilidade de ``Internet de alta velocidade'' nos Estados Unidos. O governo considera que
um setor censitário --- uma das cerca de 75.000 áreas geográficas em que os Estados Unidos
são divididos para fins de censo --- possui Internet de alta velocidade se mesmo uma única
residência tiver acesso a ela, independentemente do preço e independentemente de qualquer
residência ter efetivamente assinado uma conexão. Assim, as estimativas do governo sobre a
difusão da Internet de alta velocidade são amplamente infladas.

E o preço importa. Em grande parte do mundo, velocidades de gigabit estão disponíveis por
menos de US\$ 50 por mês. Na área de Boston, onde estamos escrevendo, custa US\$ 70 por mês
nas áreas restritas onde está disponível --- e requer um contrato de 24 meses.

O americano típico tem apenas uma ou duas opções realistas de ISP. Mais de 30\% das
residências nos Estados Unidos não têm provedores de serviço de Internet com 25 Mbps ou mais,
e menos de um quarto das residências nos Estados Unidos têm mais de uma escolha.\footnote{Jon
Brodkin, ``US Broadband: Still No ISP Choice for Many, Especially at Higher Speeds,''
\ingles{Ars Technica}, August 10, 2016,
\url{https://arstechnica.com/information-technology/2016/08/us-broadband-still-no-isp-choice-for-many-especially-at-higher-speeds/}.}

Em grande parte, a disseminação da conectividade à Internet foi deixada para o setor privado
nos Estados Unidos. Na verdade, regras em 26 estados dificultam ou proíbem que governos locais
ofereçam conectividade à Internet, forçando indivíduos para os quais nenhum serviço doméstico
acessível está disponível a se conectarem em bibliotecas públicas ou restaurantes de fast food.
\footnote{Kendra Chamberlain, ``Municipal Broadband Is Roadblocked or Outlawed in 25 States,''
\ingles{Broadband Now}, May 13, 2020,
\url{https://broadbandnow.com/report/municipal-broadband-roadblocks/}.} O código de
Montana\footnote{``Government Competition with Private Internet Services Providers Prohibited
--- Exceptions,'' Montana Code Annotated 2019,
\url{https://leg.mt.gov/bills/mca/title_0020/chapter_0170/part_0060/section_0030/0020-0170-0060-0030.html}.}
é típico: ``Exceto conforme previsto na subseção (2)(a) ou (2)(b), uma agência ou subdivisão
política do estado não pode, direta ou indiretamente, ser um provedor de serviços de Internet.''
Portanto, a menos que todos os ISPs privados se retirem, as pessoas de Browning, Montana, estão
presas às ofertas privadas de baixa qualidade e caras. Nenhum tipo de empreendedorismo municipal
pode ajudar sua população. Os lobistas das empresas privadas de telecomunicações chegaram
primeiro à legislatura estadual.

Agora, é claro, existem razões para impedir que os governos concorram com empresas privadas.
Os argumentos são familiares. A competição reduz os preços e melhora a qualidade. O dinheiro
dos contribuintes não deve ser usado para prejudicar os fornecedores privados. Os governos
não devem interferir nos mercados livres.

Mas simplesmente não há pessoas suficientes do outro lado para pagar pela conexão de longa
distância. Isso não é diferente da situação da entrega gratuita de correspondência rural
--- que se tornou lei em 1893 e chegou a Billings em 1902 --- ou da rede elétrica nas décadas
de 1930 e 1940. Conectar o país requer que as comunicações eletrônicas sejam vistas como
eletricidade ou correio postal: elas teriam que estar disponíveis para todos a um preço
acessível. Esse princípio, de fato, não é geralmente aceito. Em vez disso, a metáfora
operacional para a Internet é a televisão ou um cinema de várias telas. Os provedores de
serviços de Internet dominantes veem a Internet como uma forma de conectar provedores de
conteúdo ativo a consumidores passivos. É por isso que os ISPs oferecem pacotes que incentivam
o download e limitam o upload.

Na ausência do tipo de incentivo do governo que resultou em serviços de Internet muito
melhores na Coreia do Sul, Suíça e até na maior parte da Finlândia do que nos Estados Unidos,
por que a concorrência não reduziu os preços e aumentou a qualidade?

Alguns argumentariam que a ganância corporativa, a colusão e a corrupção são os culpados, e
embora essa perspectiva possa ter alguma validade em alguns casos, a realidade é que as redes
de comunicação crescem e se consolidam quase organicamente por motivos de eficiência. Paul
Baran próprio antecipou que esse fenômeno afetaria as redes de computadores vários anos antes
de projetar uma das primeiras redes. Em seu depoimento perante o Congresso sobre privacidade
eletrônica em 1966, ele disse,\footnote{Paul Baran, ``Full Text of `The Computer and Invasion
of Privacy','' July 26, 1966,
\url{https://archive.org/details/U.S.House1966TheComputerAndInvasionOfPrivacy/mode/2up}.}

\begin{quote}
Nossas primeiras ferrovias na década de 1830 eram rotas curtas que conectavam centros
populacionais locais. Ninguém se sentou e elaborou um plano mestre para uma rede de trilhos
de ferro. Com o tempo, um número crescente desses sistemas locais separados foi construído.
Uma rede gradualmente se formou à medida que a pressão econômica levou à construção de novos
elos para preencher as lacunas entre as rotas individuais.

Não começamos a construir uma rede telegráfica nacional no final dos anos 1840; apenas elos
telegráficos independentes. No entanto, não demorou muito para termos uma rede nacional
integrada. Até o nome, Western Union, evoca o padrão de elos independentes unidos para
fornecer um sistema mais útil.

Não começamos a construir um sistema telefônico nacional nos primeiros dias do telefone na
década de 1890. No entanto, hoje temos uma rede telefônica altamente integrada.

Tais padrões de crescimento não são acidentes. Comunicações e transporte são serviços que
historicamente tendem a formar ``monopólios naturais''. O motivo é bem compreendido. É mais
barato compartilhar o uso de uma grande entidade do que construir suas próprias instalações.
Portanto, se você observasse a Terra, digamos, a partir do ponto de vista distante da lua,
pareceria que o crescimento dessas redes integradas a partir de peças individuais é quase
biológico.
\end{quote}

Portanto, não é muito complicado. É mais valioso fazer parte de uma rede grande do que de
uma pequena, e quanto maior a rede, mais valioso é fazer parte dela. Na ausência de
resistência de alguma estrutura social com autoridade para resistir a fusões, consolidações,
aquisições e decisões estratégicas corporativas para controlar o tráfego de rede, as redes
de comunicação crescerão e se tornarão cada vez mais escassas com o tempo. Essa monopolização
não é necessariamente contrária ao interesse público --- desde que o interesse público
participe das decisões sobre distribuição e preços. Hoje em dia, isso raramente acontece.

\subsection{O carteiro pode decidir qual correspondência entregar?}
\label{gatekeepers:carteiro}

A história da Telus e seus trabalhadores em greve com a qual este capítulo começou demonstra
que a dicotomia entre links e conteúdo não é útil quando o guardião de links assume o papel
de guardião de conteúdo. A noção de que os provedores de serviços de Internet não devem
decidir quais bits entregar é conhecida como neutralidade da rede. Em princípio, pode
parecer simples e incontestável; afinal, não queremos que a companhia telefônica decida
quais conversas ela permitirá acontecer em suas linhas de voz. É verdade que, quando os
clientes não pagam suas contas, o serviço telefônico deles pode ser cortado, mas mesmo isso
é raro porque a sociedade geralmente reconhece --- ou costumava reconhecer --- que o serviço
telefônico é importante para a vida cotidiana. Mas a Internet não é exatamente como a rede
telefônica.

Por volta do mesmo período em que a Telus estava derrubando sites pró-sindicato no Canadá, um
pequeno ISP da Carolina do Norte chamado Madison River Communications fechou a Vonage, que
oferecia serviço de Voz sobre IP (VoIP). O uso da Internet para fornecer conversas de voz ao
vivo teria parecido loucura no início da Internet, porque a rede era muito lenta e os
computadores conectados a ela não conseguiam acompanhar o fluxo de pacotes para reuni-los em
uma fala compreensível. Mas os tempos mudam. À medida que as velocidades de conexão melhoraram
e novos protocolos, baseados em IP, foram otimizados para comunicações por voz, uma diferença
sistemática entre serviços telefônicos e de Internet interveio. As empresas de telefonia
cobravam a mais pelo serviço de longa distância; os provedores de serviços de Internet não se
importavam com a origem ou destino dos pacotes. Eles podiam cobrar mais por taxas de dados mais
altas, mas não por destinos mais distantes. Inevitavelmente, o software VoIP --- o Skype foi o
primeiro sucesso comercial nesse espaço --- foi desenvolvido para substituir a telefonia por
comunicações pela Internet e tornar as ``chamadas'' de longa distância virtualmente gratuitas
para qualquer pessoa com uma conexão à Internet. A Vonage estava usando o serviço de dados da
Madison River para minar o serviço telefônico da Madison River.

Quando a Vonage foi bloqueada, a empresa reclamou com a Comissão Federal de Comunicações, que
tem jurisdição sobre serviços telefônicos. O caso foi resolvido quando a Madison River
concordou em pagar uma multa e não bloquear o VoIP por três anos, mas essa resolução deixou
mais perguntas do que respostas em seu rastro. E se a Madison River fosse uma empresa de cabo
oferecendo serviços de Internet em vez de uma empresa telefônica? Por outro lado, e se a Madison
River fosse grande o suficiente para lutar contra a FCC nos tribunais? Não estava de forma
alguma claro se a FCC tinha a autoridade congressual para apoiar suas táticas autoritárias sobre
a maneira como até mesmo os ISPs de telefonia estavam escolhendo quais bits entregar.
\footnote{Scott Bradner, ``The Internet: Unblocking Pipes,'' Network World, March 14, 2005,
\url{https://www.networkworld.com/article/2319666/the-internet--unblocking-pipes.html}.}

A situação chegou ao auge em 2008, quando a FCC ordenou que o ISP Comcast parasse de
``estrangular'' --- ou seja, de reduzir a velocidade --- do BitTorrent, um serviço de
compartilhamento de arquivos peer-to-peer amplamente usado para entregar filmes para o lar.
A Comcast estava entregando filmes com lucro sobre o mesmo cabo que estava usando para
fornecer serviços de Internet, então o BitTorrent estava prejudicando seus negócios de entrega
de vídeo. A Comcast processou com sucesso a FCC, estabelecendo que a FCC realmente não tinha
autoridade para regular seu negócio de serviços de Internet. Essa decisão deu início a um
debate sobre a neutralidade da rede que se arrasta há mais de uma década.

Os detalhes são complexos. Em poucas palavras, vozes pró-neutralidade argumentaram a favor
da escolha e liberdade do consumidor; oponentes argumentaram que as forças de mercado
resolveriam quaisquer tensões, um argumento recebido com ceticismo por aqueles que observam
a escassa concorrência no espaço dos ISP. Nos Estados Unidos, as regras de neutralidade da
rede foram instituídas durante a administração Obama e revogadas durante a administração
Trump. Muitas outras nações adotaram a neutralidade da rede em princípio, mas algumas delas
permitem a cobrança com base no uso, o que pode ter o efeito de tornar certos aplicativos,
como assistir filmes, inaceitavelmente caros, alcançando assim o mesmo resultado ---
priorizando outros meios de entrega de filmes para casa --- que a Comcast havia alcançado ao
limitar os serviços peer-to-peer em 2008.