\chapter[Gatekeepers]{Gatekeepers\\\large\textit{Quem está no comando aqui?}}
\label{gatekeepers}

\section{Quem Controla o Fluxo de Bits?}
\label{gatekeepers:fluxo-bits}

Quando o Sindicato dos Trabalhadores das Telecomunicações entrou em greve contra
a Telus, a principal empresa de telecomunicações do oeste do Canadá,\footnote{Ian
Austen, ``A Canadian Telecom's Labor Dispute Leads to Blocked Web Sites and
Questions of Censorship,'' \ingles{The New York Times}, August 1, 2005,
\url{https://www.nytimes.com/2005/08/01/business/worldbusiness/a-canadian-telecoms-labor-dispute-leads-to-blocked.html}.}
uma discussão sobre a interrupção da greve surgiu em um site pró-sindicato
operado por um funcionário da Telus. Então, de repente, o site se tornou
inacessível para qualquer pessoa que estivesse usando o serviço de Internet da
Telus. Os assinantes da Telus podiam receber bits originados de lugares que vão
do Afeganistão ao Zimbábue. Eles podiam receber bits que representavam desde
orquestras sinfônicas até pornografia. Mas se eles quisessem ver a discussão
sobre a resistência aos esforços da gerência para interromper a greve, não podiam.
A Telus tomou a posição de que, como os cabos que entregavam os bits pertenciam à
empresa, ela poderia escolher entregar ou não os bits.

O sindicato estava furioso, e os especialistas jurídicos estavam confusos. Parecia
ser claro o suficiente que a Telus não poderia cortar o serviço telefônico para o
sindicato ou seus apoiadores se quisesse, mas as leis haviam sido escritas em uma
era pré-Internet. A Telus estava dentro de seus direitos ao interromper o serviço
de Internet nesse caso? A empresa observou que também havia bloqueado o site
telusscabs.ca, que mostrava fotos de gerentes e funcionários que estavam indo
trabalhar apesar da greve. A Telus afirmou que tinha responsabilidade pela
segurança deles e sentia-se obrigada a protegê-los. No entanto, descobriu-se que a
Telus bloqueou muitos mais do que esses 2 sites. O servidor web que hospedava esses
2 sites também hospedava outros 766 sites, incluindo um site de medicina
alternativa e um site de arrecadação de fundos para pesquisa sobre câncer de mama. Ao
bloquear com sucesso os 2 sites ofensivos, a Telus também bloqueou todos os outros.

A Telus voltou atrás depois de garantir que nenhum material ameaçador seria
publicado. Mas o incidente --- e outros semelhantes --- levantou questões que ainda
hoje não têm respostas claras e geralmente aceitas. Quem controla o que as pessoas
podem fazer na Internet?

\section{A Internet Aberta?}
\label{gatekeepers:internet-aberta}

Ninguém deveria estar encarregado da Internet. Nem mesmo era para ser algo que
pudesse ser possuído ou controlado. Era para ser mais como uma linguagem que
diferentes pessoas poderiam usar da maneira que pudessem imaginar, para conversar umas
com as outras, recitar poesias ou cantar músicas. Deveria ser como o ``éter luminífero'',
a substância invisível que preenchia o espaço, que os físicos costumavam pensar que
deveria existir porque a luz não poderia ir de um lugar para outro sem ele. A Internet
deveria ser um meio que possibilitasse a comunicação de qualquer pessoa com qualquer
pessoa e de qualquer lugar para qualquer lugar, mas o controle dessa comunicação era
considerado impossível, pois não haveria lugar para controlá-la. Qualquer pessoa que
quisesse participar de uma conversa poderia fazê-lo --- apenas falando a linguagem
dos protocolos da Internet.

Pergunte a Alex Jones se as coisas funcionaram dessa maneira. Um destacado teórico
da conspiração americano --- ou, como ele se considera, um ``criminoso de pensamento
contra o Grande Irmão'' --- Jones conquistou uma grande quantidade de seguidores no
YouTube, Facebook, LinkedIn e em outros sites de redes sociais. Milhões de pessoas
seguiam cada palavra dele --- e cada rumor maluco que ele promovia era enviado
instantaneamente para seus celulares para que pudessem ver. E então, de repente,
muitos sites o baniram. A Apple parou de fornecer o aplicativo de Jones aos usuários.
Você ainda pode encontrar o site dele se procurar, mas o Pinterest não vai sugerir
isso para você.

Se isso não te convencer de que a Internet não é um paraíso participativo, use a
Internet para perguntar a qualquer pessoa na China o que aconteceu em 4 de junho.
Seja por e-mail, mensagens de texto ou Weibo (a versão chinesa do Twitter), é
improvável que sua mensagem chegue a alguém, pois a menção ao dia 4 de junho foi
amplamente censurada. Pergunte a qualquer pessoa em Hong Kong, e você nem precisará
dizer qual ano; o massacre de Tiananmen em 4 de junho de 1989 ainda é vividamente
lembrado mesmo após mais de 30 anos. Mas na Internet do continente, os bilhões de
usuários nunca falam sobre o dia 4 de junho. Mencioná-lo resulta em uma conversa
sendo rapidamente apagada em vez de se espalhar rapidamente. E o governo não é o
único guardião que controla o compartilhamento eletrônico de informações na China.
Quando os manifestantes em Hong Kong usaram um aplicativo chamado HKmap.live para
se organizar, o governo chinês ficou furioso, e a Apple removeu o aplicativo de sua
App Store. O Google respondeu de forma semelhante a um pedido da polícia de Hong Kong
para retirar do ar um jogo que permitia aos usuários desempenharem o papel de
manifestantes.\footnote{Tripp Mickle et al., ``Apple, Google Pull Hong Kong Protest
Apps Amid China Uproar,'' \ingles{Wall Street Journal}, October 10, 2019,
\url{https://www.wsj.com/articles/apple-pulls-hong-kong-cop-tracking-map-app-after-china-uproar-11570681464}.}

Ou pergunte a Hasan Minhaj, um comediante americano cujo programa ``Patriot Act'' é
distribuído pela Netflix. Seus comentários críticos sobre o príncipe herdeiro saudita
Mohammed bin Salman podem ser assistidos em quase todo o mundo --- mas não onde eles
teriam mais significado, na Arábia Saudita. O governo saudita exigiu que o episódio
fosse retirado do ar, citando uma lei que criminaliza a ``produção, preparação,
transmissão ou armazenamento de material que afete a ordem pública, os valores
religiosos, a moral pública e a privacidade, através de redes de informação ou
computadores.'' A Netflix respondeu dizendo que apoia a liberdade artística --- mas,
mesmo assim, removeu o vídeo.\footnote{Jim Rutenberg, ``Netflix’s Bow to Saudi Censors
Comes at a Cost to Free Speech,'' \ingles{The New York Times}, January 6, 2019,
\url{https://www.nytimes.com/2019/01/06/business/media/netflix-saudi-arabia-censorship-hasan-minhaj.html}.}
A liberdade artística de Minhaj se mostrou incompatível com a liberdade pessoal dos
funcionários da Netflix, que poderiam ser sujeitos a dez anos ou mais de prisão pelo
crime vagamente definido de transmitir material que afete a ordem pública.

Ou pergunte a qualquer pessoa que venda algo. O mecanismo de busca do Google é
utilizado em mais de 90\% das pesquisas na Internet; o Bing da Microsoft ocupa o
segundo lugar, com 3\%. Se você quer vender ferramentas e não aparece na primeira
página de resultados quando as pessoas procuram por ``ferramenta,'' pode ser difícil
atrair atenção. ``O Google é o guardião da \ingles{World Wide Web}, da internet como
a conhecemos,'' como disse o advogado Gary Reback. ``Cada bit hoje é tão importante
como o petróleo era quando John D. Rockefeller estava o monopolizando.''\footnote{Steve
Kroft, ``How Did Google Get so Big?'' CBS News, May 21, 2018,
\url{https://www.cbsnews.com/news/how-did-google-get-so-big/}.}
O Google se defendeu respondendo que não é um monopólio porque os resultados de busca
da Amazon também influenciam os resultados das compras. (O Google poderia ter observado
que isso acontece especialmente porque a Amazon concede um selo de ``Escolha da Amazon''
aos produtos que favorece.) Mas a defesa do Google apenas enfatiza o fato de que o
número de guardiões é pequeno, mesmo que seja ligeiramente maior do que um. Pode haver
1.000 pequenas empresas fabricando e vendendo ferramentas, mas apenas algumas que
aparecem na primeira página dos resultados de busca têm chances de obter negócios na
Internet --- e elas estão competindo por visibilidade contra empresas muito maiores.

Ou pergunte a qualquer pessoa em Browning, Montana, se qualquer um pode usar a Internet
para se comunicar com qualquer pessoa. Na Reserva Indígena Blackfoot, apenas 0,1\% da
população tem acesso à Internet de alta velocidade. A conectividade mais barata de
qualquer tipo é de 10 Mbps e custa quase US\$ 780 por ano\footnote{``Internet Providers
in Browning, Montana,'' Broadband Now, accessed April 27, 2020, \url{https://broadbandnow.com/Montana/Browning}.}
--- em um lugar com uma taxa de pobreza de 35\% e uma renda média anual familiar de
menos de US\$ 22.000.\footnote{``Browning, MT,'' Data USA, accessed April 27, 2020,
\url{https://datausa.io/profile/geo/browning-mt/}.} Na maioria das cidades americanas,
obter dados pela Internet é como abrir a torneira para obter água, e o mesmo acontece
em toda a Finlândia e Japão. O serviço de Internet é abundante em lugares onde é
lucrativo para fornecedores privados ou onde é fornecido como uma questão de política
pública. Nada disso é verdade em Browning, onde o serviço de Internet é praticamente
indisponível.

Independentemente da história, da teoria e do potencial, a realidade é que algumas
corporações e alguns governos exercem um enorme controle sobre o que a maioria das
pessoas realmente vê na Internet e o que elas podem fazer com ela. Se essas empresas e
instituições não fornecerem a infraestrutura para entregar a sua mensagem, ou se eles
derem menor prioridade à entrega do seu anúncio, notícia ou crítica política, será como
se você tivesse gritado no deserto. Alguém pode ouvir o que você diz, mas provavelmente
não muitas pessoas. Isso é exatamente o oposto da forma como a Internet foi projetada
para funcionar. A Internet evoluiu desde seu design financiado publicamente como um
sistema aberto com usos potenciais ilimitados para um sistema em que algumas empresas
privadas detêm um controle quase monopolista sobre cada um de seus principais aspectos.

Este capítulo foca em três tipos de guardiões da Internet. O primeiro são os controladores
dos canais de dados pelos quais os bits fluem. Chamaremos esses de guardiões dos links.
O segundo são os controladores das ferramentas que usamos para encontrar coisas na Web.
Chamaremos esses de guardiões das buscas. E o terceiro são os controladores das conexões
sociais que são, para muitos de nós, nosso uso mais importante da Internet. Chamaremos
esses de guardiões sociais.

Os guardiões dos links controlam o meio físico através do qual os bits fluem, enquanto
os guardiões das buscas e os guardiões sociais controlam o que esses bits expressam
--- ou seja, eles são guardiões de conteúdo. No entanto, essas distinções não são tão
nítidas quanto possam parecer. Os guardiões dos links podem ser capazes, por exemplo,
de censurar ou favorecer determinados conteúdos em relação a outros ou determinados
clientes em relação a outros. Guardiões de conteúdo podem entrar no mercado de links se
acharem que será vantajoso resistir ao controle quase monopolista dos guardiões dos links
--- independentemente de consolidar o controle de links e conteúdo estar ou não no
interesse público mais amplamente concebido. Guardiões sociais adicionaram a busca dentro
de suas plataformas sociais para minar o controle quase monopolista dos guardiões das
buscas.

Nos Estados Unidos, todas as três funções dos guardiões estão em grande parte nas mãos
do setor privado. Em outras partes do mundo, os governos assumiram alguns desses papéis
de guardiões. Debates familiares sobre serviços privados versus públicos têm acontecido
como parte da Internet quase desde o seu início. Uma argumentação conhecida é que a
concorrência entre partes privadas reduz os custos e melhora a qualidade; mas outros
argumentam que a consolidação resulta em eficiências de escala que superam os efeitos
negativos da redução da concorrência. De acordo com outra narrativa, o governo deve
fornecer infraestrutura de benefício geral para o povo, custeada através de impostos
gerais em vez de compra privada; ele deve fornecer o ``éter'', da mesma forma que fornece
estradas e serviços postais, igualmente para todos. Mas tais analogias apenas levantam a
questão se a Internet realmente se assemelha às estradas públicas, em que qualquer um
pode dirigir, ou se se assemelha à televisão a cabo ou cinemas, que são mais acessíveis
em áreas urbanas do que rurais e não estão disponíveis para pessoas que não estão
dispostas a pagar as taxas.

Os resultados das diversas respostas possíveis para tais questões têm sido variadas
e dependem em certa medida de questões fundamentais de metas cívicas e econômicas.
Em regimes autoritários, comprometidos com a ``harmonia'' social em detrimento da
liberdade individual, o controle de conteúdo pode ser ainda mais centralizado do que
nos Estados Unidos. Por outro lado, investimentos substanciais do governo na
infraestrutura fora dos Estados Unidos resultaram em conectividade muito melhor em
certos países democráticos e não democráticos. Os debates sobre o nível correto de
investimento e supervisão governamental da Internet não são mais simples do que a
história do envolvimento governamental na entrega de correspondência, eletricidade,
serviço telefônico, educação ou cuidados médicos. Após contar rapidamente a história
de como a Internet aberta caiu sob o controle de oligopólios de guardiões,
levantaremos as questões com as quais a sociedade fica sobre o que, se é que alguma
coisa, fazer.

Vamos começar com como tudo funciona.

\section{Conectando os Pontos: Projetado para Compartilhar e Sobreviver}
\label{gatekeepers:conectando-pontos}

A Internet surgiu a partir do ARPANET, um projeto de rede de computadores do
Departamento de Defesa dos Estados Unidos (DoD) dos anos 1970. Por meio de sua
Agência de Projetos de Pesquisa Avançada (ARPA, posteriormente DARPA), o DoD estava
pagando direta ou indiretamente por máquinas de ponta em muitos laboratórios
acadêmicos e de pesquisa nacionais. A ARPA tinha duas preocupações.

Uma das preocupações era prosaica: a agência estava pagando por grandes e caros
computadores em todo o país, mas não havia maneira de que máquinas subutilizadas em
um local pudessem ser utilizadas para resolver problemas que os pesquisadores
precisavam resolver em outros locais. Assim, cada pesquisador queria o maior
computador possível, e muito tempo de computação estava sendo desperdiçado. Os
cientistas poderiam colocar seus dados em fitas e enviá-los pelo país por transporte
aéreo, mas não havia maneira de enviar os bits sem enviar os átomos. Então, a ARPA
queria melhorar a utilização dos computadores de pesquisa que estava financiando,
conectando-os em rede.

A outra preocupação da ARPA atingiu o cerne da missão militar. O DoD estava preocupado
há algum tempo que suas bases e navios espalhados não pudessem se comunicar se locais
críticos fossem destruídos em uma guerra nuclear. No início da década de 1960, a
preocupação era se a rede telefônica sobreviveria a um ataque que eliminasse alguns
centros de comutação chave, onde muitas linhas telefônicas de longa distância estavam
interconectadas.

Naquela época, o pesquisador Paul Baran estudou as propriedades de uma rede
descentralizada, na qual havia muitos pontos de junção, cada um conectado apenas a
alguns outros. (A rede telefônica, por outro lado, consistia de um pequeno número de
estações centrais de comutação conectadas a clientes, como raios se unindo ao centro de
uma roda até sua borda.) Na rede em forma de malha proposta por Baran, haveria muitos
caminhos entre quaisquer dois pontos, de modo que eliminar qualquer um dos pontos de
junção não impediria que outros pontos se comunicassem. Baran imaginou uma conexão
irregular de pontos de comutação, como mostrado abaixo em uma ilustração de seu artigo
de 1962.\footnote{Paul Baran, ``On Distributed Communications Networks,'' (RAND Corporation,
Santa Monica, CA, September 1962), Reprinted with permission.
\url{https://www.rand.org/pubs/papers/P2626.html}.}

%%%%%%%%%%% AQUI FICA UMA FIGURA, PÁGINA 80 %%%%
FIGURA\\
%%%%%%%%%%% AQUI FICA UMA FIGURA %%%%%%%%%%%%%%%

Baran contribuiu com uma segunda ideia importante: se um ponto de comutação ficasse
inativo, outra rota poderia ser encontrada que não passasse por ele, desde que o próprio
ponto de comutação não fosse um dos extremos da comunicação. Através da configuração
correta dos comutadores, a comunicação entre dois pontos poderia ser estabelecida ao
longo de um caminho específico. No entanto, se qualquer um dos pontos ao longo do caminho
fosse danificado, isso interromperia essa comunicação. Da mesma forma, uma falha de
hardware comum em qualquer um desses pontos intermediários também causaria interrupções.
Era importante proteger a integridade das comunicações individuais, mesmo quando os
componentes da rede falhassem de maneiras imprevisíveis.

Baran propôs dividir as comunicações em pequenos fragmentos de bits, o que hoje chamamos
de ``pacotes''. Além do ``\ingles{payload}'', um fragmento da própria comunicação, os
pacotes conteriam informações identificando a origem e o destino (semelhante às
informações de endereço em uma carta postal) e também um número de série para que o nó de
destino pudesse reuni-los na ordem correta, caso chegassem fora de sequência. Com todas
essas informações no ``envelope'', os pacotes que compõem uma única comunicação não
precisavam seguir o mesmo caminho. Se uma parte da rede estivesse indisponível, os nós da
rede poderiam direcionar os pacotes por um caminho diferente. Fazer tudo isso funcionar não
foi simples --- como os nós da rede saberiam em que direção encaminhar um pacote? --- mas,
em princípio, a ideia de Baran de uma interconexão em forma de malha e de comunicação em
forma de pacotes atenderia aos requisitos militares de sobrevivência.

\subsection{Protocolos: Como Apertar as Mãos com Estranhos}
\label{gatekeepers:protocolos}

Uma vez que a ARPANET estava operacional e conectava algumas dezenas de computadores,
começou a ficar claro que o que precisava ser conectado não eram computadores individuais,
mas redes de computadores existentes. Diferentes formas de interconectar computadores
poderiam coexistir, desde que as redes usassem uma linguagem comum para se comunicar entre
si. E nas décadas de 1970 e 1980, diferentes tipos de redes de computadores existiam, cada
uma utilizando os padrões de uma empresa de computadores diferente. A IBM tinha sua SNA
(Arquitetura de Rede de Sistemas). A Digital Equipment Corporation tinha a DECnet. A Apollo
Computer conectava suas máquinas em um anel, em vez de uma árvore ramificada ou malha. Cada
empresa divulgava as vantagens de seu esquema de rede, e algumas das reivindicações eram
válidas para casos de uso específicos. No entanto, nenhum dos fabricantes tinha incentivo
para tornar suas máquinas interoperáveis com as de outros fabricantes --- até que a ARPA
declarou que não pagaria por mais computadores a menos que eles pudessem ser interconectados.
Partindo do sucesso da ARPANET, Vinton Cerf e Robert Kahn projetaram os protocolos para
interconectar redes de computadores.\footnote{Vinton G. Cerf and Robert E Kahn, ``A Protocol
for Packet Network Intercommunication,'' \ingles{IEEE Transactions on Communications}, no. 5
(1974): 13.} Ou seja, eles projetaram a Internet.

A Internet são seus protocolos. A Internet não é uma máquina, nem mesmo uma coleção de
máquinas. Não é um software específico. É um conjunto de regras. Qualquer pessoa ou
organização pode construir hardware ou escrever software que siga essas regras e se tornar
uma parte funcional da Internet.

Protocolos são convenções de comunicação, como a convenção de que as pessoas apertam as
mãos direitas. Cumprimentar uns aos outros apertando as mãos esquerdas funcionaria
igualmente bem, mas a convenção estabelecida de aperto de mão com a mão direita permite que
estranhos se cumprimentem sem mediação prévia. Os protocolos da Internet são as convenções
pelas quais diferentes redes apertam as mãos para transmitir informações de uma rede para
outra. Cada rede pode operar internamente como desejar; apenas nos pontos em que as redes
estão conectadas é que os protocolos da Internet se tornam relevantes.

A decisão de tornar a ARPANET uma rede comutada por pacotes simplificou consideravelmente
o design da Internet. As redes eram conectadas à Internet por meio de pontos de conexão
chamados de \ingles{gateways}. Se um gateway se comportasse adequadamente, as informações
fluiriam por ele. Se ele não se comportasse corretamente, isso não causaria nenhum dano,
exceto isolar aquela rede do resto da Internet. Nenhum computador ou rede de computadores
precisava de permissão para ingressar na Internet. Se seguisse os padrões da Internet,
poderia ser compreendido por outros e interpretar mensagens direcionadas a ele.

Quando olhamos para a Internet hoje, ela parece variada e complicada: tantos tipos
diferentes de conteúdo, tantos tipos diferentes de dispositivos e tantos tipos diferentes
de conexões. Mas tudo é construído em cima de um único protocolo, conhecido simplesmente
como Protocolo de Internet (IP). É função do IP levar um único pacote de cerca de mil bits
de uma extremidade de um link de rede de comunicações para a outra. Os bits, ao serem
entregues, podem conter erros; nada físico funciona perfeitamente o tempo todo. Mas os
erros podem ser reconhecidos e, se necessário, tratados. Para fazer os pacotes atravessarem
a rede, o IP é usado repetidamente, como um estilo de corrente humana, com cada ponto de
comutação recebendo pacotes, verificando-os e, em seguida, enviando-os em direção ao destino
pretendido.

A simplicidade do design da Internet tornou possível construir protocolos sobre protocolos
para expandir o utilitário da Internet. Os usos iniciais da Internet eram para fazer login
em computadores de tempo compartilhado remotamente, mover arquivos de um lugar para outro
e enviar correio eletrônico. Todos esses serviços exigiam que os dados chegassem sem
erros, mas não necessariamente instantaneamente. Ninguém perceberia se uma transferência
de arquivo ou a entrega de um e-mail levasse uma fração de segundo extra, mas um único bit
mudar de 0 para 1 em trânsito poderia ter consequências catastróficas. Para tais
transferências, um protocolo foi desenvolvido para garantir que os pacotes enviados pela
fonte fossem recebidos corretamente e reconstituídos na ordem correta. Dada a falta de
confiabilidade dos nós intermediários da rede, isso requer algum registro tanto na origem
quanto no destino. Um pacote, uma vez recebido, é confirmado ao enviar um pacote especial
de volta do destino para a origem. A origem executa um temporizador; se um pacote enviado
não for confirmado antes que o temporizador zere, a origem conclui que o pacote se perdeu
de alguma forma e o retransmite.

Os detalhes são complicados, mas não são importantes para o panorama geral. O resultado é
que, desde que os comutadores estejam fazendo o melhor esforço para encaminhar os pacotes
em direção ao destino, qualquer mensagem enviada será recebida em perfeita ordem. O
protocolo que garante essas transmissões perfeitas é chamado de Protocolo de Controle de
Transmissão (TCP). Como o protocolo subjacente para mover pacotes ao longo de links
individuais da rede é o IP, TCP/IP é o nome comum para o par de convenções que tornam
comunicações confiáveis possíveis em uma rede não confiável.

Uma vez que não existem regras para ingressar na Internet, é válido questionar a suposição
de ``melhor esforço''. Será que um agente mal-intencionado poderia tentar sabotar a rede
adicionando comutadores que descartariam ou redirecionariam os pacotes em vez de enviá-los
para o destino pretendido? De fato, isso poderia acontecer, mas os pontos de comutação
vizinhos eventualmente perceberiam que os pacotes não estavam sendo entregues e começariam
a evitar os elementos maliciosos. O roteamento da Internet se autocorrige ao aprender a
evitar pontos problemáticos --- não apenas em caso de falhas de hardware, mas também em
caso de malícia. A Internet se torna mais confiável à medida que cresce em tamanho e se
torna mais interconectada.

A Internet funcionou porque, uma vez que um número suficientemente grande de partes
concordou em usá-la da maneira pretendida, os atores mal-intencionados poderiam, na prática,
ser excluídos, uma vez que eram poucos em número.

Além das informações de roteamento e \ingles{payload}, os pacotes também incluem alguns
bits redundantes para auxiliar na detecção de erros. Por exemplo, um único bit extra pode
ser adicionado a cada pacote para garantir que todos os pacotes transmitidos tenham um número
ímpar de bits 1. Se um pacote chegar com um número par de bits 1, ele pode ser reconhecido
como tendo sido corrompido durante a transmissão e descartado, para que o remetente possa
retransmiti-lo. Tais bits extras não podem garantir que cada pacote recebido esteja correto.
Mas eles garantem uma transmissão correta com probabilidade avassaladora e, do ponto de vista
prático, esse processo é suficiente para tornar a probabilidade de um erro não detectado
menor do que a probabilidade de uma catástrofe, como um impacto de meteorito, na fonte da
transmissão.

O IP, o protocolo de encaminhamento de pacotes de melhor esforço, também pode ser usado para
entregar mensagens de forma imperfeita, mas rápida. Por exemplo, pense em como a Internet
pode ser usada para transmitir comunicações de voz, como chamadas telefônicas. O sinal de voz
pode ser dividido em pequenos intervalos de tempo, cada um digitalizado e enviado pela
Internet. Mas em vez de usar o TCP, que garante a entrega, mas não a rapidez, um protocolo
diferente, chamado UDP, é utilizado. O UDP aceita alguma perda de pacotes em troca de uma
entrega mais rápida. Os tons de voz mudam lentamente de um instante para o próximo, de modo
que os pacotes de uma conversa telefônica podem ser um pouco embaralhados e alguns podem ser
omitidos totalmente, sem dificultar a compreensão da conversa --- desde que os pacotes que a
compõem cheguem a uma taxa aproximadamente correta.

Muitos outros protocolos foram projetados para outros fins e para serem sobrepostos a esses,
utilizando o TCP e o UDP para realizar tarefas de comunicação mais complexas. Por exemplo, o
Protocolo de Transferência de Hipertexto (HTTP) foi projetado para a comunicação entre um
navegador da web em um computador do usuário e um servidor web em qualquer outro lugar do
mundo. O HTTP depende do TCP para recuperar páginas da web com base em informações de
localização, como lewis.seas.harvard.edu. Portanto, sem precisar conhecer os detalhes de como
o TCP opera, qualquer pessoa poderia configurar um servidor web que entregasse páginas da web
em resposta a requisições recebidas.

\subsection{Quem está no comando?}
\label{gatekeepers:quem}

Não existem policiais da Internet para obrigar alguém a formatar seus pacotes como TCP, IP,
UDP ou outros protocolos estipulados. Ninguém o expulsará da Internet se você colocar o
endereço de origem onde o destino deveria estar e vice-versa. Se seus pacotes não estiverem
de acordo com os padrões, eles simplesmente não serão entregues, ou serão ignorados se forem
entregues.

No entanto, a Internet possui algumas autoridades governantes. Uma delas é o Internet
Engineering Task Force (IETF), que estabelece os padrões para os protocolos da Internet. O
IETF é uma organização notável. Está aberta a qualquer pessoa que queira participar e toma
decisões com base em ``consenso geral e código em funcionamento''. Em anos anteriores, o
IETF se reunia em uma sala e determinava o ``consenso geral'' ao fazer com que os membros
debatessem. Maiorias substanciais eram evidentes para todos, e as preferências individuais
desfrutavam de um nível de anonimato, pois em um grupo grande é difícil identificar quem
está concordando e quem não está. Como a maioria das mudanças nos protocolos da Internet são
aprimoramentos e adições que não alteram nada que já esteja funcionando, raramente há
necessidade de tomar uma decisão positiva sob pressão de tempo; o IETF pode adiar decisões,
permitir que as pessoas falem mais enquanto ajustam suas propostas e aguardar o
desenvolvimento de um consenso verdadeiro.

Assim, a Internet é aberta por design. Qualquer pessoa pode participar do processo de tomada
de decisão. Não estaria errado lembrar do utopismo comunal dos anos 1960. O membro inicial
do IETF, David Clark, popularmente disse: ``Rejeitamos reis, presidentes e votações.
Acreditamos em um consenso geral e código em funcionamento'' --- a última frase indicando a
preferência do engenheiro por provas de conceito em vez de conceitos isolados.\footnote{Pete
Resnick, ``On Consensus and Humming in the IETF,'' Internet Engineering Task Force, June 2014,
\url{https://tools.ietf.org/html/rfc7282}.} Claro, uma vez que a Internet se tornou amplamente
adotada, seria necessário fazer muita persuasão para desenvolver um consenso para mudar
qualquer coisa que se tornasse importante para muitas pessoas. Mas se você e eu estivéssemos
a meio mundo de distância um do outro e decidíssemos desenvolver nosso próprio protocolo
secreto para (digamos) duetos de xilofone transpacíficos, poderíamos programar nossos
computadores para trocar pacotes IP que mais ninguém saberia o que fazer. O IETF explica seu
papel desta maneira em sua declaração de missão:

\begin{quote}
Quando o IETF assume a propriedade de um protocolo ou função, ele aceita a responsabilidade
por todos os aspectos do protocolo, mesmo que alguns aspectos raramente ou nunca sejam vistos
na Internet. Por outro lado, quando o IETF não é responsável por um protocolo ou função, ele
não tenta exercer controle sobre ele, mesmo que às vezes possa tocar ou afetar a Internet
\footnote{Harald Tveit Alvestrand, ``A Mission Statement for the IETF,'' Internet Engineering
Task Force, October 2004, \url{https://tools.ietf.org/html/rfc3935}.}
\end{quote}

Esta é uma afirmação notável e mostra como a metáfora da ``Autoestrada da Informação'' se
desintegra quando aplicada à Internet. Se a Internet é uma estrada, é uma em que os veículos
motorizados aderem voluntariamente a certas convenções para poderem compartilhar a estrada com
segurança, mas ciclistas e skatistas também são bem-vindos a usar as vias --- embora por sua
própria conta e risco.

A Internet é aberta em outra direção também. Assim como o IP serve como a camada base para
uma hierarquia de protocolos, o IP em si é um protocolo lógico, não físico. Os pacotes da
Internet podem ser transmitidos por fios de cobre, por cabos de fibra óptica ou por ondas
de rádio. Se você é um usuário comum de computador pessoal comprando algo na Amazon, é
provável que os pacotes que vão e voltam entre você e a Amazon passem por todos os três e
mais, à medida que se movem do seu computador para o roteador sem fio, para o seu provedor
de serviços de Internet, através da Internet, para a rede corporativa da Amazon e para um
dos computadores da Amazon. Sempre que os engenheiros desenvolvem uma nova forma de
transmitir bits através de meios físicos, eles também podem desenvolver uma implementação do
IP que funcione nesse meio físico. Há até mesmo um protocolo de pombo-correio que, em
princípio, poderia ser usado para implementar o IP.

O IP, o formato pelo qual todos os pacotes passam pela Internet, desempenha um papel
semelhante ao design da tomada elétrica de 120V onipresente, com três buracos de forma e
dimensões especificadas. A fonte elétrica de um lado da tomada pode ser, em última instância
uma represa hidrelétrica a centenas de quilômetros de distância, painéis solares a apenas
alguns metros de distância ou baterias. Desde que a eletricidade esteja de acordo com os
padrões, a tomada está fazendo o seu trabalho. Os dispositivos que são conectados à tomada
podem ser geladeiras, escovas de dente, aspiradores de pó ou brocas dentárias. Desde que um
dispositivo seja equipado com a tomada certa e seja projetado para funcionar com corrente
alternada padrão, ele funcionará. Da mesma forma, o Protocolo de Internet age como um mediador
universal entre aplicativos e meios físicos.

Na verdade, a padronização do IP é a razão pela qual a Internet possui tantos usos que
inicialmente não foram previstos. O Zoom e o Facetime --- aplicações da Internet para
conectar pessoas por meio de links de áudio e vídeo ao vivo --- foram construídos com base
no IP, mesmo que não houvesse absolutamente nada sobre tais serviços no design original da
Internet. Os inventores do sistema de telefone pela Internet, o Skype --- um pequeno grupo de
engenheiros escandinavos e estonianos --- apenas precisaram adaptar os protocolos da Internet
para seus propósitos. E eles não precisaram pedir permissão à IETF ou a qualquer outra
autoridade para começar a usar o Skype ou incentivar outros a começarem a pagar por seu uso.

\section{A Internet Não Tem Guardiões?}
\label{gatekeepers:nao-tem-guardioes}

Sempre foi um exagero dizer que a Internet não possui guardiões, mas isso é menos verdade 
agora do que costumava ser. Como veremos em breve, em alguns países, os governos são os
principais guardiões, e em outros, como nos Estados Unidos, as corporações privadas
assumem papéis de guardiões. Vamos começar com as formas de guarda que existiram por
muito tempo.

\subsection{Nomes para Números: Qual é o Seu Endereço?}
\label{gatekeepers:nomes-para-numeros}

O primeiro fato da vida na Internet é que não adianta estar ``na'' Internet se ninguém
puder te encontrar. Os pacotes que fluem pela Internet possuem endereços numéricos. Alguma
entidade precisa traduzir os nomes simbólicos --- como cornell.edu e Skype.com --- em
números e acompanhar quais pontos de conexão têm quais números.

A Corporação da Internet para Atribuição de Nomes e Números (ICANN) é a entidade que
determina quais endereços numéricos são atribuídos à Universidade de Cornell ou à nação da
Austrália. Ela supervisiona a publicação de diretórios eletrônicos de tal forma que qualquer
pessoa que envie um e-mail para o endereço president@cornell.edu ou recupere uma página da
web de um endereço como http://anu.edu.au (a página inicial da Universidade Nacional
Australiana) seja direcionada para o local correto na Internet. As tabelas de tradução, de
letras para números, são mantidas em servidores do Sistema de Nomes de Domínio (DNS), que
outros computadores consultam para procurar os endereços numéricos a serem inseridos no campo
``destino'' dos pacotes IP antes de serem lançados na Internet. Se a Internet tem uma única
vulnerabilidade, é o controle sobre os servidores DNS. A nação insular de Tuvalu obtém seu
próprio domínio de alto nível na Internet, como .au para a Austrália? (Sim, e é muito valioso.
É .tv, e a nação, que costumava ganhar dinheiro vendendo selos, agora obtém receita ao permitir
que sites de vídeo como twitch.tv usem a extensão .tv.) Quem decide se a Coca-Cola tem direito
a cocacola.com ou, nesse sentido, cocacola.porcaria? Não deve ser surpresa que tais questões de
alto risco despertem grande interesse de governos e corporações multinacionais. Geralmente,
essas questões não podem ser resolvidas simplesmente concordando anonimamente ou por meio de
qualquer processo igualmente inclusivo.

No entanto, essas disputas territoriais são resolvidas sem o uso de força e sem fraturar a rede.
Na verdade, desde as primeiras máquinas conectadas à ARPANET em 1969, o número de dispositivos
conectados cresceu para bilhões hoje em dia. Qualquer um deles pode, em princípio, se conectar
a qualquer outro.\footnote{O design do protocolo de pacotes da Internet foi estabelecido antes
da era da miniaturização em massa e em um momento em que a memória do computador era limitada e
cara. Naquela época, ninguém imaginava a necessidade de conectar mais de 4 bilhões de
computadores à Internet, então apenas 32 bits foram reservados para os campos de endereço. Mas
agora relógios de pulso e refrigeradores têm seus próprios endereços IP, e o número total de
computadores conectados é maior do que pode ser distinguido usando 32 bits. Foram desenvolvidas
várias soluções alternativas, e um novo protocolo, o IPv6, que possui endereços de 128 bits,
está sendo lentamente implementado.} Os problemas sérios de guarda da Internet estão em outro
lugar.

\section{Guardiões dos Links: Conectando-se}
\label{gatekeepers:guardioes-links}

A Internet não é muito útil se você não consegue se conectar.

Se você dirigir para o oeste de Boston através do norte dos Estados Unidos, as opções nos
balcões de delicatessen dos supermercados mudam quando você chega ao Iowa. De repente, eles
apresentam saladas de gelatina em grande variedade, incorporando várias frutas picadas, camadas
coloridas e coberturas cremosas. Às vezes, a gelatina é moldada ao redor de peixe ou carne. A
moda persiste pelas Grandes Planícies e sobe pelas Montanhas Rochosas, mas desaparece nas
encostas descendentes. Quando você chega ao Oceano Pacífico, a gelatina novamente é principalmente
para crianças e pacientes hospitalares. O amor pelas criações de gelatina no coração rural é tão
prevalente que esses pratos são comumente apresentados nas capas de revistas de alimentos
disponíveis nos caixas dos supermercados costeiros e urbanos, mesmo que ninguém que compre lá
sonharia em servir tal coisa. Entre as elites costeiras e urbanas, as saladas de gelatina são
consideradas pouco sofisticadas.

A predileção do Meio-Oeste por saladas de gelatina está desaparecendo agora, à medida que a
cultura gastronômica, assim como o restante da cultura americana, está se tornando
geograficamente homogeneizada. Em algumas áreas, as receitas de salada de gelatina da vovó são
lembradas como chapéus de palha feitos à mão e vestidos de chita --- como artefatos de um
passado rural fora do lugar em tempos mais avançados. Mas as saladas de gelatina não foram
levadas para o Oeste em carroças cobertas; isso teria sido impossível, já que criá-las requer
refrigeração. Elas são, em vez disso, um subproduto de uma revolução tecnológica do século XX:
a eletrificação rural. As saladas de gelatina eram uma iguaria nas fazendas porque você não
poderia fazê-las se sua fazenda não fosse eletrificada. As pessoas que serviam saladas de
gelatina também tinham bombas de poço elétricas e luzes elétricas. Servir gelatina provava que
você estava tecnologicamente avançado.

A economia de difundir eletricidade e de difundir bits por meio de fios ou cabos é semelhante.
É caro instalar cabos em longas distâncias e, se houver poucos clientes no final da linha, não
é economicamente viável instalar o cabo. Também é caro puxar fios para muitas residências
individuais se estiverem distantes umas das outras. Os lucros são muito maiores ao conectar uma
cidade, porque assim que a linha é trazida até uma rua, cada unidade habitacional na rua se
torna um cliente, e a distância da linha principal para os clientes tende a ser curta. De fato,
as cidades foram eletrificadas primeiro, principalmente para a iluminação pública, que não exigia
a instalação de fiação dentro de edifícios particulares. Todos os outros usos da eletricidade
--- para iluminação interna, geladeiras, máquinas de lavar, lava-louças e rádios --- derivaram do
uso para a iluminação pública. Até hoje, um passeio pela Beacon Street, perto do Fenway Park em
Boston, leva você a uma estrutura que diz ``The Edison Electric Illuminating Company''.

Não teria feito sentido difundir a eletricidade em nível nacional apenas para que as pessoas
pudessem comer gelatina. Na verdade, não existia geladeira doméstica elétrica quando os primeiros
postes de iluminação elétrica foram acesos. A eletricidade doméstica era o que Jonathan Zittrain
chama de tecnologia geradora.\footnote{Jonathan Zittrain, \ingles{The Future of the Internet—and
How to Stop It} (Yale University Press, 2008).} Uma vez que a infraestrutura foi instalada,
pessoas criativas começaram a sonhar com usos para ela e desenvolveram tecnologias totalmente
novas que não poderiam ter existido sem ela. No processo, alguns setores antigos morreram, com
enormes custos para aqueles que lucraram com eles. Hoje em dia, ``caixa de gelo'' é, na melhor das
hipóteses, uma expressão nostálgica para uma geladeira, mas há um século havia um enorme setor
dedicado a cortar a superfície congelada dos lagos em blocos, transportá-los por longas distâncias
e distribuí-los nos lares americanos. As tecnologias geradoras também são tecnologias destrutivas.

A difusão da Internet seguiu em grande parte as mesmas etapas que a difusão da eletricidade já fez.
A iluminação foi a aplicação matadora para a eletricidade; as saladas de gelatina foram os vídeos
engraçados da rede elétrica. No entanto, a experiência dos Estados Unidos com a Internet tem sido
muito diferente do que foi com a rede elétrica.

Os Estados Unidos foram eletrificados rapidamente e de forma ubíqua, mas essa rápida expansão não
teria ocorrido sem um incentivo do governo federal. Fazer a instalação elétrica em áreas remotas
não era lucrativo, e teria sido ainda mais desvantajoso para um concorrente instalar um segundo
cabo para atender aos mesmos clientes. Portanto, a eletricidade estava prontamente disponível nas
cidades, mas extremamente cara em áreas remotas, quando estava disponível.

Franklin Roosevelt pôde perceber a diferença no preço da eletricidade quando se retirou de Nova
York e Washington, DC, para Warm Springs, Geórgia, onde buscava alívio e tratamentos de spa para
sua paralisia. Foi lá que ele concebeu a Lei de Eletrificação Rural e a assinou em lei em 1936.
Essa iniciativa estimulou não apenas a difusão da eletricidade, mas também a invenção de novas
maneiras para os consumidores comuns a utilizarem.

No início dos anos 1920, menos de 1\% das residências nos Estados Unidos tinham eletricidade. Seis
anos após a assinatura da Lei de Eletrificação Rural, seguindo uma terrível recessão econômica,
metade das residências nos Estados Unidos estavam eletrificadas. Em 1960, praticamente todo o país
tinha eletricidade.

Para fazer uma comparação sensata entre a difusão da Internet e a difusão da eletricidade,
precisamos definir nossos termos. A eletricidade fornecida às residências foi padronizada: nos
Estados Unidos, a eletricidade é corrente alternada, com frequência de 60 Hz e tensão de 120V.
Esses padrões são análogos ao IP para a Internet. Eles garantem que os mesmos aparelhos possam
ser conectados a tomadas em qualquer lugar do país.

Mas há outro parâmetro importante: a quantidade de energia elétrica usada por um aparelho ou
uma residência. A taxa na qual a eletricidade é usada é chamada de potência e é medida em
watts ou quilowatts (para milhares de watts). A quantidade de energia usada é medida em
quilowatt-hora; um quilowatt-hora (kWh) é a quantidade de energia consumida ao usá-la por uma
hora a uma taxa de 1.000 watts. Os códigos elétricos foram padronizados de forma que os
circuitos residenciais possam lidar com cerca de 2.000 watts; se você usar muito mais do que
isso, um disjuntor será desarmado, e o fio pode derreter se não houver um fusível ou disjuntor.
A fiação em uma casa antiga pode precisar ser atualizada quando um proprietário deseja usar
mais equipamentos elétricos ou equipamentos elétricos mais potentes, como ar-condicionado ou
uma banheira de hidromassagem. Por outro lado, os novos equipamentos tendem a usar menos
energia do que os antigos, portanto, o consumo médio por residência aumentou apenas lentamente
ao longo do tempo. As concessionárias de energia elétrica que realmente fornecem a energia
podem ter que atualizar sua rede de distribuição para acompanhar a demanda, mas uma combinação
de pressão do consumidor e normas federais geralmente tornam raro nos Estados Unidos ter
``apagões'', quando uma cidade inteira ou bairro tem energia insuficiente. Nos Estados Unidos,
a eletricidade é, em sua maioria, uma indústria regulamentada com sucesso.

O análogo da potência para a Internet é a taxa de bits, e aqui a experiência da Internet e da
rede elétrica se divergiram significativamente. O fornecimento de conectividade à Internet tem
sido deixado quase inteiramente para o setor privado, com regulamentação governamental mínima
e apoio governamental mínimo. Em quase nenhum lugar existe uma concorrência séria, então os
consumidores não podem mudar para provedores melhores. O provedor monopolista de serviços de
Internet pode oferecer uma escolha de velocidades, mas as velocidades mais altas provavelmente
serão exorbitantemente caras. Em uma palavra, em vez de fornecer Internet de alta velocidade,
a maioria dos provedores de serviços de Internet tenta nos convencer de que já a temos, e o
governo está ajudando em seu engano em vez de pressioná-los a melhorar seus serviços.

\subsection{Onde Estão os Gargalos e o Que Conta como ``Alta Velocidade''?}
\label{gatekeepers:gargalos}

A taxa na qual os bits completam sua transmissão --- por exemplo, de um servidor da web em
algum lugar para o navegador em execução no seu computador doméstico ou do seu computador de
escritório para a sala de bate-papo de vídeo na sede da sua empresa em Londres --- é a mais
lenta das taxas de qualquer um dos elos ao longo do caminho. A taxa na qual os bits fluem
através de um elo é afetada por alguns parâmetros físicos --- as propriedades elétricas ou
eletromagnéticas do cobre ou vidro dos quais o elo é composto --- e o quanto o tráfego é
pesado. Se a sua comunicação tiver o elo para si mesma, ela pode utilizar todos os bits por
segundo dos quais o meio de comunicação é capaz, mas se tiver que compartilhar essa capacidade
com um milhão de outras transmissões, a sua pode receber apenas um milionésimo.

Pense novamente no seu computador doméstico recuperando uma página da web de um servidor
pertencente a uma grande empresa. Seu pedido precisa chegar ao servidor corporativo, e os
pacotes que compõem a página da web precisam retornar ao seu computador doméstico.
Simplificando bastante, você pode pensar que os bits do seu pedido fazem três saltos. Eles
precisam ir do seu computador até a parede externa da sua casa; depois da sua casa para a
``rede de distribuição'', os cabos de longa distância que cruzam o país; e então através das
centenas ou milhares de milhas da rede de distribuição. A conexão da sua casa para a rede de
distribuição é comumente referida como a ``última milha''. A mesma hierarquia é, em princípio,
atravessada no outro extremo, exceto que a Amazon e o Google estão conectados diretamente à
rede de distribuição devido às suas enormes necessidades de capacidade. Se você estivesse se
comunicando com outra pessoa sentada em um computador doméstico, os bits teriam que percorrer
a ``última milha'' até a casa dessa pessoa.

Dentro de uma casa, a maioria das pessoas usa Wi-Fi, uma espécie de comunicação por rádio de
curta distância. As tecnologias Wi-Fi mais recentes podem atingir velocidades de gigabit, mas
na prática as conexões provavelmente serão mais lentas devido a interferências ou obstruções.
Usuários avançados ainda podem ter suas casas cabeadas para que possam conectar seus
computadores usando cabos Ethernet em vez de conexões sem fio.

Mas uma conexão sem fio lenta pode ser suficientemente rápida se a ``última milha'' já for
lenta de qualquer maneira. E nos Estados Unidos, quase certamente é lenta. Em termos globais,
o que é chamado nos Estados Unidos de ``Internet de alta velocidade'' simplesmente não o é.

A rede de distribuição da Internet é composto por cabos de fibra ótica. A fibra é incrível; o
próprio vidro possui uma capacidade de transporte de informações praticamente ilimitada. Sua
capacidade real não é limitada pelo vidro, mas sim pela eletrônica que conecta a rede nos
pontos de comutação. A eletrônica está constantemente sendo aprimorada; uma vez instalada, a
fibra nunca é substituída (a menos que seja danificada --- por exemplo, porque um barco de
pesca a pega).\footnote{For a full account of the story of fiber, see Susan Crawford, Fiber:
\ingles{The Coming Tech Revolution --- and Why America Might Miss It}, Yale University Press,
2018.}

Em algumas partes do mundo, a última milha também é composta por fibra, de modo que essas
incríveis capacidades de informações chegam diretamente às portas de residências e empresas.
Em Cingapura e na Suécia, praticamente todos têm acesso a velocidades de Internet na casa dos
bilhões de bits por segundo. Em contraste, talvez 15\% dos americanos tenham conexões de fibra
em suas residências, e essa porcentagem não está aumentando. A maioria de nós está conectada
por fios telefônicos legados, usando o chamado serviço DSL, ou por cabo coaxial que foi
instalado para trazer a TV a cabo para nossas casas. Até mesmo o serviço DSL está sendo
gradualmente descontinuado em alguns lugares por ser pouco lucrativo. E as provedoras de cabo
e telecomunicações efetivamente dividiram o mapa entre elas; em apenas poucos lugares um
consumidor pode escolher entre o serviço de cabo e DSL, muito menos entre múltiplos serviços
de cabo.

A própria Internet não se conecta à sua casa, escritório ou celular. As pessoas comuns
conectam seus dispositivos não à Internet, mas a um provedor de serviços de Internet (ISP).
Como os pacotes da Internet podem viajar por uma variedade de mídias físicas, em princípio,
não há limite para o número de ISPs que podem transportar pacotes para dentro e para fora de
sua casa, mediante pagamento. A realidade é muito diferente. Nos Estados Unidos, é provável
que seu ISP residencial seja a AT\&T, Time Warner, Comcast/Xfinity, Verizon ou Charter. A
razão pela qual há tão poucos é que cada uma delas é uma empresa de telecomunicações ou de
televisão a cabo, e elas estão usando a fiação ou cabo de fibra óptica que já trouxeram para
sua casa para fornecer serviço de telefone ou televisão. O acesso à Internet sem fio também é
possível --- por isso não se entende a conexão Wi-Fi entre seu computador e um roteador, que
está então conectado ao seu ISP, mas sim uma conexão sem fio diretamente com o ISP. A conexão
via satélite é possível em áreas rurais onde nenhum tipo de conexão com fio foi instalado,
mas a Internet via satélite é lenta e cara. Os celulares se conectam à Internet por meio da
rede de telefonia celular, mas a rede celular não é uma opção viável para uso residencial. E
os chamados sinais de rádio 5G, que estão sendo anunciados como o futuro da conectividade à
Internet, viajam apenas curtas distâncias. Portanto, uma infraestrutura 5G é realista apenas
em áreas densamente povoadas, onde é possível instalar muitos hubs de forma econômica.

Os americanos são bombardeados com anúncios de ``Internet de alta velocidade'', mas, na
realidade, até mesmo a definição do governo de ``alta velocidade'' é enganosa. A última vez
que foi atualizada foi em 2015, quando --- contra as objeções dos provedores de serviços de
Internet --- a FCC aumentou o padrão de 4 Mbps para 25 Mbps. Isso equivale a um quadragésimo
da velocidade de gigabit que agora é padrão no Japão e na Suécia, e que até mesmo a China
está difundindo em áreas rurais. E o padrão dos EUA de 25 Mbps se aplica apenas às velocidades
de download, como se a Internet fosse basicamente um meio de transmissão para os consumidores
receberem filmes da Netflix. Muitas aplicações, desde videochamadas até a transferência de
imagens médicas, exigem altas velocidades em ambas as direções. Empresas de todos os tipos
dependem da conectividade à Internet tanto para uploads quanto para downloads; eles usam a
Internet para obter informações sobre seus produtos e serviços, e até mesmo os próprios
produtos e serviços, para seus clientes. Portanto, é muito difícil iniciar um negócio em uma
área onde a conectividade é fraca ou limitada a downloads rápidos e uploads lentos. E, no
entanto, nos Estados Unidos, a Internet foi otimizada como substituta da televisão --- como
uma forma para poucos fornecerem conteúdo para muitos.

Uma segunda forma de semântica criativa distorce as estatísticas governamentais sobre a
disponibilidade de ``Internet de alta velocidade'' nos Estados Unidos. O governo considera que
um setor censitário --- uma das cerca de 75.000 áreas geográficas em que os Estados Unidos
são divididos para fins de censo --- possui Internet de alta velocidade se mesmo uma única
residência tiver acesso a ela, independentemente do preço e independentemente de qualquer
residência ter efetivamente assinado uma conexão. Assim, as estimativas do governo sobre a
difusão da Internet de alta velocidade são amplamente infladas.

E o preço importa. Em grande parte do mundo, velocidades de gigabit estão disponíveis por
menos de US\$ 50 por mês. Na área de Boston, onde estamos escrevendo, custa US\$ 70 por mês
nas áreas restritas onde está disponível --- e requer um contrato de 24 meses.

O americano típico tem apenas uma ou duas opções realistas de ISP. Mais de 30\% das
residências nos Estados Unidos não têm provedores de serviço de Internet com 25 Mbps ou mais,
e menos de um quarto das residências nos Estados Unidos têm mais de uma escolha.\footnote{Jon
Brodkin, ``US Broadband: Still No ISP Choice for Many, Especially at Higher Speeds,''
\ingles{Ars Technica}, August 10, 2016,
\url{https://arstechnica.com/information-technology/2016/08/us-broadband-still-no-isp-choice-for-many-especially-at-higher-speeds/}.}

Em grande parte, a disseminação da conectividade à Internet foi deixada para o setor privado
nos Estados Unidos. Na verdade, regras em 26 estados dificultam ou proíbem que governos locais
ofereçam conectividade à Internet, forçando indivíduos para os quais nenhum serviço doméstico
acessível está disponível a se conectarem em bibliotecas públicas ou restaurantes de fast food.
\footnote{Kendra Chamberlain, ``Municipal Broadband Is Roadblocked or Outlawed in 25 States,''
\ingles{Broadband Now}, May 13, 2020,
\url{https://broadbandnow.com/report/municipal-broadband-roadblocks/}.} O código de
Montana\footnote{``Government Competition with Private Internet Services Providers Prohibited
--- Exceptions,'' Montana Code Annotated 2019,
\url{https://leg.mt.gov/bills/mca/title_0020/chapter_0170/part_0060/section_0030/0020-0170-0060-0030.html}.}
é típico: ``Exceto conforme previsto na subseção (2)(a) ou (2)(b), uma agência ou subdivisão
política do estado não pode, direta ou indiretamente, ser um provedor de serviços de Internet.''
Portanto, a menos que todos os ISPs privados se retirem, as pessoas de Browning, Montana, estão
presas às ofertas privadas de baixa qualidade e caras. Nenhum tipo de empreendedorismo municipal
pode ajudar sua população. Os lobistas das empresas privadas de telecomunicações chegaram
primeiro à legislatura estadual.

Agora, é claro, existem razões para impedir que os governos concorram com empresas privadas.
Os argumentos são familiares. A competição reduz os preços e melhora a qualidade. O dinheiro
dos contribuintes não deve ser usado para prejudicar os fornecedores privados. Os governos
não devem interferir nos mercados livres.

Mas simplesmente não há pessoas suficientes do outro lado para pagar pela conexão de longa
distância. Isso não é diferente da situação da entrega gratuita de correspondência rural
--- que se tornou lei em 1893 e chegou a Billings em 1902 --- ou da rede elétrica nas décadas
de 1930 e 1940. Conectar o país requer que as comunicações eletrônicas sejam vistas como
eletricidade ou correio postal: elas teriam que estar disponíveis para todos a um preço
acessível. Esse princípio, de fato, não é geralmente aceito. Em vez disso, a metáfora
operacional para a Internet é a televisão ou um cinema de várias telas. Os provedores de
serviços de Internet dominantes veem a Internet como uma forma de conectar provedores de
conteúdo ativo a consumidores passivos. É por isso que os ISPs oferecem pacotes que incentivam
o download e limitam o upload.

Na ausência do tipo de incentivo do governo que resultou em serviços de Internet muito
melhores na Coreia do Sul, Suíça e até na maior parte da Finlândia do que nos Estados Unidos,
por que a concorrência não reduziu os preços e aumentou a qualidade?

Alguns argumentariam que a ganância corporativa, a colusão e a corrupção são os culpados, e
embora essa perspectiva possa ter alguma validade em alguns casos, a realidade é que as redes
de comunicação crescem e se consolidam quase organicamente por motivos de eficiência. Paul
Baran próprio antecipou que esse fenômeno afetaria as redes de computadores vários anos antes
de projetar uma das primeiras redes. Em seu depoimento perante o Congresso sobre privacidade
eletrônica em 1966, ele disse,\footnote{Paul Baran, ``Full Text of `The Computer and Invasion
of Privacy','' July 26, 1966,
\url{https://archive.org/details/U.S.House1966TheComputerAndInvasionOfPrivacy/mode/2up}.}

\begin{quote}
Nossas primeiras ferrovias na década de 1830 eram rotas curtas que conectavam centros
populacionais locais. Ninguém se sentou e elaborou um plano mestre para uma rede de trilhos
de ferro. Com o tempo, um número crescente desses sistemas locais separados foi construído.
Uma rede gradualmente se formou à medida que a pressão econômica levou à construção de novos
elos para preencher as lacunas entre as rotas individuais.

Não começamos a construir uma rede telegráfica nacional no final dos anos 1840; apenas elos
telegráficos independentes. No entanto, não demorou muito para termos uma rede nacional
integrada. Até o nome, Western Union, evoca o padrão de elos independentes unidos para
fornecer um sistema mais útil.

Não começamos a construir um sistema telefônico nacional nos primeiros dias do telefone na
década de 1890. No entanto, hoje temos uma rede telefônica altamente integrada.

Tais padrões de crescimento não são acidentes. Comunicações e transporte são serviços que
historicamente tendem a formar ``monopólios naturais''. O motivo é bem compreendido. É mais
barato compartilhar o uso de uma grande entidade do que construir suas próprias instalações.
Portanto, se você observasse a Terra, digamos, a partir do ponto de vista distante da lua,
pareceria que o crescimento dessas redes integradas a partir de peças individuais é quase
biológico.
\end{quote}

Portanto, não é muito complicado. É mais valioso fazer parte de uma rede grande do que de
uma pequena, e quanto maior a rede, mais valioso é fazer parte dela. Na ausência de
resistência de alguma estrutura social com autoridade para resistir a fusões, consolidações,
aquisições e decisões estratégicas corporativas para controlar o tráfego de rede, as redes
de comunicação crescerão e se tornarão cada vez mais escassas com o tempo. Essa monopolização
não é necessariamente contrária ao interesse público --- desde que o interesse público
participe das decisões sobre distribuição e preços. Hoje em dia, isso raramente acontece.

\subsection{O carteiro pode decidir qual correspondência entregar?}
\label{gatekeepers:carteiro}

A história da Telus e seus trabalhadores em greve com a qual este capítulo começou demonstra
que a dicotomia entre links e conteúdo não é útil quando o guardião de links assume o papel
de guardião de conteúdo. A noção de que os provedores de serviços de Internet não devem
decidir quais bits entregar é conhecida como neutralidade da rede. Em princípio, pode
parecer simples e incontestável; afinal, não queremos que a companhia telefônica decida
quais conversas ela permitirá acontecer em suas linhas de voz. É verdade que, quando os
clientes não pagam suas contas, o serviço telefônico deles pode ser cortado, mas mesmo isso
é raro porque a sociedade geralmente reconhece --- ou costumava reconhecer --- que o serviço
telefônico é importante para a vida cotidiana. Mas a Internet não é exatamente como a rede
telefônica.

Por volta do mesmo período em que a Telus estava derrubando sites pró-sindicato no Canadá, um
pequeno ISP da Carolina do Norte chamado Madison River Communications fechou a Vonage, que
oferecia serviço de Voz sobre IP (VoIP). O uso da Internet para fornecer conversas de voz ao
vivo teria parecido loucura no início da Internet, porque a rede era muito lenta e os
computadores conectados a ela não conseguiam acompanhar o fluxo de pacotes para reuni-los em
uma fala compreensível. Mas os tempos mudam. À medida que as velocidades de conexão melhoraram
e novos protocolos, baseados em IP, foram otimizados para comunicações por voz, uma diferença
sistemática entre serviços telefônicos e de Internet interveio. As empresas de telefonia
cobravam a mais pelo serviço de longa distância; os provedores de serviços de Internet não se
importavam com a origem ou destino dos pacotes. Eles podiam cobrar mais por taxas de dados mais
altas, mas não por destinos mais distantes. Inevitavelmente, o software VoIP --- o Skype foi o
primeiro sucesso comercial nesse espaço --- foi desenvolvido para substituir a telefonia por
comunicações pela Internet e tornar as ``chamadas'' de longa distância virtualmente gratuitas
para qualquer pessoa com uma conexão à Internet. A Vonage estava usando o serviço de dados da
Madison River para minar o serviço telefônico da Madison River.

Quando a Vonage foi bloqueada, a empresa reclamou com a Comissão Federal de Comunicações, que
tem jurisdição sobre serviços telefônicos. O caso foi resolvido quando a Madison River
concordou em pagar uma multa e não bloquear o VoIP por três anos, mas essa resolução deixou
mais perguntas do que respostas em seu rastro. E se a Madison River fosse uma empresa de cabo
oferecendo serviços de Internet em vez de uma empresa telefônica? Por outro lado, e se a Madison
River fosse grande o suficiente para lutar contra a FCC nos tribunais? Não estava de forma
alguma claro se a FCC tinha a autoridade congressual para apoiar suas táticas autoritárias sobre
a maneira como até mesmo os ISPs de telefonia estavam escolhendo quais bits entregar.
\footnote{Scott Bradner, ``The Internet: Unblocking Pipes,'' Network World, March 14, 2005,
\url{https://www.networkworld.com/article/2319666/the-internet--unblocking-pipes.html}.}

A situação chegou ao auge em 2008, quando a FCC ordenou que o ISP Comcast parasse de
``estrangular'' --- ou seja, de reduzir a velocidade --- do BitTorrent, um serviço de
compartilhamento de arquivos peer-to-peer amplamente usado para entregar filmes para o lar.
A Comcast estava entregando filmes com lucro sobre o mesmo cabo que estava usando para
fornecer serviços de Internet, então o BitTorrent estava prejudicando seus negócios de entrega
de vídeo. A Comcast processou com sucesso a FCC, estabelecendo que a FCC realmente não tinha
autoridade para regular seu negócio de serviços de Internet. Essa decisão deu início a um
debate sobre a neutralidade da rede que se arrasta há mais de uma década.

Os detalhes são complexos. Em poucas palavras, vozes pró-neutralidade argumentaram a favor
da escolha e liberdade do consumidor; oponentes argumentaram que as forças de mercado
resolveriam quaisquer tensões, um argumento recebido com ceticismo por aqueles que observam
a escassa concorrência no espaço dos ISP. Nos Estados Unidos, as regras de neutralidade da
rede foram instituídas durante a administração Obama e revogadas durante a administração
Trump. Muitas outras nações adotaram a neutralidade da rede em princípio, mas algumas delas
permitem a cobrança com base no uso, o que pode ter o efeito de tornar certos aplicativos,
como assistir filmes, inaceitavelmente caros, alcançando assim o mesmo resultado ---
priorizando outros meios de entrega de filmes para casa --- que a Comcast havia alcançado ao
limitar os serviços peer-to-peer em 2008.

\section{Guardiões de Busca: Se você não consegue encontrá-lo, será que ele existe?}
\label{gatekeepers:guardioes-busca}

Presciente como Baran foi, ele não poderia ter antecipado o quanto, à medida que as redes de
comunicação se tornaram acessíveis a todos, o controle sobre as informações que elas
transportam também tenderia a cair nas mãos de um pequeno número de empresas privadas. A
tecnologia de busca foi um desenvolvimento surpreendente dos anos 1990; agora é difícil
imaginar um mundo sem ela. E ainda assim não é difícil imaginar um mundo em que o Google não
controle a maioria das buscas no mundo ocidental. Apenas aconteceu dessa maneira, com
consequências preocupantes.

\subsection{Encontrado Após 70 Anos}
\label{gatekeepers:encontrado}

Rosalie Polotsky tinha 10 anos quando acenou adeus às suas primas, Sophia e Ossie, na estação
de trem de Moscou em 1937. As duas irmãs estavam fugindo da opressão da Rússia Soviética para
começar uma nova vida. A família de Rosalie ficou para trás. Ela cresceu em Moscou, ensinou
francês, casou-se com Nariman Berkovich e criou uma família. Em 1990, ela emigrou para os
Estados Unidos e se estabeleceu perto de seu filho, Sasha, em Massachusetts. Rosalie, Nariman
e Sasha sempre se perguntaram sobre o destino de Sophia e Ossie. A Cortina de Ferro havia
cortado completamente a comunicação entre parentes judeus. Quando Rosalie partiu para os
Estados Unidos, seus laços com Sophia e Ossie haviam sido rompidos por tanto tempo que ela
tinha pouca esperança de se reconectar com eles --- e, à medida que os anos passavam, tinha
menos motivos para acreditar que suas primas ainda estivessem vivas. Embora seu avô sonhasse
em encontrá-las, a busca de Sasha nos registros de imigrantes na Ilha Ellis e na Cruz Vermelha
Internacional não forneceu pistas. Talvez, viajando pela Europa em tempos de guerra, as
meninas nem sequer tenham chegado aos Estados Unidos.

Então, um dia, o primo de Sasha digitou ``Polotsky'' na janela de busca do Google e encontrou
uma pista. Uma entrada em um site genealógico mencionava ``Minacker'', o nome do pai de Sophia
e Ossie. Em pouco tempo, Rosalie, Sophia e Ossie se reuniram na Flórida, após 70 anos
separados. ``Durante todo o tempo em que ele estava vivo, ele me pediu para fazer algo para
encontrá-los'', disse Sasha, lembrando o desejo de seu avô. ``É algo mágico''.\footnote{Eva
Wolchover, ``Web Reconnects Cousins Cut off by Iron Curtain,'' Boston Herald, December 18,
2007.}

A \ingles{World Wide Web} colocou vastas quantidades de informações ao alcance de milhões de
pessoas comuns. Mas você não pode alcançar algo se não souber onde está. A maior parte desse
vasto tesouro de informações digitais poderia muito bem não existir sem uma maneira de
encontrá-lo. Na verdade, a ``\ingles{dark web}'' existe como uma espécie de universo paralelo,
com tesouros de informações invisíveis para mecanismos de busca e para usuários que não sabem
onde procurar.

A busca tanto realiza sonhos quanto molda o conhecimento humano. As ferramentas de busca que
nos ajudam a encontrar agulhas no palheiro digital são as lentes através das quais
visualizamos a paisagem digital. Mas as ``lentes'' não são passivas. Elas ativamente
influenciam o que vemos, selecionando o que nos mostrar na primeira página de resultados e na
ordem em que os resultados são apresentados. Quem controla o mecanismo de busca molda --- e
distorce --- a realidade que vemos através dele. O Google, que é usado em mais de 90\% das
buscas do mundo,\footnote{``Search Engine Market Share Worldwide 2019,'' Statista, accessed
April 27, 2020, \url{https://www.statista.com/statistics/216573/worldwide-market-share-of-search-engines/}.}
é suportado por publicidade, então surgem inevitavelmente questões sobre se os resultados
otimizam os lucros do Google ou a satisfação dos usuários. O Bing da Microsoft não é menos
eficaz na produção de resultados, mas tem menos de 5\% do mercado. O DuckDuckGo, que oferece
proteções de privacidade muito mais fortes do que o Google ou o Bing, mas produz resultados
menos direcionados, tem uma participação de mercado negligenciável.\footnote{Nathaniel Popper,
``A Feisty Google Adversary Tests How Much People Care About Privacy,'' \ingles{The New York
Times}, July 15, 2019, \url{https://www.nytimes.com/2019/07/15/technology/duckduckgo-private-search.html}.}
O Baidu é o mecanismo de busca dominante na China, mas por uma razão: ele censura fortemente,
como qualquer mecanismo de busca no mercado chinês deve fazer. Como essas estatísticas
desequilibradas surgiram e quais são suas consequências?

\subsection{A Queda da Hierarquia}
\label{gatekeepers:a-queda}

Desde o início da escrita até cerca de 1994, havia apenas duas maneiras de organizar
informações para que pudessem ser recuperadas rapidamente. Você poderia colocá-las em uma
hierarquia ou poderia criar um índice.

Uma hierarquia permite que você coloque coisas em categorias e divida essas categorias em
subcategorias. Aristóteles tentou classificar tudo. Seres vivos, por exemplo, eram ou plantas
ou animais. Os animais tinham ou não sangue vermelho; os animais de sangue vermelho eram ou
vivíparos ou ovíparos; os vivíparos eram ou humanos ou outros mamíferos; os ovíparos nadavam
ou voavam; e assim por diante. Esponjas, morcegos e baleias apresentaram enigmas de
classificação, sobre os quais Aristóteles não achou que tinha a última palavra. No início do
Iluminismo, Linnaeus forneceu uma maneira mais útil de classificar os seres vivos, usando uma
abordagem que ganhou validade científica intrínseca uma vez que refletiu linhas evolutivas de
descendência.

Nossas tradições de classificação hierárquica são evidentes em todos os lugares. Adoramos
estruturas de esquema. A lei contra a quebra de proteção de direitos autorais é o Título 17,
Seção 1201, parágrafo (a), parte (1), subparte (A). No sistema da Biblioteca do Congresso,
cada livro está em uma das 26 categorias principais, designadas por uma letra maiúscula, e
essas categorias principais são divididas internamente de maneira semelhante; por exemplo, na
categoria B, filosofia, você encontra BQ, Budismo.

Se as categorias forem claras, pode ser possível usar uma hierarquia de organização para
encontrar o que você está procurando. Isso requer que a pessoa que está fazendo a pesquisa
não apenas conheça o sistema de classificação, mas também seja habilidosa em tomar todas as
decisões necessárias. Por exemplo, se o conhecimento sobre seres vivos fosse organizado da
maneira que Aristóteles o fez, qualquer pessoa que quisesse saber sobre baleias teria que saber
previamente se uma baleia é um peixe ou um mamífero para seguir o ramo adequado da árvore de
classificação. À medida que mais e mais conhecimento precisa ser encaixado na árvore, a árvore
cresce e brotam galhos, que com o tempo se tornam ramos que brotam mais galhos. O problema de
classificação se torna complicado, e o problema de recuperação se torna praticamente impossível.

Em 1991, quando a Internet era pouco conhecida fora de círculos acadêmicos e governamentais,
alguns pesquisadores acadêmicos lançaram um programa chamado Gopher. Esse programa fornecia
um diretório hierárquico de muitos sites, organizando os diretórios fornecidos pelos sites
individuais em um grande esquema. Encontrar coisas usando o Gopher era tedioso pelos padrões
de hoje e dependia das habilidades organizacionais dos colaboradores. O Yahoo! foi fundado em
1994 como um diretório online da Internet, com editores humanos colocando produtos e serviços
em categorias, fazendo recomendações e tentando tornar a Internet acessível para pessoas não
técnicas. Embora hoje seja um site de busca e notícias, o nome ``Yahoo'' originalmente dizia-se
ser um acrônimo para ``\ingles{Yet Another Hierarchical Organized Oracle}''.

As limitações práticas das árvores de organização hierárquica foram previstas há 60 anos, muito
antes do crescimento explosivo da \ingles{World Wide Web} e das inúmeras mudanças diárias nela.
Durante a Segunda Guerra Mundial, o presidente Franklin Roosevelt nomeou Vannevar Bush do MIT
para servir como diretor do \ingles{Office of Strategic Research and Development} (OSRD). O
OSRD coordenava a pesquisa científica em apoio ao esforço de guerra. Foi um grande esforço, com
30.000 pessoas e centenas de projetos cobrindo o espectro da ciência e engenharia. O Projeto
Manhattan, que produziu a bomba atômica, era apenas uma pequena parte dele.

Do ponto de vista de Bush, ele viu um grande obstáculo para o progresso científico contínuo.
Estávamos produzindo informações mais rapidamente do que podiam ser consumidas --- ou mesmo
classificadas. Décadas antes de os computadores se tornarem comuns, ele escreveu sobre esse
problema em um artigo visionário intitulado ``\ingles{As We May Think}''.\footnote{Vannevar Bush,
``As We May Think,'' \ingles{The Atlantic}, July 1, 1945,
\url{https://www.theatlantic.com/magazine/archive/1945/07/as-we-may-think/303881/}.} Este artigo
foi publicado na Atlantic Monthly --- uma revista famosa, não um jornal técnico. Como Bush viu,

\begin{quote}
    A dificuldade parece ser, não tanto que publicamos de forma excessiva\ldots mas sim que a
    publicação foi estendida muito além de nossa capacidade atual de fazer uso real do registro.
    A soma da experiência humana está se expandindo a uma taxa prodigiosa, e os meios que usamos
    para navegar por esse labirinto resultante até o item momentaneamente importante são os
    mesmos que eram usados nos dias dos navios de velas quadradas\ldots Nossa inaptidão para
    acessar o registro é em grande parte causada pela artificialidade dos sistemas de indexação.
\end{quote}

A aurora da era digital estava nesse momento apenas um vislumbre no horizonte. Mas Bush imaginou
uma máquina, que ele chamou de ``memex'', que ampliaria a memória humana armazenando e
recuperando todas as informações necessárias. Seria um ``suplemento íntimo aumentado'' para a
memória humana, que poderia ser ``consultado com velocidade e flexibilidade excepcionais''.

Bush claramente percebeu o problema, mas as tecnologias disponíveis na época --- microfilme e
válvulas de vácuo --- não podiam resolvê-lo. Ele compreendeu que o problema de encontrar
informações eventualmente sobrecarregaria o progresso da ciência na criação e registro do
conhecimento, e ele antecipou que seria possível buscar usando vários termos para isolar tipos
especiais de informações:

\begin{quote}
    Surgirão formas completamente novas de enciclopédias, prontas com uma rede de trilhas
    associativas que as atravessam, prontas para serem inseridas no memex e ampliadas lá\ldots
    
    O historiador, com um vasto relato cronológico de um povo, o compara a uma trilha de saltos
    que só para nos itens mais relevantes e pode seguir, a qualquer momento, trilhas
    contemporâneas que o levem por toda a civilização em uma época específica. Surgiu uma nova
    profissão de desbravadores de trilhas, aqueles que encontram prazer na tarefa de estabelecer
    trilhas úteis através da enorme massa de registros comuns. A herança do mestre se torna não
    apenas suas adições ao registro mundial, mas para seus discípulos, todo o andaime pelo qual
    foram erguidos.
\end{quote}

Bush estava intensamente ciente de que a própria civilização havia sido ameaçada pela guerra,
mas ele acreditava que deveríamos avançar com otimismo em relação ao que o registro de nosso
vasto conhecimento poderia nos trazer:

\begin{quote}
    Presumivelmente, o espírito humano deveria ser elevado se ele puder revisar melhor seu
    passado sombrio e analisar de forma mais completa e objetiva seus problemas atuais. Ele
    construiu uma civilização tão complexa que precisa mecanizar seus registros de forma mais
    completa se quiser levar seu experimento a uma conclusão lógica e não apenas ficar atolado
    no meio do caminho sobrecarregando sua memória limitada. Suas excursões podem ser mais
    agradáveis se ele puder recuperar o privilégio de esquecer as inúmeras coisas de que não
    precisa imediatamente à mão, com alguma garantia de que poderá encontrá-las novamente se
    elas se mostrarem importantes.
    
    \ldots Ele pode perecer em conflitos antes de aprender a usar esse registro para o seu
    verdadeiro bem. No entanto, na aplicação da ciência às necessidades e desejos do homem,
    parece ser uma etapa singularmente infeliz para encerrar o processo ou perder a esperança
    quanto ao resultado.
\end{quote}

\begin{tcolorbox}[title={Um Precedente Futurista}]
    Em 1937, H. G. Wells antecipou a visão de 1945 de Vannevar Bush sobre um memex. Wells
    escreveu de forma ainda mais clara sobre a possibilidade de indexar tudo e o que isso
    significaria para a civilização:
    
    \begin{quote}
        Não existe nenhum obstáculo prático agora para a criação de um índice eficiente de
        todos os conhecimentos, ideias e realizações humanas, para a criação, ou seja, de uma
        memória planetária completa para toda a humanidade. E não apenas um índice; a
        reprodução direta da própria coisa pode ser convocada para qualquer local adequadamente
        preparado\ldots Isso, por si só, é um fato de tremenda importância. Isso antecipa uma
        verdadeira unificação intelectual de nossa raça. A memória humana inteira pode ser, e
        provavelmente em pouco tempo será, acessível a cada indivíduo\ldots Isso não é um sonho
        distante, nem uma fantasia.\footnote{H. G. Wells, \ingles{World Brain} (Methuen, 1938),
        pp. 60–61.}
    \end{quote}
\end{tcolorbox}

As capacidades que Bush não poderia ter previsto claramente são comuns agora. Computadores
digitais, armazenamento vasto e redes de alta velocidade tornam a busca e recuperação de
informações necessárias. Eles também tornam isso possível. A Web é uma realização do memex de
Bush, e a pesquisa é fundamental para torná-la útil.

\subsection{Históricos de Pesquisa}
\label{gatekeepers:historicos}

Bush não imaginou que todos teriam um memex, mas ele previu que as ``trilhas associativas''
continuariam. Vale a pena analisar um pouco mais de perto o que isso implica sobre a forma como
os motores de busca funcionam. O que Bush viu como uma nova estrutura de conhecimento importante
acabou sendo algo mais parecido com um resíduo digital --- um efeito colateral em grande parte
inofensivo dos sofisticados mecanismos digitais que usamos para realizar tarefas.

``Buscar'' é um termo um tanto inadequado para o que o Google e outros motores de busca
realmente fazem. Quando você digita algo em um motor de busca, o mecanismo não verifica a
\ingles{World Wide Web} inteira em busca disso. Ele procura o termo de busca em um índice que
já foi criado. É um índice muito grande e muito bem organizado para que possa ser atualizado
constantemente e para que as pesquisas possam se concentrar em vários termos, mas fundamentalmente
não é diferente do índice no final de um livro\ldots exceto que quando você procura algo no
índice de um livro, só você sabe que o fez. Se você pede ao Google para procurar algo em seu
índice, ele se lembra de que você o fez.

Existem boas razões para o Google lembrar o que você procurou. A informação pode ser útil para
ajudar o Google a responder a futuras pesquisas de maneira mais apropriada. Certamente seria útil
para ajudar o Google a direcionar publicidade para você. Você pode usar um mecanismo de busca que
preserve a privacidade (como o DuckDuckGo, mencionado anteriormente), mas talvez não fique tão
satisfeito com a qualidade dos resultados. A esmagadora dominação do Google sugere que as pessoas
estão dispostas a trocar sua privacidade por qualidade --- ou simplesmente estão indo com o nome
conhecido e não percebem a troca que estão fazendo.

Casey Anthony pode não ter pensado na permanência dos históricos de busca quando ela ou alguém 
usando seu computador pesquisou no Google ``quebrar pescoço'' e ``como fazer clorofórmio'' antes
da morte misteriosa de sua filha Caylee em 2012.\footnote{``Shady Web Searches in Missing Girl
Case,'' CBS News, November 26, 2008, \url{https://www.cbsnews.com/news/shady-web-searches-in-missing-girl-case/}.}
A divulgação desse histórico de pesquisa em seu julgamento não resultou em sua condenação; mais
tarde, descobriu-se que o mesmo computador havia sido usado para pesquisar ``sufocação infalível''
no dia em que a menina desapareceu.\footnote{Tony Pipitone, ``Cops, Prosecutors Botched Casey
Anthony Evidence,'' WKMG, November 28, 2012,
\url{https://www.clickorlando.com/news/2012/11/28/cops-prosecutors-botched-casey-anthony-evidence/}.}
(Os detetives não perceberam isso porque a pesquisa foi feita usando o navegador Firefox, em vez do
navegador Internet Explorer que revelou as informações sobre as outras pesquisas.) Alguns anos
antes, James Petrick foi condenado por matar sua esposa em parte com base em pesquisas que ele
havia feito sobre termos como ``pescoço'' e ``quebra rápida'', e por informações topográficas sobre
o lago onde o corpo dela foi encontrado.\footnote{K. C. Jones, ``Ex-Computer Consultant Convicted
in `Google Murder’ Trial,'' \ingles{InformationWeek}, November 30, 2005,
\url{https://www.informationweek.com/ex-computer-consultant-convicted-ingoog/174403074}.} Um tribunal
de apelações em Illinois manteve as condenações por assassinato e incitação ao assassinato de Steven
Louis Zirko\footnote{\ingles{People v. Zirko}, 2012 IL App (1st) 092158.} em parte com base em
pesquisas feitas no computador de Zirko por termos como ``mercenário para alugar'' e os horários em
que a criança de uma de suas vítimas estaria na escola.

Todos esses casos envolveram buscas policiais no computador doméstico de alguém. Mas há outra
maneira de obter informações sobre o histórico de pesquisa: perguntando ao Google. O Google não
fornecerá essas informações a qualquer pessoa que perguntar, mas você pode ver por si mesmo o que o
Google está lembrando sobre suas pesquisas e outras atividades realizadas enquanto você estava
conectado ao Google. (Ou pelo menos o que o Google diz que está lembrando; provavelmente está
lembrando muito mais.) Na página da sua conta do Google, há uma tela de ``Dados e personalização'',
onde você pode desativar o registro do histórico de pesquisa, por exemplo. Você até pode editar o
histórico sem se livrar dele completamente, se desejar --- como aparentemente fez o Dr. Brent
Dennis para enganar as autoridades. Ele disse à polícia que sua esposa morreu por beber
anticongelante, mas foi ele mesmo quem pesquisou ``anticongelante'' e depois contratou alguém
para limpar o histórico de pesquisa.\footnote{George Knapp and Matt Adams, ``I-Team: Details the
Night Attorney Susan Winters Died,'' 8NewsNow, February 10, 2017,
\url{https://www.8newsnow.com/news/i-team-details-the-night-attorney-susan-winters-died/}.}

\begin{tcolorbox}
    Sempre que você acessa um novo site que pede para você criar uma conta ou ``entrar usando
    o Google'' ou ``entrar usando o Facebook'', você pode ficar satisfeito por economizar tempo
    e ter uma senha a menos para lembrar. No entanto, o que você está realmente fazendo se
    optar por usar suas credenciais de login existentes do Google ou do Facebook é dar permissão
    ao Google ou ao Facebook para adicionar às informações enormes que já possuem sobre você as
    novas informações que eles obtêm com base em sua atividade no novo site.
\end{tcolorbox}

E o governo pode forçar o Google a fornecer o que sabe sobre você --- como seu histórico de
pesquisa, por exemplo. Não é nem mesmo tão complicado. Em 2018, o Google recebeu cerca de 130.000
solicitações de tribunais e outras agências governamentais e cumpriu, pelo menos em parte, cerca
de dois terços delas.\footnote{``Requests for User Information,'' Google Transparency Report,
accessed April 27, 2020, \url{https://transparencyreport.google.com/user-data/overview}.} O Google
diz que irá informá-lo quando uma agência estiver buscando seus registros, mas não está obrigado
a fazê-lo nem a cumprir seus desejos se você se opuser.

Para obter essas informações do seu laptop, as autoridades precisariam de um mandado de busca; ou
seja, teriam que apresentar um caso a um juiz de que seus direitos da Quarta Emenda contra buscas
não razoáveis não estavam sendo violados. Por que a polícia pode obter mais facilmente as mesmas
informações diretamente do Google?

\begin{tcolorbox}[title={A polícia pode buscar suas pesquisas?}]
    As autoridades podem obter informações sobre suas buscas mesmo se você não tiver feito nada
    de errado? Isso parece ter acontecido em Edina, Minnesota, no início de 2017.\footnote{``US
    Judge Asks Reports of Google Searches,'' SEL, accessed April 27, 2020,
    \url{https://searchenginelaw.net/security/103-us-judge-asks-reports-of-google-searches}.}
    Alguém que se passava por um homem chamado Douglas ligou para a Spire Credit Union e
    persuadiu o atendente a transferir US\$ 28.500 do dinheiro de Douglas para outro banco. O
    falso Douglas forneceu adequadamente o nome, a data de nascimento, o número do Seguro Social
    e uma cópia por fax de seu passaporte --- ou pelo menos um passaporte que tinha a foto de
    Douglas, que correspondia aos registros do banco e completou sua autenticação.
    
    Quando Douglas percebeu que seu dinheiro havia desaparecido, ele entrou em contato com as
    autoridades. O detetive David Lindman usou a pesquisa de imagens do Google para encontrar
    uma foto correspondente de Douglas online. Lindman pediu ao juiz do condado de Hennepin que
    emitisse um mandado de busca para o Google em busca de registros de qualquer pessoa em Edina
    que tivesse pesquisado pela imagem de Douglas durante um período de cinco semanas antes do
    incidente. Portanto, pelo menos em alguns casos, um mandado de busca pode ser emitido não
    contra um indivíduo específico, mas para o conjunto de indivíduos que realizaram um
    determinado tipo de pesquisa.
    
    O Google, o guardião, acaba tendo portões que se abrem em ambas as direções. Enquanto você
    faz uma pesquisa, ele está determinando quais informações mostrar a você e, ao mesmo tempo,
    está coletando informações sobre você. Pode usar essas informações para seus próprios fins
    publicitários e, sob ordem judicial, pode abrir o portão para terceiros.
\end{tcolorbox}

O princípio legal subjacente é simples: na ausência de qualquer legislação mais específica, se
as autoridades policiais perguntarem ao Google o que você procurou, ele pode revelar as
informações sobre você devido à ``Doutrina de Terceiros''. Você pode me contar um segredo, e o
governo não pode nos obrigar a divulgá-lo. Mas se você usar o Gmail --- essencialmente pedindo
ao Google para passar seu segredo para mim --- então o Google é um terceiro e não está vinculado
pela Quarta Emenda a respeitar seu desejo e o meu de manter a informação em segredo, assim como
se você compartilhasse seu segredo com um estranho na rua. O mesmo se aplica a outras
informações que você confiou ao Google --- por exemplo, os termos que você pediu a ele para
pesquisar. Essas pesquisas são propriedade do Google, não suas. Ele pode usá-las para gerar
publicidade direcionada e para outros fins.

Em 2017, o Google anunciou que não escanearia os e-mails dos usuários para melhorar o
direcionamento de anúncios, mas isso foi uma decisão de política, não uma resposta a qualquer
lei dos EUA. Na verdade, os Estados Unidos carecem de leis abrangentes de privacidade, e as
políticas corporativas não precisam ser imutáveis ou consistentes com as expectativas dos
usuários. Quando o Gmail foi lançado em 2004, o Google explicou sua prática de escanear e-mails
dos usuários como sendo útil para direcionar anúncios e compensar o custo de oferecer um
serviço gratuito. Uma década de críticas crescentes e alguns processos resultaram na decisão
do Google de parar de escanear e-mails. O que o Google não explicou na época foi que estava
permitindo que certos parceiros corporativos escaneassem e-mails --- e às vezes tivessem humanos
lendo-os.

Navideh Forghani de Phoenix, Arizona, parece não ter conhecimento dessas práticas quando se
cadastrou no Earny, um serviço de economia de dinheiro.\footnote{Douglas MacMillan, ``Tech’s
`Dirty Secret': The App Developers Sifting Through Your Gmail,'' \ingles{Wall Street Journal},
July 2, 2018,
\url{https://www.wsj.com/articles/techs-dirty-secret-the-app-developers-sifting-through-your-gmail-1530544442}.}
Earny verifica a caixa de entrada do cliente em busca de recibos de itens que ela comprou,
pesquisa na web para ver se consegue localizar os mesmos itens a preços mais baixos, registra
um pedido de reembolso com a empresa do cartão de crédito para a diferença de preço e depois
divide os lucros com o cliente, tudo discretamente nos bastidores. Depois que Navideh se
cadastrou, a única coisa que ela precisava fazer era observar os créditos aparecerem em suas
faturas de cartão de crédito.

Earny é uma empresa privada independente do Google, e o Google não está escaneando seu email.
Mas quando ela clicou em um botão para dar ao Earny acesso à sua caixa de entrada, ela estava
autorizando o Earny a fazer exatamente isso. Por sua vez, o Earny estava compartilhando seu
email com a Return Path, uma empresa com a qual o Earny havia feito parceria para fazer a
verificação real.

O Google, o Earny e a Return Path explicaram que não fizeram nada de errado, pois essas
práticas foram autorizadas pelas políticas de privacidade das empresas. Navideh reconhece que
não leu a política de privacidade do Earny\footnote{\url{https://hashsnap.zendesk.com/hc/en-us/articles/218609757-Privacy-Policy}.}.
Não é surpresa: ela tinha quase 3.000 palavras quando eu a verifiquei em 2019. A política de
privacidade do Earny é suficientemente densa para ser difícil para muitos leitores; e, para
piorar, ela vincula a várias outras políticas de privacidade, tornando quase impossível
entender o que você está renunciando quando se inscreve para o serviço deles. A política de
privacidade da Return Path tem quase 6.000 palavras.

A conclusão é que seus dados têm valor e cada conveniência que você aceita tem um preço. Como
afirmou Marc Rotenberg, presidente da EPIC, uma importante organização de defesa da privacidade,
``O modelo de política de privacidade está simplesmente quebrado além de qualquer conserto.
Simplesmente não há como os usuários do Gmail imaginarem que seus dados pessoais seriam
transferidos para terceiros.''\footnote{John D. McKinnon and Douglas MacMillan, ``Google Says
It Continues to Allow Apps to Scan Data from Gmail Accounts,'' \ingles{Wall Street Journal},
September 20, 2018,
\url{https://www.wsj.com/articles/google-says-it-continues-to-allow-apps-to-scan-data-from-gmail-accounts-1537459989}.}
Quando um produto tem tão pouca concorrência e é tão útil para a vida cotidiana e para a
condução dos negócios, o protocolo de ``aviso e consentimento'' não protege realisticamente
os usuários contra o uso de seus dados de maneiras inesperadas.

\subsection{Como o Google ficou tão grande?}
\label{gatekeepers:como-o-google}

Conforme a Web cresceu no início dos anos 1990, as estruturas hierárquicas, nunca satisfatórias
para encontrar informações não classificáveis, rapidamente não conseguiram acompanhar o tamanho
da Web. Vários motores de busca baseados em um índice automaticamente construído começaram a
aparecer, e alguns tiveram algum sucesso. Mas muito rapidamente o Google se tornou dominante, a
ponto de o nome do mecanismo de busca e da empresa se tornar um verbo sinônimo de ``busca na web''.

Em 1996, os fundadores do Google, Larry Brin e Sergey Brin, tiveram uma ideia brilhante enquanto
estavam na pós-graduação. Uma página da web importante é aquela referenciada --- ou seja,
vinculada --- por muitas páginas importantes. Isso parece uma definição circular, mas se toda a
estrutura da Web pode ser capturada e analisada, algumas matemáticas bastante simples podem ser
usadas para obter uma medida consistente da importância de cada página da web. Essa matemática,
juntamente com alguma engenharia sólida para organizar e processar todos os dados no
armazenamento limitado disponível na época, fez a empresa decolar. Sua interface
consideravelmente simples --- basta digitar algo e receber respostas, sem opções, sinos ou
assobios --- tranquilizou até mesmo os usuários mais ingênuos e os atraiu para usá-lo ainda mais.

O mecanismo de busca do Google era bom, mas não era dez vezes melhor do que os outros disponíveis
quando a empresa foi fundada em 1998. Por exemplo, nessa época, o AltaVista estava em operação há
três anos\footnote{Peter H. Lewis, ``Digital Equipment Offers Web Browsers Its `Super Spider',''
\ingles{The New York Times}, December 18, 1995,
\url{https://www.nytimes.com/1995/12/18/business/digital-equipment-offers-web-browsers-its-super-spider.html}.}
e processava centenas de milhões de consultas de pesquisa como um serviço gratuito para o público.

A Digital Equipment Corporation, que desenvolveu o AltaVista, nunca descobriu como torná-lo
lucrativo. A Digital era principalmente uma empresa de hardware e vendeu o AltaVista para outra
empresa. (A própria Digital foi comprada pela Compaq logo depois.) O AltaVista mudou de mãos
novamente e finalmente foi silenciosamente desativado. A Microsoft não lançou seu mecanismo de
busca Bing até 2009. Até então, o Google tinha uma vantagem aparentemente insuperável, apesar do
fato de os usuários poderem trocar de mecanismo de busca do Google para o Bing com um esforço
mínimo.

O Google ganhou sua vantagem ao veicular publicidade desde o início. Os anúncios são gerados em
resposta aos termos de pesquisa; pesquise ``celulares'' e você provavelmente verá anúncios de
produtos e serviços relacionados a celulares. Quais anúncios aparecem, entre todos os anunciantes
que desejam promover seus produtos para pessoas interessadas em celulares, são determinados por
um leilão contínuo. Anunciantes dispostos a pagar mais por seus anúncios têm mais chances de
vê-los aparecer. Os leilões ocorrem automaticamente e invisivelmente, e o resultado é um sistema
de eficiência sem precedentes. Um anunciante em um jornal, revista ou estação de rádio precisa
esperar que entre a massa indiferenciada de pessoas expostas ao anúncio, algumas se interessem
pelo produto anunciado. Os anunciantes podem tentar inclinar as probabilidades a seu favor,
colocando anúncios relacionados a esportes em estações de rádio que atendem a fãs de esportes,
por exemplo. Mas associar a publicidade à pesquisa direciona os anúncios apenas para aquelas
pessoas que mostraram pelo menos interesse suficiente em um tópico para pesquisá-lo.

Os próprios fundadores do Google reconheceram\footnote{Sergey Brin and Lawrence Page, ``The
Anatomy of a Large-Scale Hypertextual Web Search Engine,'' \ingles{Computer Networks and ISDN
Systems} 30, no. 1–7 (April 1998): 107–117,
\url{https://snap.stanford.edu/class/cs224w-readings/Brin98Anatomy.pdf}.} o lado negativo de
misturar publicidade com pesquisa, o que já estava sendo feito por alguns dos outros motores
de busca em uso na época. Isso diminuiria a confiança nos resultados da pesquisa, por exemplo,
se os usuários suspeitassem que os resultados da pesquisa em si estivessem enviesados para
favorecer os anunciantes:

\begin{quote}
    Por exemplo, em nosso motor de busca protótipo, um dos principais resultados para ``celular''
    é ``O Efeito do Uso de Celular na Atenção do Motorista'', um estudo que explica em grande
    detalhe as distrações e riscos associados a conversar ao celular enquanto dirige. Este
    resultado de pesquisa apareceu primeiro devido à sua alta importância, conforme julgado
    pelo algoritmo PageRank, uma aproximação da importância das citações na web. É evidente que
    um mecanismo de busca que estivesse recebendo dinheiro para exibir anúncios de celulares
    teria dificuldade em justificar a página que nosso sistema retornou aos seus anunciantes
    pagantes.
\end{quote}

Depois de mencionar alguns outros exemplos de conflitos de interesse entre retornar resultados
de pesquisa úteis e obter receitas publicitárias, Page e Brin concluíram: ``acreditamos que a
questão da publicidade cria incentivos mistos suficientes para que seja crucial ter um mecanismo
de busca competitivo que seja transparente e no âmbito acadêmico''. Seja como for, nenhum
mecanismo de busca desse tipo é amplamente usado hoje em dia. As enormes receitas do Google
derivam em grande parte desse tipo de publicidade, e em 2017 a União Europeia multou o Google
em 2,4 bilhões de euros por favorecer seus anunciantes nos resultados de pesquisa.\footnote{Mark
Scott, ``Google Fined Record \$2.7 Billion in E.U. Antitrust Ruling,'' \ingles{The New York Times},
June 27, 2017, \url{https://www.nytimes.com/2017/06/27/technology/eu-google-fine.html}.} E sem
saber exatamente o que está acontecendo dentro do código do Google, é difícil saber se os
resultados estão sendo enviesados. Brin e Page também anteciparam isso: ``Por exemplo, um mecanismo
de busca poderia adicionar um pequeno fator aos resultados de pesquisa de empresas `amigas' e
subtrair um fator dos resultados de concorrentes. Esse tipo de viés é muito difícil de detectar,
mas ainda pode ter um efeito significativo no mercado.'' Como os interesses conflitantes e a
falta de transparência serão resolvidos permanece desconhecido, mas as apostas são extremamente
altas.

\section{Guardiões Sociais: Conhecidos pela Companhia que Mantêm}
\label{gatekeepers:guardioes-sociais}

Quando foi criada, a Internet era um meio de conectar um computador a outro e, em última
análise, uma rede de computadores a outra (daí o nome ``Internet''). Ela evoluiu de conectar
máquinas para conectar usuários à informação. A complexidade dessas conexões levou ao papel do
guardião de busca. Em sua fase mais recente de conectividade, a Internet facilitou a conexão das
pessoas umas com as outras em um nível e com implicações que nem mesmo os criadores das soluções
dominantes imaginavam.

\subsection{A Rede Social: Facebook e Mais}
\label{gatekeepers:a-rede-social}

A explosão digital nunca foi tão poderosa quanto no crescimento do Facebook. Como foi
glamorizado no filme ``A Rede Social'', o sucesso do Facebook pareceria ser resultado da sorte
de adolescentes e da crueldade capitalista. A história completa é mais interessante e mais
reveladora em termos das maneiras como as pessoas compartilham informações.

Houve redes sociais online antes do Facebook. A mais antiga foi o Sixdegrees.com em 1997. O nome
foi derivado da peça e filme do início dos anos 1990 ``Seis Graus de Separação''. O site cresceu
para incluir milhões de usuários, mas eventualmente parou e encerrou em 2000 devido à falta de um
modelo de negócios sustentável e porque não havia muito para as pessoas fazerem nele depois de se
conectarem umas com as outras.\footnote{Danah M Boyd and Nicole B. Ellison, ``Social Network Sites:
Definition, History, and Scholarship,'' \ingles{Journal of Computer-Mediated Communication} 13,
no. 1 (October 1, 2007): 210–30, \url{https://doi.org/10.1111/j.1083-6101.2007.00393.x}.}

O Friendster foi lançado em 2002 e rapidamente se tornou um dos sites mais populares na Web.
Inicialmente, se apresentava como um lugar onde os usuários poderiam conhecer novas pessoas para
namorar, fazer novos amigos ou ajudar amigos a conhecer novas pessoas.\footnote{``Friendster,''
June 11, 2004, \url{https://web.archive.org/web/20040611192459}.} Em seu auge, ele permitia uma
busca fácil em todo o banco de dados de membros, que continha dezenas de milhões de pessoas.

%%%%%%%%%%% AQUI FICA UMA FIGURA, PÁGINA 123 %%%%
FIGURA\\
%%%%%%%%%%% AQUI FICA UMA FIGURA %%%%%%%%%%%%%%%

O Friendster entrou em colapso em 2006, vítima de seu próprio sucesso. O crescimento foi tão
explosivo que o site foi atormentado por problemas técnicos. As pessoas têm uma tolerância muito
limitada para esperar páginas carregarem; eles abandonarão completamente um site se ele não
funcionar e não tiverem uma forte razão para continuar tentando. O Friendster também fez muito de
seu principal mérito, que era facilitar a conexão com pessoas que você ainda não conhecia, mas
que poderia ter interesse em conhecer. Além disso, surgiram problemas sociais não previstos porque
era muito fácil ver os perfis de outros usuários. Por exemplo, descobriu-se que nem tudo o que os
usuários colocavam em seus perfis para estimular conexões sociais era algo que eles queriam que
seus chefes soubessem.

O Myspace foi lançado em 2003 como concorrente do Friendster. Ele atraiu usuários do Friendster
que estavam insatisfeitos ou foram expulsos do site. O Myspace tinha um grupo leal de bandas de
rock indie e seus seguidores que se recusaram a seguir as regras do Friendster. Logo, o Myspace
tinha mais visitantes do que o Google ou qualquer outro site, embora culturalmente ele mantivesse
a sensação travessa e criativa de suas origens rebeldes. No entanto, em poucos anos, os usuários
o abandonaram em massa após uma série de encontros altamente divulgados entre adultos e crianças
que se encontraram online. O pânico moral levou o governo dos EUA a considerar a legislação para
controlar as redes sociais online (veja o quadro).\footnote{Pete Cashmore, ``MySpace, America’s
Number One,'' Mashable, July 11, 2006, \url{https://mashable.com/2006/07/11/myspace-americas-number-one/}.}

\begin{tcolorbox}[[%
    enhanced,
    breakable,
    skin first=enhanced,
    skin middle=enhanced,
    skin last=enhanced,
    title={Você Sabe Onde Seu Filho Está na Internet Esta Noite?}
    ]
    Foi o pior pesadelo de todos os pais. Katherine Lester, uma estudante de honra de 16 anos
    de Fairgrove, Michigan, desapareceu em junho de 2006. Seus pais não tinham ideia do que
    tinha acontecido com ela; ela nunca lhes tinha dado um momento de preocupação. Eles chamaram
    a polícia. Em seguida, as autoridades federais se envolveram.
    
    Após três dias de ausência aterrorizante, ela foi encontrada, em segurança, em Amã, na
    Jordânia.
    
    Fairgrove é pequena demais para ter uma agência dos correios, e os Lesters moravam na última
    casa de uma rua sem saída. Em outro momento, a escola de Katherine, a 6 milhas de distância,
    poderia ter sido o limite máximo de seu universo. Mas, através da Internet, seu universo era
    o mundo inteiro. Katherine conheceu um homem palestino, Abdullah Jinzawi, de Jericó, na
    Cisjordânia. Ela encontrou o perfil dele na rede social Myspace e enviou-lhe uma mensagem:
    ``vc é fofo''. Eles rapidamente aprenderam tudo sobre o outro por meio de mensagens online.
    Lester enganou sua mãe para conseguir um passaporte e depois partiu para o Oriente Médio.
    Quando as autoridades dos EUA a encontraram em Amã, ela concordou em voltar para casa e pediu
    desculpas a seus pais pelo sofrimento que causou a eles.
    
    Um mês depois, a deputada Judy Biggert, de Illinois, levantou-se na Câmara para co-patrocinar
    a Lei de Exclusão de Predadores Online (DOPA). ``O MySpace.com e outros sites de redes
    sociais tornaram-se novos locais de caça para predadores de crianças'', disse ela, observando
    que ``todos nós ficamos horrorizados'' com a história de Katherine Lester. ``Pelo menos, vamos
    dar aos pais algum conforto de que seus filhos não cairão nas mãos de predadores enquanto
    usam a Internet em escolas e bibliotecas que recebem financiamento federal para serviços de
    Internet.'' A lei exigiria que essas instituições impedissem que as crianças usassem
    computadores no local para acessar salas de bate-papo e sites de redes sociais sem supervisão
    de adultos.
    
    Orador após orador levantou-se na Câmara para enfatizar a importância de proteger as crianças
    dos predadores online, mas nem todos apoiaram o projeto de lei. A linguagem era
    ``excessivamente ampla e ambígua'', disse um deles. Como originalmente redigida, parecia
    abranger não apenas o Myspace, mas também sites como Amazon e Wikipedia. Esses sites possuem
    algumas das mesmas características do Myspace: os usuários podem criar perfis pessoais e
    compartilhar continuamente informações uns com os outros por meio da Web. Embora a lei
    pudesse impedir que crianças em escolas e bibliotecas acessassem ``locais'' onde encontram
    amigos (e às vezes predadores), também impediria o acesso a enciclopédias online e
    livrarias, que dependem do conteúdo postado pelos usuários.
    
    Em vez de dedicar tempo para desenvolver uma definição mais precisa do que exatamente deveria
    ser proibido, os patrocinadores do DOPA rapidamente reformularam a lei para omitir a
    definição, deixando para a Comissão Federal de Comunicações decidir posteriormente o que a
    lei abrangeria. Alguns murmuravam que as próximas eleições de meio de mandato estavam
    motivando os patrocinadores a apresentar um esforço mal pensado e vistoso para proteger as
    crianças --- um esforço que provavelmente seria ineficaz e tão vago a ponto de ser
    inconstitucional.
    
    As crianças usam computadores em muitos lugares; restringir o que acontece nas escolas e
    bibliotecas dificilmente dissuadiria adolescentes determinados de acessar o Myspace. Apenas
    os pais mais controladores poderiam responder honestamente à pergunta feita pelo USA Today
    em seu artigo sobre ``ciberpredadores'': ``São 23h. Você sabe onde seu filho está na internet
    esta noite?''
    
    As estatísticas sobre o que pode dar errado eram certamente assustadoras. O Departamento de
    Justiça fez milhares de prisões por ``entendimento virtual'' --- quase sempre homens mais
    velhos usando sites de redes sociais para atrair adolescentes para encontros, alguns dos
    quais terminam muito mal. No entanto, como afirmou a \ingles{American Library Association}
    em oposição ao DOPA, a educação, não a proibição, é a ``chave para o uso seguro da
    Internet''. Os estudantes precisam aprender a cooperar online porque o uso da internet e
    todas as interações humanas que ele permite são ferramentas básicas do novo mundo
    globalmente interconectado dos negócios, educação e cidadania --- e talvez até do mundo
    globalmente interconectado do amor verdadeiro.
    
    A história de Katherine Lester tomou um rumo inesperado. Desde o momento em que ela foi
    encontrada na Jordânia, Lester afirmou consistentemente que pretendia se casar com Jinzawi.
    Jinzawi, que tinha 20 anos quando ele e Lester fizeram contato pela primeira vez, afirmou
    estar apaixonado por ela --- e sua mãe também a adorava. Jinzawi implorou para que Lester
    contasse a verdade a seus pais antes de ir encontrá-lo, mas ela se recusou. Após seu retorno,
    as autoridades acusaram Lester de ser uma criança fugitiva e confiscaram seu passaporte. No
    entanto, em 12 de setembro de 2007, ao atingir a maioridade legal ao completar 18 anos, ela
    embarcou novamente em um avião para o Oriente Médio, finalmente para encontrar seu amado
    pessoalmente. O caso finalmente terminou algumas semanas depois em uma troca de acusações e
    negações, bem como uma insinuação de que uma terceira pessoa havia atraído a atenção de
    Lester. Não houve drama de alta tecnologia na separação --- exceto pelo fato de que foi
    televisionada no programa Dr. Phil.
\end{tcolorbox}

Este foi o ambiente no qual o Facebook foi lançado em um quarto de dormitório de Harvard em
2004. Zuckerberg estava estudando sociologia, psicologia e redes de computadores, e ele criou
rapidamente um site simples que ele propôs chamar de ``Seis Graus para Harry Lewis''. Como ele
escreveu para Lewis,

\begin{quote}
    Professor, tenho interesse em teoria dos grafos e suas aplicações em redes sociais há algum
    tempo, então fiz algumas pesquisas\ldots que têm a ver com a ligação entre pessoas através
    dos artigos em que aparecem no Crimson. Pensei que as pessoas achariam isso interessante,
    então criei um site preliminar que permite que as pessoas encontrem a conexão (através de
    pessoas e artigos) de qualquer pessoa com a pessoa mais mencionada no período de tempo que
    eu examinei. Essa pessoa é você. Eu gostaria de pedir sua permissão para colocar este site
    no ar, pois ele tem o seu nome em seu título.
\end{quote}

Lewis hesitou brevemente. ``Posso vê-lo antes de dizer sim? Tudo é informação pública, mas de
alguma forma, há um ponto em que a agregação de informações públicas parece uma invasão de
privacidade.'' Pouco depois, pensando que o projeto parecia educacional, Lewis respondeu: ``Claro,
por que não. Parece inofensivo.'' O thefacebook, como era originalmente conhecido, foi lançado
uma semana depois.

Em menos de dois anos, o Facebook havia superado o Myspace. Isso foi alcançado por meio de uma
combinação de boas decisões:

\begin{itemize}
    \item Boa engenharia (geralmente era confiável mesmo em sua fase de crescimento mais explosivo).
    \item Design que equilibrava, de forma mais bem-sucedida do que seus concorrentes, as
        imperativas opostas de conectar o mundo e proporcionar espaços para conversas mais íntimas
        entre pessoas semelhantes.
    \item Uma interface mais tranquila e padronizada do que a do Myspace, talvez refletindo que o
        Facebook havia começado como uma comunidade de estudantes universitários, em vez de amantes de
        música indie.
    \item Um modelo de publicidade em grande parte palatável para seus usuários (em parte porque
        eles não tinham conhecimento da extensão em que seus dados estavam sendo reutilizados).
    \item Um grande número de aquisições estratégicas que tiveram o efeito combinado de fazer do
        Facebook uma plataforma única não apenas para redes sociais, mas também para mensagens de
        texto, busca de vídeos, armazenamento e visualização de fotos, compras e jogos, entre outras
        coisas.
\end{itemize}

O resultado foi um crescimento surpreendente.\footnote{``Company Info,'' \ingles{About Facebook},
accessed April 28, 2020, \url{https://about.fb.com/company-info/}.} O Facebook foi lançado em 4
de fevereiro de 2004, como um site exclusivo para estudantes de Harvard, substituindo os ``livros
de rosto'' impressos pelas residências estudantis de Harvard para familiarizar os alunos entre si.
Um mês depois, a rede foi estendida para Stanford, Yale e Columbia, e até o final do ano tinha
mais de 1 milhão de usuários. O site pode ter se tornado mais popular porque começou com uma aura
de exclusividade. Durante 2005, adicionou centenas de outras faculdades, bem como escolas
secundárias, e até o final de 2006 qualquer pessoa podia se juntar, e a base de usuários era de
mais de 12 milhões. Um ano depois, eram 60 milhões, e atingiu 500 milhões até meados de 2010. No
momento desta escrita, o Facebook afirma ter 1,59 bilhão de usuários diários, com 2,41 bilhões que
o utilizam pelo menos uma vez por mês. Isso representa cerca de um terço da população da Terra,
incluindo bebês. O número ainda está aumentando, apesar da publicidade negativa devido ao uso
indevido dos dados da empresa.

De fato, o Facebook possui muitos dados sobre seus usuários e aprendeu muito ``no trabalho'' sobre
como lidar com eles. Consciente, desde cedo, de que a privacidade seria importante para os usuários,
afirmou inequivocamente em 2007: ``Não usamos e não usaremos cookies para coletar informações
privadas de nenhum usuário''.\footnote{thefacebook, ``Privacy Policy,'' January 7, 2005,
\url{https://web.archive.org/web/20050107221705/}.} No entanto, apenas alguns meses depois, o
usuário do Facebook Sean Lane comprou um anel de eternidade de diamante online, e sua esposa ficou
sabendo instantaneamente, pelo Facebook. O Facebook havia lançado recentemente um novo recurso
chamado \ingles{Beacon}. Na tentativa de manter os amigos do Facebook atualizados sobre o que os
usuários estavam fazendo, e também expandir as oportunidades de publicidade no Facebook, o Beacon
postava informações sobre o que os usuários estavam comprando em sites externos ao Facebook nos
feeds de notícias de seus amigos. A esposa de Lane não apenas ficou sabendo do anel que ele havia
comprado, mas também que ele havia conseguido um desconto de 51\% nele. ``Para quem é o anel?'',
ela perguntou.\footnote{Bill Goodwin and Sebastian Klovig Skelton, ``Facebook’s Privacy Game ---
How Zuckerberg Backtracked on Promises to Protect Personal Data,'' \url{ComputerWeekly.com}, July
1, 2019,
\url{https://www.computerweekly.com/feature/Facebooks-privacy-U-turn-how-Zuckerberg-backtracked-on-promises-to-protect-personal-data}.}
Era para ela, então apenas a surpresa foi estragada, e não o casamento!

O Facebook havia feito parceria com outros sites em um esquema de compartilhamento de informações.
Quando os usuários faziam uma compra em um site parceiro, o Facebook era informado e, às vezes,
inseria a informação nos feeds de notícias dos amigos. Os usuários podiam optar por não participar
--- se notassem o pequeno quadrado no site do parceiro e entendessem o que estavam sendo convidados
a não participar. Dezenas de milhares de usuários ficaram furiosos e pediram à empresa que
removesse o recurso. As coisas pioraram quando um pesquisador descobriu que, em certas
circunstâncias, as informações eram enviadas ao Facebook mesmo quando o usuário estava desconectado
e nenhuma opção de desativação estava sendo exibida. As negações por parte dos porta-vozes da
empresa acabaram sendo imprecisas. Processos judiciais foram iniciados. Zuckerberg primeiro se
desculpou,\footnote{Facebook, ``Thoughts on Beacon,'' December 5, 2007,
\url{https://www.facebook.com/notes/facebook/thoughts-on-beacon/7584397130/}.} teve que pagar
milhões de dólares para resolver as ações judiciais e depois desativou completamente o recurso
Beacon.

Mas a penalização financeira não evitou novos deslizes de privacidade. No final de 2009, quando a
rede havia crescido para 350 milhões de usuários, sua política de privacidade foi atualizada sem
aviso prévio. Envolto em um anúncio\footnote{``Facebook Asks More Than 350 Million Users Around the
World to Personalize Their Privacy,'' \ingles{About Facebook}, December 10, 2009,
\url{https://about.fb.com/news/2009/12/facebook-asks-more-than-350-million-users-around-the-world-to-personalize-their-privacy/}.}
que destacava que os usuários agora seriam incentivados a ``personalizar sua privacidade'', a
empresa observou que as configurações padrão de privacidade haviam mudado. Anteriormente, apenas
o nome do usuário e sua ``rede'' eram visíveis para o mundo exterior. (``Redes'' eram vestígios
dos dias em que apenas membros de certos grupos podiam ingressar no Facebook --- a rede de um
usuário era sua faculdade ou escola secundária, por exemplo.) De acordo com a nova política,
\footnote{Kurt Opsahl, ``Facebook’s Eroding Privacy Policy: A Timeline,'' Electronic Frontier
Foundation, April 28, 2010, \url{https://www.eff.org/deeplinks/2010/04/facebook-timeline}.}

\begin{quote}
    Determinadas categorias de informações, como seu nome, foto de perfil, lista de amigos e
    páginas das quais você é fã, gênero, região geográfica e redes às quais você pertence, são
    consideradas publicamente disponíveis para todos, incluindo aplicativos aprimorados pelo
    Facebook, e, portanto, não possuem configurações de privacidade. No entanto, você pode limitar
    a capacidade de outras pessoas de encontrar essas informações por meio das configurações de
    privacidade de pesquisa.
\end{quote}

O anúncio observou que a maioria das pessoas torna essas informações públicas de qualquer maneira.
Talvez seja verdade, mas havia uma grande diferença entre o que algumas pessoas escolhiam fazer e
o que outras esperavam. Em pouco tempo, ficou evidente o quão reveladoras as listas de amigos e as
páginas de fãs poderiam ser. Pesquisadores do MIT, por exemplo, descobriram que não era difícil
descobrir, com alta precisão, quem era gay, mesmo na ausência de qualquer informação explícita sobre
orientação sexual:

\begin{quote}
    Informações públicas sobre colegas de trabalho, amigos, família e conhecidos, bem como suas
    associações com eles, implicitamente revelam informações privadas\ldots Nossa pesquisa
    demonstra um método para prever com precisão a orientação sexual de usuários do Facebook por
    meio da análise das associações de amizade. Após analisar 4.080 perfis do Facebook da rede do
    MIT, determinamos que a porcentagem de amigos de um determinado usuário que se autodeclaram
    homossexuais masculinos está fortemente correlacionada com a orientação sexual desse usuário,
    e desenvolvemos um classificador de regressão logística com forte poder preditivo.\footnote{Carter
    Jernigan and Behram F. T. Mistree, ``Gaydar: Facebook Friendships Expose Sexual Orientation,''
    \ingles{First Monday}, September 25, 2009,
    \url{https://firstmonday.org/ojs/index.php/fm/article/view/2611}.}
\end{quote}

A agregação de informações públicas suficientes de fato constitui uma invasão de privacidade, algo
que Zuckerberg parece não ter pensado profundamente. Após a mudança nas configurações padrão do
Facebook, ele descobriu que suas próprias fotografias haviam se tornado públicas e rapidamente
redefiniu suas configurações de privacidade.\footnote{Kashmir Hill, ``Either Mark Zuckerberg Got a
Whole Lot Less Private or Facebook’s CEO Doesn’t Understand the Company’s New Privacy Settings,''
\ingles{Forbes}, December 10, 2009,
\url{https://www.forbes.com/sites/kashmirhill/2009/12/10/either-mark-zuckerberg-got-a-whole-lot-less-private-or-facebooks-ceo-doesnt-understand-the-companys-new-privacy-settings/}.}
Outro executivo do Facebook sugeriu que os usuários que não desejassem que sua cidade natal fosse
tornada pública deveriam mentir sobre isso, aparentemente esquecendo que o Facebook exige que
essas informações sejam verdadeiras.\footnote{Julia Angwin, ``How Facebook Is Making Friending
Obsolete,'' \ingles{The Wall Street Journal}, December 15, 2009,
\url{https://www.wsj.com/articles/SB126084637203791583}.}

A reação foi intensa, não apenas dos usuários, mas também de autoridades governamentais e
organizações de privacidade. Em 27 de abril de 2010, Zuckerberg recebeu uma carta educada, porém
ameaçadora, de quatro senadores dos Estados Unidos que concluía:

\begin{quote}
    Estamos ansiosos para que a FTC examine este problema, mas, enquanto isso, acreditamos que o
    Facebook pode tomar medidas rápidas e produtivas para aliviar as preocupações de seus usuários.
    Fornecer mecanismos de aceitação em vez de esperar que os usuários passem por processos de
    recusa longos e complicados é um passo fundamental para manter a clareza e a transparência.
    \footnote{Politico Staff, ``Senators’ Letter to Facebook,'' Politico, April 27, 2010,
    \url{https://www.politico.com/news/stories/0410/36406.html}.}
\end{quote}

Em maio de 2010, o Facebook reverteu as configurações padrão, para que apenas o nome, foto,
gênero e redes fossem automaticamente públicos.\footnote{``Facebook Redesigns Privacy,''
\ingles{About Facebook}, May 26, 2010, \url{https://about.fb.com/news/2010/05/facebook-redesigns-privacy/}.}

Apesar das reclamações dos usuários e da sugestão de intervenção de uma agência federal, nos
poucos meses entre a mudança mal considerada das configurações de privacidade e a decisão de
retornar às suposições anteriores, o Facebook ganhou 50 milhões de usuários. As pessoas estavam
reclamando, mas achavam o Facebook muito útil para desistir. Cada novo membro da rede tornava-a
ainda mais valiosa para se juntar. As pessoas iam ao Facebook não para fazer novos amigos, mas
porque todos os seus amigos já estavam lá. Esse fenômeno é conhecido como efeito de rede: conforme
Paul Baran havia previsto, o valor para um indivíduo de estar na rede aumenta à medida que o
tamanho da rede como um todo aumenta. Após 2010, o Facebook não teve concorrência séria de redes
sociais nos Estados Unidos, embora em algumas partes do mundo, o Facebook nem mesmo fosse
permitido cadastrar usuários.

À medida que a maior parte dos Estados Unidos e um grande número de indivíduos não americanos
aderiram, o efeito de rede foi impulsionado pela diversificação do produto. O Facebook adicionou
um serviço de mensagens de texto em 2008 e adquiriu o Instagram em 2012 e o WhatsApp em 2014. O
Facebook se tornou uma plataforma completa para todos os tipos de comunicação --- boa, ruim e
fraudulenta. Em 2019, foi relatado que o Facebook era usado em 90\% das instâncias relatadas de
compartilhamento de pornografia infantil.\footnote{Jennifer Valentino-DeVries and Gabriel J. X. Dance,
``Facebook Encryption Eyed in Fight Against Online Child Sex Abuse,'' \ingles{The New York Times},
October 2, 2019, \url{https://www.nytimes.com/2019/10/02/technology/encryption-online-child-sex-abuse.html}.}
A maior parte dessa comunicação ocorria por meio do aplicativo Messenger do Facebook, e quando o
Facebook anunciou que adicionaria criptografia de ponta a ponta ao Messenger, o que tornaria
impossível para qualquer pessoa, exceto o destinatário, decifrar as comunicações do Messenger
durante o envio, o procurador-geral dos Estados Unidos e seus homólogos em outros países pediram
firmemente a Zuckerberg que não seguisse em frente.\footnote{Priti Patel et al., ``Open Letter to
Facebook,'' October 4, 2019, \url{https://www.justice.gov/opa/press-release/file/1207081/download}.}
A carta citou um dos milhares de exemplos de como as autoridades usaram a vigilância eletrônica
para pegar um criminoso:

\begin{quote}
    Para citar um exemplo, o Facebook enviou uma denúncia prioritária ao NCMEC, tendo identificado
    uma criança que havia enviado material de abuso sexual infantil autoproduzido a um homem
    adulto. O Facebook localizou vários bate-papos entre os dois que indicavam abuso sexual
    histórico e contínuo. Quando os investigadores conseguiram localizar e entrevistar a criança,
    ela relatou que o adulto havia abusado sexualmente dela centenas de vezes ao longo de quatro
    anos, começando quando ela tinha 11 anos. Ele também exigia regularmente que ela lhe enviasse
    imagens sexualmente explícitas de si mesma. O agressor, que ocupava um cargo de confiança com
    a criança, foi condenado a 18 anos de prisão. Sem as informações do Facebook, o abuso contra
    essa menina poderia continuar até hoje.
\end{quote}

A criptografia tornaria impossível pegar esses infratores. ``Portanto, pedimos ao Facebook e a
outras empresas'', continuava a carta, ``seja qual for a forma de criptografia que eles usem, que
permitam às autoridades o acesso legal ao conteúdo em um formato legível e utilizável.''

O Capítulo 5, ``Segredos Revelados'', traça a história da criptografia. A maioria dos especialistas
em privacidade e segurança concorda que permitir que as autoridades tenham acesso a comunicações
criptografadas aumenta muito o risco de que outros também consigam acessá-las. Mas uma questão
ainda maior está em jogo: o Facebook e o Google têm mais informações sobre a maioria dos seres
humanos e todas as formas de atividade humana do que qualquer governo. Os erros que eles cometem,
como vazamentos de dados e interrupções de serviço, por exemplo, podem afetar porções significativas
da população da Terra. No caso do Facebook, Mark Zuckerberg sozinho pode, em princípio, tomar as
decisões por si mesmo.

Como um veículo para muitas formas de comunicação, o Facebook não tem a obrigação de permitir todas
as formas legais de discurso, e dadas as proteções que as empresas de Internet desfrutam sob a
Seção 230 (consulte o Capítulo 7, ``Você Não Pode Dizer Isso na Internet'') do Ato de Decência nas
Comunicações, sua responsabilidade mesmo para formas ilegais de discurso é limitada nos Estados
Unidos. No entanto, eles reconhecem tanto um interesse comercial quanto um interesse político em
limitar algumas formas de discurso, mesmo nos Estados Unidos, e no exterior a obrigação de censurar
é muito mais explícita. Anúncios políticos fraudulentos podem ter afetado eleitores nas eleições
presidenciais dos EUA de 2016. Vários atiradores em massa estavam imersos em conteúdo violento nas
redes sociais.

No entanto, essas empresas podem, em grande parte, estabelecer políticas --- o que e se censurar, e
se e como criptografar mensagens, por exemplo --- como acharem adequado. Sem dúvida, eles tenderão
a tomar essas decisões com os interesses de seus acionistas em mente; na verdade, seria inadequado
para eles agirem de outra forma rotineiramente. Claro, esses interesses são melhor servidos ao
seguir políticas que o público geralmente aprovará. Mas o público em si tem interesse em intervir,
como sugere o Procurador Geral Barr, e como o senador Schumer sugeriu anteriormente?

E ter uma política não é o mesmo que executá-la perfeitamente. A revisão humana de cada comentário,
anúncio ou vídeo postado no Facebook é impossível; até mesmo responder de forma oportuna aos que os
usuários reclamam seria incrivelmente difícil. Inevitavelmente, os guardiões têm recorrido ao
software de inteligência artificial (IA) para fazer parte da triagem. No entanto, onde a intenção
e o contexto são importantes, a IA ainda não se equipara aos leitores humanos.

E se uma palavra ou frase em uma postagem violar a proibição de ``discurso de ódio'' do Facebook,
mas for retirada do contexto? Um algoritmo não pode fazer esse julgamento. Relutante em ser acusado
de viés político, o Facebook anunciou que geralmente não removerá a publicidade política, mesmo
quando for sabido que é flagrantemente falsa. Sua decisão, diz ele, ``se baseia na crença fundamental
do Facebook na liberdade de expressão, no respeito ao processo democrático e na crença de que, em
democracias maduras com imprensa livre, o discurso político é possivelmente o discurso mais
escrutinado que existe''.\footnote{Cecilia Kang, ``Facebook’s Hands-off Approach to Political
Speech Gets Impeachment Test,'' \ingles{The New York Times}, October 8, 2019,
\url{https://www.nytimes.com/2019/10/08/technology/facebook-trump-biden-ad.html}.} Isso é o que
melhor servirá ao interesse público? Não é surpreendente que nem todos concordem. O que fazer em
vez disso é muito menos claro.

Ou essas empresas são simplesmente muito grandes? Essa noção se tornou popular entre alguns dos
candidatos presidenciais durante o ciclo eleitoral de 2020. Elizabeth Warren acha que o Facebook
deveria ser dividido, talvez desfazendo algumas de suas aquisições. Se isso seria legal ou útil
será motivo de intensa discussão. Outra ideia seria deixá-los intactos, mas regulá-los mais
rigorosamente --- embora o diabo esteja nos detalhes. Não seria um pequeno passo por parte do
governo tratar uma empresa privada cujo produto são bits como se fosse um serviço público cujo
produto é água.

Uma parte do governo tem preocupações específicas sobre um aspecto específico dos serviços
oferecidos por empresas de tecnologia. Como resultado de uma das descobertas mais notáveis do
século XX --- apenas um pouco de aritmética sobre bits --- cidadãos comuns podem e realmente
trocam mensagens criptografadas pela Internet pública, que as autoridades podem interceptar, mas
não decodificar. Como isso aconteceu e o que isso pressagia é o assunto do próximo capítulo.