\documentclass{book}
\usepackage{url}
\newcommand{\ingles}[1]{\textit{#1}}

\begin{document}

\chapter[Quem é o Dono da Sua Privacidade?]{Quem é o Dono da Sua Privacidade?\\\large\textit{A Comercialização dos Dados Pessoais}}
\label{quem}

\section{Que tipo de vegetal você é?}
\label{quem:vegetal}
Não causou espanto quando Aleksandr Kogan ofereceu \ingles{``This Is Your Digital Life''}
como um aplicativo de questionário do Facebook. Aplicativos de questionário são
uma parte essencial do marketing no Facebook, atraindo usuários para participar
e, em seguida, colhendo dados de marketing. Esses aplicativos---que são atrativos,
sedutores e altamente eficazes---deram origem a uma sub indústria inteira de
ferramentas e especialistas em marketing de questionários.

Aproximadamente 270.000 usuários do Facebook instalaram o aplicativo de Kogan
e fizeram o teste de personalidade, dando ao aplicativo o acesso aos seus contatos
para convidá-los a fazer o mesmo. A motivação ostensiva de Kogan era a pesquisa
acadêmica---estudar como os emojis são usados para transmitir emoções. Mas o que
ele fez com todos os dados que coletou foi bastante diferente. Através do
aplicativo de Kogan, a empresa Cambridge Analytica colheu dados de mais de 50
milhões de pessoas. A Cambridge Analytica utilizou essas informações para ajudar
a campanha do candidato presidencial Donald Trump a direcionar públicos para    
publicidade digital e captação de recursos, modelar a participação dos eleitores,
identificar mercados para veicular anúncios de televisão e até mesmo planejar as
viagens de Trump. A Cambridge Analytica afirmou que seus ``perfis psicográficos''
ajudaram a identificar eleitores prováveis e os tipos de mensagens que os
persuadiriam a votar em Trump.\footnote{Matthew Rosenberg, Nicholas Confessore,
and Carole Cadwalladr, ``Firm That Assisted Trump Exploited Data of Millions'',
New York Times, March 18, 2018: A1,
\url{https://www.nytimes.com/2018/03/17/us/politics/cambridge-analytica-trump-campaign.html}.}

Mas como é que um quarto de milhão de pessoas baixando um aplicativo se transformou
em vazamento de dados de 50 milhões? Através do modelo de privacidade poroso dos
aplicativos do Facebook. Cada um dos 270.000 usuários que instalaram o aplicativo
estava conectado a uma média de 200 amigos. \ingles{``This Is Your Digital Life''}
baseava sua avaliação não tanto no questionário, mas sim na história das páginas
``curtidas''. O questionário era um pretexto para obter acesso às curtidas dos
usuários e aos de seus contatos. O Facebook permitiu essa coleta de dados em 
2015---embora afirme que Kogan violou os termos do programa ao compartilhar dados
de perfil com a Cambridge Analytica.

Sua privacidade não é sua própria. Mesmo que você tenha rejeitado o \ingles{``This
Is Your Digital Life''}, qualquer um dos seus amigos---ou os aplicativos que eles
instalaram---poderiam ter comprometido seus dados. Isso tem paralelos também no
mundo não digital, é claro. (Considere o velho ditado ``Duas pessoas podem guardar
um segredo se uma delas estiver morta.'') Mas offline, você pode ter melhores
intuições sobre isso. Você sabe que não deve compartilhar uma história com o
vizinho fofoqueiro até estar pronto para responder perguntas de estranhos no
supermercado. Online, levou muito tempo para as configurações de privacidade do
Facebook adquirirem controles de público simples e somente após o escândalo da
Cambridge Analytica é que a rede social deixou de permitir que os aplicativos
percorressem o grafo social, capturando a rede de conexões de amigos.

\subsection{Me deixe em paz}
\label{quem:me}
Há mais de um século, dois advogados alertaram sobre o impacto da tecnologia
e da mídia na privacidade pessoal: 

\begin{quote}
Fotografias instantâneas e a empresa jornalística invadiram os santuários da
vida privada e doméstica; e numerosos dispositivos mecânicos ameaçam concretizar
a previsão de que ``o que é sussurrado no armário será proclamado nos telhados.'' 
\end{quote}

Esta declaração é do artigo seminal da \ingles{Harvard Law Review} sobre privacidade,
publicado em 1890 pelo advogado de Boston, Samuel Warren, e seu parceiro de
advocacia, Louis Brandeis, que mais tarde se tornaria um juiz da Suprema Corte
dos Estados Unidos (onde, como vimos, ele discordou em defesa da privacidade
no caso \ingles{Olmstead v. U.S.}).\footnote{Samuel A. Warren and Louis D.
Brandeis, ``The Right to Privacy'', Harvard Law Review 4, no. 5 (December 15, 1890),
\url{https://groups.csail.mit.edu/mac/classes/6.805/articles/privacy/Privacy_brand_warr2.html}.}
Warren e Brandeis continuaram dizendo: 

\begin{quote}
A fofoca já não é apenas recurso dos ociosos e dos viciosos, mas se tornou
um negócio, que é exercido com diligência e descaramento. Para satisfazer um gosto
lascivo, os detalhes das relações sexuais são espalhados amplamente nas colunas
dos jornais diários. Para ocupar os ociosos, coluna após coluna é preenchida com
fofocas vazias, que só podem ser obtidas invadindo o círculo doméstico.
\end{quote}

Novas tecnologias facilitaram a produção desse lixo, e então a oferta criou
a demanda. E aquelas fotografias sinceras e colunas de fofoca não eram apenas
de mau gosto; eram ruins. Soando como críticos de programas de TV
sem sentido, Warren e Brandeis indignaram-se ao afirmar que a sociedade 
estava indo de mal a pior por conta de todas aquelas coisas que estavam
sendo espalhadas:

\begin{quote}
Mesmo a fofoca aparentemente inofensiva, quando amplamente e persistentemente
divulgada, tem o poder de causar o mal. Ela menospreza e perverte. Menospreza
ao inverter a importância relativa das coisas, diminuindo assim os pensamentos
e aspirações de um povo. Quando fofocas pessoais alcançam a dignidade de serem
publicadas e ocupam o espaço disponível para assuntos de real interesse para a
comunidade, não é surpreendente que os ignorantes e desavisados confundam sua
importância relativa. Fácil de compreender, apelando para o lado fraco da
natureza humana que nunca será totalmente abalado pelas desgraças e fraquezas de
nossos vizinhos, ninguém pode se surpreender que ela usurpe o lugar de interesse
em mentes capazes de outras coisas. A trivialidade destrói imediatamente a
robustez do pensamento e a delicadeza do sentimento. Nenhum entusiasmo pode
florescer, nenhum impulso generoso pode sobreviver sob sua influência arrasadora.
\end{quote}

O problema percebido por Warren e Brandeis era que era difícil dizer exatamente
por que essas invasões de privacidade deveriam ser ilegais. Em casos individuais,
você poderia dizer algo sensato, mas as decisões legais individuais não faziam
parte de um regime geral. Os tribunais certamente aplicariam sanções legais
por difamação---publicar fofocas maliciosas que eram falsas---mas e quanto às
fofocas maliciosas que eram verdadeiras? Alguns tribunais impuseram
penalidades por publicar as cartas pessoais de um indivíduo---mas com base
na lei de propriedade, como se a posse de alguém tivesse sido roubada ao invés
das palavras em suas cartas.
Não, eles concluíram, tais argumentos não chegavam à essência da questão. Quando
algo privado é publicado sobre você, algo lhe foi tirado, você é vítima de 
roubo---mas a coisa roubada de você faz parte de sua identidade como pessoa. Na
verdade, a privacidade era um direito, afirmaram eles, um ``direito geral do 
indivíduo de ser deixado em paz''. Esse direito há muito tempo estava presente nas 
decisões judiciais, mas as novas tecnologias haviam trazido essa questão à tona. Ao
articular esse novo direito, Warren e Brandeis estavam, segundo eles afirmaram,
fundamentando-o no princípio de ``integridade da pessoa inviolável'', a santidade
da identidade individual.

\subsection{Privacidade e liberdade}
\label{quem:privacidade}

A articulação de Warren--Brandeis sobre a privacidade como um direito de ser
deixado em paz foi influente, mas nunca foi realmente completa. Ao longo do
século XX, havia simplesmente muitas boas razões para \emph{não} deixar as pessoas
em paz, e também havia muitas maneiras pelas quais as pessoas \emph{preferiam} não
ser deixadas em paz. E nos Estados Unidos, os direitos da Primeira Emenda
entravam em tensão com os direitos de privacidade. Como regra geral, o governo
não pode me impedir de dizer qualquer coisa que seja verdadeira. Em particular,
geralmente não pode me impedir de dizer o que descobri legalmente sobre seus
assuntos privados. No entanto, a definição de Warren--Brandeis funcionou bem
o suficiente por muito tempo, porque, como Robert Fano colocou, ``O ritmo do
progresso tecnológico foi por muito tempo suficientemente lento para permitir
que a sociedade aprendesse pragmaticamente como explorar novas tecnologias e
evitar seu abuso, mantendo seu equilíbrio na maior parte do tempo''.\footnote{Robert
Fano, ``Review of Alan Westin’s Privacy and Freedom'', Scientific American
(May 1968): 148–152.} Até o final da década de 1950, as emergentes tecnologias
eletrônicas, tanto computadores quanto comunicação, destruíram esse equilíbrio.
A sociedade não podia mais se ajustar pragmaticamente porque as tecnologias de
vigilância estavam se desenvolvendo muito rapidamente.

O resultado foi um estudo importante sobre privacidade realizado pela Associação
dos Advogados da Cidade de Nova York, que culminou na publicação, em 1967, do
livro de Alan Westin, intitulado \ingles{``Privacy and Freedom''}.\footnote{
Alan F. Westin, Privacy and Freedom (Atheneum, 1967).} (Fano estava
revisando o livro de Westin quando descreveu o quadro de desequilíbrio social
causado pela rápida mudança tecnológica.) Westin propôs uma mudança crucial de foco.

Brandeis e Warren haviam visto a perda de privacidade como uma forma de lesão
pessoal, que poderia ser tão grave a ponto de causar ``dor mental e angústia,
muito maiores do que poderiam ser infligidas por mera lesão corporal''. Os
indivíduos precisavam assumir a responsabilidade de se proteger. ``Cada pessoa
é responsável apenas por seus próprios atos e omissões''. Mas a lei tinha que
fornecer as armas para resistir a invasões de privacidade.

Westin reconheceu que a formulação de Brandeis--Warren era muito absoluta, diante
dos direitos de expressão de outras pessoas e das práticas legítimas de coleta
de dados da sociedade. A proteção poderia não vir de escudos protetores, mas
do controle sobre os usos que poderiam ser feitos das informações pessoais.
``Privacidade'', escreveu Westin, ``é o direito de indivíduos, grupos ou instituições
determinarem por si mesmos quando, como e em que medida informações sobre eles
são comunicadas a outros.'' Westin propôs:

\begin{quote}
\ldots O que é necessário é um processo estruturado e racional de ponderação,
com critérios definidos, que as autoridades públicas e privadas possam aplicar
ao comparar o pedido de vigilância ou divulgação por meio de novos
dispositivos, com a pedido de privacidade. Os seguintes passos são
sugeridos como a base desse processo: avaliar a gravidade da necessidade de
conduzir a vigilância; decidir se existem métodos alternativos para atender
à necessidade; decidir qual grau de confiabilidade será exigido do instrumento
de vigilância; determinar se foi concedido um verdadeiro consentimento para a
vigilância; e avaliar a capacidade de limitação e controle da vigilância se
for permitida.\footnote{Ibid.}  
\end{quote}

Portanto, mesmo que houvesse uma razão legítima pela qual o governo ou outra
parte pudesse saber algo sobre você, seu direito à privacidade poderia limitar
o que a parte informada poderia fazer com essa informação.

Essa compreensão mais sutil da privacidade surgiu dos importantes papéis sociais
que a privacidade desempenha. Privacidade não é, como Warren e Brandeis afirmaram,
o direito de ser isolado da sociedade; privacidade é um direito que faz a
sociedade funcionar.

Fano mencionou três papéis sociais da privacidade. Primeiro, ``o direito de manter
a privacidade de sua personalidade pode ser considerado parte do direito de
autodefesa''---o direito de guardar para si mesmo os julgamentos equivocados da
adolescência e os conflitos pessoais, desde que não tenham uma importância
duradoura para sua posição final na sociedade. Segundo, a privacidade é a forma
como a sociedade permite desvios das normas sociais predominantes, considerando
que nenhum conjunto de normas sociais é universal e permanentemente satisfatório---e, 
de fato, dado que o progresso social requer experimentação social. E terceiro,
a privacidade é essencial para o desenvolvimento do pensamento independente; ela
permite que o indivíduo se desvincule da sociedade de forma que os pensamentos
possam ser compartilhados em círculos limitados e ensaiados antes de serem
expostos publicamente.

A filósofa Helen Nissenbaum também fundamenta a privacidade no ser social,
descrevendo-a como ``integridade contextual''.\footnote{Helen Nissenbaum, Privacy
in Context: Technology, Policy, and the Integrity of Social Life (Stanford Law
Books, 2009).} A privacidade depende de uma correspondência entre o fluxo de dados
e as expectativas e normas do ambiente em que as informações foram geradas e
compartilhadas. Quando o Facebook o convida a adicionar seu terapeuta ou um colega
paciente como amigo, isso representa uma violação de contexto. Os espaços online
oferecem a oportunidade de multiplicar contextos:\\

%%%%%%%%%%%%%%%%%%%%%%%%%%%%%%%% BLOCO IMAGEM %%%%%%%%%%%%%%%%%%%%%%%%%%%%%%%%
A privacidade é a forma como a sociedade permite desvios das normas sociais
predominantes, considerando que o progresso social requer experimentação social.\\
%%%%%%%%%%%%%%%%%%%%%%%%%%%%%%%% BLOCO IMAGEM %%%%%%%%%%%%%%%%%%%%%%%%%%%%%%%%

Você pode ser uma pessoa em sua conta do Instagram e outra na sala de aula. Mas
os espaços online também ameaçam o colapso de contexto, como Stacy Snyder
descobriu há muito tempo, na era do \ingles{Myspace}, quando sua foto com a legenda
``pirata bêbada'', em uma postagem que ela pensava ser apenas social, custou-lhe 
um diploma de ensino.\footnote{``Judge Sides with University Against Student-Teacher
with ‘Drunken Pirate’ Photo'', The Chronicle of Higher Education, December 4,
2008, \url{https://www.chronicle.com/article/Judge-Sides-With-University/42066}.}

O crescimento explosivo das tecnologias digitais alterou radicalmente nossas
expectativas sobre o que será privado e mudou nosso pensamento sobre o que
\emph{deveria} ser privado. Isso tornou as violações de privacidade mais fáceis e 
potencialmente mais numerosas. De fato, é notável que já não nos surpreendemos
com invasões que há uma década teriam parecido chocantes. Ao contrário da
história do segredo, não houve um único evento tecnológico que tenha causado
essa mudança, nenhuma descoberta revolucionária da privacidade---apenas um
avanço constante em várias frentes tecnológicas que, finalmente, ultrapassou
um ponto de inflexão.

Os dispositivos sensoriais se tornaram mais baratos, melhores e menores.
Pequenas câmeras, unidades de GPS e microfones passaram de objetos de museus
de espionagem para a banalidade do uso diário. Uma vez que eles se tornaram
bens de consumo úteis, parecemos nos preocupar menos com seus usos como
dispositivos de vigilância. Em vez de tentar criar uma teoria unificada sobre
a privacidade e seu valor, nos encontramos montando a privacidade a partir de
sentimentos de desconforto e arrependimento diante da abundância. Isso fica
ainda mais difícil quando somos nós mesmos que trazemos espiões para nossas
próprias casas e das nossas amizades, quando trocamos a privacidade por
convivialidade e conveniência.

\subsection{Sorria enquanto tiramos a foto}
\label{quem:sorria}

\ingles{Big Brother} tinha suas legiões de câmeras, e a Cidade de Londres tem 
as suas hoje em dia. Mas em termos de onipresença fotográfica, nada supera as 
câmeras nos celulares nas mãos das pessoas comuns. Ao voar antes do feriado de 
Quatro de Julho, Helen foi solicitada a trocar de lugar com outra mulher que 
queria sentar ao lado do namorado. Ela ocupou o assento uma fileira acima e começou
uma conversa com seu novo companheiro de assento, sem saber que a fileira de
trás estava filmando-os como um casal apaixonado. O casal que ela ajudou começou
a twittar sobre o voo, usando a hashtag \#PlaneBae, e a história logo foi parar
nos programas de televisão matinais. Pode parecer diversão inocente, mas não para
Helen, que afirmou (por meio de advogados),

\begin{quote}
Sem o meu conhecimento ou consentimento, outros passageiros me fotografaram e
gravaram minha conversa com uma pessoa ao lado. Eles postaram imagens e gravações
nas redes sociais e especularam injustamente sobre minha conduta privada.

Desde então, minhas informações pessoais foram amplamente divulgadas online.
Estranhos discutiram publicamente minha vida privada com base em informações
claramente falsas.

Eu fui vítima de doxing, envergonhamento, insultos e assédio. \ingles{Voyeurs}
vieram me procurar online e no mundo real.\footnote{Taylor Lorenz, ``Unidentified
Plane-Bae Woman’s Statement Confirms the Worst'', The Atlantic, July 13, 2018, 
\url{https://www.theatlantic.com/technology/archive/2018/07/unidentified-plane-bae-womans-statement-confirms-the-worst/565139/}.}
\end{quote}

%Aqui eu não soube traduzir corretamente o termo "Little-Brotherism",
%Portanto, optei por manter o termo original.
A disseminação massiva de câmeras baratas, combinada com o acesso universal
à Internet, possibilita uma espécie de justiça vigilante---um onipresente 
\ingles{``Little-Brotherism''}---, no qual todos podemos ser detetives, juízes
e oficiais correcionais. Blogueiros podem chamar a atenção global para cidadãos
comuns.

Para cada lente direcionada intencionalmente, há muitas outras observando sem
serem notadas: observação e vigilância pública e privada. As ruas principais
estão repletas de câmeras de segurança espiando pelas vitrines das lojas e
câmeras de vigilância da polícia, algumas das quais até mesmo oferecem
visualização pública. Até nos bairros mais tranquilos você pode estar sendo
observado, graças às redes de segurança ``Ring'' e vizinhos vigilantes nos
grupos do aplicativo ``Nextdoor''. Aliado ao reconhecimento facial automatizado,
as ruas conectadas podem estar compilando dossiês sobre todos nós.

Olhar imagens na Web é agora uma atividade de lazer que qualquer pessoa pode
fazer a qualquer momento e em qualquer lugar do mundo. Usando o Google Street
View, você pode sentar em um café no Tajiquistão e identificar um carro que
estava estacionado em minha garagem quando a câmera do Google passou por lá
(talvez meses atrás). De Seul, você pode ver o que está acontecendo agora mesmo,
atualizado a cada poucos segundos, em Piccadilly Circus ou na avenida
de Las Vegas. Essas vistas sempre estiveram disponíveis ao público, mas câmeras
e a Internet mudam o significado de ``público''.

Algumas das invasões à nossa privacidade ocorrem devido aos efeitos colaterais
inesperados e invisíveis das coisas que fazemos voluntariamente. Embora a Quarta
Emenda proteja contra o excesso de vigilância governamental, nos Estados Unidos
há apenas uma consideração legal fragmentada sobre a coleta de informações
privadas. Empresas rotineiramente coletam e inferem informações sobre indivíduos
e as utilizam para personalizar ofertas de produtos e anúncios. Como diz o ditado,
se você não está pagando, você é o produto.

\section{Pegadas e impressões digitais}
\label{pegadas}
Enquanto conduzimos nossos negócios diários e levamos nossas vidas privadas,
deixamos pegadas e impressões digitais. Podemos ver nossas pegadas na lama no
chão, na areia, e na neve ao ar livre. Não nos surpreenderíamos se alguém se
desse ao trabalho de combinar nossos sapatos com nossas pegadas e pudesse
determinar, ou supor, onde estivemos. Impressões digitais são diferentes. Nem
sequer nos ocorre que estamos deixando-as enquanto abrimos portas e bebemos em
copos. Aqueles que têm consciência culpada podem pensar sobre as impressões
digitais e se preocupar com onde estão deixando-as, mas o resto de nós não.\\

%%%%%%%%%%%%%%%%%%%%%%%%%%%%%%%% BLOCO IMAGEM %%%%%%%%%%%%%%%%%%%%%%%%%%%%%%%%
O OLHAR INDESEJADO

O livro \ingles{``The Unwanted Gaze''} de Jeffrey Rosen (Vintage, 2000) detalha
muitas maneiras pelas quais o sistema legal tem contribuído para a nossa perda
de privacidade.\\
%%%%%%%%%%%%%%%%%%%%%%%%%%%%%%%% BLOCO IMAGEM %%%%%%%%%%%%%%%%%%%%%%%%%%%%%%%%

No mundo digital, todos nós deixamos pegadas eletrônicas e impressões
digitais---trilhas de dados que deixamos intencionalmente e trilhas de dados das
quais não estamos cientes ou inconscientes. Os dados identificadores podem ser
úteis para fins forenses. No entanto, como a maioria de nós não se considera
criminoso, tendemos a não nos preocupar com isso. O que não consideramos é que
as várias pequenas marcas que deixamos na paisagem digital podem ser úteis para
outra pessoa---alguém que queira usar os dados que deixamos para ganhar dinheiro
ou obter algo de nós. Portanto, é importante entender como e onde deixamos essas
pegadas digitais e impressões digitais.

\subsection{Papel de rastreamento}
\label{pegadas:papel}
Se eu enviar um e-mail ou baixar um conteúdo da web, não deve ser surpresa que eu
tenha deixado algumas pegadas digitais. Afinal, os bits têm que chegar até mim,
então alguma parte do sistema sabe onde estou. Nos velhos tempos, se eu quisesse
ser anônimo, poderia escrever uma nota, mas minha caligrafia poderia ser reconhecível,
e eu poderia deixar impressões digitais (do tipo oleoso) no papel. Eu poderia ter
digitado, mas Perry Mason regularmente resolvia crimes combinando uma nota digitada
com a assinatura única da máquina de escrever do suspeito. Mais impressões digitais.

Então, hoje em dia eu poderia imprimir a carta a laser e usar luvas. Mas mesmo isso
pode não ser suficiente para me disfarçar. Pesquisadores da Universidade Purdue
desenvolveram técnicas para associar a saída impressa a laser a uma impressora
específica.\footnote{Emil Venere, ``Printer Forensics to Aid Homeland Security,
Tracing Counterfeiters'', Purdue University, October 12, 2004,
\url{https://www.purdue.edu/uns/html4ever/2004/041011.Delp.forensics.html}.}
Eles analisam as folhas impressas e detectam características únicas de
cada fabricante e cada impressora individual---impressões digitais que podem ser
usadas, assim como as manchas das antigas teclas de máquinas de escrever, para
associar a saída à fonte. Pode ser desnecessário colocar o microscópio em letras
individuais para identificar qual impressora produziu uma página.

A \ingles{Electronic Frontier Foundation} demonstrou que muitas impressoras
coloridas codificam quase invisivelmente o número de série, a data e a hora da
impressora em cada página que imprimem (veja a Figura 3.1). Portanto, quando
você imprime um relatório, não deve assumir que ninguém pode saber quem o imprimiu.\\

%%%%%%%%%%% AQUI FICA A FIGURA 3.1 %%%%%%%%%%%%%%%
FIGURA 3.1\\
%%%%%%%%%%% AQUI FICA A FIGURA 3.1 %%%%%%%%%%%%%%%

FIGURA 3.1: Impressão digital deixada por uma impressora a laser colorida Xerox
DocuColor 12. Os pontos são muito difíceis de ver a olho nu; a fotografia foi
tirada sob luz azul. O padrão de pontos codifica a data (2005-05-21), o horário
(12:50) e o número de série da impressora (21052857).

Havia uma razão sensata por trás dessa tecnologia. O governo queria garantir que
as impressoras de escritório não pudessem ser usadas para produzir notas falsas
de cem dólares. A tecnologia que foi criada para frustrar falsificadores torna
possível rastrear cada página impressa em impressoras a laser coloridas até a
fonte. Tecnologias úteis muitas vezes têm consequências não intencionais.

Muitas pessoas, por razões perfeitamente legais e válidas, gostariam de proteger
sua anonimidade. Elas podem ser denunciantes ou dissidentes. Talvez estejam
apenas protestando contra injustiças em seus locais de trabalho. Será que as
tecnologias que enfraquecem o anonimato no discurso político também sufocarão a
liberdade de expressão? Um certo grau de anonimato é essencial em uma democracia
saudável---e nos Estados Unidos, a anonimidade tem sido uma arma usada para
promover a liberdade de expressão desde a época da Revolução. Podemos lamentar o
completo abandono da anonimidade em favor de tecnologias de comunicação que deixam
pegadas.

O problema não é apenas a existência das impressões digitais, mas sim que ninguém
nos disse que estamos criando-as.\\

%%%%%%%%%%%%%%%%%%%%%%%%%%%%%%%% BLOCO TEXTO %%%%%%%%%%%%%%%%%%%%%%%%%%%%%%%%
O problema não é apenas a existência das impressões digitais, mas sim que ninguém
nos disse que estamos criando-as.\\
%%%%%%%%%%%%%%%%%%%%%%%%%%%%%%%% BLOCO TEXTO %%%%%%%%%%%%%%%%%%%%%%%%%%%%%%%%

Quando a contratante da \ingles{NSA (National Security Agency)}, Reality Winner,
vazou informações classificadas para o \ingles{The Intercept}, talvez ela tenha
pensado que enviar uma cópia em papel impediria tentativas de rastrear as origens
do vazamento.\footnote{Michael M. Grynbaum and John Koblin, ``Journalists Fear
Effects of Arrest'', New York Times, June 7, 2017: A19,
\url{https://www.nytimes.com/2017/06/06/business/media/intercept-reality-winner-russia-trump-leak.html}.} 
O The Intercept havia compartilhado o documento com a NSA para
verificar sua autenticidade, e Winner foi presa poucos dias depois. Relatos
iniciais especularam que ela foi rastreada através de microdots da impressora,
mas a verdade parece ter sido ainda mais banal: os registros da NSA mostraram
que apenas seis contas, incluindo a de Winner, tiveram acesso ao documento,
e Winner havia usado uma conta pessoal para entrar em contato com o The
Intercept pouco antes disso.\footnote{Jake Swearingen, ``Did the Intercept
Betray Its NSA Source?'', New York Magazine, June 6, 2017.
\url{https://nymag.com/intelligencer/2017/06/intercept-nsa-leaker-reality-winner.html}.}

\subsection{Publicidade}
\label{pegadas:publicidade}
Se você utilizar o metrô T em Boston, verá muitos anúncios de programas
universitários e de pós-graduação. Todos eles têm números de telefone e URLs,
e muitos direcionam você para páginas como ``college.edu/recruiting/redline''.
O endereço da web não está informando que a faculdade possui um programa
especial na Linha Vermelha (nome de uma linha do metrô), mas sim que eles têm
um programa especial de publicidade lá. O termo ``redline'' no final do URL
permite que a faculdade saiba que você foi encaminhado para essa página pelo
anúncio no metrô. Eles podem usar essa informação para direcioná-lo aos
programas específicos anunciados no cartaz e para acompanhar a eficácia dessa
campanha publicitária.

Anúncios na Web utilizam a página de referência como apenas um dos muitos
indicadores; outros são menos visíveis do que a decoração de URL visível no
cartaz do metrô. Quando você segue um link para abrir uma página da web em seu
navegador, esse clique inicia uma série de eventos que começa com uma solicitação
eletrônica da página da web e um pedido de quaisquer cookies que o site possa
ter configurado anteriormente. Todas as páginas, exceto as mais simples, acionarão
solicitações de mais sub-recursos: imagens, fontes, scripts para tornar a página
dinâmica. Um site comercial pode ter dezenas de anúncios e pixels de rastreamento,
ou ``bugs da web''---elementos invisíveis que fazem seu computador se conectar a
outra fonte com o objetivo de rastrear suas atividades.\\

%%%%%%%%%%%%%%%%%%%%%%%%%%%%%%%% BLOCO IMAGEM %%%%%%%%%%%%%%%%%%%%%%%%%%%%%%%%
COMO SITES SABEM QUEM VOCÊ É (UMA LISTA INCOMPLETA)

\textbf{1. Você informa a eles.} Ao fazer login no Gmail, Amazon ou eBay, você
está dizendo exatamente quem você é para eles.

\textbf{2. Eles deixaram cookies em uma de suas visitas anteriores.} Um cookie
é um pequeno arquivo de texto armazenado em seu disco rígido local que contém
informações que um determinado site deseja ter disponível durante sua sessão
atual (por exemplo, sobre seu carrinho de compras) ou de uma sessão para outra.
Os cookies fornecem informações persistentes aos sites para rastreamento e
personalização. Seu navegador tem um comando para mostrar cookies; se você o
usar, poderá se surpreender com quantos sites os deixaram!

\textbf{3. Eles têm o endereço IP do seu dispositivo.} O servidor da web precisa
saber onde você está para enviar suas páginas da web para você. Seu endereço IP
é um número como 66.82.9.88 que localiza seu computador na Internet. Esse
endereço pode mudar de um dia para o outro. Mas, em um ambiente residencial, seu
provedor de serviços de Internet (ISP; geralmente a empresa de telefone ou cabo)
sabe quem foi atribuído a cada endereço IP em qualquer momento. Esses registros
são frequentemente solicitados em casos judiciais.

\textbf{4. Você se parece com alguém que eles já reconhecem.} Eles identificam
pessoas que têm características semelhantes. Quando os usuários fazem login no
Facebook, eles compartilham detalhes sobre suas vidas, interesses e conexões
sociais. Com base nesses dados, o Facebook cria grupos de usuários que têm
características em comum, mesmo que essas informações não tenham sido fornecidas
diretamente por cada usuário. Esses grupos são chamados de \ingles{``shadow audiences''}.

\textbf{5. Eles criaram uma impressão digital do seu navegador e o vincularam a
perfis de visitas anteriores.} Os sites podem acessar muitos detalhes aparentemente
inofensivos sobre o seu navegador (tipo, versão, codificação gráfica, idioma e
muito mais). Essas informações tendem a ser relativamente estáticas e, muitas
vezes, identificam de forma única uma instância específica do navegador.
Essa técnica é simples, mas surpreendentemente precisa e eficaz.\\
%%%%%%%%%%%%%%%%%%%%%%%%%%%%%%%% BLOCO IMAGEM %%%%%%%%%%%%%%%%%%%%%%%%%%%%%%%%

Se você está curioso para saber quem está usando um determinado endereço IP, você
pode verificar o \ingles{American Registry of Internet Numbers} (\url{www.arin.net}).
Serviços como \url{whatismyip.com}, \url{whatismyip.org} e \url{ipchicken.com}
também permitem que você verifique seu próprio endereço IP. E o \url{www.whois.net}
permite que você verifique quem é o proprietário de um nome de domínio, como
\url{harvard.com}, que, como se constata, é a \ingles{Harvard Bookstore}, uma
livraria privada bem em frente à universidade.

Infelizmente, as informações do endereço IP não revelarão quem está enviando spam
para você, pois os \ingles{spammers} rotineiramente falsificam a origem dos
e-mails que enviam. Além disso, entre o momento em que você solicita uma página
da web e seus anúncios são exibidos em seu navegador, geralmente ocorre um leilão
em tempo real, no qual seus ``olhos'' (ou seja, os espaços de anúncios na página
da web que seu navegador está prestes a exibir) são vendidos para o maior lance.
Redes de publicidade coletam informações de pixels de rastreamento e contexto da
página para determinar quais anúncios oferecer e quanto oferecer para colocá-los
nesses leilões.

Por que esses sapatos estão me perseguindo? Talvez você os tenha visto no Instagram,
marcado-os no Pinterest ou procurado por um novo par de tênis no site da sua loja
favorita. Talvez você até tenha colocado os sapatos em um carrinho de compras
antes de decidir que eles não cabiam no seu orçamento no momento. Agora, parece
que você não consegue escapar dos sapatos: seja lendo notícias ou conversando
com amigos no Facebook, lá estão os sapatos, perseguindo você nos banners de
anúncios, incentivando você a clicar em ``comprar''.

Conhecidos no mercado como \ingles{``retargeting''}, esses anúncios são alguns
dos produtos de lances em tempo real. O profissional de marketing que inseriu
um cookie de rastreamento em seu navegador durante uma sessão de navegação
anterior ou a visita de compras que você interrompeu está usando-o para
identificá-lo como um comprador interessado em sapatos e fazendo lances para
mostrar esses anúncios na esperança de atraí-lo de volta para a compra. Se
você clicar em qualquer um dos anúncios, o profissional de marketing
registrará uma ``conversão'' e levará esses dados em consideração para o seu
perfil, visando futuras oportunidades de anúncios.

Os usuários de navegação na web não aceitaram tudo isso tranquilamente. A
revista \ingles{The Economist} chama os dados de ``o novo petróleo'' e os
navegadores que não desejam ser vistos como fontes desse petróleo estão
baixando bloqueadores de anúncios. Desde o início de 2020, todos os
principais navegadores da web incorporaram recursos de bloqueio de
rastreadores ou anunciaram planos para limitar cookies de terceiros.

Arvind Narayanan e sua equipe da Universidade de Princeton criaram um
laboratório para medição na web\footnote{``Web Privacy—Arvind Narayanan'',
accessed May 18, 2020,
\url{https://www.cs.princeton.edu/~arvindn/web-privacy/}.} e descobriram novas
técnicas de rastreamento de navegadores. Por meio de \ingles{``crawls''} na
web, eles encontraram técnicas de rastreamento usadas no mundo real para
identificar usuários e reidentificar aqueles que pensaram que haviam eliminado
todas as interações anteriores. Um dos paradoxos da privacidade na web é que
os navegadores podem ser identificados pela sua ``impressão digital'', por
conta de suas características únicas, incluindo recursos que o usuário pode
ativar com o objetivo de obter maior privacidade. Isso significa que ativar
tais proteções pode fazer com que o usuário em busca de privacidade acabe por
se destacar. Em tais casos, a privacidade depende das ações de muitos para
fornecer um grupo no qual o navegador em busca de privacidade possa se misturar.
Processos padronizados e configurações padrão bem pensadas são necessárias
para preservar as oportunidades de privacidade.

\subsection{A Target sabe que você está grávida}
\label{pegadas:target}
Em 2012, conforme relatado por Charles Duhigg no \ingles{New York Times},\footnote{
Charles Duhugg, ``How Companies Learn Your Secrets'', The New York Times, February 16, 2012,
\url{https://www.nytimes.com/2012/02/19/magazine/shopping-habits.html}.}
um homem entrou em uma loja da Target na região de Minneapolis e pediu para falar
com o gerente, furiosamente dizendo: ``Minha filha recebeu isso pelo correio!
Ela ainda está no ensino médio, e vocês estão enviando cupons para roupas de
bebê e berços? Vocês estão tentando incentivá-la a engravidar?''

O gerente da loja pediu desculpas ao homem de Minneapolis pelo aparente erro
deles, mas ele retornou algumas semanas depois com um pedido de desculpas
próprio: sua filha estava, de fato, grávida. Os modelos preditivos da Target
haviam reconhecido a gravidez da jovem mulher mesmo antes que seu pai soubesse.
Os modelos da Target não tinham acesso às informações privadas dela. Eles
tinham o poder de ferramentas analíticas e dados prontamente disponíveis.

Assim como muitas outras lojas com cartões de fidelidade ou contas de usuário,
a Target construiu modelos estatísticos do comportamento dos compradores para
prever produtos populares para estoque e precificação, e para fazer recomendações.
A Target correlacionava o histórico de compras dos clientes com base em um ID
de hóspede interno e adquiria dados externos para complementar seus registros.
A partir desses registros, o estatístico da empresa poderia derivar padrões,
percebendo, por exemplo, que mulheres no segundo trimestre da gravidez
frequentemente compravam loções hidratantes sem perfume e suplementos. Depois
de observar esse padrão muitas vezes, a loja podia antecipar futuras compras
de roupas de bebê e fraldas com base nas compras anteriores de loção sem perfume---e
anunciar para a futura mãe em um momento em que seus hábitos de compra estavam
em mudança---respondendo a um sinal que ela nem sabia que estava enviando.

Como podemos resolver um problema de privacidade que resulta de muitos
desenvolvimentos, mas nenhum deles é realmente um problema em si?

\subsection{Você paga pelo microfone, nós apenas ouvimos}
\label{pegadas:voce}
Plantar microfones minúsculos onde poderiam captar conversas de figuras do
submundo costumava ser um trabalho arriscado para as autoridades federais. Agora,
existem alternativas muito mais seguras, já que a maioria das pessoas carrega
seus próprios microfones equipados com rádio o tempo todo ou convida Alexa,
Siri, Cortana ou Google para suas casas.

Muitos celulares podem ser reprogramados remotamente para que o microfone esteja
sempre ligado e o telefone esteja transmitindo, mesmo que você pense que o
desligou. O FBI usou essa técnica em 2004 para ouvir as conversas de John Tomero
com outros membros de sua família do crime organizado. Um tribunal federal
considerou que esse \ingles{roving bug}, instalado após devida autorização,
constituía uma forma legal de escuta telefônica. Tomero poderia ter impedido
isso removendo a bateria, e agora alguns executivos de negócios apreensivos,
rotineiramente fazem exatamente isso.

O microfone em um carro da \ingles{General Motors} equipado com o sistema
\ingles{OnStar} também pode ser ativado remotamente, uma funcionalidade que pode
salvar vidas quando os operadores do OnStar entram em contato com o motorista
após receberem um sinal de colisão. O OnStar adverte: ``O OnStar cooperará com
ordens judiciais oficiais relacionadas a investigações criminais das forças
policiais e de outras agências'', e de fato, o FBI usou esse método para
interceptar conversas realizadas dentro dos carros. Em um caso, um tribunal
federal se opôs a essa forma de coleta de provas---mas não por questões de
privacidade. O roving bug desativou o funcionamento normal do OnStar, e
o tribunal simplesmente considerou que o FBI interferiu no direito contratual
do proprietário do veículo de conversar com os operadores do OnStar!

Danielle, cliente do Amazon Echo em Portland, Oregon, ficou assustada ao receber
uma ligação de um colega de trabalho de seu marido, alertando que ela estava
sendo hackeada.\footnote{``Amazon Alexa Can Accidentally Record and Share Your
Conversations'', Vanity Fair, May 24, 2018,
\url{https://www.vanityfair.com/news/2018/05/yes-amazons-alexa-can-secretly-record-and-share-conversations}.}
O dispositivo, que deveria gravar apenas após o comando ``Alexa'',
captou também um comando para enviar mensagem durante a conversa de Danielle.
Sua discussão sobre pisos de madeira se transformou em uma mensagem de voz para
um conhecido comercial. Evento incomum, mas que pode se repetir à medida que
gravadores em rede se tornam mais comuns. Autoridades alemãs proibiram a boneca
falante ``My Friend Cayla''\footnote{Katie Collins, ``That Smart Doll Could
be a Spy. Parents, Smash!'', CNET, February 17, 2018,
\url{https://www.cnet.com/tech/computing/parents-told-to-destroy-connected-dolls-over-hacking-fears/}.}
por temores de espionagem e coleta de dados. Cayla
enviava sons pela internet para interagir com crianças. Pais foram instruídos
a destruir o ``aparelho de espionagem ilegal''. Enquanto isso, nos Estados
Unidos, sua smart TV pode estar observando seus hábitos de visualização para
personalizar a publicidade. O CTO da Vizio disse na \ingles{Consumer Electronics
Show} que as TVs custariam mais se não fosse por essa fonte de receita.\footnote{
Ben Gilbert, ``There’s a simple reason your new smart TV was so affordable:
It’s collecting and selling your data, and serving you ads'', Business Insider,
April 5, 2019,
\url{https://www.businessinsider.com/smart-tv-data-collection-advertising-2019-1}.}

\subsection{Venmo: Tudo se soma}
\label{pegadas:venmo}
Anteriormente, discutimos o rastreamento que os cartões de crédito permitem em
agências de relatórios de crédito e empresas de análise de dados. Novas
tecnologias de pagamento trazem o relatório diretamente para você. O Venmo
permite que você envie dinheiro a alguém ou divida uma conta inserindo o número
de telefone da pessoa. É tão fácil que ao enviar dinheiro para amigos ou
colegas de quarto usando o aplicativo Venmo, você pode não perceber que essas
transações de pagamento são públicas, incluindo qualquer nota que você escreva
junto com o pagamento.

Um pesquisador encontrou padrões em algumas das milhões de transações no
``Venmo stories'':\footnote{Hang Do Thi Duc, Public By Default, Venmo Stories of
2017, \url{https://publicbydefault.fyi/}.} 
o hábito de fast food de um estudante, as vendas de um vendedor de cannabis, um
relacionamento em desenvolvimento? Você pode não se importar em compartilhar
sua paixão por elote (milho temperado), mas pode se sentir diferente em relação
a compras recreativas de maconha, mesmo em estados onde isso é legal. O pesquisador,
Hang Do Thi Duc, anonimizou os detalhes, mas observa que o feed, que inclui tudo,
exceto os valores em dólares, permaneceu acessível para qualquer visitante da API
pública do Venmo. (Cada página do site desenvolvido por Duc, \url{publicbydefault.fyi},
incentiva os usuários do Venmo a alterarem suas configurações de privacidade do
padrão para tornar as transações privadas entre remetente e destinatário.)

\subsection{DNA: A Última Impressão Digital}
\label{pegadas:dna}
Em abril de 2018, o estado da Califórnia acusou Joseph James DeAngelo de uma
série de crimes de assassinato e estupro ocorridos décadas atrás. O Assassino
do Estado Dourado havia sido um caso antigo até que um investigador carregou
DNA de uma cena de crime em um site público de genealogia, o GEDmatch. O
investigador criou um perfil falso para o assassino desconhecido cujo DNA
foi recuperado na cena do crime, e carregado no site. Depois que o GEDmatch comparou o DNA dessa pessoa
com seu banco de dados existente para identificar correspondências genéticas
parciais, ele mostrou perfis de pessoas que provavelmente eram parentes
distantes do assassino. Esses nomes levaram a árvores genealógicas e à
genealogia que pôde ser rastreada ainda mais através de registros censitários,
obituários, sepulturas e bancos de dados comerciais e de órgãos de aplicação da
lei. Após essas pesquisas revelarem o nome do criminoso, os investigadores
confirmaram suas suspeitas rastreando-o e obtendo outra amostra de DNA, a
partir de células de pele que ele deixou na porta do carro quando estacionou
em um estacionamento do Hobby Lobby. Esse DNA correspondeu às amostras
originais da cena do crime.\footnote{Avi Selk, ``The ingenious and ‘dystopian’
DNA technique police used to hunt the ‘Golden State Killer’ suspect'',
Washington Post, April 28, 2018,
https://www.washingtonpost.com/news/true-crime/wp/2018/04/27/golden-state-killer-dna-website-gedmatch-was-used-to-identify-joseph-deangelo-as-suspect-police-say/.}

DeAngelo não havia postado no site de ancestralidade, mas como um dos pais passa
aproximadamente metade de seus genes para uma criança (com algumas mutações ao
longo do caminho), grande parte do registro genético de DeAngelo poderia ser
lida ou revelada por parentes. Se seus familiares explorarem seus perfis genéticos
e árvores genealógicas no GEDmatch, eles também estarão expondo informações sobre
traços que você pode compartilhar. Sua privacidade pode ser invadida sem nenhuma
ação sua. Embora a Lei de Não Discriminação de Informações Genéticas proíba que
empregadores ou seguradoras de saúde discriminem com base no DNA, a lei não
restringe as inúmeras outras maneiras pelas quais o DNA pode ser usado.

O caso do Assassino do Estado Dourado iniciou um boom na genealogia forense de DNA.
Até o final de 2018, mais de uma dúzia de criminosos violentos e perpetradores
de agressão sexual haviam sido identificados por meio do GEDmatch. No entanto, o
site também ouviu alarmes sobre privacidade e alterou seus termos de serviço para
proibir que as forças policiais correspondessem perfis de DNA, a menos que os
usuários optassem pôr seus próprios registros.

\end{document}