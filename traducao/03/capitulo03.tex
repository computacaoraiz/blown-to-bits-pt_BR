\documentclass{book}
\usepackage{url}
\newcommand{\ingles}[1]{\textit{#1}}

\begin{document}

\chapter[Quem é o Dono da Sua Privacidade?]{Quem é o Dono da Sua Privacidade?\\\large\textit{A Comercialização dos Dados Pessoais}}
\label{quem}

\section{Que tipo de vegetal você é?}
\label{quem:vegetal}
Não causou espanto quando Aleksandr Kogan ofereceu \ingles{``This Is Your Digital Life''}
como um aplicativo de questionário do Facebook. Aplicativos de questionário são
uma parte essencial do marketing no Facebook, atraindo usuários para participar
e, em seguida, colhendo dados de marketing. Esses aplicativos---que são atrativos,
sedutores e altamente eficazes---deram origem a uma sub indústria inteira de
ferramentas e especialistas em marketing de questionários.

Aproximadamente 270.000 usuários do Facebook instalaram o aplicativo de Kogan
e fizeram o teste de personalidade, dando ao aplicativo o acesso aos seus contatos
para convidá-los a fazer o mesmo. A motivação ostensiva de Kogan era a pesquisa
acadêmica---estudar como os emojis são usados para transmitir emoções. Mas o que
ele fez com todos os dados que coletou foi bastante diferente. Através do
aplicativo de Kogan, a empresa Cambridge Analytica colheu dados de mais de 50
milhões de pessoas. A Cambridge Analytica utilizou essas informações para ajudar
a campanha do candidato presidencial Donald Trump a direcionar públicos para    
publicidade digital e captação de recursos, modelar a participação dos eleitores,
identificar mercados para veicular anúncios de televisão e até mesmo planejar as
viagens de Trump. A Cambridge Analytica afirmou que seus ``perfis psicográficos''
ajudaram a identificar eleitores prováveis e os tipos de mensagens que os
persuadiriam a votar em Trump.\footnote{Matthew Rosenberg, Nicholas Confessore,
and Carole Cadwalladr, ``Firm That Assisted Trump Exploited Data of Millions'',
New York Times, March 18, 2018: A1,
\url{https://www.nytimes.com/2018/03/17/us/politics/cambridge-analytica-trump-campaign.html}.}

Mas como é que um quarto de milhão de pessoas baixando um aplicativo se transformou
em vazamento de dados de 50 milhões? Através do modelo de privacidade poroso dos
aplicativos do Facebook. Cada um dos 270.000 usuários que instalaram o aplicativo
estava conectado a uma média de 200 amigos. \ingles{``This Is Your Digital Life''}
baseava sua avaliação não tanto no questionário, mas sim na história das páginas
``curtidas''. O questionário era um pretexto para obter acesso às curtidas dos
usuários e aos de seus contatos. O Facebook permitiu essa coleta de dados em 
2015---embora afirme que Kogan violou os termos do programa ao compartilhar dados
de perfil com a Cambridge Analytica.

Sua privacidade não é sua própria. Mesmo que você tenha rejeitado o \ingles{``This
Is Your Digital Life''}, qualquer um dos seus amigos---ou os aplicativos que eles
instalaram---poderiam ter comprometido seus dados. Isso tem paralelos também no
mundo não digital, é claro. (Considere o velho ditado ``Duas pessoas podem guardar
um segredo se uma delas estiver morta.'') Mas offline, você pode ter melhores
intuições sobre isso. Você sabe que não deve compartilhar uma história com o
vizinho fofoqueiro até estar pronto para responder perguntas de estranhos no
supermercado. Online, levou muito tempo para as configurações de privacidade do
Facebook adquirirem controles de público simples e somente após o escândalo da
Cambridge Analytica é que a rede social deixou de permitir que os aplicativos
percorressem o grafo social, capturando a rede de conexões de amigos.

\section{Me deixe em paz}
\label{quem:paz}
Há mais de um século, dois advogados alertaram sobre o impacto da tecnologia
e da mídia na privacidade pessoal: 

\begin{quote}
Fotografias instantâneas e a empresa jornalística invadiram os santuários da
vida privada e doméstica; e numerosos dispositivos mecânicos ameaçam concretizar
a previsão de que ``o que é sussurrado no armário será proclamado nos telhados.'' 
\end{quote}

Esta declaração é do artigo seminal da \ingles{Harvard Law Review} sobre privacidade,
publicado em 1890 pelo advogado de Boston, Samuel Warren, e seu parceiro de
advocacia, Louis Brandeis, que mais tarde se tornaria um juiz da Suprema Corte
dos Estados Unidos (onde, como vimos, ele discordou em defesa da privacidade
no caso \ingles{Olmstead v. U.S.}).\footnote{Samuel A. Warren and Louis D.
Brandeis, ``The Right to Privacy'', Harvard Law Review 4, no. 5 (December 15, 1890),
\url{https://groups.csail.mit.edu/mac/classes/6.805/articles/privacy/Privacy_brand_warr2.html}.}
Warren e Brandeis continuaram dizendo: 

\begin{quote}
A fofoca já não é apenas recurso dos ociosos e dos viciosos, mas se tornou
um negócio, que é exercido com diligência e descaramento. Para satisfazer um gosto
lascivo, os detalhes das relações sexuais são espalhados amplamente nas colunas
dos jornais diários. Para ocupar os ociosos, coluna após coluna é preenchida com
fofocas vazias, que só podem ser obtidas invadindo o círculo doméstico.
\end{quote}

Novas tecnologias facilitaram a produção desse lixo, e então a oferta criou
a demanda. E aquelas fotografias sinceras e colunas de fofoca não eram apenas
de mau gosto; eram ruins. Soando como críticos de programas de TV
sem sentido, Warren e Brandeis indignaram-se ao afirmar que a sociedade 
estava indo de mal a pior por conta de todas aquelas coisas que estavam
sendo espalhadas:

\begin{quote}
Mesmo a fofoca aparentemente inofensiva, quando amplamente e persistentemente
divulgada, tem o poder de causar o mal. Ela menospreza e perverte. Menospreza
ao inverter a importância relativa das coisas, diminuindo assim os pensamentos
e aspirações de um povo. Quando fofocas pessoais alcançam a dignidade de serem
publicadas e ocupam o espaço disponível para assuntos de real interesse para a
comunidade, não é surpreendente que os ignorantes e desavisados confundam sua
importância relativa. Fácil de compreender, apelando para o lado fraco da
natureza humana que nunca será totalmente abalado pelas desgraças e fraquezas de
nossos vizinhos, ninguém pode se surpreender que ela usurpe o lugar de interesse
em mentes capazes de outras coisas. A trivialidade destrói imediatamente a
robustez do pensamento e a delicadeza do sentimento. Nenhum entusiasmo pode
florescer, nenhum impulso generoso pode sobreviver sob sua influência arrasadora.
\end{quote}

O problema percebido por Warren e Brandeis era que era difícil dizer exatamente
por que essas invasões de privacidade deveriam ser ilegais. Em casos individuais,
você poderia dizer algo sensato, mas as decisões legais individuais não faziam
parte de um regime geral. Os tribunais certamente aplicariam sanções legais
por difamação---publicar fofocas maliciosas que eram falsas---mas e quanto às
fofocas maliciosas que eram verdadeiras? Alguns tribunais impuseram
penalidades por publicar as cartas pessoais de um indivíduo---mas com base
na lei de propriedade, como se a posse de alguém tivesse sido roubada ao invés
das palavras em suas cartas.
Não, eles concluíram, tais argumentos não chegavam à essência da questão. Quando
algo privado é publicado sobre você, algo lhe foi tirado, você é vítima de 
roubo---mas a coisa roubada de você faz parte de sua identidade como pessoa. Na
verdade, a privacidade era um direito, afirmaram eles, um ``direito geral do 
indivíduo de ser deixado em paz''. Esse direito há muito tempo estava presente nas 
decisões judiciais, mas as novas tecnologias haviam trazido essa questão à tona. Ao
articular esse novo direito, Warren e Brandeis estavam, segundo eles afirmaram,
fundamentando-o no princípio de ``integridade da pessoa inviolável'', a santidade
da identidade individual.

\section{Privacidade e Liberdade}
\label{quem:privacidade}

A articulação de Warren--Brandeis sobre a privacidade como um direito de ser
deixado em paz foi influente, mas nunca foi realmente completa. Ao longo do
século XX, havia simplesmente muitas boas razões para \emph{não} deixar as pessoas
em paz, e também havia muitas maneiras pelas quais as pessoas \emph{preferiam} não
ser deixadas em paz. E nos Estados Unidos, os direitos da Primeira Emenda
entravam em tensão com os direitos de privacidade. Como regra geral, o governo
não pode me impedir de dizer qualquer coisa que seja verdadeira. Em particular,
geralmente não pode me impedir de dizer o que descobri legalmente sobre seus
assuntos privados. No entanto, a definição de Warren--Brandeis funcionou bem
o suficiente por muito tempo, porque, como Robert Fano colocou, ``O ritmo do
progresso tecnológico foi por muito tempo suficientemente lento para permitir
que a sociedade aprendesse pragmaticamente como explorar novas tecnologias e
evitar seu abuso, mantendo seu equilíbrio na maior parte do tempo''.\footnote{Robert
Fano, ``Review of Alan Westin’s Privacy and Freedom'', Scientific American
(May 1968): 148–152.} Até o final da década de 1950, as emergentes tecnologias
eletrônicas, tanto computadores quanto comunicação, destruíram esse equilíbrio.
A sociedade não podia mais se ajustar pragmaticamente porque as tecnologias de
vigilância estavam se desenvolvendo muito rapidamente.

O resultado foi um estudo importante sobre privacidade realizado pela Associação
dos Advogados da Cidade de Nova York, que culminou na publicação, em 1967, do
livro de Alan Westin, intitulado \ingles{``Privacy and Freedom''}.\footnote{
Alan F. Westin, Privacy and Freedom (Atheneum, 1967).} (Fano estava
revisando o livro de Westin quando descreveu o quadro de desequilíbrio social
causado pela rápida mudança tecnológica.) Westin propôs uma mudança crucial de foco.

Brandeis e Warren haviam visto a perda de privacidade como uma forma de lesão
pessoal, que poderia ser tão grave a ponto de causar ``dor mental e angústia,
muito maiores do que poderiam ser infligidas por mera lesão corporal''. Os
indivíduos precisavam assumir a responsabilidade de se proteger. ``Cada pessoa
é responsável apenas por seus próprios atos e omissões''. Mas a lei tinha que
fornecer as armas para resistir a invasões de privacidade.

Westin reconheceu que a formulação de Brandeis--Warren era muito absoluta, diante
dos direitos de expressão de outras pessoas e das práticas legítimas de coleta
de dados da sociedade. A proteção poderia não vir de escudos protetores, mas
do controle sobre os usos que poderiam ser feitos das informações pessoais.
``Privacidade'', escreveu Westin, ``é o direito de indivíduos, grupos ou instituições
determinarem por si mesmos quando, como e em que medida informações sobre eles
são comunicadas a outros.'' Westin propôs:

\begin{quote}
\ldots o que é necessário é um processo estruturado e racional de ponderação,
com critérios definidos, que as autoridades públicas e privadas possam aplicar
ao comparar o pedido de divulgação ou vigilância por meio de novos
dispositivos, com a pedido de privacidade. Os seguintes passos são
sugeridos como a base desse processo: avaliar a gravidade da necessidade de
conduzir a vigilância; decidir se existem métodos alternativos para atender
à necessidade; decidir qual grau de confiabilidade será exigido do instrumento
de vigilância; determinar se foi concedido um verdadeiro consentimento para a
vigilância; e avaliar a capacidade de limitação e controle da vigilância se
for permitida.\footnote{Ibid.}  
\end{quote}

Portanto, mesmo que houvesse uma razão legítima pela qual o governo ou outra
parte pudesse saber algo sobre você, seu direito à privacidade poderia limitar
o que a parte informada poderia fazer com essa informação.

Essa compreensão mais sutil da privacidade surgiu dos importantes papéis sociais
que a privacidade desempenha. Privacidade não é, como Warren e Brandeis afirmaram,
o direito de ser isolado da sociedade; privacidade é um direito que faz a
sociedade funcionar.

Fano mencionou três papéis sociais da privacidade. Primeiro, ``o direito de manter
a privacidade de sua personalidade pode ser considerado parte do direito de
autodefesa''---o direito de guardar para si mesmo os julgamentos equivocados da
adolescência e os conflitos pessoais, desde que não tenham uma importância
duradoura para sua posição final na sociedade. Segundo, a privacidade é a forma
como a sociedade permite desvios das normas sociais predominantes, considerando
que nenhum conjunto de normas sociais é universal e permanentemente satisfatório---e, 
de fato, dado que o progresso social requer experimentação social. E terceiro,
a privacidade é essencial para o desenvolvimento do pensamento independente; ela
permite que o indivíduo se desvincule da sociedade de forma que os pensamentos
possam ser compartilhados em círculos limitados e ensaiados antes de serem
expostos publicamente.

A filósofa Helen Nissenbaum também fundamenta a privacidade no ser social,
descrevendo-a como ``integridade contextual''.\footnote{Helen Nissenbaum, Privacy
in Context: Technology, Policy, and the Integrity of Social Life (Stanford Law
Books, 2009).} A privacidade depende de uma correspondência entre o fluxo de dados
e as expectativas e normas do ambiente em que as informações foram geradas e
compartilhadas. Quando o Facebook o convida a adicionar seu terapeuta ou um colega
paciente como amigo, isso representa uma violação de contexto. Os espaços online
oferecem a oportunidade de multiplicar contextos:

\vspace{1cm}
%%%%%%%%%%%%%%%%%%%%%%%%%%%%%%%% BLOCO IMAGEM %%%%%%%%%%%%%%%%%%%%%%%%%%%%%%%%
A privacidade é a forma como a sociedade permite desvios das normas sociais
predominantes, considerando que o progresso social requer experimentação social.
%%%%%%%%%%%%%%%%%%%%%%%%%%%%%%%% BLOCO IMAGEM %%%%%%%%%%%%%%%%%%%%%%%%%%%%%%%%
\vspace{1cm}

Você pode ser uma pessoa em sua conta do Instagram e outra na sala de aula. Mas
os espaços online também ameaçam o colapso de contexto, como Stacy Snyder
descobriu há muito tempo, na era do Myspace, quando sua foto com a legenda
``pirata bêbada'', em uma postagem que ela pensava ser apenas social, custou-lhe 
um diploma de ensino.\footnote{``Judge Sides with University Against Student-Teacher
with ‘Drunken Pirate’ Photo'', The Chronicle of Higher Education, December 4,
2008, \url{https://www.chronicle.com/article/Judge-Sides-With-University/42066}.}

O crescimento explosivo das tecnologias digitais alterou radicalmente nossas
expectativas sobre o que será privado e mudou nosso pensamento sobre o que
\emph{deveria} ser privado. Isso tornou as violações de privacidade mais fáceis e 
potencialmente mais numerosas. De fato, é notável que já não nos surpreendemos
com invasões que há uma década teriam parecido chocantes. Ao contrário da
história do segredo, não houve um único evento tecnológico que tenha causado
essa mudança, nenhuma descoberta revolucionária da privacidade---apenas um
avanço constante em várias frentes tecnológicas que, finalmente, ultrapassou
um ponto de inflexão.

Os dispositivos sensoriais se tornaram mais baratos, melhores e menores.
Pequenas câmeras, unidades de GPS e microfones passaram de objetos de museus
de espionagem para a banalidade do uso diário. Uma vez que eles se tornaram
bens de consumo úteis, parecemos nos preocupar menos com seus usos como
dispositivos de vigilância. Em vez de tentar criar uma teoria unificada sobre
a privacidade e seu valor, nos encontramos montando a privacidade a partir de
sentimentos de desconforto e arrependimento diante da abundância. Isso fica
ainda mais difícil quando somos nós mesmos que trazemos espiões para nossas
próprias casas e das nossas amizades, quando trocamos a privacidade por
convivialidade e conveniência.

\end{document}