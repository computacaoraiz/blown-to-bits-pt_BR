\documentclass{book}
\usepackage{url}
\newcommand{\ingles}[1]{\textit{#1}}

\begin{document}

\chapter[Quem é o Dono da Sua Privacidade?]{Quem é o Dono da Sua Privacidade?\\\large\textit{A Comercialização dos Dados Pessoais}}
\label{cap3:quem}

\section{Que tipo de vegetal você é?}
\label{cap3:quem-que}
Não causou espanto quando Aleksandr Kogan ofereceu \ingles{``This Is Your Digital Life''}
como um aplicativo de questionário do Facebook. Aplicativos de questionário são
uma parte essencial do marketing no Facebook, atraindo usuários para participar
e, em seguida, colhendo dados de marketing. Esses aplicativos---que são atrativos,
sedutores e altamente eficazes---deram origem a uma sub indústria inteira de
ferramentas e especialistas em marketing de questionários.

Aproximadamente 270.000 usuários do Facebook instalaram o aplicativo de Kogan e
fizeram o teste de personalidade, dando ao aplicativo o acesso aos seus
contatos para convidá-los a fazer o mesmo. A motivação ostensiva de Kogan era a
pesquisa acadêmica---estudar como os emojis são usados para transmitir emoções.
Mas o que ele fez com todos os dados que coletou foi bastante diferente.
Através do aplicativo de Kogan, a empresa Cambridge Analytica colheu dados de
mais de 50 milhões de pessoas. A Cambridge Analytica utilizou essas informações
para ajudar a campanha do candidato presidencial Donald Trump a direcionar
públicos para publicidade digital e captação de recursos, modelar a
participação dos eleitores, identificar mercados para veicular anúncios de
televisão e até mesmo planejar as viagens de Trump. A Cambridge Analytica
afirmou que seus ``perfis psicográficos'' ajudaram a identificar eleitores
prováveis e os tipos de mensagens que os persuadiriam a votar em
Trump.\footnote{Matthew Rosenberg, Nicholas Confessore, and Carole Cadwalladr,
``Firm That Assisted Trump Exploited Data of Millions'', New York Times, March
18, 2018: A1,
\url{https://www.nytimes.com/2018/03/17/us/politics/cambridge-analytica-trump-campaign.html}.}

Mas como é que um quarto de milhão de pessoas baixando um aplicativo se
transformou em vazamento de dados de 50 milhões? Através do modelo de
privacidade poroso dos aplicativos do Facebook. Cada um dos 270.000 usuários
que instalaram o aplicativo estava conectado a uma média de 200
amigos.~\ingles{``This Is Your Digital Life''} baseava sua avaliação não tanto
no questionário, mas sim na história das páginas ``curtidas''. O questionário
era um pretexto para obter acesso às curtidas dos usuários e aos de seus
contatos. O Facebook permitiu essa coleta de dados em 2015---embora afirme que
Kogan violou os termos do programa ao compartilhar dados de perfil com a
Cambridge Analytica.

Sua privacidade não é sua própria. Mesmo que você tenha rejeitado o
\ingles{``This Is Your Digital Life''}, qualquer um dos seus amigos---ou os
aplicativos que eles instalaram---poderiam ter comprometido seus dados. Isso
tem paralelos também no mundo não digital, é claro. (Considere o velho ditado
``Duas pessoas podem guardar um segredo se uma delas estiver morta.'') Mas
offline, você pode ter melhores intuições sobre isso. Você sabe que não deve
compartilhar uma história com o vizinho fofoqueiro até estar pronto para
responder perguntas de estranhos no supermercado. Online, levou muito tempo
para as configurações de privacidade do Facebook adquirirem controles de
público simples e somente após o escândalo da Cambridge Analytica é que a rede
social deixou de permitir que os aplicativos percorressem o grafo social,
capturando a rede de conexões de amigos.

\subsection{Me deixe em paz}
\label{cap3:quem-que-me}
Há mais de um século, dois advogados alertaram sobre o impacto da tecnologia
e da mídia na privacidade pessoal:

\begin{quote}
    Fotografias instantâneas e a empresa jornalística invadiram os santuários da
    vida privada e doméstica; e numerosos dispositivos mecânicos ameaçam concretizar
    a previsão de que ``o que é sussurrado no armário será proclamado nos telhados.''
\end{quote}

Esta declaração é do artigo seminal da \ingles{Harvard Law Review} sobre
privacidade, publicado em 1890 pelo advogado de Boston, Samuel Warren, e seu
parceiro de advocacia, Louis Brandeis, que mais tarde se tornaria um juiz da
Suprema Corte dos Estados Unidos (onde, como vimos, ele discordou em defesa da
privacidade no caso \ingles{Olmstead v. U.S.}).\footnote{Samuel A. Warren and
Louis D. Brandeis, ``The Right to Privacy'', Harvard Law Review 4, no. 5
(December 15, 1890),
\url{https://groups.csail.mit.edu/mac/classes/6.805/articles/privacy/Privacy_brand_warr2.html}.}
Warren e Brandeis continuaram dizendo:

\begin{quote}
    A fofoca já não é apenas recurso dos ociosos e dos viciosos, mas se tornou
    um negócio, que é exercido com diligência e descaramento. Para satisfazer um gosto
    lascivo, os detalhes das relações sexuais são espalhados amplamente nas colunas
    dos jornais diários. Para ocupar os ociosos, coluna após coluna é preenchida com
    fofocas vazias, que só podem ser obtidas invadindo o círculo doméstico.
\end{quote}

Novas tecnologias facilitaram a produção desse lixo, e então a oferta criou a
demanda. E aquelas fotografias sinceras e colunas de fofoca não eram apenas de
mau gosto; eram ruins. Soando como críticos de programas de TV sem sentido,
Warren e Brandeis indignaram-se ao afirmar que a sociedade estava indo de mal a
pior por conta de todas aquelas coisas que estavam sendo espalhadas:

\begin{quote}
    Mesmo a fofoca aparentemente inofensiva, quando amplamente e persistentemente
    divulgada, tem o poder de causar o mal. Ela menospreza e perverte. Menospreza
    ao inverter a importância relativa das coisas, diminuindo assim os pensamentos
    e aspirações de um povo. Quando fofocas pessoais alcançam a dignidade de serem
    publicadas e ocupam o espaço disponível para assuntos de real interesse para a
    comunidade, não é surpreendente que os ignorantes e desavisados confundam sua
    importância relativa. Fácil de compreender, apelando para o lado fraco da
    natureza humana que nunca será totalmente abalado pelas desgraças e fraquezas de
    nossos vizinhos, ninguém pode se surpreender que ela usurpe o lugar de interesse
    em mentes capazes de outras coisas. A trivialidade destrói imediatamente a
    robustez do pensamento e a delicadeza do sentimento. Nenhum entusiasmo pode
    florescer, nenhum impulso generoso pode sobreviver sob sua influência arrasadora.
\end{quote}

O problema percebido por Warren e Brandeis era que era difícil dizer exatamente
por que essas invasões de privacidade deveriam ser ilegais. Em casos
individuais, você poderia dizer algo sensato, mas as decisões legais
individuais não faziam parte de um regime geral. Os tribunais certamente
aplicariam sanções legais por difamação---publicar fofocas maliciosas que eram
falsas---mas e quanto às fofocas maliciosas que eram verdadeiras? Alguns
tribunais impuseram penalidades por publicar as cartas pessoais de um
indivíduo---mas com base na lei de propriedade, como se a posse de alguém
tivesse sido roubada ao invés das palavras em suas cartas. Não, eles
concluíram, tais argumentos não chegavam à essência da questão. Quando algo
privado é publicado sobre você, algo lhe foi tirado, você é vítima de
roubo---mas a coisa roubada de você faz parte de sua identidade como pessoa. Na
verdade, a privacidade era um direito, afirmaram eles, um ``direito geral do
indivíduo de ser deixado em paz''. Esse direito há muito tempo estava presente
nas decisões judiciais, mas as novas tecnologias haviam trazido essa questão à
tona. Ao articular esse novo direito, Warren e Brandeis estavam, segundo eles
afirmaram, fundamentando-o no princípio de ``integridade da pessoa
inviolável'', a santidade da identidade individual.

\subsection{Privacidade e liberdade}
\label{cap3:quem-que-privacidade}

A articulação de Warren-Brandeis sobre a privacidade como um direito de ser
deixado em paz foi influente, mas nunca foi realmente completa. Ao longo do
século XX, havia simplesmente muitas boas razões para \emph{não} deixar as
pessoas em paz, e também havia muitas maneiras pelas quais as pessoas
\emph{preferiam} não ser deixadas em paz. E nos Estados Unidos, os direitos da
Primeira Emenda entravam em tensão com os direitos de privacidade. Como regra
geral, o governo não pode me impedir de dizer qualquer coisa que seja
verdadeira. Em particular, geralmente não pode me impedir de dizer o que
descobri legalmente sobre seus assuntos privados. No entanto, a definição de
Warren-Brandeis funcionou bem o suficiente por muito tempo, porque, como Robert
Fano colocou, ``O ritmo do progresso tecnológico foi por muito tempo
suficientemente lento para permitir que a sociedade aprendesse pragmaticamente
como explorar novas tecnologias e evitar seu abuso, mantendo seu equilíbrio na
maior parte do tempo''.\footnote{Robert Fano, ``Review of Alan Westin's Privacy
and Freedom'', Scientific American (May 1968): 148152.} Até o final da década
de 1950, as emergentes tecnologias eletrônicas, tanto computadores quanto
comunicação, destruíram esse equilíbrio. A sociedade não podia mais se ajustar
pragmaticamente porque as tecnologias de vigilância estavam se desenvolvendo
muito rapidamente.

O resultado foi um estudo importante sobre privacidade realizado pela
Associação dos Advogados da Cidade de Nova York, que culminou na publicação, em
1967, do livro de Alan Westin, intitulado \ingles{``Privacy and
Freedom''}.\footnote{ Alan F. Westin, Privacy and Freedom (Atheneum, 1967).}
(Fano estava revisando o livro de Westin quando descreveu o quadro de
desequilíbrio social causado pela rápida mudança tecnológica.) Westin propôs
uma mudança crucial de foco.

Brandeis e Warren haviam visto a perda de privacidade como uma forma de lesão
pessoal, que poderia ser tão grave a ponto de causar ``dor mental e angústia,
muito maiores do que poderiam ser infligidas por mera lesão corporal''. Os
indivíduos precisavam assumir a responsabilidade de se proteger. ``Cada pessoa
é responsável apenas por seus próprios atos e omissões''. Mas a lei tinha que
fornecer as armas para resistir a invasões de privacidade.

Westin reconheceu que a formulação de Brandeis-Warren era muito absoluta,
diante dos direitos de expressão de outras pessoas e das práticas legítimas de
coleta de dados da sociedade. A proteção poderia não vir de escudos protetores,
mas do controle sobre os usos que poderiam ser feitos das informações pessoais.
``Privacidade'', escreveu Westin, ``é o direito de indivíduos, grupos ou
instituições determinarem por si mesmos quando, como e em que medida
informações sobre eles são comunicadas a outros.'' Westin propôs:

\begin{quote}
    \ldots O que é necessário é um processo estruturado e racional de ponderação,
    com critérios definidos, que as autoridades públicas e privadas possam aplicar
    ao comparar o pedido de vigilância ou divulgação por meio de novos
    dispositivos, com a pedido de privacidade. Os seguintes passos são
    sugeridos como a base desse processo: avaliar a gravidade da necessidade de
    conduzir a vigilância; decidir se existem métodos alternativos para atender
    à necessidade; decidir qual grau de confiabilidade será exigido do instrumento
    de vigilância; determinar se foi concedido um verdadeiro consentimento para a
    vigilância; e avaliar a capacidade de limitação e controle da vigilância se
    for permitida.\footnote{Ibid.}
\end{quote}

Portanto, mesmo que houvesse uma razão legítima pela qual o governo ou outra
parte pudesse saber algo sobre você, seu direito à privacidade poderia limitar
o que a parte informada poderia fazer com essa informação.

Essa compreensão mais sutil da privacidade surgiu dos importantes papéis
sociais que a privacidade desempenha. Privacidade não é, como Warren e Brandeis
afirmaram, o direito de ser isolado da sociedade; privacidade é um direito que
faz a sociedade funcionar.

Fano mencionou três papéis sociais da privacidade. Primeiro, ``o direito de
manter a privacidade de sua personalidade pode ser considerado parte do direito
de autodefesa''---o direito de guardar para si mesmo os julgamentos equivocados
da adolescência e os conflitos pessoais, desde que não tenham uma importância
duradoura para sua posição final na sociedade. Segundo, a privacidade é a forma
como a sociedade permite desvios das normas sociais predominantes, considerando
que nenhum conjunto de normas sociais é universal e permanentemente
satisfatório---e, de fato, dado que o progresso social requer experimentação
social. E terceiro, a privacidade é essencial para o desenvolvimento do
pensamento independente; ela permite que o indivíduo se desvincule da sociedade
de forma que os pensamentos possam ser compartilhados em círculos limitados e
ensaiados antes de serem expostos publicamente.

A filósofa Helen Nissenbaum também fundamenta a privacidade no ser social,
descrevendo-a como ``integridade contextual''.\footnote{Helen Nissenbaum,
Privacy in Context: Technology, Policy, and the Integrity of Social Life
(Stanford Law Books, 2009).} A privacidade depende de uma correspondência entre
o fluxo de dados e as expectativas e normas do ambiente em que as informações
foram geradas e compartilhadas. Quando o Facebook o convida a adicionar seu
terapeuta ou um colega paciente como amigo, isso representa uma violação de
contexto. Os espaços online oferecem a oportunidade de multiplicar contextos:\\

%%%%%%%%%%%%%%%%%%%%%%%%%%%%%%%% BLOCO IMAGEM %%%%%%%%%%%%%%%%%%%%%%%%%%%%%%%%
A privacidade é a forma como a sociedade permite desvios das normas sociais
predominantes, considerando que o progresso social requer experimentação
social.\\
%%%%%%%%%%%%%%%%%%%%%%%%%%%%%%%% BLOCO IMAGEM %%%%%%%%%%%%%%%%%%%%%%%%%%%%%%%%

Você pode ser uma pessoa em sua conta do Instagram e outra na sala de aula. Mas
os espaços online também ameaçam o colapso de contexto, como Stacy Snyder
descobriu há muito tempo, na era do \ingles{Myspace}, quando sua foto com a
legenda ``pirata bêbada'', em uma postagem que ela pensava ser apenas social,
custou-lhe um diploma de ensino.\footnote{``Judge Sides with University Against
Student-Teacher with `Drunken Pirate' Photo'', The Chronicle of Higher
Education, December 4, 2008,
\url{https://www.chronicle.com/article/Judge-Sides-With-University/42066}.}

O crescimento explosivo das tecnologias digitais alterou radicalmente nossas
expectativas sobre o que será privado e mudou nosso pensamento sobre o que
\emph{deveria} ser privado. Isso tornou as violações de privacidade mais fáceis
e potencialmente mais numerosas. De fato, é notável que já não nos
surpreendemos com invasões que há uma década teriam parecido chocantes. Ao
contrário da história do segredo, não houve um único evento tecnológico que
tenha causado essa mudança, nenhuma descoberta revolucionária da
privacidade---apenas um avanço constante em várias frentes tecnológicas que,
finalmente, ultrapassou um ponto de inflexão.

Os dispositivos sensoriais se tornaram mais baratos, melhores e menores.
Pequenas câmeras, unidades de GPS e microfones passaram de objetos de museus de
espionagem para a banalidade do uso diário. Uma vez que eles se tornaram bens
de consumo úteis, parecemos nos preocupar menos com seus usos como dispositivos
de vigilância. Em vez de tentar criar uma teoria unificada sobre a privacidade
e seu valor, nos encontramos montando a privacidade a partir de sentimentos de
desconforto e arrependimento diante da abundância. Isso fica ainda mais difícil
quando somos nós mesmos que trazemos espiões para nossas próprias casas e das
nossas amizades, quando trocamos a privacidade por convivialidade e
conveniência.

\subsection{Sorria enquanto tiramos a foto}
\label{cap3:quem-que-sorria}

\ingles{Big Brother} tinha suas legiões de câmeras, e a Cidade de Londres tem
as suas hoje em dia. Mas em termos de onipresença fotográfica, nada supera as
câmeras nos celulares nas mãos das pessoas comuns. Ao voar antes do feriado de
Quatro de Julho, Helen foi solicitada a trocar de lugar com outra mulher que
queria sentar ao lado do namorado. Ela ocupou o assento uma fileira acima e começou
uma conversa com seu novo companheiro de assento, sem saber que a fileira de
trás estava filmando-os como um casal apaixonado. O casal que ela ajudou começou
a twittar sobre o voo, usando a hashtag \#PlaneBae, e a história logo foi parar
nos programas de televisão matinais. Pode parecer diversão inocente, mas não para
Helen, que afirmou (por meio de advogados),

\begin{quote}
    Sem o meu conhecimento ou consentimento, outros passageiros me fotografaram e
    gravaram minha conversa com uma pessoa ao lado. Eles postaram imagens e gravações
    nas redes sociais e especularam injustamente sobre minha conduta privada.

    Desde então, minhas informações pessoais foram amplamente divulgadas online.
    Estranhos discutiram publicamente minha vida privada com base em informações
    claramente falsas.

    Eu fui vítima de doxing, envergonhamento, insultos e assédio.~\ingles{Voyeurs}
    vieram me procurar online e no mundo real.\footnote{Taylor Lorenz,
    ``Unidentified Plane-Bae Woman's Statement Confirms the Worst'', The Atlantic,
    July 13, 2018,
    \url{https://www.theatlantic.com/technology/archive/2018/07/unidentified-plane-bae-womans-statement-confirms-the-worst/565139/}.}
\end{quote}

%Aqui eu não soube traduzir corretamente o termo "Little-Brotherism",
%Portanto, optei por manter o termo original.
A disseminação massiva de câmeras baratas, combinada com o acesso universal à
Internet, possibilita uma espécie de justiça vigilante---um onipresente
\ingles{``Little-Brotherism''}---, no qual todos podemos ser detetives, juízes
e oficiais correcionais. Blogueiros podem chamar a atenção global para cidadãos
comuns.

Para cada lente direcionada intencionalmente, há muitas outras observando sem
serem notadas: observação e vigilância pública e privada. As ruas principais
estão repletas de câmeras de segurança espiando pelas vitrines das lojas e
câmeras de vigilância da polícia, algumas das quais até mesmo oferecem
visualização pública. Até nos bairros mais tranquilos você pode estar sendo
observado, graças às redes de segurança ``Ring'' e vizinhos vigilantes nos
grupos do aplicativo ``Nextdoor''. Aliado ao reconhecimento facial
automatizado, as ruas conectadas podem estar compilando dossiês sobre todos
nós.

Olhar imagens na Web é agora uma atividade de lazer que qualquer pessoa pode
fazer a qualquer momento e em qualquer lugar do mundo. Usando o Google Street
View, você pode sentar em um café no Tajiquistão e identificar um carro que
estava estacionado em minha garagem quando a câmera do Google passou por lá
(talvez meses atrás). De Seul, você pode ver o que está acontecendo agora
mesmo, atualizado a cada poucos segundos, em Piccadilly Circus ou na avenida de
Las Vegas. Essas vistas sempre estiveram disponíveis ao público, mas câmeras e
a Internet mudam o significado de ``público''.

Algumas das invasões à nossa privacidade ocorrem devido aos efeitos colaterais
inesperados e invisíveis das coisas que fazemos voluntariamente. Embora a
Quarta Emenda proteja contra o excesso de vigilância governamental, nos Estados
Unidos há apenas uma consideração legal fragmentada sobre a coleta de
informações privadas. Empresas rotineiramente coletam e inferem informações
sobre indivíduos e as utilizam para personalizar ofertas de produtos e
anúncios. Como diz o ditado, se você não está pagando, você é o produto.

\section{Pegadas e impressões digitais}
\label{cap3:quem-pegadas}
Enquanto conduzimos nossos negócios diários e levamos nossas vidas privadas,
deixamos pegadas e impressões digitais. Podemos ver nossas pegadas na lama no
chão, na areia, e na neve ao ar livre. Não nos surpreenderíamos se alguém se
desse ao trabalho de combinar nossos sapatos com nossas pegadas e pudesse
determinar, ou supor, onde estivemos. Impressões digitais são diferentes. Nem
sequer nos ocorre que estamos deixando-as enquanto abrimos portas e bebemos em
copos. Aqueles que têm consciência culpada podem pensar sobre as impressões
digitais e se preocupar com onde estão deixando-as, mas o resto de nós não.\\

%%%%%%%%%%%%%%%%%%%%%%%%%%%%%%%% BLOCO IMAGEM %%%%%%%%%%%%%%%%%%%%%%%%%%%%%%%%
O OLHAR INDESEJADO

O livro \ingles{``The Unwanted Gaze''} de Jeffrey Rosen (Vintage, 2000) detalha
muitas maneiras pelas quais o sistema legal tem contribuído para a nossa perda
de privacidade.\\
%%%%%%%%%%%%%%%%%%%%%%%%%%%%%%%% BLOCO IMAGEM %%%%%%%%%%%%%%%%%%%%%%%%%%%%%%%%

No mundo digital, todos nós deixamos pegadas eletrônicas e impressões
digitais---trilhas de dados que deixamos intencionalmente e trilhas de dados
das quais não estamos cientes ou inconscientes. Os dados identificadores podem
ser úteis para fins forenses. No entanto, como a maioria de nós não se
considera criminoso, tendemos a não nos preocupar com isso. O que não
consideramos é que as várias pequenas marcas que deixamos na paisagem digital
podem ser úteis para outra pessoa---alguém que queira usar os dados que
deixamos para ganhar dinheiro ou obter algo de nós. Portanto, é importante
entender como e onde deixamos essas pegadas digitais e impressões digitais.

\subsection{Papel rastreador}
\label{cap3:quem-pegadas-papel}
Se eu enviar um e-mail ou baixar um conteúdo da web, não deve ser surpresa que eu
tenha deixado algumas pegadas digitais. Afinal, os bits têm que chegar até mim,
então alguma parte do sistema sabe onde estou. Nos velhos tempos, se eu quisesse
ser anônimo, poderia escrever uma nota, mas minha caligrafia poderia ser reconhecível,
e eu poderia deixar impressões digitais (do tipo oleoso) no papel. Eu poderia ter
digitado, mas Perry Mason regularmente resolvia crimes combinando uma nota digitada
com a assinatura única da máquina de escrever do suspeito. Mais impressões digitais.

Então, hoje em dia eu poderia imprimir a carta a laser e usar luvas. Mas mesmo
isso pode não ser suficiente para me disfarçar. Pesquisadores da Universidade
Purdue desenvolveram técnicas para associar a saída impressa a laser a uma
impressora específica.\footnote{Emil Venere, ``Printer Forensics to Aid
Homeland Security, Tracing Counterfeiters'', Purdue University, October 12,
2004,
\url{https://www.purdue.edu/uns/html4ever/2004/041011.Delp.forensics.html}.}
Eles analisam as folhas impressas e detectam características únicas de cada
fabricante e cada impressora individual---impressões digitais que podem ser
usadas, assim como as manchas das antigas teclas de máquinas de escrever, para
associar a saída à fonte. Pode ser desnecessário colocar o microscópio em
letras individuais para identificar qual impressora produziu uma página.

A \ingles{Electronic Frontier Foundation} demonstrou que muitas impressoras
coloridas codificam quase invisivelmente o número de série, a data e a hora da
impressora em cada página que imprimem (veja a Figura 3.1). Portanto, quando
você imprime um relatório, não deve assumir que ninguém pode saber quem o
imprimiu.\\

%%%%%%%%%%% AQUI FICA A FIGURA 3.1 %%%%%%%%%%%%%%%
FIGURA 3.1\\
%%%%%%%%%%% AQUI FICA A FIGURA 3.1 %%%%%%%%%%%%%%%

FIGURA 3.1: Impressão digital deixada por uma impressora a laser colorida Xerox
DocuColor 12. Os pontos são muito difíceis de ver a olho nu; a fotografia foi
tirada sob luz azul. O padrão de pontos codifica a data (2005--05--21), o
horário (12:50) e o número de série da impressora (21052857).

Havia uma razão sensata por trás dessa tecnologia. O governo queria garantir
que as impressoras de escritório não pudessem ser usadas para produzir notas
falsas de cem dólares. A tecnologia que foi criada para frustrar falsificadores
torna possível rastrear cada página impressa em impressoras a laser coloridas
até a fonte. Tecnologias úteis muitas vezes têm consequências não intencionais.

Muitas pessoas, por razões perfeitamente legais e válidas, gostariam de
proteger sua anonimidade. Elas podem ser denunciantes ou dissidentes. Talvez
estejam apenas protestando contra injustiças em seus locais de trabalho. Será
que as tecnologias que enfraquecem o anonimato no discurso político também
sufocarão a liberdade de expressão? Um certo grau de anonimato é essencial em
uma democracia saudável---e nos Estados Unidos, a anonimidade tem sido uma arma
usada para promover a liberdade de expressão desde a época da Revolução.
Podemos lamentar o completo abandono da anonimidade em favor de tecnologias de
comunicação que deixam pegadas.

O problema não é apenas a existência das impressões digitais, mas sim que
ninguém nos disse que estamos criando-as.\\

%%%%%%%%%%%%%%%%%%%%%%%%%%%%%%%% BLOCO TEXTO %%%%%%%%%%%%%%%%%%%%%%%%%%%%%%%%
O problema não é apenas a existência das impressões digitais, mas sim que
ninguém nos disse que estamos criando-as.\\
%%%%%%%%%%%%%%%%%%%%%%%%%%%%%%%% BLOCO TEXTO %%%%%%%%%%%%%%%%%%%%%%%%%%%%%%%%

Quando a contratante da \ingles{NSA (National Security Agency)}, Reality
Winner, vazou informações classificadas para o \ingles{The Intercept}, talvez
ela tenha pensado que enviar uma cópia em papel impediria tentativas de
rastrear as origens do vazamento.\footnote{Michael M. Grynbaum and John Koblin,
``Journalists Fear Effects of Arrest'', New York Times, June 7, 2017: A19,
\url{https://www.nytimes.com/2017/06/06/business/media/intercept-reality-winner-russia-trump-leak.html}.}
O The Intercept havia compartilhado o documento com a NSA para verificar sua
autenticidade, e Winner foi presa poucos dias depois. Relatos iniciais
especularam que ela foi rastreada através de microdots da impressora, mas a
verdade parece ter sido ainda mais banal: os registros da NSA mostraram que
apenas seis contas, incluindo a de Winner, tiveram acesso ao documento, e
Winner havia usado uma conta pessoal para entrar em contato com o The Intercept
pouco antes disso.\footnote{Jake Swearingen, ``Did the Intercept Betray Its NSA
Source?'', New York Magazine, June 6, 2017.
\url{https://nymag.com/intelligencer/2017/06/intercept-nsa-leaker-reality-winner.html}.}

\subsection{Publicidade}
\label{cap3:quem-pegadas-publicidade}
Se você utilizar o metrô T em Boston, verá muitos anúncios de programas
universitários e de pós-graduação. Todos eles têm números de telefone e URLs,
e muitos direcionam você para páginas como ``college.edu/recruiting/redline''.
O endereço da web não está informando que a faculdade possui um programa
especial na Linha Vermelha (nome de uma linha do metrô), mas sim que eles têm
um programa especial de publicidade lá. O termo ``redline'' no final do URL
permite que a faculdade saiba que você foi encaminhado para essa página pelo
anúncio no metrô. Eles podem usar essa informação para direcioná-lo aos
programas específicos anunciados no cartaz e para acompanhar a eficácia dessa
campanha publicitária.

Anúncios na Web utilizam a página de referência como apenas um dos muitos
indicadores; outros são menos visíveis do que a decoração de URL visível no
cartaz do metrô. Quando você segue um link para abrir uma página da web em seu
navegador, esse clique inicia uma série de eventos que começa com uma
solicitação eletrônica da página da web e um pedido de quaisquer cookies que o
site possa ter configurado anteriormente. Todas as páginas, exceto as mais
simples, acionarão solicitações de mais sub-recursos: imagens, fontes, scripts
para tornar a página dinâmica. Um site comercial pode ter dezenas de anúncios e
pixels de rastreamento, ou ``bugs da web''---elementos invisíveis que fazem seu
computador se conectar a outra fonte com o objetivo de rastrear suas
atividades.\\

%%%%%%%%%%%%%%%%%%%%%%%%%%%%%%%% BLOCO IMAGEM %%%%%%%%%%%%%%%%%%%%%%%%%%%%%%%%
COMO SITES SABEM QUEM VOCÊ É (UMA LISTA INCOMPLETA)

\textbf{1. Você informa a eles.} Ao fazer login no Gmail, Amazon ou eBay, você
está dizendo exatamente quem você é para eles.

\textbf{2. Eles deixaram cookies em uma de suas visitas anteriores.} Um cookie
é um pequeno arquivo de texto armazenado em seu disco rígido local que contém
informações que um determinado site deseja ter disponível durante sua sessão
atual (por exemplo, sobre seu carrinho de compras) ou de uma sessão para outra.
Os cookies fornecem informações persistentes aos sites para rastreamento e
personalização. Seu navegador tem um comando para mostrar cookies; se você o
usar, poderá se surpreender com quantos sites os deixaram!

\textbf{3. Eles têm o endereço IP do seu dispositivo.} O servidor da web precisa
saber onde você está para enviar suas páginas da web para você. Seu endereço IP
é um número como 66.82.9.88 que localiza seu computador na Internet. Esse
endereço pode mudar de um dia para o outro. Mas, em um ambiente residencial, seu
provedor de serviços de Internet (ISP;~geralmente a empresa de telefone ou cabo)
sabe quem foi atribuído a cada endereço IP em qualquer momento. Esses registros
são frequentemente solicitados em casos judiciais.

\textbf{4. Você se parece com alguém que eles já reconhecem.} Eles identificam
pessoas que têm características semelhantes. Quando os usuários fazem login no
Facebook, eles compartilham detalhes sobre suas vidas, interesses e conexões
sociais. Com base nesses dados, o Facebook cria grupos de usuários que têm
características em comum, mesmo que essas informações não tenham sido fornecidas
diretamente por cada usuário. Esses grupos são chamados de \ingles{``shadow audiences''}.

\textbf{5. Eles criaram uma impressão digital do seu navegador e o vincularam a
perfis de visitas anteriores.} Os sites podem acessar muitos detalhes aparentemente
inofensivos sobre o seu navegador (tipo, versão, codificação gráfica, idioma e
muito mais). Essas informações tendem a ser relativamente estáticas e, muitas
vezes, identificam de forma única uma instância específica do navegador.
Essa técnica é simples, mas surpreendentemente precisa e eficaz.\\
%%%%%%%%%%%%%%%%%%%%%%%%%%%%%%%% BLOCO IMAGEM %%%%%%%%%%%%%%%%%%%%%%%%%%%%%%%%

Se você está curioso para saber quem está usando um determinado endereço IP,
você pode verificar o \ingles{American Registry of Internet Numbers}
(\url{www.arin.net}). Serviços como \url{whatismyip.com}, \url{whatismyip.org}
e \url{ipchicken.com} também permitem que você verifique seu próprio endereço
IP.~E o \url{www.whois.net} permite que você verifique quem é o proprietário de
um nome de domínio, como \url{harvard.com}, que, como se constata, é a
\ingles{Harvard Bookstore}, uma livraria privada bem em frente à universidade.

Infelizmente, as informações do endereço IP não revelarão quem está enviando
spam para você, pois os \ingles{spammers} rotineiramente falsificam a origem
dos e-mails que enviam. Além disso, entre o momento em que você solicita uma
página da web e seus anúncios são exibidos em seu navegador, geralmente ocorre
um leilão em tempo real, no qual seus ``olhos'' (ou seja, os espaços de
anúncios na página da web que seu navegador está prestes a exibir) são vendidos
para o maior lance. Redes de publicidade coletam informações de pixels de
rastreamento e contexto da página para determinar quais anúncios oferecer e
quanto oferecer para colocá-los nesses leilões.

Por que esses sapatos estão me perseguindo? Talvez você os tenha visto no
Instagram, marcado-os no Pinterest ou procurado por um novo par de tênis no
site da sua loja favorita. Talvez você até tenha colocado os sapatos em um
carrinho de compras antes de decidir que eles não cabiam no seu orçamento no
momento. Agora, parece que você não consegue escapar dos sapatos: seja lendo
notícias ou conversando com amigos no Facebook, lá estão os sapatos,
perseguindo você nos banners de anúncios, incentivando você a clicar em
``comprar''.

Conhecidos no mercado como \ingles{``retargeting''}, esses anúncios são alguns
dos produtos de lances em tempo real. O profissional de marketing que inseriu
um cookie de rastreamento em seu navegador durante uma sessão de navegação
anterior ou a visita de compras que você interrompeu está usando-o para
identificá-lo como um comprador interessado em sapatos e fazendo lances para
mostrar esses anúncios na esperança de atraí-lo de volta para a compra. Se você
clicar em qualquer um dos anúncios, o profissional de marketing registrará uma
``conversão'' e levará esses dados em consideração para o seu perfil, visando
futuras oportunidades de anúncios.

Os usuários de navegação na web não aceitaram tudo isso tranquilamente. A
revista \ingles{The Economist} chama os dados de ``o novo petróleo'' e os
navegadores que não desejam ser vistos como fontes desse petróleo estão
baixando bloqueadores de anúncios. Desde o início de 2020, todos os principais
navegadores da web incorporaram recursos de bloqueio de rastreadores ou
anunciaram planos para limitar cookies de terceiros.

Arvind Narayanan e sua equipe da Universidade de Princeton criaram um
laboratório para medição na web\footnote{``Web Privacy—Arvind Narayanan'',
accessed May 18, 2020,
\url{https://www.cs.princeton.edu/~arvindn/web-privacy/}.} e descobriram novas
técnicas de rastreamento de navegadores. Por meio de \ingles{``crawls''} na
web, eles encontraram técnicas de rastreamento usadas no mundo real para
identificar usuários e reidentificar aqueles que pensaram que haviam eliminado
todas as interações anteriores. Um dos paradoxos da privacidade na web é que os
navegadores podem ser identificados pela sua ``impressão digital'', por conta
de suas características únicas, incluindo recursos que o usuário pode ativar
com o objetivo de obter maior privacidade. Isso significa que ativar tais
proteções pode fazer com que o usuário em busca de privacidade acabe por se
destacar. Em tais casos, a privacidade depende das ações de muitos para
fornecer um grupo no qual o navegador em busca de privacidade possa se
misturar. Processos padronizados e configurações padrão bem pensadas são
necessárias para preservar as oportunidades de privacidade.

\subsection{A Target sabe que você está grávida}
\label{cap3:quem-pegadas-target}
Em 2012, conforme relatado por Charles Duhigg no \ingles{New York Times},\footnote{
Charles Duhugg, ``How Companies Learn Your Secrets'', The New York Times, February 16, 2012,
\url{https://www.nytimes.com/2012/02/19/magazine/shopping-habits.html}.}
um homem entrou em uma loja da Target na região de Minneapolis e pediu para falar
com o gerente, furiosamente dizendo: ``Minha filha recebeu isso pelo correio!
Ela ainda está no ensino médio, e vocês estão enviando cupons para roupas de
bebê e berços? Vocês estão tentando incentivá-la a engravidar?''

O gerente da loja pediu desculpas ao homem de Minneapolis pelo aparente erro
deles, mas ele retornou algumas semanas depois com um pedido de desculpas
próprio: sua filha estava, de fato, grávida. Os modelos preditivos da Target
haviam reconhecido a gravidez da jovem mulher mesmo antes que seu pai soubesse.
Os modelos da Target não tinham acesso às informações privadas dela. Eles
tinham o poder de ferramentas analíticas e dados prontamente disponíveis.

Assim como muitas outras lojas com cartões de fidelidade ou contas de usuário,
a Target construiu modelos estatísticos do comportamento dos compradores para
prever produtos populares para estoque e precificação, e para fazer
recomendações. A Target correlacionava o histórico de compras dos clientes com
base em um ID de hóspede interno e adquiria dados externos para complementar
seus registros. A partir desses registros, o estatístico da empresa poderia
derivar padrões, percebendo, por exemplo, que mulheres no segundo trimestre da
gravidez frequentemente compravam loções hidratantes sem perfume e suplementos.
Depois de observar esse padrão muitas vezes, a loja podia antecipar futuras
compras de roupas de bebê e fraldas com base nas compras anteriores de loção
sem perfume---e anunciar para a futura mãe em um momento em que seus hábitos de
compra estavam em mudança---respondendo a um sinal que ela nem sabia que estava
enviando.

Como podemos resolver um problema de privacidade que resulta de muitos
desenvolvimentos, mas nenhum deles é realmente um problema em si?

\subsection{Você paga pelo microfone, nós apenas ouvimos}
\label{cap3:quem-pegadas-voce}
Plantar microfones minúsculos onde poderiam captar conversas de figuras do
submundo costumava ser um trabalho arriscado para as autoridades federais. Agora,
existem alternativas muito mais seguras, já que a maioria das pessoas carrega
seus próprios microfones equipados com rádio o tempo todo ou convida Alexa,
Siri, Cortana ou Google para suas casas.

Muitos celulares podem ser reprogramados remotamente para que o microfone
esteja sempre ligado e o telefone esteja transmitindo, mesmo que você pense que
o desligou. O FBI usou essa técnica em 2004 para ouvir as conversas de John
Tomero com outros membros de sua família do crime organizado. Um tribunal
federal considerou que esse \ingles{roving bug}, instalado após devida
autorização, constituía uma forma legal de escuta telefônica. Tomero poderia
ter impedido isso removendo a bateria, e agora alguns executivos de negócios
apreensivos, rotineiramente fazem exatamente isso.

O microfone em um carro da \ingles{General Motors} equipado com o sistema
\ingles{OnStar} também pode ser ativado remotamente, uma funcionalidade que
pode salvar vidas quando os operadores do OnStar entram em contato com o
motorista após receberem um sinal de colisão. O OnStar adverte: ``O OnStar
cooperará com ordens judiciais oficiais relacionadas a investigações criminais
das forças policiais e de outras agências'', e de fato, o FBI usou esse método
para interceptar conversas realizadas dentro dos carros. Em um caso, um
tribunal federal se opôs a essa forma de coleta de provas---mas não por
questões de privacidade. O roving bug desativou o funcionamento normal do
OnStar, e o tribunal simplesmente considerou que o FBI interferiu no direito
contratual do proprietário do veículo de conversar com os operadores do OnStar!

Danielle, cliente do Amazon Echo em Portland, Oregon, ficou assustada ao
receber uma ligação de um colega de trabalho de seu marido, alertando que ela
estava sendo hackeada.\footnote{``Amazon Alexa Can Accidentally Record and
Share Your Conversations'', Vanity Fair, May 24, 2018,
\url{https://www.vanityfair.com/news/2018/05/yes-amazons-alexa-can-secretly-record-and-share-conversations}.}
O dispositivo, que deveria gravar apenas após o comando ``Alexa'', captou
também um comando para enviar mensagem durante a conversa de Danielle. Sua
discussão sobre pisos de madeira se transformou em uma mensagem de voz para um
conhecido comercial. Evento incomum, mas que pode se repetir à medida que
gravadores em rede se tornam mais comuns. Autoridades alemãs proibiram a boneca
falante ``My Friend Cayla''\footnote{Katie Collins, ``That Smart Doll Could be
a Spy. Parents, Smash!'', CNET, February 17, 2018,
\url{https://www.cnet.com/tech/computing/parents-told-to-destroy-connected-dolls-over-hacking-fears/}.}
por temores de espionagem e coleta de dados. Cayla enviava sons pela internet
para interagir com crianças. Pais foram instruídos a destruir o ``aparelho de
espionagem ilegal''. Enquanto isso, nos Estados Unidos, sua smart TV pode estar
observando seus hábitos de visualização para personalizar a publicidade. O CTO
da Vizio disse na \ingles{Consumer Electronics Show} que as TVs custariam mais
se não fosse por essa fonte de receita.\footnote{ Ben Gilbert, ``There's a
simple reason your new smart TV was so affordable: It's collecting and selling
your data, and serving you ads'', Business Insider, April 5, 2019,
\url{https://www.businessinsider.com/smart-tv-data-collection-advertising-2019-1}.}

\subsection{Venmo: Tudo se soma}
\label{cap3:quem-pegadas-venmo}
Anteriormente, discutimos o rastreamento que os cartões de crédito permitem em
agências de relatórios de crédito e empresas de análise de dados. Novas
tecnologias de pagamento trazem o relatório diretamente para você. O Venmo
permite que você envie dinheiro a alguém ou divida uma conta inserindo o número
de telefone da pessoa. É tão fácil que ao enviar dinheiro para amigos ou
colegas de quarto usando o aplicativo Venmo, você pode não perceber que essas
transações de pagamento são públicas, incluindo qualquer nota que você escreva
junto com o pagamento.

Um pesquisador encontrou padrões em algumas das milhões de transações no
``Venmo stories'':\footnote{Hang Do Thi Duc, Public By Default, Venmo Stories
of 2017, \url{https://publicbydefault.fyi/}.} o hábito de fast food de um
estudante, as vendas de um vendedor de cannabis, um relacionamento em
desenvolvimento? Você pode não se importar em compartilhar sua paixão por elote
(milho temperado), mas pode se sentir diferente em relação a compras
recreativas de maconha, mesmo em estados onde isso é legal. O pesquisador, Hang
Do Thi Duc, anonimizou os detalhes, mas observa que o feed, que inclui tudo,
exceto os valores em dólares, permaneceu acessível para qualquer visitante da
API pública do Venmo. (Cada página do site desenvolvido por Duc,
\url{publicbydefault.fyi}, incentiva os usuários do Venmo a alterarem suas
configurações de privacidade do padrão para tornar as transações privadas entre
remetente e destinatário.)

\subsection{DNA:~A Impressão Digital Definitiva}
\label{cap3:quem-pegadas-dna}
Em abril de 2018, o estado da Califórnia acusou Joseph James DeAngelo de uma
série de crimes de assassinato e estupro ocorridos décadas atrás. O Assassino
do Estado Dourado havia sido um caso antigo até que um investigador carregou
DNA de uma cena de crime em um site público de genealogia, o GEDmatch. O
investigador criou um perfil falso para o assassino desconhecido cujo DNA
foi recuperado na cena do crime, e carregado no site. Depois que o GEDmatch comparou o DNA dessa pessoa
com seu banco de dados existente para identificar correspondências genéticas
parciais, ele mostrou perfis de pessoas que provavelmente eram parentes
distantes do assassino. Esses nomes levaram a árvores genealógicas e à
genealogia que pôde ser rastreada ainda mais através de registros censitários,
obituários, sepulturas e bancos de dados comerciais e de órgãos de aplicação da
lei. Após essas pesquisas revelarem o nome do criminoso, os investigadores
confirmaram suas suspeitas rastreando-o e obtendo outra amostra de DNA, a
partir de células de pele que ele deixou na porta do carro quando estacionou
em um estacionamento do Hobby Lobby. Esse DNA correspondeu às amostras
originais da cena do crime.\footnote{Avi Selk, ``The ingenious and `dystopian'
DNA technique police used to hunt the `Golden State Killer' suspect'',
Washington Post, April 28, 2018,
\url{https://www.washingtonpost.com/news/true-crime/wp/2018/04/27/golden-state-killer-dna-website-gedmatch-was-used-to-identify-joseph-deangelo-as-suspect-police-say/}.}

DeAngelo não havia postado no site de ancestralidade, mas como um dos pais
passa aproximadamente metade de seus genes para uma criança (com algumas
mutações ao longo do caminho), grande parte do registro genético de DeAngelo
poderia ser lida ou revelada por parentes. Se seus familiares explorarem seus
perfis genéticos e árvores genealógicas no GEDmatch, eles também estarão
expondo informações sobre traços que você pode compartilhar. Sua privacidade
pode ser invadida sem nenhuma ação sua. Embora a Lei de Não Discriminação de
Informações Genéticas proíba que empregadores ou seguradoras de saúde
discriminem com base no DNA, a lei não restringe as inúmeras outras maneiras
pelas quais o DNA pode ser usado.

O caso do Assassino do Estado Dourado iniciou um boom na genealogia forense de
DNA.~Até o final de 2018, mais de uma dúzia de criminosos violentos e
perpetradores de agressão sexual haviam sido identificados por meio do
GEDmatch. No entanto, o site também ouviu alarmes sobre privacidade e alterou
seus termos de serviço para proibir que as forças policiais correspondessem
perfis de DNA, a menos que os usuários optassem pôr seus próprios registros.

\section{Princípios de práticas de informação justas}
\label{cap3:quem-principios}
Uma antiga revolução da informação, que aconteceu em salas cheias de
unidades de disco que brotaram em prédios governamentais e corporativos
na década de 1960, desencadeou um debate sobre o significado prático dos
direitos de privacidade. O que, na prática, aqueles que detêm um grande
banco de dados devem considerar ao coletar, lidar e compartilhar os dados
com outros?

Em 1973, o Departamento de Saúde, Educação e Bem-Estar emitiu os ``Princípios
de práticas de informação justas'' (\ingles{Fair Information Practice Principles}),
conforme abaixo:

\begin{quote}
    \textbf{Transparência.} Não deve haver sistemas de registro de dados pessoais
    cuja própria existência seja secreta.

    \textbf{Divulgação.} Deve haver um meio para uma pessoa descobrir quais
    informações sobre ela estão em um registro e como elas são utilizadas.

    \textbf{Uso secundário.} Deve haver um meio para uma pessoa impedir que
    informações sobre ela, obtidas para um propósito, sejam usadas ou
    disponibilizadas para outros fins sem o consentimento da pessoa.

    \textbf{Correção.} Deve haver um meio para uma pessoa corrigir ou alterar um
    registro de informações identificáveis sobre ela.

    \textbf{Segurança.} Qualquer organização que crie, mantenha, utilize ou divulgue
    registros de dados pessoais identificáveis deve garantir a confiabilidade dos
    dados para seu uso pretendido e deve tomar precauções para evitar o uso indevido
    dos dados.
\end{quote}

Esses princípios foram propostos para os dados médicos nos Estados Unidos, mas
nunca foram adotados. No entanto, eles têm sido a base para muitas políticas de
privacidade corporativa. Variações desses princípios foram codificadas em
acordos comerciais internacionais pela Organização para a Cooperação e
Desenvolvimento Econômico (OCDE) em 1980 e dentro da União Europeia (UE) em
1995. Nos Estados Unidos, ecos desses princípios podem ser encontrados em
algumas leis estaduais, mas as leis federais geralmente tratam a privacidade
caso a caso, ou de maneira ``setorial''. A Lei de Privacidade de 1974 se aplica
a transferências de dados entre agências dentro do governo federal, mas não
impõe limitações ao manuseio de dados no setor privado. A Lei de Relatórios de
Crédito Justos se aplica apenas a dados de crédito do consumidor, mas não se
aplica a dados médicos. A Lei de Privacidade de Vídeo se aplica apenas a
locações de fitas de vídeo, mas não a downloads de filmes sob demanda, que não
existiam quando a lei foi promulgada. Por fim, poucas leis federais ou
estaduais se aplicam aos enormes bancos de dados nos arquivos e sistemas de
computador de cidades e municípios. O governo americano é descentralizado, e a
autoridade sobre os dados do governo também é descentralizada.

Os Estados Unidos não carecem de leis de privacidade. No entanto, a privacidade
tem sido legislada de maneira inconsistente e confusa e em termos dependentes
de contingências tecnológicas. Não há consenso nacional sobre o que deve ser
protegido e como as proteções devem ser aplicadas. Sem um julgamento coletivo
mais profundamente informado sobre os benefícios e custos da privacidade, o
atual mosaico legislativo pode piorar nos Estados Unidos.

A discrepância entre os padrões de privacidade de dados americanos e europeus
ameaçou a participação dos EUA no comércio internacional, porque uma diretiva
da UE proibiria transferências de dados para nações, como os Estados Unidos,
que não atendessem ao ``padrão de adequação'' europeu para a proteção da
privacidade. Em 2000, a Comissão Europeia criou um ``porto seguro'' para
empresas americanas com operações multinacionais, mas o Tribunal de Justiça
Europeu considerou inadequado para proteger os direitos dos cidadãos europeus.
Em 2016, a FTC desenvolveu uma alternativa, o \ingles{Privacy Shield}, com uma
diferença de aplicação saliente: ``Embora aderir ao Privacy Shield Framework
seja voluntário, uma vez que uma empresa elegível faça o compromisso público de
cumprir os requisitos do Framework, o compromisso se tornará aplicável sob a
lei dos EUA''.\footnote{``Fact Sheet: Overview of the EU-U.S. Privacy Shield
Framework for Interested Participants'', U.S. Department of Commerce, July 12,
2016,
\url{https://2014-2017.commerce.gov/sites/commerce.gov/files/media/files/2016/fact_sheet-_eu-us_privacy_shield_7-16_sc_cmts.pdf}.}

Em 2020, o Tribunal de Justiça da União Europeia (TJUE) decidiu que mesmo o
Privacy Shield era inadequado, porque os dados de cidadãos europeus nos Estados
Unidos estariam sujeitos à vigilância do governo dos EUA.\footnote{Court of
Justice of the European Union, Press Release No 91/20, Luxembourg, July 16,
2020, Judgment in Case C-311/18: Data Protection Commissioner v Facebook
Ireland and Maximillian Schrems.
\url{https://curia.europa.eu/jcms/upload/docs/application/pdf/2020-07/cp200091en.pdf}.}

\subsection{Privacidade como um direito básico}
\label{cap3:quem-principios-basico}
Ao navegar na internet durante uma visita à Europa, você pode notar uma profusão
de pop-ups e banners. Parece que todos os sites desejam que você consinta ao uso
de cookies e ao ``processamento de seus dados'', supostamente para melhorar sua
experiência de navegação. Enquanto a lei europeia considera uma visão mais forte
da privacidade pessoal como um direito fundamental, os anunciantes europeus estão
tão ansiosos quanto aqueles nos Estados Unidos para coletar dados pessoais. Esses
banners são meios de solicitar ``consentimento para o processamento de dados'',
conforme exigido pela Diretiva de Privacidade Eletrônica.

Em 2018, o Regulamento Geral de Proteção de Dados (GDPR) estabeleceu direitos
individuais específicos em relação a dados pessoais e obrigou as empresas a dar
às pessoas (``titulares de dados'') a capacidade de controlar o uso desses
dados. Aqueles que coletam ou processam dados pessoais devem ser capazes de
justificar a intrusão na privacidade com base no consentimento ou em outro
``propósito legítimo''; por exemplo, um provedor de e-mail precisa dos
endereços de e-mail dos seus contatos para enviar e-mails ao destino, mas não
precisa dos endereços residenciais deles. Os indivíduos têm até o direito de
retirar o consentimento, exigindo que os provedores apaguem os dados coletados
sobre eles. Como o GDPR afirma ter alcance extraterritorial, se aplica a
cidadãos europeus onde quer que estejam fisicamente localizados, muitos
provedores fora da Europa também adotaram solicitações de consentimento de
cookies e adaptaram o tratamento de dados para poder responder a solicitações
de exclusão de dados.

Apesar da promessa teórica da lei europeia, até 2020, a aplicação tem sido
limitada. Apenas uma multa significativa foi aplicada, contra o Google, no
valor de 50 milhões de euros (aproximadamente US\$ 54 milhões), ou cerca de um
décimo do que o Google gera em vendas de anúncios em um único dia. Sem a
investigação das centenas de reclamações apresentadas por cidadãos às suas
autoridades nacionais de proteção de dados, é difícil dizer se os europeus têm
mais privacidade online ou apenas mais pop-ups para clicar.

Infelizmente, é fácil debater se a abordagem abrangente europeia é mais
fundamentada do que a abordagem fragmentada dos EUA, quando a verdadeira
questão é se qualquer uma das abordagens alcança o que queremos que alcance. A
Lei de Privacidade de 1974 nos assegurou que declarações obscuras seriam
enterradas no Registro Federal, fornecendo o aviso oficial necessário sobre os
enormes planos de coleta de dados governamentais; foi melhor do que nada, mas
proporcionou ``transparência'' apenas em um sentido restrito e técnico. A
maioria das grandes corporações que fazem negócios com o público possui avisos
de privacidade, e praticamente ninguém os lê. Apenas 0,3\% dos usuários do
Yahoo! leram seu aviso de privacidade em 2002, por exemplo. No meio de uma
publicidade negativa em massa naquele ano, quando o Yahoo! alterou sua política
de privacidade para permitir mensagens publicitárias, o número de usuários que
acessaram a política de privacidade aumentou apenas para 1\%. Nenhuma das
muitas leis de privacidade dos EUA impediu o programa de escutas telefônicas
sem mandado instituído pela administração George W. Bush, nem a colaboração com
ele por parte das principais empresas de telecomunicações dos EUA.

De fato, a cooperação entre o governo federal e a indústria privada parece mais
essencial do que nunca para a coleta de informações sobre o tráfico de drogas e
o terrorismo internacional, devido a mais um desenvolvimento tecnológico. Há
vinte anos, a maioria das chamadas telefônicas de longa distância passava pelo
menos parte do tempo pelo ar, viajando por ondas de rádio entre torres de
antenas de micro-ondas ou entre o solo e um satélite de comunicação. Os agentes
do governo poderiam simplesmente ouvir. Agora, muitas chamadas telefônicas
passam por cabos de fibra óptica, e o governo está interceptando essa
infraestrutura de propriedade privada.

Altos padrões de privacidade têm um custo. Eles podem limitar a utilidade
pública dos dados. A preocupação pública com a divulgação de informações
médicas pessoais levou a grandes remédios legislativos. A Lei de Portabilidade
e Responsabilidade de Seguro de Saúde (HIPAA) tinha como objetivo incentivar o
uso da troca eletrônica de dados para informações de saúde e impor penalidades
severas para a divulgação de ``informações de saúde protegidas'', uma categoria
muito ampla que inclui não apenas históricos médicos, mas também, por exemplo,
pagamentos médicos. A lei exige a remoção de qualquer coisa que possa ser usada
para reconectar registros médicos à sua fonte. A HIPAA apresenta problemas em
um ambiente de dados onipresentes e computação poderosa.\\

%%%%%%%%%%%%%%%%%%%%%%%%%%%%%%%% BLOCO IMAGEM %%%%%%%%%%%%%%%%%%%%%%%%%%%%%%%%
\textbf{JÁ LEU AQUELES DOCUMENTOS DE ``CONCORDO''?}\\
As empresas podem fazer praticamente o que quiserem com suas informações,
contanto que você concorde. Parece difícil discordar desse princípio, mas
a situação pode ser desfavorável para o consumidor que está ``concordando''
com os termos da empresa. A \ingles{Sears Holding Corporation} (SHC), a
controladora da Sears, Roebuck e Kmart, deu aos consumidores a oportunidade
de se juntarem à ``Minha Comunidade da Sears Holding'', que a empresa
descreve como ``algo novo, algo diferente\ldots uma comunidade online
dinâmica e altamente interativa\ldots onde sua voz é ouvida e sua opinião
importa''. Quando você ia online para se inscrever, os termos apareciam em
uma janela na tela.

A caixa de rolagem continha apenas 10 linhas de texto, e o acordo era extenso e
volumoso. No meio dos termos, havia um detalhe: você estava permitindo que a
Sears instalasse software em seu PC que ``monitora todo o comportamento na
Internet que ocorre no computador\ldots, incluindo\ldots preencher um carrinho
de compras, completar um formulário de inscrição ou verificar suas\ldots
informações financeiras ou de saúde pessoal''. Assim, seu computador poderia
enviar seu histórico de crédito e resultados de testes de AIDS para a SHC, e
você havia dito que isso estava tudo bem!\\
%%%%%%%%%%%%%%%%%%%%%%%%%%%%%%%% BLOCO IMAGEM %%%%%%%%%%%%%%%%%%%%%%%%%%%%%%%%

Reunir informações de diferentes fontes e fazer conexões entre elas torna muito
difícil alcançar o nível de anonimato que a HIPAA pretendia garantir. No
entanto, você pode contar com assistência disponível, mediante pagamento, de
uma indústria inteira de consultores especializados em conformidade com a
HIPAA.~Se você fizer uma pesquisa online sobre a HIPAA, provavelmente
encontrará anúncios de serviços que o ajudarão a proteger seus dados e também a
evitar problemas legais.

Ao mesmo tempo em que o HIPAA e outras leis de privacidade têm protegido nossas
informações pessoais, eles tornaram a pesquisa médica cara e, por vezes,
impossível de conduzir. É provável que estudos clássicos como o Estudo do
Coração de Framingham, no qual muitas políticas públicas sobre doenças
cardíacas foram fundamentadas, não pudessem ser repetidos no ambiente atual de
regras de privacidade fortalecidas. A Dra. Roberta Ness, presidente do
\ingles{American College of Epidemiology}, relatou que ``há uma percepção de
que o HIPAA pode até estar tendo um efeito negativo nas práticas de vigilância
de saúde pública''.\footnote{``Privacy Rule Slows Scientific Discovery and Adds
Cost To Research, Scientists Say'', University of Pittsburgh Schools of the
Health Sciences.
\url{https://www.sciencedaily.com/releases/2007/11/071113165648.htm}. Also:
Roberta B. Ness, MD, MPH, ``Influence of the HIPAA Privacy Rule on Health
Research'', JAMA.~2007;298(18):2164--2170.~doi:10.1001/jama.298.18.2164.}

Os cinco princípios das Práticas Equitativas de Informação (FIPP) e o espírito
de transparência e controle pessoal que estão por trás deles, sem dúvida,
levaram a melhores práticas de privacidade. No entanto, eles foram
sobrecarregados pela explosão digital, juntamente com a insegurança do mundo e
todas as mudanças sociais e culturais que ocorreram na vida cotidiana. Fred H.
Cate, um estudioso da privacidade na \ingles{Indiana University}, caracteriza
os princípios FIPP como praticamente um fracasso completo:

\begin{quote}
    A lei moderna de privacidade costuma ser cara, burocrática, pesada e oferece
    surpreendentemente pouca proteção para a privacidade. Ela substituiu o controle
    individual das informações, que na verdade raramente é alcançado, pela proteção
    da privacidade. Em um mundo que está rapidamente se tornando mais global através
    das tecnologias de informação, do comércio multinacional e das viagens rápidas,
    as leis de proteção de dados tornaram-se mais fragmentadas e protecionistas.
    Essas leis se desvincularam de suas bases principais, e os princípios sobre os
    quais estão fundamentados se tornaram tão variados e procedimentais que nossa
    contínua repetição do mantra das Práticas Equitativas de Informação (FIPP) já
    não têm mais uma base sólida ou eficaz para sustentar a proteção da
    privacidade.\footnote{Fred H. Cate, ``The failure of Fair Information Practice
    Principles'', in Jane K. Winn, ed., Consumer Protection in the Age of the
    ``Information Economy'' (Ashgate, 2006).}
\end{quote}

Apenas seitas como os Amish ainda vivem sem eletricidade. É quase tão incomum
viver sem conectividade à internet, com todas as pegadas que ela deixa de suas
pesquisas diárias, logins e downloads. Até mesmo a antiga TV ``aberta'' está
desaparecendo rapidamente em favor das comunicações
digitais.\footnote{https://www.nielsen.com/pt/insights/2019/nielsen-local-watch-report-the-evolving-ota-home/}

A TV digital traz as vantagens do vídeo sob demanda, mas com um alto custo em
termos de privacidade. Seu provedor de serviço de televisão registra tudo o que
você assiste e quando. É tão atraente poder assistir ao que queremos quando
queremos, que não sentimos falta nem do inconveniente nem do anonimato dos dias
em que todas as emissoras de TV preenchiam sua casa com suas ondas de radio.
Você não podia escolher os horários de transmissão, mas pelo menos ninguém
sabia qual sinal você estava captando do ar.

\subsection{Privacidade como um direito de controlar informação}
\label{cap3:quem-principios-controlar}
A privacidade é complexa e está sob ataque de nossos colegas, de nossos próprios
dispositivos e de governos e profissionais de marketing corporativos. Os bits
estão por toda parte; simplesmente não há como prendê-los, e ninguém realmente
deseja mais fazer isso. O significado da privacidade mudou, e não temos uma boa
maneira de descrevê-lo. Não é o direito de ser deixado em paz, porque nem mesmo
as medidas mais extremas desconectarão nossos eu digitais do resto do mundo.
Não é o direito de manter nossas informações privadas para nós mesmos, porque
os bilhões de fatos atômicos não se prestam a ser classificados de maneira
simples e única como privados ou públicos.

Qual deles preferiríamos: o novo mundo com impressões digitais por toda parte e
a constante consciência de que estamos sendo rastreados, ou o antigo mundo com
poucas pegadas digitais e um senso maior de segurança contra olhares curiosos?
E qual é o sentido de sequer fazer essa pergunta quando o mundo não pode ser
devolvido ao seu antigo bloqueio de informações?\\

%%%%%%%%%%%%%%%%%%%%%%%%%%%%%%%% BLOCO IMAGEM %%%%%%%%%%%%%%%%%%%%%%%%%%%%%%%%
Os bits estão por toda parte;\\ Simplesmente não há como prendê-los, e ninguém
realmente deseja mais fazer isso.\\
%%%%%%%%%%%%%%%%%%%%%%%%%%%%%%%% BLOCO IMAGEM %%%%%%%%%%%%%%%%%%%%%%%%%%%%%%%%

Em um mundo que foi além da antiga noção de privacidade como uma parede em
torno do indivíduo, poderíamos em vez disso regular aqueles que usariam
informações sobre nós de maneira inadequada. Se eu postar um vídeo no YouTube
de mim dançando nu, devo esperar sofrer algumas consequências pessoais. No
final das contas, como Warren e Brandeis disseram, os indivíduos têm que
assumir a responsabilidade por suas ações. Mas a sociedade já traçou limites no
passado em torno de quais fatos são relevantes para determinadas decisões e
quais não são. Enquanto a fronteira da privacidade se torna cada vez mais
porosa, a fronteira da relevância passa ser ainda mais forte. Como explica
Daniel Weitzner:

\begin{quote}
    Novas leis de privacidade devem enfatizar restrições de uso para prevenir
    discriminação injusta com base em informações pessoais, mesmo que elas estejam
    publicamente disponíveis. Por exemplo, um empregador em potencial poderia
    encontrar um vídeo de um candidato a emprego entrando em uma clínica de AIDS ou
    em uma mesquita. Embora o indivíduo possa ter tornado tais fatos públicos, novas
    proteções de privacidade impediriam o empregador de tomar uma decisão de
    contratação com base nessa informação e imporiam penalidades reais para tal
    abuso.\footnote{Daniel. J. Weitzner, ``Beyond Secrecy: New Privacy Protection
    Strategies for Open Information Spaces'', in IEEE Internet Computing 11, no. 5
    (September-October 2007): 96--95,
    \url{https://dl.acm.org/doi/10.1109/MIC.2007.101}.}
\end{quote}

Ainda podem existir princípios de responsabilidade pelo uso indevido de
informações. Algumas pesquisas em andamento estão delineando uma possível nova
tecnologia web para ajudar a garantir que as informações sejam usadas
adequadamente quando são conhecidas. Talvez ferramentas automatizadas de
classificação e raciocínio, desenvolvidas para ajudar a conectar os pontos em
sistemas de informação em rede, possam ser redirecionadas para limitar o uso
inadequado de informações em rede. No entanto, é provável que uma guerra de
fronteiras continue sendo travada ao longo de uma frente de liberdade de
expressão existente: a linha que separa meu direito de contar a verdade sobre
você do seu direito de não ter essa informação usada contra você. No campo da
privacidade, a explosão digital deixou questões profundamente incertas.

Paul Ohm postula um ``banco de dados de ruína'':

\begin{quote}
    Quase todas as pessoas no mundo desenvolvido podem ser relacionadas com pelo
    menos um fato em um banco de dados de computador que um adversário poderia usar
    para chantagem, discriminação, assédio ou roubo financeiro ou de
    identidade.\footnote{Paul Ohm, ``Broken Promises of Privacy: Responding to the
    Surprising Failure of Anonymization'', SSRN Scholarly Paper (Social Science
    Research Network, August 13, 2009),
    \url{https://papers.ssrn.com/abstract=1450006}.}
\end{quote}

Devemos, por meio de uma combinação de leis, tecnologia e normas de
comportamento, encontrar maneiras de evitar uma destruição mútua da privacidade
assegurada.

Alguns raios de esperança vêm de legisladores estaduais, especialmente na
Califórnia, e de uma cultura crescente de privacidade entre engenheiros.
Algumas notificações de privacidade corporativa ainda são padrão, mas outras
dão a impressão de que a privacidade é uma característica do produto, projetada
para adicionar valor aos usuários e atender às suas necessidades.

\section{Sempre ligado}
\label{cap3:quem-sempre}
Em 1984, a tecnologia invasiva e ubíqua poderia ser desligada:

\begin{quote}
    Enquanto O'Brien passava pela ``teletela'', um pensamento pareceu atingi-lo.
    Ele parou, virou-se para o lado e apertou um interruptor na parede. Houve um
    estalo nítido. A voz parou.

    Julia emitiu um som mínimo, uma espécie de chiado de surpresa. Mesmo no meio do
    pânico, Winston ficou tão surpreso que não conseguiu segurar a língua.

    ``Você pode desligá-lo!'' ele disse.

    ``Sim'', disse O'Brien, ``nós podemos desligá-lo. Temos esse privilégio\ldots
    Sim, tudo está desligado. Estamos sozinhos''.
\end{quote}

Às vezes, ainda podemos desligá-lo hoje em dia---e devemos. Mas na maioria das
vezes, não queremos. Não queremos ficar sozinhos; queremos estar conectados.
Achamos conveniente deixá-lo ligado, deixar nossas pegadas e impressões
digitais por toda parte, para que sejamos reconhecidos quando voltarmos. Não
queremos ter que digitar novamente nosso nome e endereço ao retornar a um site.
Gostamos quando o restaurante lembra nosso nome, talvez porque nosso número de
telefone apareceu no identificador de chamadas e está vinculado ao nosso registro
em seu banco de dados. Apreciamos comprar uvas por \$1.95/lb em vez de \$3.49,
apenas deixando a loja saber que as compramos. Podemos querer deixá-lo ligado
para nós mesmos porque sabemos que está ligado para criminosos. Ser observado
nos lembra que eles também estão sendo observados. Ser observado também
significa que estamos sendo vigiados.

E talvez não nos importemos muito sobre o quanto é conhecido de nós, porque era
assim que a sociedade humana costumava ser: em grupos de parentesco e pequenos
assentamentos, saber tudo sobre todos era uma questão de sobrevivência. Ter
tudo isso ligado o tempo todo pode ressoar com preferências inatas que
adquirimos milênios atrás, antes que a vida urbana tornasse a anonimidade
possível. Ainda assim, hoje, privacidade significa algo muito diferente em uma
pequena cidade rural do que no \ingles{Upper East Side} de Manhattan.

Não podemos saber qual será o custo de tê-lo ligado o tempo todo. Tão
preocupante quanto a ameaça de medidas autoritárias para restringir a liberdade
pessoal é a ameaça da conformidade voluntária. Como Fano observou
perspicazmente, a privacidade permite experimentação social limitada---as
divergências das normas sociais que são muito arriscadas para o indivíduo sob a
exposição pública, mas que podem ser, e frequentemente foram no passado, as
bordas progressistas das mudanças sociais. Com ele sempre ligado, podemos
preferir não tentar nada convencional e estagnar socialmente pela inação
coletiva.

Na maior parte das vezes, é tarde demais, realisticamente, para desligá-lo.
Pode ser que já tenhamos tido o privilégio de desligá-lo, mas esse privilégio
não existe mais. Precisamos resolver nossos problemas de privacidade de outra
maneira.

A explosão digital está despedaçando antigas suposições sobre quem sabe o quê.
Os bits se movem rapidamente, baratos e em múltiplas cópias perfeitas.
Informações que costumavam ser públicas a princípio---por exemplo, registros em
um tribunal, o preço que você pagou pela sua casa ou histórias em um jornal de
uma cidade pequena---agora estão disponíveis para todos no mundo. Informações
que costumavam ser privadas e disponíveis para quase ninguém---registros
médicos e fotos pessoais, por exemplo---podem se tornar igualmente difundidas
por descuido ou malícia. As normas, práticas comerciais e leis da sociedade
ainda não acompanharam essa mudança.
\end{document}