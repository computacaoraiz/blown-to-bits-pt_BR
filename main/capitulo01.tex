%%%%%%%%%%%%%%%%%%%%%%%%%%%%%%%%%%%%%%%%%%%%%%%%%%%%%%%%%%%%%%%%%%%%%%%%%%%%%%%%
\chapter[Explosão Digital]{Explosão Digital\\\large\textit{Por que está
acontecendo e o que está em jogo?}}
\label{cap1:exp-dig}

Este livro não é sobre computadores. É sobre a sua vida e a minha. É sobre como
o terreno embaixo de nós mudou de maneiras fundamentais. Todos nós sabemos que
isso está acontecendo. Vemos isso ao nosso redor todos os dias. Todos nós
precisamos entender isso melhor.

A explosão digital está mudando tudo. Neste livro, falamos sobre o que está
acontecendo e como. Explicamos a tecnologia em si --- porque ela cria tantas 
surpresas e porque as coisas muitas vezes não funcionam como esperamos. Também 
é sobre as coisas que a explosão da informação está destruindo: pressupostos
antigamente dados como consolidados sobre nossa privacidade, sobre nossa
identidade e sobre quem controla nossas vidas. É sobre como chegamos a esse
ponto, o que estamos perdendo e o que ainda resta para a sociedade ter a chance
de corrigir.

A explosão digital está criando oportunidades e riscos. Muitos deles 
desaparecerão em uma década, resolvidos de uma forma ou de outra. Governos, 
corporações e outras autoridades estão aproveitando o caos, e a maioria de nós 
nem mesmo percebe que isso está acontecendo. No entanto, todos nós temos
interesse no resultado. Além da ciência, da história, da lei e da política, este
livro é um alerta. As forças que moldam o seu futuro são digitais, e você 
precisa entendê-las.

Este livro trata das histórias que ouvimos e lemos todos os dias. Histórias que 
se referem ao impacto profundo e muitas vezes inesperado que a tecnologia 
digital está tendo em nossas vidas. Vamos começar com a história de Nicolette 
Vartuli\index{Vartuli, Nicolette|(}.

\index{IA (inteligência artificial)!para decisões de contratação|(}
\index{história de bits!Nicolette Vartuli (processo de contratação|(}
Nicolette não conseguia entender porque não conseguiu o emprego. Ela é uma
estudante universitária com GPA de 3,5\footnote{[NT] GPA, do inglês
\ingles{Grade Point Average}, é um número que indica o rendimento, o
aproveitamento acadêmico nas universidades americanas. É equivalente ao CR
(coeficiente de rendimento) utilizado nas universidades brasileiras. O GPA é
calculado, grosso modo, como uma média ponderada das notas de cada uma das
disciplinas e, em geral, varia de 0 a 4. Para saber mais:
\url{https://en.wikipedia.org/wiki/Grading_in_education}.}, 
se preparou para a entrevista no banco de investimentos e se manteve
positiva durante todo o processo. Ela manteve a cabeça erguida, sorriu e falou
com confiança. Mas quando a empresa entrou em contato com a resposta, foi uma má
notícia. Ela não seguiria adiante no processo de contratação\footnote{Drew
Harwell, ``A Face-Scanning Algorithm Increasingly Decides Whether You Deserve
the Job,'' \textit{Washington Post}, November 6, 2019, \url{https://www.washingtonpost.com/technology/2019/10/22/ai-hiring-face-scanning-algorithm-increasingly-decides-whether-you-deserve-job/}.}. 

\index{decisão de contratação, software de IA para|(} Nicolette queria saber o que
ela havia feito de errado, mas ninguém pôde 
explicar o motivo dela ter sido rejeitada --- porque ninguém realmente sabe.
Ela foi entrevistada por um computador que usava um software de inteligência
artificial da HireVue para avaliar sua adequação. Esse software a rejeitou não
porque ela não tinha alguma qualificação específica, mas porque, como a HireVue
afirmava, o software conseguia detectar padrões em pessoas que tinham sucesso no
trabalho --- e o que observou em Nicolette não correspondia. É fácil entender
ser rejeitado porque você não tem três anos de experiência exigida ou alguma
habilidade específica. Isso é diferente. E assustador --- especialmente porque
não há nenhuma explicação sobre o que o software estava procurando. E pode ser
que nenhuma explicação pudesse ser oferecida, mesmo que a HireVue estivesse
disposta a divulgar seus algoritmos proprietários (ela não está).%
\index{Vartuli, Nicolette|)}\index{Nicolette Vartuli|see{Vartuli, Nicolette}}

As empresas gostam dessa nova tecnologia. É mais barata e eficiente do que 
entrevistas humanas. Na verdade, a HireVue, apenas uma das muitas fornecedoras, 
já realizou mais de 10 milhões de entrevistas. Muitos candidatos, por outro 
lado, não gostam desses assistentes de contratação automatizados. Não é apenas o
fato de se sentir desumanizado ao ser julgado por uma máquina. As empresas que 
oferecem esse serviço argumentam que, ao usar a tecnologia, mais pessoas podem 
conseguir entrevistas, e a probabilidade de preconceito inerente por parte dos
entrevistadores é reduzida. Eles afirmam que a tecnologia está abrindo
oportunidades, não limitando-as --- mas como podemos ter certeza disso?

A antipatia instintiva à triagem automatizada de empregos não é causada porque
as pessoas não querem que os computadores tomem decisões críticas em situações
de risco à vida. Muitas dessas decisões são tomadas por computadores hoje em
dia, por exemplo: os sistemas de avião e de radioterapia são em grande parte 
automatizados; computadores agora superam radiologistas altamente treinados na 
detecção de tumores de câncer em mamografias\footnote{Scott Mayer McKinney et
al., ``International Evaluation of an AI System for Breast Cancer Screening,''
\textit{Nature} 577, no. 7788 (January 2020): 89--94,
\url{https://doi.org/10.1038/s41586-019-1799-6}.}. Alguém iria preferir
avaliadores humanos menos precisos? Mas os julgamentos da HireVue são de um tipo
diferente. O programa tomou uma decisão sobre a humanidade de Nicolette. Decidiu
que ela não era o tipo de pessoa que a empresa deveria contratar, e fez isso sem
explicar a ela ou a qualquer outra pessoa qual seria o tipo de pessoa adequada
para a contratação e como Nicolette não se encaixava.

Muitos outros sistemas estão fazendo julgamentos semelhantes em outros domínios
humanos hoje em dia. Juízes consultam computadores para avaliar o risco de réus 
criminais não comparecerem aos seus julgamentos --- novamente comparando os 
indivíduos com outros que foram presos no passado e tiveram o benefício de 
evitar a detenção antes do julgamento\footnote{Julia Angwin et al., ``Machine
Bias,'' \textit{ProPublica}, May 23, 2016,
\url{https://www.propublica.org/article/machine-bias-risk-assessments-in-criminal-sentencing}.}. Agentes
imobiliários usam computadores para avaliar quais potenciais inquilinos
provavelmente serão devedores\footnote{Elizabeth Fernandez, ``Will Machine
Learning Algorithms Erase the Progress of the Fair Housing Act?''
\textit{Forbes}, November 17, 2019,
\url{https://www.forbes.com/sites/fernandezelizabeth/2019/11/17/will-machine-learning-algorithms-erase-the-progress-of-the-fair-housing-act/}.}.

A maioria desses sistemas é proprietária, e as empresas que os desenvolvem não 
precisam divulgar como eles funcionam. E afinal, argumentam eles,
entrevistadores humanos também não podem ser considerados como um padrão de
julgamento imparcial. Eles estão sujeitos a todo tipo de preconceitos e
tendências infelizes. É por isso que as audições para músicos instrumentais
agora são comumente realizadas sem que os ouvintes vejam os candidatos: quando
os intérpretes podiam ser vistos, as mulheres eram sistematicamente julgadas de
forma mais rigorosa do que os homens\footnote{Claudia Goldin and Cecilia Rouse,
``Orchestrating Impartiality: The Impact of `Blind' Auditions on Female
Musicians,'' \textit{The American Economic Review} 90, no. 4 (September 2000),
\url{https://pubs.aeaweb.org/doi/pdfplus/10.1257/aer.90.4.715}.}. Ao comparar as
habilidades dos candidatos nas entrevistas às dos trabalhadores existentes, a
HireVue afirma estar eliminando a parte mais falível do sistema. São os
recrutadores humanos, diz a HireVue, que são a ``caixa-preta final''. Talvez ---
exceto que a HireVue afirma estar combinando os candidatos ao perfil dos
melhores funcionários atuais do banco. Como alguém saberia se o software está
simplesmente replicando, agora automaticamente, todos os preconceitos que deram
ao banco a força de trabalho que ele tem agora?

O que torna toda essa história particularmente importante não é apenas o fato de
Nicolette ter sido considerada inadequada por uma máquina mas, sim, o fato de
que ninguém --- nem um gerente de recursos humanos, nem mesmo um programador ---
informou ao software da HireVue quais critérios usar. Ele determinou tudo
sozinho. O software assistiu a vídeos de funcionários existentes e escolheu seus
próprios critérios.

\index{história de bits!definição}
A história da candidatura de emprego rejeitada de Nicolette é o que chamamos de 
uma ``história de bits''. Não é apenas uma história de busca de emprego; é uma
história sobre a coleta, armazenamento, análise, transmissão e uso de trilhões
de trilhões de trilhões de $0$s e $1$s individuais. Ao analisar cuidadosamente
essas histórias, podemos entender não apenas a tecnologia por trás delas, mas
também suas implicações e riscos.

\begin{tcolorbox}
\index{transparência algorítmica}
A ``transparência algorítmica'' é o princípio de que devemos saber como os
computadores estão tomando decisões sobre nós. Nas palavras da EPIC
(\ingles{Electronic Privacy Information Center} --- Centro de Informações sobre
Privacidade Eletrônica): ``O público tem o direito de conhecer os processos de
dados que impactam suas vidas, para que possam corrigir erros e contestar
decisões tomadas por algoritmos.''\footnote{``Algorithmic Transparency: End
Secret Profiling,'' \textit{Electronic Privacy Information Center}, March 1,
2020, \url{https://epic.org/algorithmic-transparency/}.}
\end{tcolorbox}

Os bits representavam a imagem de Nicolette enquanto fluíam de seu próprio
computador para o da HireVue, por meio de fios, cabos e provavelmente vários
tipos de ondas de rádio. Os bits foram desmontados, remontados e analisados
pelos programas da HireVue. Eles foram de alguma forma comparados a trilhões de
trilhões de trilhões de outros bits que representam vídeos de outras pessoas e,
em seguida, um único bit, um simples sim ou não, foi emitido: continuar para a
próxima etapa do processo de contratação ou rejeitar imediatamente. Esse bit foi
um $0$ para Nicolette, e foi tudo o que ela ouviu da empresa. Mas a HireVue
manteve todos os bits da entrevista malsucedida de Nicolette; ela teve que ceder
seus direitos sobre eles para conseguir a entrevista em primeiro lugar.

Essas novas tecnologias interagem de maneiras estranhas com os padrões em
evolução de privacidade, práticas de comunicação e legislação criminal. A
história de Nicolette, embora importante para ela, é apenas uma das milhares de
histórias de bits que poderiam ser contadas sobre qualquer um de nós. Todos os
dias encontramos consequências inesperadas do fluxo de dados que não poderiam
ter ocorrido há apenas alguns anos atrás.\index{decisão de contratação, software de IA para|)}

Quando você terminar de ler este livro, você verá o mundo de uma maneira 
diferente. Você ouvirá uma história de um amigo ou em um noticiário e dirá a si 
mesmo: ``Isso é, na verdade, uma história de bits'', mesmo que ninguém mencione
nada digital. Os movimentos de objetos físicos e as ações de seres humanos de 
carne e osso são apenas a superfície. Para entender o que realmente está 
acontecendo, você precisa enxergar o mundo virtual, o sinistro fluxo de bits 
que está direcionando os eventos de nossas vidas.

Este livro é seu guia para este novo mundo.
\index{IA (inteligência artificial)!para decisões de contratação|)}
\index{história de bits!Nicolette Vartuli (processo de contratação|)}


%%%%%%%%%%%%%%%%%%%%%%%%%%%%%%%%%%%%%%%%
\section{A explosão de bits e tudo mais}
\label{cap1:exp-dig-bits}
\index{explosão digital|(}

\index{bits!explosão digital de|(}
O mundo mudou muito repentinamente. Quase tudo está armazenado em algum 
computador, em algum lugar. Registros judiciais, compras de supermercado,
fotos importantes de sua família, filmes inestimáveis de Hollywood, programas de
televisão sem sentido\ldots\ Os computadores armazenam uma enorme quantidade de
coisas que não são úteis atualmente, mas que alguém acredita que possam ser
úteis algum dia, no futuro. Tudo está sendo reduzido a $0$s e $1$s ---
``bits''. Os bits são armazenados em discos de computadores domésticos, nos
\ingles{data centers} de grandes corporações e nas agências
governamentais.\index{discos!definição de} Muitos desses discos nem mesmo são
objetos redondos e 
giratórios --- são um tipo diferente de mídia de armazenamento, chamados de
``discos'' por razões históricas. A maioria dos discos hoje em dia está ``na
nuvem'' --- apenas um nome sofisticado para discos pertencentes a grandes 
empresas como a Amazon, alugados para quem precisa de espaço para armazenar 
coisas. Os discos podem armazenar tantos bits que não há necessidade de escolher
o que deve ser lembrado.

\begin{tcolorbox}
\index{bits!definição}
``Bit'' é uma abreviação de ``dígito binário''. O sistema de numeração binária
utiliza apenas dois algarismos, $0$ e $1$, em vez dos dez algarismos $0$, $1$,
$2$, $3$, $4$, $5$, $6$, $7$, $8$ e $9$, usados no sistema decimal. A primeira
declaração clara dos princípios da notação binária foi feita por Gottfried
Wilhelm Leibniz em 1679.\index{Leibniz, Gottfried Wilhelm}%
\index{Gottfried Wilhelm Leibniz|see{Leibniz, Gottfried Wilhelm}}
\end{tcolorbox}

Tanta informação digital, desinformação, dados e lixo estão sendo armazenados 
que a maioria disso será visto apenas pelos computadores, nunca por olhos
humanos. E os computadores estão ficando cada vez melhores em extrair 
significado de todos esses bits --- encontrando padrões que às vezes resolvem
crimes, diagnosticam doenças e fazem sugestões úteis --- e às vezes revelam
coisas sobre nós que não gostaríamos que os outros soubessem.

\index{história de bits!Edward Snowden (vigilância governamental)}
A história de Edward Snowden, que vazou milhares de documentos altamente
secretos do governo em 2013, é uma história de bits. Ele levou os documentos 
para fora dos Estados Unidos em seu laptop; apenas alguns anos antes, ele
precisaria carregar centenas de quilos de papel. E tudo o que ele revelou estava
relacionado à vigilância eletrônica do governo, levantando questões fundamentais
sobre os \ingles{trade-offs} entre privacidade e segurança.

\index{história de bits!aviões 737 Max (mal funcionamento de software)}
O impedimento do 737 Max em 2019 não foi apenas uma história da aviação. Foi
também uma história de bits. Os motores dos primeiros modelos do 737 haviam
sido movidos, alterando a distribuição de peso da aeronave; e o software
desenvolvido para processar os dados dos sensores e controlar automaticamente os
movimentos do avião não funcionou como previsto\footnote{Benjamin Zhang, ``The
Boeing 737 Max Is Likely to Be the Last Version of the Best-Selling Airliner of
All Time,'' \textit{Business Insider}, March 19, 2019,
\url{https://www.businessinsider.com/boeing-737-max-design-pushed-to-limit-2019-3}.}.

Mas não são apenas os eventos de importância global que são histórias de bits; 
são as histórias do dia a dia da vida comum. A experiência perturbadora da 
corredora amadora Rosie Spinks é uma história de bits\index{Spinks, Rosie}.
\index{Rosie Spinks|see{Spinks, Rosie}}
\index{história de bits!Rosie Spinks (app de fitness)} Spinks usava um
aplicativo\index{aplicativo de fitness} em seu telefone para rastrear suas rotas
e tempos, e ela pensava que sua localização estava sendo mantida em segredo
porque ela tinha ativado a configuração de ``Privacidade Aprimorada'' do
aplicativo. Mas quando estranhos começaram a ``curtir'' seus treinos enquanto
ela estava viajando para o exterior, ela percebeu que a ``Privacidade
Aprimorada'', na verdade, significava ``contar para homens aleatórios sobre meus
treinos se eu estiver na lista dos melhores''. O aplicativo de condicionamento
físico também era um aplicativo de rede social, e os dados de Rosie estavam
sendo comercializados\footnote{Rosie Spinks, ``Confused About How to Use Strava
Safely? You Are Not Alone,'' \textit{Quartz}, January 29, 2018,
\url{https://qz.com/1191431/strava-privacy-concerns-here-is-how-to-safely-use-the-app}.}.

Uma vez que algo está em um computador, pode se replicar e se mover pelo mundo 
num piscar de olhos. Fazer um milhão de cópias perfeitas leva apenas um
instante --- cópias de coisas que gostaríamos de divulgar para todo mundo, mas
também cópias de coisas que não eram para ser copiadas de jeito nenhum.

A explosão digital está mudando o mundo tanto quanto a impressão já fez --- e 
algumas das mudanças estão nos pegando desprevenidos, desmontando nossas 
suposições sobre como o mundo funciona.

A explosão digital pode parecer benigna, divertida ou até mesmo utópica. Em vez 
de enviar fotos pelo correio para a vovó, compartilhamos imagens de nossos 
filhos no Instagram. Assim, não apenas a vovó pode vê-las, mas também os amigos 
da vovó e qualquer outra pessoa. Desfrutamos dos benefícios, mas quais são os 
riscos? As fotos para a vovó são fofas e inofensivas. Mas suponha que um turista
tire uma foto de férias e você apareça ao fundo, em um restaurante onde ninguém
sabia que você estava jantando. Se o turista enviar a foto para a Internet e
torná-la pública, o mundo inteiro poderá saber onde você estava, e quando. O
\index{reconhecimento facial|(}\index{fotos!reconhecimento facial|(}
reconhecimento facial, que há apenas alguns anos
estava além das capacidades dos 
computadores, agora é suficientemente poderoso para que a foto do turista possa
até ser marcada com o seu nome. A proibição desse tipo de coisa pode ocorrer por
causa de uma política ou lei, mas não por limitações tecnológicas. A
identificação automática de rostos em multidões é um problema resolvido, e esse
tipo de software está sendo usado na China e em outros regimes autoritários para
desencorajar protestos públicos e rastrear a localização dos cidadãos.%
\index{rastreamento de localização} E essa
tecnologia também está sendo usada nos Estados Unidos: com a ajuda de bilhões de
fotos rotuladas coletadas do Facebook e de outras mídias sociais, uma pequena
empresa chamada Clearview AI\index{Clearview AI|(}, de repente, se tornou uma
ferramenta de muitas 
agências de aplicação da lei e até mesmo de empresas privadas preocupadas
com a segurança\footnote{Kashmir Hill, ``The Secretive Company That Might End
Privacy as We Know It,'' \textit{The New York Times}, January 18, 2020,
\url{https://www.nytimes.com/2020/01/18/technology/clearview-privacy-facial-recognition.html}.}.
Não é nem mesmo muito difícil de fazer --- basta que uma companhia
empreendedora esteja disposta a esticar os limites do uso apropriado das enormes
bases de dados fotográficas que o Facebook e outras redes sociais
coletaram.\index{Clearview AI|)}\index{reconhecimento facial|)}%
\index{fotos!reconhecimento facial|)}

E antes de deixarmos o assunto das fotos de família, lembra quando elas eram 
todas impressas em papel e duravam décadas? Não é mais assim. Os maravilhosos 
benefícios das imagens digitais também as tornam inacessíveis. Você não pode 
colocar imagens digitais em uma caixa embaixo da sua cama para que seus netos as
encontrem. Todas essas memórias familiares podem se perder no futuro. Há um lado
bom e um lado ruim para praticamente tudo no mundo digital.

Vazamentos de dados. Os registros de cartões de crédito deveriam ficar trancados
em um \ingles{data warehouse}, um armazém de dados, mas eles acabam caindo nas
mãos de ladrões de identidade. E nós fornecemos informações apenas porque
recebemos algo em troca. Uma empresa lhe dará chamadas telefônicas gratuitas
para qualquer lugar do mundo --- se você não se importar em assistir anúncios
dos produtos sobre os quais seus computadores ouvem você falar. O Google
sugerirá restaurantes que você pode gostar --- se você deixar a localização
ativada para que o Google saiba quais restaurantes você frequenta. Se você fizer
uma refeição, o Google perguntará se você gostou --- nenhum software, por
enquanto, é capaz de descobrir \textbf{isso} sozinho --- e lá vai sua resposta
para dentro da ``boca'' de dados para ajudar o Google a fazer recomendações para
você e outros (e ganhar um pouco de dinheiro no caminho).

E isso é apenas uma parte do que está acontecendo hoje. A explosão digital e a
disruptura social que ela causará mal começaram.

Já vivemos em um mundo onde há memória suficiente \textbf{apenas nos celulares}
para armazenar cada palavra de cada livro da Biblioteca do Congresso, bilhões de
vezes.\index{armazenamento!quantidade de} Todos os dias, os vídeos enviados para
o YouTube são suficientes para
gravar todos os momentos de uma vida humana inteira. E esse crescimento
explosivo ainda está acontecendo. A cada ano podemos armazenar mais informações,
movê-las mais rapidamente e fazer coisas muito mais engenhosas com essas
informações do que era possível ser feito no ano anterior. Agora que geladeiras
e aspiradores de pó geram dados, a taxa crescente na qual os dados são criados é
quase inimaginável. A maioria dos dados que já existiu foi criada no ano passado
--- e isso será verdade novamente no próximo ano e nos anos seguintes.

Tanto armazenamento em disco \index{discos!disponibilidade de armazenamento nos}
está sendo produzido a cada ano que poderia ser 
usado para registrar uma página de informação, a cada poucos segundos, sobre 
\textbf{você e cada outro ser humano na Terra}. Um comentário feito há muito 
tempo pode voltar para assombrar um candidato político, e uma carta escrita 
rapidamente pode ser uma descoberta chave para um biógrafo. Imagine o que 
significaria gravar cada palavra que cada ser humano fala ou escreve em uma 
vida inteira. A barreira tecnológica para isso já foi removida: há armazenamento
suficiente para lembrar tudo. O YouTube diz que 500 horas de vídeo são enviadas 
a cada minuto\footnote{J. Clement, ``Hours of Video Uploaded to YouTube Every
Minute, 2007–2019,'' \textit{Statista}, August 9, 2019,
\url{https://www.statista.com/statistics/259477/hours-of-video-uploaded-to-youtube-every-minute/}.}. Deveria
haver alguma barreira social no caminho?

Às vezes, as coisas parecem funcionar melhor e pior do que costumavam. Um
\index{registro público, disponibilidade de|(}``registro público'', agora, é
\textbf{muito} público. Antes de ser contratado
em Nashville, Tennessee, seu empregador pode descobrir se você foi pego há dez
anos fazendo uma conversão ilegal à esquerda em Lubbock, Texas. A antiga noção
de um ``registro jurídico selado'' é, em grande, parte uma fantasia em um mundo
onde pequenas informações são duplicadas, catalogadas e movidas
infinitamente. Na Europa, um novo ``direito ao esquecimento'' foi adicionado à
lista de direitos humanos, destinado a proteger as pessoas de terem que carregar
para sempre cada indiscrição juvenil; mas, nos Estados Unidos, o direito à
liberdade de expressão continua dominante e a colisão entre esses direitos
conflitantes é inevitável. No mundo dos bits, o oceano Atlântico pode ser
atravessado em microssegundos.\index{registro público, disponibilidade de|)}

Com centenas de estações de TV e rádio, e milhões de sites, os americanos adoram
a variedade de fontes de notícias, mas eles se adaptaram, de modo
desconfortável, à substituição de fontes mais autoritárias. Na China, a situação
é invertida: a tecnologia cria um maior controle do governo sobre as informações
que seus cidadãos recebem, além de melhores ferramentas para monitorar seu
comportamento.
\index{bits!explosão digital de|)}
\index{explosão digital|)}

%%%%%%%%%%%%%%%%%%%%%%%%%%%%%%%%%%%%%%%%
\section{Os koans de bits}
\label{cap1:exp-dig-koans}
\index{bits!koans de|(}
\index{koans de bits|(}

Os bits se comportam de maneira estranha. Eles viajam quase instantaneamente e 
ocupam quase nenhum espaço para armazenar. Temos que usar metáforas físicas para
torná-los compreensíveis\index{metáforas!para bits}. Os comparamos à dinamite
explodindo ou água fluindo. \index{bits!metáforas para}
Também usamos metáforas sociais para os bits. Falamos sobre dois computadores
concordando sobre alguns bits e sobre pessoas usando ferramentas de arrombamento
para roubar bits. Encontrar uma metáfora correta é importante, mas também é
importante reconhecer as limitações de nossas metáforas. Uma metáfora imperfeita
pode enganar tanto quanto uma metáfora adequada pode iluminar.

\boxazul{Claude Shannon}{%
%\begin{tcolorbox}[title={Claude Shannon}]
\begin{minipage}[t]{0.5\linewidth}
\vspace*{0pt}
Claude Shannon (1916--2001) é, de forma indisputável, a figura fundadora da
teoria da informação e comunicação. Quando ele trabalhava nos \ingles{Bell}
\ingles{Telephone Laboratories} após a Segunda Guerra Mundial, escreveu o artigo
seminal ``\ingles{A Mathematical Theory of Communication}'', que antecipou
grande parte do desenvolvimento futuro das tecnologias digitais. Publicado em
1948, este artigo deu origem à compreensão universal de que o bit é a unidade
natural de informação e ao uso do termo.%
\index{A Mathematical Theory of Communication (Shannon)@``A Mathematical Theory of Communication'' (Shannon)}
\end{minipage}\hfill%
\begin{minipage}[t]{0.46\linewidth}
\vspace*{0pt}
\captionof{figure}{Claude Shannon}
\label{fig:shannon}
\vspace{-0.3cm}
\includegraphics[width=\linewidth]{imagens/shannon.jpeg}
\\%
\scriptsize{Uso permitido pela Nokia Corporation e AT\&T Archives
(\url{http://www bell-labs com/news/2001/february/26/shannon2_lg.jpeg})}
\end{minipage}%
%\end{tcolorbox}
}

Apresentaremos agora sete verdades sobre os bits. Nós chamamos essas verdades de
``koans'': no Zen Budismo um koan é uma anedota ou um enigma verbal paradoxal
utilizado para demonstrar a inadequação de um raciocínio lógico, para provocar
meditação e iluminação. Esses koans são simplificações e generalizações
excessivas. Eles descrevem um mundo que está em desenvolvimento mas que ainda
não emergiu completamente. Mas mesmo hoje eles são mais verdadeiros do que
frequentemente percebemos. Esses temas ecoarão em nossos relatos sobre a
explosão digital.


%%%%%%%%%%%%%%%%%%%%
\subsubsection*{Koan 1: Tudo é formado apenas por Bits}
\label{cap1:exp-dig-koans:1}
\index{bits!koans de!``tudo é formado apenas por bits''|(}
\index{koans de bits!``tudo é formado apenas por bits''|(}
Seu computador e seu smartphone (que na realidade é apenas outro computador)
criam com sucesso a ilusão de que contêm fotografias, cartas, músicas e
filmes. Tudo que eles realmente contêm são bits --- muitos deles --- organizados
de maneiras que você não consegue ver. Seu computador foi projetado para
armazenar apenas bits; todos os arquivos e pastas e diferentes tipos de dados
são ilusões criadas por programadores de computador. Quando você envia uma
mensagem contendo uma fotografia, os computadores que gerenciam sua mensagem à
medida que ela flui pela Internet não têm ideia de que estão lidando com uma
parte de texto e outra gráfica. Chamadas telefônicas também são apenas bits, e
isso ajudou a criar competição: empresas de telefone tradicionais, empresas de
celular, empresas de TV a cabo e provedores de serviço de voz sobre IP (VoIP)
podem trocar bits entre si para completar chamadas. A Internet foi projetada
para lidar apenas com bits, não e-mails ou anexos, que são invenções de
engenheiros de software. Não poderíamos viver sem esses conceitos mais
intuitivos, mas eles são artifícios. Por baixo, são apenas bits.

\index{Naral Pro-Choice America versus Verizon Wireless|(}
\index{mensagens de texto!Naral versus Verizon|(}
\index{Verizon Wireless versus Naral Pro-Choice America|(}
Este koan tem mais consequências do que você pode pensar. Considere a história
da Naral Pro-Choice America e da Verizon Wireless. A Naral queria formar um
grupo de mensagens de texto para enviar alertas a seus membros, mas dependia da
Verizon para fornecer o serviço. A Verizon decidiu não permitir, citando as
coisas ``controversas ou desagradáveis'' que as mensagens poderiam
conter\footnote{Adam Liptak, ``Verizon Blocks Messages of Abortion Rights
Group,'' \textit{The New York Times}, September 27, 2007,
\url{https://www.nytimes.com/2007/09/27/us/27verizon.html}.}. A Verizon permitia
grupos de alerta de mensagem de texto para candidatos políticos, mas não para
causas políticas que considerasse controversas. Se a Naral simplesmente quisesse
um serviço telefônico ou um número 0800, a Verizon não poderia recusar. As
empresas de telefone foram declaradas, há muito tempo, como ``meios de
transporte''.\index{meios de transporte} Como as ferrovias, as empresas de
telefone são legalmente proibidas de escolher clientes dentre aqueles que querem
seus serviços. No mundo dos bits, não há diferença entre uma mensagem de texto e
uma chamada de celular. São todos apenas bits, viajando pelo ar por ondas de
rádio. Mas a lei ainda não acompanhou a tecnologia. Ela não trata todos os bits
da mesma forma, e as regras de transporte comum para bits de voz não se aplicam
a bits de mensagem de texto.

\begin{tcolorbox}[title={Exclusivos e rivais}]
\index{bits!como não exclusivos e não rivais}
Economistas diriam que os bits, a menos que controlados de alguma forma, tendem
a ser não exclusivos (uma vez que algumas pessoas os têm, é difícil impedir que
outros os tenham) e não rivais (quando alguém os pega de mim, não tenho
menos). Em uma carta que escreveu sobre a natureza das ideias, Thomas Jefferson
declarou eloquentemente ambas as propriedades: ``Se a natureza fez alguma coisa
menos suscetível do que todas as outras de propriedade exclusiva, é a ação do
poder do pensamento chamada de ideia, que um indivíduo pode possuir
exclusivamente enquanto a mantiver para si; mas no momento em que é divulgada,
ela se impõe à posse de todos, e o receptor não pode se desfazer dela. Seu
caráter peculiar, também, é que ninguém possui menos, porque todo outro possui o
todo dela''.\footnote{``Article 1, Section 8, Clause 8: Thomas Jefferson to
Isaac McPherson,'' in Andrew A. Lipscomb and Albert Ellery Bergh, eds.,
\textit{The Writings of Thomas Jefferson} (Thomas Jefferson Memorial
Association, 1905),
\url{http://press-pubs.uchicago.edu/founders/documents/a1_8_8s12.html}.}
\end{tcolorbox}

A Verizon recuou no caso da Naral, mas não em relação ao princípio. Uma empresa 
de telefonia pode fazer o que achar que maximizará seus lucros ao decidir quais 
mensagens distribuir. No entanto, nenhuma distinção de engenharia sensata pode 
ser feita entre mensagens de texto, chamadas telefônicas e quaisquer outros bits
que viajam pelas ondas digitais.
\index{bits!koans de!``tudo é formado apenas por bits''|)}
\index{koans de bits!``tudo é formado apenas por bits''|)}
\index{Naral Pro-Choice America versus Verizon Wireless|(}
\index{mensagens de texto!Naral versus Verizon|(}
\index{Verizon Wireless versus Naral Pro-Choice America|(}

%%%%%%%%%%%%%%%%%%%%
\label{cap1:exp-dig-koans:2}
\subsubsection*{Koan 2: A perfeição é normal}
\index{bits!koans de!``a perfeição é normal''|(}
\index{koans de bits!``a perfeição é normal''|(}
Errar é humano. Quando os livros eram laboriosamente transcritos à mão em 
\ingles{scriptoria}\footnote{[NT] Do latim \ingles{scriptorium}, que significa
um local para escrever.} antigos e mosteiros medievais, erros se infiltravam a
cada cópia. Computadores e redes funcionam de forma diferente. Cada cópia é
perfeita.\index{cópias!perfeição das}\index{cópias!propriedade intelectual das}
Se você enviar um email com uma fotografia para um amigo, o amigo não receberá
uma versão mais desfocada do que o original. A cópia será idêntica, até níveis
de detalhe muito pequenos para serem percebidos pelo olho humano.

Computadores falham, é claro. Redes também quebram. Se a energia elétrica cai e
não há bateria de reserva, nada funciona. Portanto, a afirmação de que as cópias
são normalmente perfeitas é apenas relativamente verdadeira. Cópias digitais são
perfeitas apenas na medida em que podem ser comunicadas. E sim, é possível em
teoria que um único bit de uma grande mensagem chegue incorretamente --- mas
também é possível que um vulcão entre em erupção sob você e você não receba a
mensagem de jeito nenhum. As chances de um bit errôneo são menores do que as
chances de uma catástrofe física, e isso é suficiente para todos os propósitos
práticos.

As redes não apenas transferem bits de um lugar para outro. Elas verificam se os
bits parecem ter sido danificados durante a transmissão e os corrigem ou 
retransmitem se parecerem incorretos. Como resultado desses mecanismos de 
detecção e correção de erros, as chances de um erro real --- por exemplo, um
caractere errado em um e-mail --- são tão baixas que seria mais sensato nos
preocuparmos com um meteoro atingindo nosso computador, mesmo que sejam
improváveis os impactos de meteoros.

O fenômeno das cópias perfeitas mudou drasticamente a lei, como é contado no
Capítulo 6, ``Equilíbrio Derrubado''. Nos tempos em que a música era distribuída
em fita cassete, os adolescentes não eram processados por fazer cópias de 
músicas, porque as cópias não eram tão boas quanto as originais, e as cópias das
cópias seriam ainda piores. A razão pela qual as pessoas hoje em dia preferem
mais assinar serviços de música do que possuir suas próprias cópias das
gravações é que as cópias são perfeitas --- não apenas tão boas quanto a
original, mas idênticas à original, de forma que até mesmo a noção de
``original'' é insignificante. As consequências da interrupção digital da
``propriedade intelectual'' ainda não acabaram. Os bits são uma espécie peculiar
de propriedade. Uma vez que eu os libere, todos os têm. E se eu lhe der meus
bits, eu não tenho menos deles.
\index{bits!koans de!``a perfeição é normal''|)}
\index{koans de bits!``a perfeição é normal''|)}

%%%%%%%%%%%%%%%%%%%%
\subsubsection*{Koan 3: Há carência no meio da abundância}
\label{cap1:exp-dig-koans:3}
\index{bits!koans de!``há carência no meio da abundância''|(}
\index{inacessibilidade da informação|(}
\index{informação!inacessibilidade da|(}
\index{koans de bits!``há carência no meio da abundância''|(}
Por mais que o armazenamento de dados em todo o mundo seja grande hoje, daqui a
dois anos ele será o dobro. No entanto, a explosão de informações significa, 
paradoxalmente, a perda de informações que não estão online. Um de nós viu um 
novo médico em uma clínica que ele frequentava há décadas. Ela mostrou ao médico
gráficos detalhados da química do seu sangue, dados transferidos do dispositivo
médico doméstico para o computador da clínica --- mais dados do que qualquer
especialista poderia ter tido à disposição há cinco anos. O médico então
perguntou se ela já havia feito um teste de estresse e o que o teste havia 
mostrado. Esses registros deveriam estar lá, explicou a paciente, no prontuário
médico. Mas as informações estavam no arquivo em \textbf{papel}, ao qual o 
médico não tinha acesso. Não estava na memória do \textbf{computador}, e a
memória do paciente estava sendo usada como um substituto inadequado. Os dados
antigos poderiam muito bem não ter existido, já que não eram digitais.

Mesmo as informações que existem em formato digital são inúteis se não houver 
dispositivos para lê-las. O rápido progresso da engenharia de armazenamento fez 
com que os dados armazenados em dispositivos obsoletos deixassem de existir 
efetivamente. Uma versão digital criada no século XX para o ``The Domesday
Book'' britânico do século XI\index{Domesday Book}, mostrado na
Figura~\ref{fig:domesday}, já era inútil quando tinha apenas um sexagésimo da
idade do original\footnote{Robin McKie and Vanessa Thorpe, ``Digital Domesday
  Book lasts 15 years not 1000,'' \textit{Guardian Unlimited}, March 3, 2002.}.

\begin{figure}[h]
\centering
\caption{The Domesday Book de 1806.}
\label{fig:domesday}
\vspace{-0.3cm}
%\fbox{
\includegraphics[scale=0.5]{imagens/domesday.jpg}
%}
\\
\scriptsize{UK National Archives
(\url{https://www.worldhistory.org/image/9476/great-domesday-book/}). Uma versão
digital comemorativa do 900º aniversário não podia ser mais lida 15 anos após
o lançamento.}
\end{figure}

Ou considere a busca, um dos assuntos do Capítulo 4, ``Guardiões''. No início,
os motores de busca, como o Google, eram conveniências interessantes que algumas
pessoas usavam para fins especiais. Com o crescimento da World Wide Web e a 
explosão de informações online, os motores de busca se tornaram o primeiro lugar
onde muitas pessoas procuram por informações --- antes mesmo de consultarem
livros ou perguntarem a amigos. Aparecer em destaque nos resultados de busca se
tornou uma questão de vida ou morte para as empresas. Podemos passar a comprar 
de um concorrente se não encontrarmos o site que queríamos logo nas primeiras 
páginas de resultados. Podemos presumir que algo não aconteceu se não 
conseguirmos encontrá-lo --- rapidamente --- em uma fonte de notícias online.
Se algo não pode ser encontrado rapidamente, é como se não existisse de forma
alguma.\index{motores de busca!inacessibilidade da informação}

E algumas informações não são verdadeiras. Todos os mecanismos que permitem a
comunicação e o armazenamento de fatos também funcionam para falsidades. A 
feiura e a crueldade são capturadas tão facilmente em bits quanto a beleza e a 
bondade. A economia de mercado da informação muda quando todos podem ser 
editores e ninguém precisa de um editor. Inundações de desinformação, informação
falsa e lixo podem sobrecarregar a verdade e a beleza.\index{informação falsa
versus informação verdadeira}\index{informação!falsa versus verdadeira}
Sociedades autoritárias podem ser capazes de gerenciar o fluxo de bits de forma
mais eficiente do que sociedades livres, que correm o risco de serem minadas por
seus próprios princípios de liberdade de informação.
\index{bits!koans de!``há carência no meio da abundância''|)}
\index{inacessibilidade da informação|)}
\index{informação!inacessibilidade da|)}
\index{koans de bits!``há carência no meio da abundância''|)}

%%%%%%%%%%%%%%%%%%%%
\subsubsection*{Koan 4: Processamento é poder}
\label{cap1:exp-dig-koans:4}
\index{bits!koans de!``processamento é poder''|(}
\index{koans de bits!``processamento é poder''|(}
\index{poder de processamento|(}
A velocidade de um computador geralmente é medida pelo número de operações 
básicas, como adições, que podem ser realizadas em um segundo. Os computadores 
mais rápidos disponíveis no início dos anos 1940 conseguiam realizar cerca de 
cinco operações por segundo. Os mais rápidos de hoje podem realizar cerca de um 
trilhão. Compradores de computadores pessoais sabem que uma máquina que parece 
rápida hoje parecerá lenta daqui a um ano ou dois.

Por pelo menos três décadas, o aumento na velocidade dos processadores era 
exponencial. Os computadores ficavam duas vezes mais rápidos a cada dois anos. 
Esses aumentos eram uma consequência da Lei de Moore\index{Lei de Moore|(}
(ver o quadro ``Lei de Moore'', na página~\pageref{qd:moore}).

Desde 2001, a velocidade do processador não tem seguido a Lei de Moore; na 
verdade, os processadores mal ficaram mais rápidos. Mas isso não significa que
os computadores não continuarão a evoluir. Novos projetos de chips
incluem vários processadores no mesmo chip, para que o trabalho possa ser
dividido e executado em paralelo. Essas inovações prometem alcançar o mesmo
efeito que aumentos contínuos na velocidade bruta do processador. E essas
melhorias tecnológicas, além de tornam os computadores mais rápidos, os
tornam mais baratos.

\begin{tcolorbox}[title={Lei de Moore}]
\label{qd:moore}
\index{Moore, Gordon}
Gordon Moore, fundador da Intel Corporation, observou que a densidade dos 
circuitos integrados parecia dobrar a cada dois anos. Essa observação é 
conhecida como ``Lei de Moore''. Claro, não é uma lei natural, como a lei da
gravidade. Em vez disso, é uma observação empírica do progresso da engenharia e 
um desafio aos engenheiros para continuarem sua inovação. Em 1965, Moore previu 
que esse crescimento exponencial continuaria por muito tempo\footnote{G. E.
Moore, ``Cramming More Components onto Integrated Circuits,'' \textit{Proceedings
of the IEEE 86}, no. 1 (January 1998): 82--85, \url{https://doi.org/10.1109/JPROC.1998.658762}.}.
O fato de ter continuado por mais de 40 anos é uma das grandes maravilhas da
engenharia. Nenhum outro esforço na história sustentou uma taxa de crescimento
nem perto dessa.
\end{tcolorbox}
\index{Lei de Moore|)}

O rápido aumento na capacidade de processamento significa que as invenções saem 
dos laboratórios e entram rapidamente nos bens de consumo. Aspiradores de pó 
robóticos e veículos com estacionamento automático eram teoricamente possíveis 
há uma década, mas agora se tornaram produtos de consumo. Tarefas que hoje 
parecem exigir habilidades exclusivamente humanas não são mais apenas objeto de 
projetos de pesquisa em laboratórios corporativos ou acadêmicos; elas estão 
incorporadas em produtos de consumo. Reconhecimento facial e reconhecimento de 
voz estão aqui e agora; telefones sabem quem está ligando, e câmeras de 
vigilância não precisam de humanos para observá-las. O poder vem não apenas dos 
bits, mas da capacidade de fazer coisas com os bits.
\index{bits!koans de!``processamento é poder''|)}
\index{koans de bits!``processamento é poder''|)}
\index{poder de processamento|)}

%%%%%%%%%%%%%%%%%%%%
\subsubsection*{Koan 5: Mais do mesmo pode ser uma coisa totalmente nova}
\label{cap1:exp-dig-koans:5}
\index{bits!koans de!``mais do mesmo pode ser uma coisa totalmente nova''|(}
\index{crescimento exponencial|(}
\index{koans de bits!``mais do mesmo pode ser uma coisa totalmente nova''|(}
O crescimento explosivo é um crescimento exponencial --- dobrando em uma taxa 
constante. Imagine ganhar 100\% de juros anuais em sua conta poupança: em 10 
anos, seu dinheiro teria aumentado mais de mil vezes, e em 20 anos, mais de um 
milhão de vezes. Uma taxa de juros mais razoável de 5\% atingirá os mesmos 
pontos de crescimento, mas o fará 14 vezes mais lentamente. Epidemias 
inicialmente se espalham de forma exponencial, à medida que cada pessoa 
infectada infecta várias outras.

Quando algo cresce exponencialmente, por um longo tempo pode parecer que não 
está aumentando. Se não o observarmos constantemente, parecerá que ocorreu algo
descontínuo e radical enquanto não estávamos olhando.

É por isso que as epidemias no início passam despercebidas, independentemente de
quão catastróficas possam ser quando estão em pleno andamento. Imagine uma
pessoa doente infectando duas pessoas saudáveis, e no dia seguinte cada uma
dessas duas infecta mais duas, e no dia seguinte cada uma dessas quatro infecta 
mais duas, e assim por diante. O número de novas infecções cresce de 2 para 4 
para 8. Em uma semana, 128 pessoas contraem a doença em um único dia, e o dobro 
desse número agora está doente, mas em uma população de 10 milhões, ninguém 
percebe. Mesmo depois de duas semanas, apenas cerca de 3 pessoas em 1.000 estão 
doentes. Mas após mais uma semana, 40\% da população está doente e a sociedade 
entra em colapso. A pandemia de coronavírus de 2019--2020 seguiu mais ou menos 
esse padrão em partes do mundo onde as sociedades não reagiram rapidamente. No 
início da epidemia em Wuhan, o número de casos dobrava aproximadamente a cada 
três dias\footnote{Steven Sanche et al., ``High Contagiousness and Rapid Spread
of Severe Acute Respiratory Syndrome Coronavirus 2,'' \textit{Emerging
Infectious Diseases} 26, no. 7 (July 2020): 1470--1477,
\url{https://dx.doi.org/10.3201/eid2607.200282.}}.

O crescimento exponencial é na verdade suave e constante; leva muito pouco 
tempo para passar de uma mudança imperceptível para algo altamente visível. O 
crescimento exponencial de qualquer coisa pode fazer o mundo parecer 
completamente diferente do que era antes. Quando esse limiar é ultrapassado, 
mudanças que são ``apenas'' quantitativas podem parecer qualitativas.

Outra maneira de entender o aparente modo abrupto do crescimento exponencial ---
sua força explosiva --- é pensar em quão pouco tempo temos para responder a ele.
Nossa epidemia hipotética levou três semanas para sobrecarregar a população. Em 
que ponto ela seria apenas metade tão devastadora? A resposta \textbf{não} é
``uma semana e meia''. A resposta é no \textbf{penúltimo} dia. Suponha que
levasse uma semana para desenvolver e administrar uma vacina. Então, perceber a
epidemia após uma semana e meia deixaria tempo de sobra para evitar o
desastre. Mas isso exigiria compreender que \textbf{havia} uma epidemia quando
apenas 2.000 pessoas em 10 milhões estavam infectadas.

A história da informação está repleta de exemplos de mudanças não percebidas 
seguidas por explosões deslocadoras. Aqueles com a visão de futuro para notar a 
explosão apenas um pouco mais cedo do que todos os outros podem obter benefícios 
enormes. Aqueles que se movem um pouco mais devagar podem ser sobrecarregados 
quando tentam responder. Tome o caso da fotografia digital.

Em 1983, os compradores de Natal podiam comprar câmeras digitais para conectar 
aos seus computadores domésticos IBM PC e Apple II. O potencial estava lá para 
qualquer um ver; não estava escondido em laboratórios corporativos secretos. Mas 
a fotografia digital não decolou. Economicamente e praticamente, não era viável.
As câmeras eram muito volumosas para caber no bolso, e as memórias digitais eram
muito pequenas para armazenar muitas imagens. Mesmo 14 anos depois, a fotografia
em filme ainda era uma indústria robusta. \index{Kodak} No início de 1997, as
ações da Kodak atingiram um preço recorde, com um aumento de 22\% no lucro
trimestral, ``impulsionado pelas vendas saudáveis de filmes e papel\ldots\ [e]
pelo negócio de filmes de cinema'', de acordo com um relatório de
notícias\footnote{``Kodak, GE, Digital Report Strong Quarterly Results,''
\textit{Atlanta Constitution}, January 17, 1997.}. A empresa aumentou seu
dividendo pela primeira vez em oito anos. Mas em 2007, as memórias digitais
haviam se tornado enormes, os processadores digitais haviam se tornado rápidos e
compactos, e ambos eram baratos. Como resultado, as câmeras se tornaram pequenos
computadores. A empresa que um dia foi sinônimo de fotografia era apenas uma
sombra do que já foi. \index{fotografia digital!crescimento exponencial da}
A Kodak anunciou que seu
quadro de funcionários seria reduzido para 30.000, pouco  mais de um quinto do
tamanho que tinha nos bons tempos do final dos anos 
1980\footnote{Claudia H. Deutsch, ``Shrinking Pains at Kodak,'' \textit{The New
York Times}, February 9, 2007.}. Em 2018, esse número estava em cerca de 5.400.
Bits tiraram 90\% dos empregos. A Lei de Moore avançou mais rapidamente do que a
Kodak.

No mundo em rápida mudança dos bits, vale a pena notar até mesmo pequenas
mudanças e fazer algo a respeito delas.
\index{bits!koans de!``mais do mesmo pode ser uma coisa totalmente nova''|)}
\index{crescimento exponencial|)}
\index{koans de bits!``mais do mesmo pode ser uma coisa totalmente nova''|)}

%%%%%%%%%%%%%%%%%%%%
\subsubsection*{Koan 6: Nada vai embora}
\label{cap1:exp-dig-koans:6}
\index{bits!koans de!``nada vai embora''|(}
\index{políticas de retenção de dados|(}
\index{armazenamento!políticas de retenção de dados|(}
\index{koans de bits!``nada vai embora''|(}
\num{25000000000000000000}.

Esse é o número de bits que foram criados e armazenados todos os dias em 2019,
de acordo com uma estimativa da indústria. A capacidade dos discos seguiu sua
própria versão da Lei de Moore, duplicando a cada dois ou três anos. E muito
mais dados são criado por todos os tipos de dispositivos, mas não armazenados.

Nas indústrias financeiras, as leis federais agora \textbf{exigem} a retenção
massiva de dados para auxiliar em auditorias e investigações de corrupção. Em
muitos outros negócios, a competitividade econômica leva as empresas a salvar
todos os dados que coletam e a buscar novos dados para reter. Dezenas de milhões
de transações ocorrem nas lojas Walmart todos os dias, e cada uma delas é salva:
data, hora, item, loja, preço, quem fez a compra e como --- crédito, débito,
dinheiro ou vale-presente. Esses dados são tão valiosos para planejar a cadeia
de suprimentos que as lojas pagarão aos próprios clientes para conseguir obter
mais dados deles. É realmente isso que os programas de fidelidade em
supermercados e outras lojas são: os compradores pensam que a loja está lhes
concedendo um desconto em agradecimento pela sua fidelidade mas, na verdade, a
loja está pagando por informações sobre seus padrões de compra. Poderia ser
melhor pensar em um imposto de privacidade: nós pagaríamos o preço regular
\textbf{a menos que} queiramos manter informações sobre nossas compras de
alimentos, álcool e produtos farmacêuticos longe do mercado; para manter nossos
hábitos para nós mesmos, pagaríamos mais.

Os enormes bancos de dados desafiam nossas expectativas sobre o que acontecerá
com os dados a nosso respeito. Pegue algo tão simples quanto uma estadia em um
hotel. Quando você faz o \ingles{check-in}, recebe um cartão-chave, e não uma
chave mecânica\index{chaves de hotel!informação associada à}. Na verdade alguns
hotéis foram um passo adiante, fazendo com que 
você use seu próprio celular como chave do quarto. Como os cartões-chave podem
ser desativados instantaneamente, não há mais um grande risco associado à perda
de sua chave, contanto que você a comunique como perdida rapidamente. Por outro
lado, o hotel agora tem um registro, preciso até o segundo, de cada vez que você
entrou em seu quarto, usou a academia, o centro de negócios, ou entrou pela
porta dos fundos após o horário comercial. O mesmo banco de dados poderia
identificar cada coquetel e bife que você cobrados na conta de seu quarto, quais
outros quartos você ligou e quando, e as marcas de absorventes ou laxantes que
você comprou na loja de presentes do hotel. Esses dados podem ser mesclados com
bilhões de bits de outros dados, analisados e transferidos para a empresa-mãe,
que possui restaurantes e academias, bem como hotéis. Eles também podem ser
perdidos, roubados ou intimados em um processo judicial.

\index{certidão de nascimento!informações coletadas na}
A facilidade de armazenar informações resultou em que mais dados fossem
solicitados e armazenados. Antigamente as certidões de nascimento incluíam
apenas informações sobre os nomes da criança e dos pais, local e data de
nascimento e a profissão dos pais. Agora o registro eletrônico de nascimento
inclui o quanto a mãe bebeu e fumou durante a gravidez, se ela teve herpes
genital ou uma variedade de outras condições médicas, e os números do Seguro
Social de ambos os pais. As oportunidades para pesquisa são abundantes, assim
como as oportunidades para mau uso e perdas catastróficas acidentais de dados.

E os dados serão mantidos para sempre, a menos que haja políticas para se livrar
deles. Por enquanto, pelo menos, os dados permanecem. E como os bancos de dados
são intencionalmente duplicados --- salvos como backup para segurança ou
compartilhados para análises --- não é certo pensar que os dados possam ser
permanentemente expurgados, mesmo que desejemos que isso aconteça. A Internet
consiste em milhões de computadores interconectados; uma vez que os dados vazam,
não há como recuperá-los. Vítimas de roubo de identidade vivenciam diariamente o
estresse de ter que remover informações erradas de seus registros. Parece que
esse problema nunca desaparecerá.

\begin{tcolorbox}
Os dados serão mantidos para sempre, a menos que existam políticas para que eles
sejam excluídos.
\end{tcolorbox}
\index{bits!koans de!``nada vai embora''|)}
\index{políticas de retenção de dados|)}
\index{koans de bits!``nada vai embora''|)}
\index{armazenamento!políticas de retenção de dados|)}

%%%%%%%%%%%%%%%%%%%%
\subsubsection*{Koan 7: Os bits se movem mais rápido que o pensamento}
\label{cap1:exp-dig-koans:7}
\index{bits!koans de!``os bits se movem mais rápido que o pensamento''|(}
\index{Internet!velocidade da|(}
\index{koans de bits!``os bits se movem mais rápido que o pensamento''|(}
A Internet passou a existir antes mesmo dos computadores pessoais. Ela antecede
os cabos de comunicação de fibra ótica que agora a mantêm unida. Quando começou
por volta de 1970, a ARPANET, como era chamada, foi projetada para conectar
alguns computadores universitários e militares. Ninguém imaginava uma rede
conectando dezenas de milhões de computadores e enviando informações ao redor do
mundo num piscar de olhos (de fato, ninguém imaginou que tantos computadores
sequer existiriam). Até o engenheiro encarregado de projetar os
\ingles{gateways} que conectariam os computadores lembra sua reação à ideia de
uma rede de computadores: ``Parece um trabalho de engenharia simples; certamente
poderíamos fazer, mas não consigo imaginar porque alguém iria querer uma coisa
dessas''\footnote{Harry R. Lewis, ``A Science Is Born,'' \ingles{Harvard
Magazine}, September-October 2020: 42,
\url{https://harvardmagazine.com/2020/09/features-a-science-is-born}.}. Junto
com o poder de processamento e capacidade de armazenamento, as redes
experimentaram seu próprio crescimento exponencial no número de computadores
interconectados e na taxa na qual os dados podem ser enviados a longas
distâncias, do espaço para a terra e dos provedores de serviços para residências
privadas.

A Internet causou mudanças drásticas nas práticas dos negócios. Ligações para o
atendimento ao cliente são terceirizadas para a Índia hoje em dia, não apenas
porque os custos de mão de obra são baixos lá. Os custos de mão de obra
\textbf{sempre} foram baixos na Índia, mas as chamadas telefônicas
internacionais costumavam ser caras. Ligações sobre reservas de passagens aéreas
e pedidos de devoluções de lingerie são atendidas na Índia hoje porque, agora,
não leva quase nenhum tempo e não custa quase nada para enviar os bits que
representam a sua voz para a Índia. O mesmo princípio se aplica a serviços
profissionais. Quando você faz um raio-X em seu hospital local em Iowa, o
radiologista que lê o raio-X pode estar do outro lado do mundo. O raio-X digital
circula pelo mundo e volta mais rápido do que um raio-X físico poderia ser
movido entre os andares de um hospital. Quando você faz um pedido em um
\ingles{drive-through} de uma lanchonete, a pessoa que anota o pedido pode estar
em outro estado. Ela insere o pedido para que ele apareça em uma tela de
computador na cozinha, a poucos metros do seu carro, e você nem percebe. Tais
desenvolvimentos estão causando mudanças massivas na economia global, à medida
que as indústrias descobrem como manter seus trabalhadores em um lugar e enviar
seus negócios em bits.

No mundo dos bits, onde as mensagens fluem instantaneamente, às vezes parece que
a distância não importa em nada. As consequências disso podem ser
surpreendentes. Um de nós, enquanto decano de uma faculdade americana,
presenciou o choque de um pai recebendo condolências pela morte de sua
filha. Esse acontecimento é triste e, de certa forma, familiar para todos nós,
exceto que nesta situação havia uma reviravolta surpreendente. Pai e filha
estavam ambos em Massachusetts, mas as condolências chegaram de meio mundo de
distância antes que o pai soubesse que sua filha havia morrido. Notícias, até
mesmo as mais íntimas, viajam rápido no mundo dos bits assim que são divulgadas.

Quando todos estão conectados o tempo todo, as pessoas podem se organizar como
nunca antes. Aqueles afetados por doenças raras ou inspirados por interesses
idiossincráticos podem teclar algumas vezes e compartilhar suas experiências,
mesmo que estejam separados por oceanos e nunca se encontrem pessoalmente. E
aqueles unidos por uma causa comum podem se organizar para expressar suas
queixas, como a juventude digitalmente habilidosa de Hong Kong fez durante os
protestos pró-democracia de 2014.\index{Manifestantes de Hong Kong} Mas, nas
mãos das autoridades, os bits que os manifestantes trocaram tornaram-se
evidências contra eles. Na época dos protestos de Hong Kong em 2019, os
organizadores haviam abandonado o Facebook e estavam recorrendo a aplicativos de
mensagens criptografadas menos convenientes --- e estavam usando máscaras
faciais para confundir os sistemas de vigilância facial do governo\footnote{Lily
Kuo, ``Hong Kong's Digital Battle: Tech That 
Helped Protesters Now Used Against Them,'' \ingles{The Guardian}, June 14,
2019,\url{https://www.theguardian.com/world/2019/jun/14/hong-kongs-digital-battle-technology-that-helped-protesters-now-used-against-them}.}.

E se a vigilância falhar, os governos podem simplesmente desligar a
Internet. Isso aconteceu no estado de maioria muçulmana na Caxemira, na Índia,
em 2019;\index{Internet!desligamento da} não houve Internet por 7 meses, em nome
da ``segurança pública''\footnote{Billy Perrigo, ``India's Supreme Court Orders
Review of Internet Shutdown in Kashmir. But for Now, It Continues,''
\ingles{Time}, January 10, 2020,
\url{https://time.com/5762751/internet-kashmir-supreme-court}.}.
Desligamentos semelhantes ocorreram em 2019 no Irã, Congo, Bangladesh e em mais
de uma dúzia de outros países\footnote{Samuel Woodhams and Simon Migliano, ``The
Global Cost of Internet Shutdowns in 2019,'' \ingles{Top10VPN}, January 7,
2020,
\url{https://www.top10vpn.com/cost-of-internet-shutdowns}.}.
\index{Internet!leis que controlam o fluxo de informação|(} E nos Estados
Unidos, a Seção 706 do \ingles{Communications Act} de 1934 autoriza o presidente
a desligar ``uma instalação para comunicação por fio'' em caso de ``estado ou
ameaça de guerra'' --- uma autorização muito ampla, até agora nunca invocada
para assumir o controle da Internet.

A comunicação instantânea de quantidades massivas de informações criou a falsa
impressão de que existe um lugar chamado ``ciberespaço'',\index{ciberespaço|(}
uma terra sem fronteiras onde todas as pessoas do mundo podem estar
interconectadas como se fossem residentes da mesma cidade. Esse conceito foi
decisivamente refutado pelas ações dos tribunais mundiais. Fronteiras nacionais
e estaduais ainda importam --- e muito. Se um livro é comprado online na
Inglaterra, o editor e o autor estão sujeitos às leis de difamação britânicas, e
não às leis do país de origem do autor ou editor. Segundo a lei britânica, os
réus têm que provar sua inocência; nos Estados Unidos, os demandantes têm que
provar a culpa dos réus. Um lado negativo da explosão da informação digital e de
seu movimento ao redor do mundo é que a informação pode se tornar menos
disponível mesmo onde seria legalmente protegida (retornamos a este assunto no
Capítulo 7, ``Você Não Pode Dizer Isso na Internet''). Leis que tratam do
``direito de ser esquecido'' podem exigir que as informações desapareçam --- não
apenas no país onde um indivíduo pediu que algum erro passado fosse retirado de
seu registro eletrônico, mas em todos os lugares. Tal lei pode parecer
inaplicável, mas as empresas que disponibilizam a informação --- como o Google,
por exemplo --- operam internacionalmente, e se violarem a lei em qualquer
lugar, arriscam-se a ter funcionários assediados ou presos sempre que estiverem
em uma jurisdição onde a lei foi violada ou ignorada. Da mesma forma, o mundo
editorial foi fragmentado. Costumava ser possível publicar uma edição censurada
de um livro ou uma edição editada de um jornal em países com códigos de
expressão restritos, mas agora os bits podem fluir facilmente de regiões menos
censoras para mais. Pode ser mais simples publicar apenas uma única versão de um
trabalho à venda em todos os lugares, uma edição que omita informações que em
algum lugar possam provocar um processo.
\index{bits!koans de!``os bits se movem mais rápido que o pensamento''|)}
\index{bits!koans de|)}
\index{koans de bits|)}
\index{ciberespaço|)}
\index{Internet!leis que controlam o fluxo de informação|)}
\index{Internet!velocidade da|)}
\index{koans de bits!``os bits se movem mais rápido que o pensamento''|)}

%%%%%%%%%%%%%%%%%%%%%%%%%%%%%%%%%%%%%%%%
\section{Bem e mal, promessa e perigo}
\label{cap1:exp-dig-perigo}

A explosão digital colocou muitas coisas em jogo, e todos nós temos uma
participação em quem faz a ``captura''. A maneira como a tecnologia nos é
oferecida, como a usamos e as consequências da vasta disseminação de informações
digitais são assuntos não apenas nas mãos de especialistas em
tecnologia. Governos, corporações, universidades e outras instituições sociais
também têm voz ativa. E os cidadãos comuns, aos quais estas instituições prestam
contas, podem influenciar suas decisões. Escolhas importantes são feitas
todos os anos, em escritórios governamentais e de legisladores, em reuniões
municipais e delegacias, nos escritórios corporativos de bancos e companhias de
seguros, nos departamentos de compras de cadeias de lojas e farmácias. Todos nós
podemos ajudar a elevar o nível do discurso e do entendimento. Todos nós podemos
ajudar a garantir que decisões técnicas sejam tomadas em um contexto de padrões
éticos.

Oferecemos duas questões morais básicas. Primeiro, a tecnologia da informação
não é inerentemente nem boa nem má; pode ser usada para o bem ou para o mal,
para nos libertar ou para nos aprisionar. Segundo, novas tecnologias trazem
mudanças sociais, e a mudança vem com riscos e oportunidades. Todos nós, e todas
as nossas agências públicas e instituições privadas, têm algo a dizer sobre se a
tecnologia será usada para o bem ou para o mal e se vamos cair presas aos seus
riscos ou prosperar com as oportunidades que cria.

%%%%%%%%%%%%%%%%%%%%
\subsubsection*{A tecnologia não é boa nem ruim}
\label{cap1:exp-dig-perigo:goodbad}
\index{neutralidade ética das tecnologias de informação|(}
\index{tecnologia da informação!neutralidade ética da|(}
Qualquer tecnologia pode ser usada para o bem ou para o mal; as tecnologias
digitais, em particular, podem ser simultaneamente boas e más. Reações nucleares
criam energia elétrica e armas de destruição em massa. Estes dois usos
compartilham um núcleo comum, mas são de outra forma bastante distintos. Não é
assim no mundo após a explosão digital.

A mesma tecnologia de criptografia que permite que você envie e-mails para seus
amigos com confiança de que nenhum bisbilhoteiro será capaz de decifrar sua
mensagem também possibilita que terroristas planejem ataques sem serem
descobertos. A mesma tecnologia da Internet que facilita a distribuição ampla de
obras educacionais para estudantes pobres em locais remotos também permite a
violação massiva de direitos autorais. As ferramentas de manipulação de fotos
que aprimoram suas imagens também são utilizadas por pornógrafos infantis para
escaparem da justiça.

As mesmas tecnologias podem ser usadas para monitorar indivíduos, rastrear seus
comportamentos e controlar as informações que recebem. Os motores de busca não
precisam retornar resultados imparciais. Muitos usuários de navegadores web não
percebem que os sites que visitam arquivam suas ações. Provavelmente há um
registro exato do que você acessou e quando, enquanto navega por catálogos de
lojas de roupas ou livros, um site que vende produtos farmacêuticos ou um
serviço que oferece conselhos sobre contracepção ou overdose de drogas. Há
oportunidades vastas para usar essas informações para fins invasivos, mas
relativamente benignos, como marketing, e também para fins mais questionáveis,
como listas negras e chantagem.

A chave para gerenciar as consequências éticas e morais da tecnologia, ao mesmo
tempo em que se fomenta o crescimento econômico, é \textbf{regular o uso} da
tecnologia \textbf{sem proibir ou restringir sua criação}.

Poucas regulamentações exigem a divulgação de que as informações estão sendo
coletadas ou restringem o uso que pode ser dado a esses dados. O \ingles{USA
PATRIOT Act} e outras leis federais concedem às agências governamentais ampla
autoridade para vasculhar dados, basicamente inocentes, em busca de sinais de
``atividade suspeita'' de potenciais terroristas --- e também para notar
transgressões menores no processo. Embora a Internet alcance milhões de
residências, as regras e regulamentações que a governam não são muito melhores
do que as de uma cidade sem lei da fronteira do Velho Oeste.
\index{neutralidade ética das tecnologias de informação|)}
\index{tecnologia da informação!neutralidade ética da|)}

%%%%%%%%%%%%%%%%%%%%
\subsubsection*{Novas tecnologias trazem riscos e oportunidades}
\label{cap1:exp-dig-perigo:risco}
\index{tecnologia da informação!riscos versus oportunidades|(}
Os mesmos grandes meios de armazenamento que permitem a qualquer pessoa analisar
milhões de estatísticas de beisebol também permitem que qualquer um com acesso a
informações confidenciais coloque em risco a segurança dessas informações. O
acesso a mapas aéreos pela Internet possibilita que criminosos planejem roubos a
casas de alto padrão, mas policiais tecnologicamente sofisticados sabem que os
registros de tais consultas também podem ser usados para resolver crimes.

\index{Facebook!conseqüências da interconectividade}
Ferramentas de redes sociais como Facebook e Twitter tornaram seus fundadores
bastante ricos e deram origem a muitos milhares de novas amizades, casamentos e
outros empreendimentos. No entanto, a interconectividade também tem efeitos
colaterais inesperados. Uma mulher na Inglaterra descobriu que seu noivo era
casado quando o Facebook sugeriu a esposa dele como alguém que ela poderia
querer como amiga\footnote{``Mum-of-Three Uncovered her Cheating Fiance's Double
Life After His Wife Came Up as a Friend Suggestion on Facebook,'' \textit{The
Sun}, September 7, 2017,
\url{https://www.thesun.co.uk/fabulous/4411305/mum-of-three-uncovered-her-cheating-fiances-double-life-after-his-wife-came-up-as-a-friend-suggestion-on-facebook}.}. E
em 2019, um estudante universitário de Massachusetts
cometeu suicídio pulando do quarto andar de uma garagem de estacionamento, tendo
recebido cerca de 47.000 mensagens de texto, muitas supostamente abusivas, de
sua namorada nos dois meses anteriores. Ela foi acusada de homicídio
involuntário --- o mesmo crime com que poderia ter sido acusada se tivesse
atingido ele com seu carro enquanto dirigia e enviava mensagens de
texto\footnote{Julia Jones, ``Girlfriend Charged in Boston College Student's
Death After Telling Him `Hundreds of Times' to Kill Himself, prosecutors
say,'' \textit{CNN}, October 29, 2019,
\url{https://www.cnn.com/2019/10/28/us/boston-college-student-suicide-charges/index.html}.}. Em
uma nação profundamente comprometida com a livre expressão como um direito
legal, quais males da Internet devem ser crimes e quais são apenas errados?

As vastas redes de dados tornaram possível transferir trabalho para onde as
pessoas estão, em vez de mover pessoas para o trabalho. Os resultados são
enormes oportunidades de negócios para empreendedores que aproveitam essas
tecnologias e novas empresas ao redor do globo, e também o outro lado da moeda:
empregos perdidos para a terceirização.

A diferença que cada um de nós pode fazer, em nosso local de trabalho ou em
outra instituição, pode ser perguntar no momento certo sobre os riscos de alguma
nova inovação tecnológica --- ou apontar a possibilidade de fazer algo no futuro
próximo que há alguns anos teria sido completamente impossível. 

Começamos nosso passeio pela paisagem digital com uma olhada em nossa
privacidade, uma estrutura social que a explosão deixou em pedaços. Enquanto
desfrutamos dos benefícios de informações onipresentes, também sentimos a perda
do abrigo que a privacidade uma vez nos deu. E não sabemos o que queremos
construir em seu lugar. O bem e o mal da tecnologia, e sua promessa e perigo,
estão todos misturados quando informações sobre nós são espalhadas por toda
parte. No mundo pós-privacidade, estamos expostos ao brilho da luz do meio-dia
--- e às vezes isso parece estranhamente agradável.
\index{tecnologia da informação!riscos versus oportunidades|)}