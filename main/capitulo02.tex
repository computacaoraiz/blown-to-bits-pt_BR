%%%%%%%%%%%%%%%%%%%%%%%%%%%%%%%%%%%%%%%%%%%%%%%%%%%%%%%%%%%%%%%%%%%%%%%%%%%%%%%%
\chapter[Despido à Luz do Sol]{Despido à Luz do Sol\\\large\textit{Privacidade
Perdida, Privacidade Abandonada}}
\label{despido}


%%%%%%%%%%%%%%%%%%%%%%%%%%%%%%%%%%%%%%%%
\section{1984 está aqui e nós gostamos disso}
\label{despido:1984}

Os fãs que compareceram ao concerto lotado de Taylor Swift no Rose Bowl na
primavera de 2018 viram-na subir ao palco em uma nuvem de neblina para cantar os
sucessos de \ingles{Reputation}. Enquanto eles entravam ou se misturavam entre
os sets, alguns desses fãs visitaram quiosques de vídeo para assistir a trechos
das performances anteriores e ensaios da estrela, para ter um vislumbre dos
bastidores de um artista favorito. O que eles não sabiam era que o quiosque
também estava os observando. A cabine de vídeo foi equipada com uma câmera que
enviava as imagens de seus visitantes de volta para um ``posto de comando'' em
Nashville, onde um software de reconhecimento facial os escaneava, aparentemente
procurando por correspondências em um banco de dados de pessoas que já haviam
perseguido Swift no passado\footnote{Sopan Deb and Natasha Singer, ``Taylor
Swift Keeping An Eye Out For Stalkers,'' New York Times, December 15, 2018, C6,
\url{https://www.nytimes.com/2018/12/13/arts/music/taylor-swift-facial-recognition.html}.}.
Essas imagens foram mantidas, ou foram excluídas com segurança? Nós não sabemos,
assim como não sabemos quantas outras câmeras nos capturam todos os
dias. Escâneres como o de Swift têm sido vistos nas entradas de arenas
esportivas, salas de concertos e outros locais de entretenimento. O público
muitas vezes fica no escuro quanto à sua existência --- e sobre as políticas
relacionadas ao uso, armazenamento ou compartilhamento das imagens e outros
dados capturados.

O livro \ingles{1984} de George Orwell foi publicado em 1948. Ao longo dos anos
seguintes, o livro se tornou sinônimo de um mundo de vigilância permanente, uma
sociedade desprovida de privacidade e liberdade:

\begin{quote}
\ldots parecia não haver cor em nada, exceto nos cartazes que estavam
colados por toda parte. O rosto de bigode preto olhava para baixo de
todos os cantos dominantes. Havia um na fachada da casa logo em frente.
O GRANDE IRMÃO ESTÁ DE OLHO EM VOCÊ\footnote{George Orwell, 1984 (Signet
Classic, 1977), p. 2.}.
\end{quote}

O verdadeiro ano de 1984 já passou há décadas. Hoje, as telas bidirecionais do
Grande Irmão seriam consideradas brinquedos amadores. A Londres imaginada por
Orwell tinha câmeras por toda parte. A cidade atual tem pelo menos meio milhão
delas. Em todo o Reino Unido, há uma câmera de vigilância para cada dez
pessoas\footnote{Silkie Carlo, ``Britain Has More Surveillance Cameras per
Person Than Any Country Except China. That’s a Massive Risk to Our Free
Society,'' Time, May 17, 2019,
\url{https://news.yahoo.com/britain-more-surveillance-cameras-per-151641361.html}.}.
O londrino médio é fotografado centenas de vezes por dia por olhos 
eletrônicos nas laterais dos prédios e nos postes de utilidade.

No entanto, há muito sobre o mundo digital que Orwell não imaginou. Ele não
previu que as câmeras estariam longe de serem a forma mais presente das
tecnologias de rastreamento atual. Existem dezenas de outras fontes de dados, e
os dados que elas produzem são armazenados e analisados. As empresas de
telefonia celular sabem não apenas os números que você chama, mas também por
onde você levou seu telefone. As empresas de cartão de crédito sabem não apenas
onde você gastou seu dinheiro, mas também no que gastou. Seu amigável banco
mantém registros eletrônicos de suas transações, não apenas para manter seu
saldo correto, mas porque precisa informar ao governo se você faz retiradas
grandes. Quando você vai a um restaurante ou uma loja, um aplicativo que tem
acompanhado silenciosamente sua localização pergunta como você gostou do local,
para alimentar sua resposta em seu mecanismo de recomendação.

A explosão digital espalhou os fragmentos de nossas vidas por toda parte:
registros das roupas que usamos, dos sabonetes com que nos lavamos, das ruas que
percorremos e dos carros que dirigimos e onde os dirigimos. E embora o Grande
Irmão de Orwell tivesse câmeras, ele não tinha mecanismos de busca para juntar
os fragmentos, para encontrar as agulhas nos palheiros. Onde quer que vamos,
deixamos rastros digitais, e computadores de capacidade impressionante
reconstróem nossos movimentos a partir dessas pistas. Computadores reconstituem
as pistas para formar uma imagem abrangente de quem somos, o que fazemos, onde
estamos fazendo isso e com quem estamos discutindo sobre isso. 

Talvez nada disso surpreendesse Orwell. Se ele tivesse conhecido a
miniaturização eletrônica, poderia ter suposto que desenvolveríamos uma
surpreendente variedade de tecnologias de rastreamento. Mas há algo mais
fundamental que distingue o mundo de 1984 do mundo atual. Nos apaixonamos por
esse mundo sempre conectado. Aceitamos a perda de privacidade em troca de
eficiência, conveniência e pequenos descontos. 

As atitudes mudaram na última década. Em um relatório do projeto Pew/Internet de
2007, 60\% dos usuários da Internet afirmaram ``não se preocupar com a quantidade
de informações disponíveis sobre eles on-line'', mas em 2018, essa proporção se
inverteu, e mais de 60\% ``gostariam de fazer mais para proteger sua
privacidade''; apenas 9\% acreditam que têm ``muito controle'' sobre as informações
que são coletadas sobre eles.4 Embora estejamos ficando mais preocupados com a
perda de controle sobre informações pessoais, não temos certeza de que há muito
que possamos fazer a respeito.

